% !TeX encoding = UTF-8
%% \textbf{重庆大学}通用毕业论文\LaTeXe{}模板
%%% 使用前请先阅读使用文档和用户协议,内有详细介绍。Happy Texing! :)
%% =======================================================
\documentclass%
	[type=master, bilinguallist=apart, 
	printmode=oneside, openany]{cquthesis}%
% 可用选项:
% type=[bachelor|master|doctor],      % 必选,毕业论文类型,以下项目不填时为默认
% liberalformat,                      % 可选,仅适用本科生,使用文学类论文标题格式,默认未打开
% proffesionalmaster=[true|false],    % 可选,仅适用研究生,是(true)否(false)专业硕士,默认为否
% printmode=[oneside|twoside|auto],	  % 可选,论文打印方式,默认采用auto按页数要求自动判定
% openany,|openright,                 % 可选,双面打印时每章的第一页仅右页开启,默认右页开启(openright)
% bilinguallist=[off|combined|apart], % 可选,图录表录等分别按双语题注混编(combined),分开编录(apart),默认关(off)
% blindtrail,                         % 可选,盲审模式,开启后封面姓名和致谢部分会隐藏,详情请参阅用户文档,默认关
% draft,                              % 写作期间可选,不渲染图片,关闭外围功能,加快预览速度,默认未开启


% 请在cquthesis.sty文件中定义其他会用到的宏包和自己的变量
% 这样可以防止main.tex太过臃肿。
\usepackage{cquthesis}
 
% 定义所有的图片文件在 figures 子目录下
\graphicspath{{figures/}}

% 定义数字圆
\usepackage{tikz}
\newcommand*\circled[1]{\tikz[baseline=(char.base)]{
            \node[shape=circle,draw,inner sep=1pt] (char) {#1};}}

%*** 写作时,使用这个命令只渲染你想查看的部分,提升工作效率,定稿时注释掉整行
%\includeonly{contents/experiment,contents/analysis,}


\begin{document}

\cqusetup{
%	************	注意	************
%	* 1. \cqusetup{}中不能出现全空的行,如果需要全空行请在行首注释
%	* 2. 不需要的配置信息可以放心地坐视不理、留空、删除或注释(都不会有影响)
%	*
%	********************************
% ===================
%	论文的中英文题目
% ===================
  ctitle = {车载信息物理融合系统建模与优化关键技术研究},
  etitle = {Research on Key Techniques for Vehicular Cyber-Physical Systems},
% ===================
% 作者部分的信息
% \secretize{}为盲审标记点,在打开盲审开关时内容会自动被替换为***输出,盲审开关默认关闭
% ===================
  cauthor = \secretize{许新操},	% 你的姓名,以下每项都以英文逗号结束
  eauthor = \secretize{Xincao~Xu},	% 姓名拼音,~代表不会断行的空格
  studentid = \secretize{},	% 仅本科生,学号
  csupervisor = \secretize{刘~~~~凯~~~~教授},	% 导师的姓名
  esupervisor = \secretize{{Prof.~Kai Liu}},	% 导师的姓名拼音
  cassistsupervisor = \secretize{}, % 本科生可选,助理指导教师姓名,不用时请留空为{}
  cextrasupervisor = \secretize{}, % 本科生可选,校外指导教师姓名,不用时请留空为{}
  eassistsupervisor = \secretize{}, % 本科生可选,助理指导教师或/和校外指导教师姓名拼音,不用时请留空为{}
  cpsupervisor = \secretize{}, % 仅专硕,兼职导师姓名
  epsupervisor = \secretize{},	% 仅专硕,兼职导师姓名拼音
  cclass = \secretize{\rmfamily{2023}\heiti{年}\rmfamily{6}\heiti{月}},	% 博士生和学硕填学科门类,学硕填学科类型
  research_direction = \zihao{3}{车联网},
  edgree = {},	% 专硕填Professional Degree,其他按实情填写
% 提示:如果内容太长,可以用\zihao{}命令控制字号,作用范围:{}内
  cmajor = 工~~~~学,	% 专硕不需填,填写专业名称
  emajor = , % % 专硕不需填,填写专业英文名称
  cmajora = \zihao{3}{计算机科学与技术},
  cmajorb = \zihao{3}{车联网},
  cmajorc = \secretize{尚明生~~~~教授},
%  cmajord = 2023年6月,
% ===================
% 底部的学院名称和日期
% ===================
  cdepartment = ,	%学院名称
  edepartment = ,	%学院英文名称
% ===================
% 封面的日期可以自动生成(注释掉时),也可以解除注释手动指定,例如:二〇一六年五月
% ===================
%	mycdate = {2023年6月},
%	myedate = {June 2023},
}% End of \cqusetup
% ===================
%
% 论文的摘要
%
% ===================
\begin{cabstract}	% 中文摘要

% 随着感知模式、通讯技术和计算范式的发展,传统汽车正朝着智能化、
% 网联化和协同化方向迅速演进。以智能网联汽车为抓手,
% 车联网驱动的智能交通系统(Intelligent Transportation System, ITS)
% 有望实现更安全、高效和可持续发展的交通运输。
% 车载信息物理融合系统(Vehicular Cyber-Physical System, VCPS)是实现ITS应用的基础和关键。
% 然而,车联网高异构、高动态和分布式等特征以及ITS应用的多元化需求给VCPS的实现带来了巨大挑战。
% 首先,面向车联网高动态物理环境,创新的服务架构和高效的数据感知与质量评估模型是VCPS的架构基础和驱动核心。
% 其次,面向车联网分布式异构节点资源,先进的任务调度与资源分配策略是进一步优化VCPS服务质量的技术支撑。
% 再次,面向智能交通系统多元应用需求,创新的服务质量与系统开销均衡策略是实现高质量、低成本和可扩展VCPS的理论保障。最后,面向动态复杂车联网环境,原型系统的设计和实现是针对VCPS必要的验证手段。因此,从架构融合与系统建模、协同资源优化、质量-开销均衡,以及原型系统实现四个方面,对车载信息物理融合系统进行了理论、技术和系统上的协同创新。主要创新成果包括:

% \circled{1} 基于分层车联网架构的车载信息物理融合系统建模。
% 首先,设计了分层车联网服务架构,其融合了软件定义网络和移动边缘计算范式,
% 并最大化逻辑集中控制与边缘分布式服务的协同效应。
% 在此基础上,提出了分布式感知与多源信息融合场景,其中边缘节点融合感知信息并构建逻辑视图。
% 其次,建立了基于多类M/G/1优先队列的信息排队模型,并针对多源信息需求对车载信息物理融合质量进行建模,
% 设计了 Age of View 指标来定量评估视图质量,并形式化定义了VCPS质量最大化问题。
% 再次,提出了基于差分奖励的多智能体深度强化学习(Multi-Agent Difference-Reward-based Deep Reinforcement Learning, MADR)算法,
% 通过确定信息感知频率、上传优先级,以及车与基础设施通信(Vehicle-to-Infrastructure, V2I)带宽,
% 以实现VCPS质量最大化。最后,构建了仿真实验模型并进行了性能评估,证明了 MADR 算法的优越性。

% \circled{2} 面向车载信息物理融合的通信与计算资源协同优化。
% 首先,提出了协同通信与计算卸载场景,其中边缘节点协同调度通信与计算资源来实现VCPS实时任务处理。
% 其次,考虑非正交多址接入(Non-Orthogonal Multiple Access, NOMA)车联网中同一边缘内与不同边缘间的干扰,
% 并建立了V2I传输模型。形式化定义了协同资源优化问题,旨在最大化服务率。
% 再次,提出了基于博弈理论的多智能体深度强化学习(Multi-Agent Game-Theoretic Deep Reinforcement Learning, MAGT)算法,
% 将原问题分解为任务卸载和资源分配两个子问题,其中,任务卸载子问题建模为严格势博弈并实现纳什均衡,资源分配子问题分解为两个独立凸优化问题,
% 并分别利用基于梯度的迭代方法和KKT条件得到最优解,以实现异构资源协同优化。
% 最后,构建了仿真实验模型并进行了性能评估,证明了 MAGT 算法的优越性。

% \circled{3} 面向车载信息物理融合的质量-开销均衡优化。
% 首先,提出了协同感知与 V2I 上传场景,其中车辆进行协同感知与上传,而边缘节点在构建视图时会同时考虑视图质量与开销。
% 其次,考虑边缘视图中多源信息的及时性和一致性,建立了VCPS质量模型。
% 同时,考虑到视图信息冗余度、感知开销以及传输开销,建立了VCPS开销模型。
% 在此基础上,形式化定义了双目标优化问题,以最大化VCPS质量和最小化VCPS开销。
% 再次,提出了基于多目标的多智能体深度强化学习(Multi-Agent Multi-Objective Deep Reinforcement Learning, MAMO)算法,
% 其中系统奖励包含VCPS质量和VCPS利润,并通过决斗评论家网络基于状态价值和动作优势来评估智能体动作,
% 以实现质量-开销均衡。最后,构建了仿真实验模型并进行了性能评估,证明了 MAMO 算法的优越性。

% \circled{4} 面向车载信息物理融合的超视距碰撞预警原型系统设计与实现。
% 首先,提出了超视距(None-Light-of-Sight, NLOS)碰撞预警场景,
% 其中交叉路口的车辆由于视线遮挡而具有潜在碰撞风险。
% 其次,提出了基于车载信息物理融合系统优化的碰撞预警
% (Vehicular Cyber-Physical System Optimization based Collision Warning, VOCW)算法,
% 建立了V2I 应用层传输时延拟合模型,设计了数据包丢失检测机制,通过丢包检测与时延补偿实现更加实时准确的逻辑视图以提高碰撞预警系统性能。再次,构建了仿真实验模型并进行了性能评估,证明了 VOCW 算法的优越性。最后,搭建了基于车载终端和路侧设备的硬件在环试验平台,并进一步在真实的车联网环境中实现了超视距碰撞预警原型系统,并验证了所提系统的可行性与有效性。

\end{cabstract}
% 中文关键词,请使用英文逗号分隔:
\ckeywords{车联网,信息物理融合系统,车载边缘计算,优化算法,多智能体深度强化学习}

\begin{eabstract}	% 英文摘要

% With the development of sensing patterns, communication technologies, and computing paradigms, traditional vehicles are rapidly evolving towards intelligence, networking, and collaboration. By leveraging intelligent connected vehicles as the starting point, the intelligent transportation system (ITS) driven by vehicle-to-everything (V2X) communications is expected to achieve safer, more efficient, and sustainable transportation. The vehicular cyber-physical system (VCPS) is the foundation and key to implement ITS applications. However, the implementation of VCPS faces significant challenges due to the highly heterogeneous, dynamic, and distributed nature of vehicular networks, along with the diverse requirements of ITS applications. First, an innovative service architecture and efficient data sensing and quality evaluation models tailored to the highly heterogeneous and dynamic physical environment of vehicular networks are the architecture foundation and driving force of VCPS. Second, advanced task scheduling and resource allocation towards distributed heterogeneous resources in vehicular networks is the technical support for further optimizing the quality of VCPS services. Third, a novel equilibrium strategy for system quality and cost towards the diversified application demands of ITS is the theoretical guarantee for achieving high-quality, low-cost, and scalable VCPS. Finally, the design and implementation of a prototype system towards the real-world dynamical vehicular network environment is a necessary verification method for VCPS. Therefore, this thesis focuses on the theoretical, technological and systematic innovations of the VCPS from four aspects: architecture integration and system modeling, collaborative resource optimization, quality-cost tradeoff, and prototype system implementation. The main innovative contributions are as follows:

% \circled{1} Vehicular cyber-physical fusion modeling based on vehicular hierarchical architecture. First, this thesis designs a hierarchical service architecture that integrates software defined network and mobile edge computing paradigms to maximize their synergistic effects. Based on this, distributed sensing and heterogeneous information fusion scenarios are proposed, where edge nodes fuse sensing information and construct logical views. Second, this thesis establishes an information queuing model based on multi-class M/G/1 priority queues and models the quality of VCPS for various requirements of heterogeneous information. Specifically, the Age of View metric is designed to quantitatively evaluate the quality of views, and the VCPS quality maximization problem is formulated. Third, a multi-agent difference-reward-based deep reinforcement learning (MADR) algorithm is proposed to achieve VCPS quality maximization. The system state includes vehicle sensing information, edge cached information, and view requirements. The action space of the vehicle includes information sensing frequencies and uploading priorities, while the edge node allocates vehicle-to-infrastructure (V2I) bandwidth according to vehicular predicted trajectories and view requirements. Finally, this thesis constructs a simulation model and gives a comprehensive performance evaluation, which conclusively demonstrates the superiority of the MADR algorithm.

% \circled{2} Cooperative optimization for communication and computing resources in vehicular cyber-physical fusion. First, this thesis proposes a collaborative communication and computing offloading scenario, where edge nodes collaborate to schedule communication and computing resources to achieve real-time task processing for VCPS. Second, this thesis considers intra-edge and inter-edge interferences in non-orthogonal multiple access (NOMA)-based vehicular networks, and establishes a V2I transmission model. The cooperative resource optimization (CRO) problem is formulated to maximize the service ratio for VCPS tasks. Third, a multi-agent game-theoretic deep reinforcement learning (MAGT) algorithm is proposed to achieve cooperative optimization for heterogeneous resources. Specifically, the CRO problem is decomposed into two subproblems, i.e., task offloading and resource allocation. The task offloading subproblem is modeled as an exact potential game and the Nash equilibrium is achieved by the MAGT algorithm. The resource allocation subproblem is decomposed into two independent convex optimization problems and solved by gradient-based iteration methods and KKT conditions, respectively. Finally, this thesis builds the simulation model and gives a comprehensive performance evaluation, which conclusively demonstrates the superiority of the MAGT algorithm.

% \circled{3} Quality-cost tradeoff optimization for vehicular cyber-physical fusion. First, this thesis proposes a collaborative sensing and V2I uploading scenario, in which vehicles perform collaborative sensing and uploading, and edge nodes take into account both the quality and cost of the view when constructing it. Second, this thesis considers the timeliness and consistency of heterogeneous information in logical views and establishes a VCPS quality model. Meanwhile, considering the redundancy of view information, sensing cost, and transmission cost, a VCPS cost model is established. On this basis, a bi-objective optimization problem is formulated to maximize VCPS quality and minimize VCPS cost. Third, a multi-agent multi-objective deep reinforcement learning (MAMO) algorithm is proposed to achieve quality-cost tradeoff. Specifically, the system reward is a one-dimensional vector containing VCPS quality and VCPS profit. The thesis also proposes a dueling critic network to evaluate agent actions based on state-value and action-advantage. Finally, this thesis constructs a simulation model and gives a comprehensive performance evaluation, demonstrating the superiority of the MAMO algorithm.

% \circled{4} Design and implementation of a non-line-of-sight collision warning prototype system. First, this thesis introduces a none-line-of-sight (NLOS) collision warning scenario, where vehicles at a crossroads have potential collision risks due to line-of-sight obstructions. Second, this thesis proposes an application-layer V2I communication delay fitting model and a packet loss detection mechanism, and proposes a vehicular cyber-physical system optimization based collision warning (VOCW) algorithm that achieves real-time and accurate logical view construction via packet loss detection and delay compensation to further improve system performance. Third, this thesis constructs a simulation model and conducts performance evaluation to prove the superiority of the VOCW algorithm. Finally, this thesis builds a hardware-in-the-loop test platform based on onboard units and roadside units and further implements a prototype system for NLOS collision warning in a real-world vehicle network environment, verifying the feasibility and effectiveness of the proposed system.
 
\end{eabstract}
% 英文关键词,请使用英文逗号分隔,关键词内可以空格:
\ekeywords{Vehicular Networks, Cyber-Physical Systems, Vehicular Edge Computing, Optimization Algorithm, Multi-Agent Deep Reinforcement Learning
}

% 封面和摘要配置完成

\frontmatter %%%前置部分(封面后绪论前)
%\cquauthpage[contents/cover1.pdf]
%\cquauthpage[contents/cover2.pdf]
%\cquauthpage[contents/cover3.pdf]
%\cquauthpage[contents/cover4.pdf]

%% 原创声明和授权说明书,可选:用扫描页替换
%\cquauthpage[authscan.pdf]
%\cquauthpage

%% 摘要
\makeabstract

%% 目录,注意需要多次编译才能更新
\setlength{\cftbeforetoctitleskip}{0pt}
\setlength{\cftaftertoctitleskip}{20pt}
\tableofcontents

\setlength{\cftbeforelottitleskip}{0em}
%% 插图索引,可选,如不用可注释掉
%\renewcommand*{\listfigurename}{插图索引}
% \clearpage
% \phantomsection  
% \addcontentsline{toc}{chapter}{插图索引}
% \listoffigures
%\listoffiguresEN
%% 表格索引,可选
%\renewcommand*{\listtablename}{表格索引}
% \clearpage
% \phantomsection  
% \addcontentsline{toc}{chapter}{表格索引}
% \listoftables
% %\listoftablesEN
% %% 公式索引,可选
% %\listofequations
% %\listofequationsEN
% %% 符号对照表,可选
% \clearpage
% \phantomsection 
% \addcontentsline{toc}{chapter}{主要符号对照表}
% \input{contents/denotation}
% %% 缩略语对照表,可选
% \clearpage
% \phantomsection 
% \addcontentsline{toc}{chapter}{缩略语对照表}
% \input{contents/abbreviate}

\mainmatter %%% 主体部分(绪论开始,结论为止)
%* 子文件的多少和内容由你决定(最好以章为单位),基本原则是提速预览、脉络清晰、管理容易。

% 设置字号为小四
\renewcommand{\normalsize}{\fontsize{12pt}{20pt}\selectfont}
% 设置小四正文行间距为 20 磅
\setstretch{1.312}

% \chapter[\hspace{0pt}绪\hskip\ccwd{}论]{{\CJKfontspec{SimHei}\zihao{3}\hspace{-5pt}绪\hskip\ccwd{}论}}
\section[\hspace{-2pt}引言]{{\CJKfontspec{SimHei}\zihao{-3} \hspace{-8pt}引言}}\label{section 1-1}

随着国民经济和社会的高速发展,以及人们对美好生活的不懈追求,汽车已成为人们日常生活中不可或缺的交通工具之一。
据统计,截至2021年底,我国民用汽车保有量高达30151万辆 \cite{gou2022zhong}。然而,汽车数量的急剧增长也给人类社会和自然环境带来了许多挑战。据世卫组织数据,全球每年约有130万人因道路交通事故死亡,另外约有2000至5000万人因事故受到非致命伤害,如致残等 \cite{shi2022dao}。同时,日益严峻的城市交通拥堵问题也给经济发展造成了巨大损失。此外,汽车也是空气污染物排放的主要贡献者之一,仅在2021年,全国汽车污染物排放总量就超过了1401.9万吨 \cite{shen2022zhong}。近年来,随着传感模式、通信技术和计算范式的发展,传统汽车正在向智能化、网联化和协同化方向迅猛发展。以智能网联汽车(Intelligent Connected Vehicle, ICV)为抓手,车联网(Internet of Vehicles, IoV)驱动的智能交通系统(Intelligent Transportation System, ITS)正致力于实现更加安全、高效和可持续发展的下一代交通运输。

近年来,车联网及其推动的智能网联汽车和智能交通系统已上升为我国的重要战略。
2019年9月,国务院发布了《交通强国建设纲要》,提出要加强智能网联汽车的研发,通过新基建形成自主可控的车联网核心技术和生态产业链 \cite{zhong2019jiao}。2020年2月,国家发改委等11个部委联合发布了《智能汽车创新发展战略》,明确指出发展智能网联汽车对我国具有重要战略意义,并将突破关键核心技术作为首要战略任务 \cite{guo2020zi}。2022年8月,科技部发文支持建设包括智能港口、智能矿山和自动驾驶在内的十个新一代智能示范应用场景\cite{ke2022ke}。同时,车联网商业化也是业界关注的热点领域。2019年7月,华为发布了首款采用第五代移动通信(The 5th Generation Mobile Communication, 5G)技术的车载通信模组 MH5000,并与一汽、上汽、广汽等18家车企共同成立\qthis{5G汽车生态圈},加速5G技术在汽车产业的商业进程。2020年10月,超过100家包括传统汽车制造商、芯片模组与硬件制造商、地图与定位服务提供商在内的相关企业,在中国上海开展了蜂窝车联网(Cellular-Vehicle-to-Everything, C-V2X)\qthis{新四跨}(跨芯片模组、跨终端、跨整车和跨安全平台)应用示范活动。截至2023年2月,已有包括一汽、上汽、广汽、通用、比亚迪和蔚来等十余家车企推出了C-V2X量产车型。

国内外许多一流高校和科研机构围绕车联网、车路协同、无人驾驶、智能交通系统等领域展开了深入探索与研究。
清华大学汽车安全与节能国家重点实验室的李克强院士团队在智能网联汽车\qthis{中国方案}技术体系的提出和推动方面做出了重要贡献\cite{wang2023design, li2017dynamical, zheng2016stability}。中科院复杂系统管理与控制国家重点实验室的王飞跃院士团队在智能交通的信息物理融合方面取得了显著进展\cite{li2023sharing, liu2021cyber, lv2021guest}。无线移动通信国家重点实验室的陈山枝博士团队致力于C-V2X标准的制定和关键技术的研究,极大推动了车联网的产业化进程\cite{chen2023cellular, chen2020a, chen2017vehicle}。西安电子科技大学综合业务网理论与关键技术国家重点实验室的毛国强教授团队在车联网的高效数据分发、实时感知、ITS应用等方面取得了具有国际影响力的科研成果\cite{zhang2022new, hao2022dhcloc, yue2022towards}。深圳大学Victor C.M Leung 教授团队专注于车联网边缘缓存、任务卸载和资源分配等领域的研究,并取得了系列重要科研成果\cite{sun2023federated, ju2023joint, wang2022efficient}。长安大学赵祥模教授团队在高速公路场景下智能车路协同体系架构以及相关运行安全性与适应性评估技术方面做出了重要贡献\cite{fang2022a, fang2022on, jing2022integrated}。

国际上,加拿大滑铁卢大学Sherman Shen 教授团队在车联网安全\cite{chen2022adaptive}、车路协同\cite{liu2022real}和资源优化\cite{li2022cost}等领域取得了重要的研究突破。
瑞典奥斯陆大学Yan Zhang教授团队在车联网边缘计算\cite{dai2022adaptive}、内容缓存\cite{zhang2022digital}和资源分配\cite{sun2022dynamic}等领域做出了突出的贡献。香港理工大学Jiannong Cao 教授团队在车联网边缘计算\cite{yang2022delegating}、计算卸载\cite{dai2023a}和数据分发\cite{yang2020efficient}等领域取得了重要的研究成果。澳大利亚悉尼大学Abbas Jamalipour 教授团队在面向下一代网络中车联网通信\cite{qi2022energy}、感知\cite{iranmanesh2022a}和计算\cite{alam2022multi}等方面取得了重要的研究突破。美国休斯敦大学Zhu Han 教授团队在车联网中安全\cite{khan2023federated}、无线资源分配\cite{zhang2023mean}以及博弈论应用\cite{kang2021joint}等领域展开了深入研究并取得了系列重要成果。加拿大卡尔顿大学F.Richard Yu 教授团队在智能网联汽车中网络安全\cite{alladi2023ambient, liang2023a, bai2022detection}等领域进行了深入研究,并取得了重要的科研成果。香港理工大学Song Guo 教授团队在车联网边缘智能\cite{wang2922imitation, ren2021blockchain, wang2022design}等领域做出了突出贡献。日本东北大学Nei Kato教授团队在车联网中安全\cite{tang2020future}、智能反射面\cite{zhu2022intelligent}和边缘计算\cite{liu2020smart}等领域进行了全面深入的研究,并获得了系列重要科研成果。香港中文大学Guoliang Xing 教授团队在自动驾驶\cite{he2021vi}和信息物理融合系统(Cyber-Physical System, CPS)\cite{shi2022vips}等领域取得了重要研究成果。

2006年,美国国家科学基金会启动了信息物理融合系统研究计划\cite{nfs2006cps},为CPS领域的发展提供了重要支持和契机,自此以后,CPS领域在全球范围内得到了广泛的关注和发展\cite{lee2016introduction}。
2011年,LI等人\cite{li2011human}首次将信息物理融合系统应用于车联网中,车载信息物理融合系统(Vehicular Cyber-Physical System, VCPS)\cite{xia2019zi}
 已成为国内外学术界热门的研究领域之一。车载信息物理融合系统利用智能网联汽车的多模态感知能力、车联网通信技术以及端边云的计算、存储和通信资源,形成集智能网联汽车、车联网、边缘计算、云计算等多种技术于一体的综合系统,并实现感知、计算、传输和控制的一体化。然而,车联网具有网络异构高动态、节点资源动态分布、ITS应用需求多元、真实环境复杂等特点,实现车载信息物理融合系统仍然面临巨大挑战。首先,未来车联网是多计算范式、服务架构共存的高动态网络,融合不同范式并最大化其协同效应,在此基础上,融合异质感知信息并评估其质量是 VCPS 的架构基础与驱动核心。其次,车联网中节点资源异构且受限,实现异构资源协同优化以最大化资源利用率是进一步优化 VCPS 服务质量的技术支持。再次,多元化ITS应用对VCPS的质量/开销需求具有差异性,实现VCPS质量-开销均衡是实现高质量低成本可扩展VCPS的理论保障。最后,在动态复杂车联网环境中,设计和实现基于VCPS的典型应用是验证VCPS的必要手段。因此,本文将结合车联网高异构、高动态、分布式特征和智能交通系统的多样化应用需求,从架构融合与系统建模、异构资源协同优化、质量-开销均衡,以及原型系统实现方面进行理论、技术和系统上的综合创新。
i
\section[\hspace{-2pt}研究背景]{{\CJKfontspec{SimHei}\zihao{-3} \hspace{-8pt}研究背景}}\label{section 1-2}

本章节将首先介绍车联网的相关概念及其发展历程。接着,以智慧全息路口为例,介绍车载信息物理融合系统,并分析其中所面临的挑战。

\begin{figure}[h]
	\centering
	\captionsetup{font={small, stretch=1.312}}
\includegraphics[width=1\columnwidth]{Fig1-1-V2X.pdf}
	\bicaption[车联网演进方向]{车联网演进方向}[Evolution direction of the Internet of Vehicles]{Evolution direction of the Internet of Vehicles}
	\label{fig 1-1}
\end{figure}

车联网是物联网(Internet of Things, IoT)技术在汽车领域的应用形式。早在2G/3G移动网络时代,车联网已应用于利用全球导航卫星系统(Global Navigation Satellite System, GNSS)的定位信息为车辆提供防盗和救援服务。如今,智能网联汽车(如宝马、比亚迪、福特、通用、蔚来以及特斯拉等)都支持空中下载(Over-the-Air, OTA)技术对车机系统进行在线更新。如图 \ref{fig 1-1} 所示,随着汽车朝着智能化、网联化、协同化方向发展,传统的面向信息服务的\qthis{车联网}已经转变为与万物互联互通的V2X(Vehicle-to-Everything)车联网。具体而言,V2X车联网是指多种通讯方式的融合,包括车辆间通讯(Vehicle-to-Vehicle, V2V)、车辆与行人通讯(Vehicle-to-Pedestrian, V2P)、车辆与基础设施通讯(Vehicle-to-Infrastructure, V2I)以及车辆与云端通讯(Vehicle-to-Cloud, V2C)。车联网利用实时数据分发,实现人、车、路等交通要素的协同配合,最终实现\qthis{聪明的车、智慧的路、协同的云}。此外,车联网还能促进基于单车智能的自动驾驶技术发展,通过车联网通信协助自动驾驶发现潜在危险,提升道路安全。随着我国车联网产业在政策规划、标准体系建设、关键技术研发、应用示范和基础设施建设等多方面的稳步发展,车联网的内涵和外延也在不断发展演进。依托快速落地的新型基础设施建设,车联网不仅广泛服务于智能网联汽车的辅助驾驶、自动驾驶等不同应用,还拓展服务于智慧矿山、智慧港口等企业生产环节以及智慧交通、智慧城市等社会治理领域\cite{zhong2021che}。

\begin{figure}[h]
	\centering
	\captionsetup{font={small, stretch=1.312}}
\includegraphics[width=0.95\columnwidth]{Fig1-2-V2X-evolution.pdf}
	\bicaption[3GPP C-V2X 标准演进]{3GPP C-V2X 标准演进}[3GPP C-V2X standard evolution]{3GPP C-V2X standard evolution}
	\label{fig 1-2}
\end{figure}

在车联网通信标准方面,电气和电子工程师协会(Institute of Electrical and Electronics Engineers, IEEE)在2003年提出了专用短距通信技术(Dedicated Short-Range Communication, DSRC)。2010年,IEEE发布了名为无线接入车载环境(Wireless Access in Vehicular Environments, WAVE)的协议栈,其中包括IEEE 802.11p、IEEE 1609.1/.2/.3/.4协议族和SAE J2735消息集字典 \cite{wu2013vehicular}。同时,基于长期演进(Long-Term Evolution, LTE)的V2X通信已形成完善的技术标准体系和产业链\cite{chen2016lte}。此外,IMT-2020(5G)推进组成立了C-V2X工作组,加速基于5G的V2X通信的演进。如图 \ref{fig 1-2} 所示,国际标准组织第三代合作伙伴计划(The 3rd Generation Partnership Project, 3GPP)在2018年启动基于5G新空口(New Radio, NR)的V2X标准研究,并在2020年完成了Rel-16版本的5G NR-V2X标准\cite{saad2021advancements},Rel-17版本进一步优化了功率控制、资源调度等相关技术。5G 汽车协会(5G Automotive Association, 5GAA)、下一代移动通信网络(Next Generation Mobile Network, NGMN)联盟以及5G Americas对IEEE 802.11p和C-V2X进行了技术对比,表\ref{table 1_1}显示C-V2X在传输时延、范围、速率以及可靠性等方面具有显著优势。目前,我国LTE-V2X产业蓬勃发展,与DSRC技术路线之争取得了重大进展。我国已建成基于LTE-V2X技术的完备产业链,芯片、模组、车载终端(Onboard Unit, OBU)、路侧设备(Roadside Unit, RSU)等均已成熟且经过了\qthis{三跨}\qthis{四跨}\qthis{新四跨}以及大规模测试,满足了商用部署条件。

\begin{table}[h]\small
\setstretch{1.245} %设置具有指定弹力的橡皮长度(原行宽的1.2倍)
\captionsetup{font={small, stretch=1.512}}
\centering
\bicaption[C-V2X和IEEE 802.11p技术对比]{C-V2X和IEEE 802.11p技术对比\cite{cheng2021feng}}[Technical comparisons of C-V2X and IEEE 802.11p]{Technical comparisons of C-V2X and IEEE 802.11p \cite{cheng2021feng}}
\label{table 1_1}
\resizebox{\columnwidth}{!}{%
\begin{tabular}{@{}ccccc@{}}
\toprule
\begin{tabular}[c]{@{}c@{}}C-V2X\\ 技术优势\end{tabular} &
 \begin{tabular}[c]{@{}c@{}}具体技术\\ 或性能\end{tabular} &
IEEE 802.11p &
\begin{tabular}[c]{@{}c@{}}LTE-V2X\\ (3GPP R14/R15)\end{tabular} &
\begin{tabular}[c]{@{}c@{}}NR-V2X\\ (3GPP R16)\end{tabular} \\ \midrule
低时延 &
  时延 &
  不确定时延 &
  \begin{tabular}[c]{@{}c@{}}R14: 20ms\\ R15: 10ms\end{tabular} &
  3ms \\ 
\begin{tabular}[c]{@{}c@{}}低时延/\\ 高可靠\end{tabular} &
  \begin{tabular}[c]{@{}c@{}}资源分配\\ 机制\end{tabular} &
  CSMA/CA &
  \begin{tabular}[c]{@{}c@{}}支持感知+半持续\\ 调度和动态调度\end{tabular} &
  \begin{tabular}[c]{@{}c@{}}支持感知+半持续\\ 调度和动态调度\end{tabular} \\ 
\multirow{3}{*}[3.4ex]{高可靠} &
  可靠性 &
  不保证可靠性 &
  \begin{tabular}[c]{@{}c@{}}R14: \textgreater{}90\%\\ R15: \textgreater{}95\%\end{tabular} &
  支持99.999\% \\  
 &
  信道编码 &
  卷积码 &
  Turbo &
  LDPC \\ 
 &
  重传机制 &
  不支持 &
  \begin{tabular}[c]{@{}c@{}}支持HARQ,\\ 固定2次传输\end{tabular} &
  \begin{tabular}[c]{@{}c@{}}支持HARQ,\\ 传输次数灵活,\\ 最大支持32次传输\end{tabular} \\ 
\multirow{2}{*}[0ex]{\begin{tabular}[c]{@{}c@{}}更远传输\\ 范围\end{tabular}} &
  通信范围 &
  100m &
  \begin{tabular}[c]{@{}c@{}}R14: 320m\\ R15: 500m\end{tabular} &
  1000m \\  
 &
  波形 &
  OFDM &
  \begin{tabular}[c]{@{}c@{}}单载波频分复用\\ (SC-FDM)\end{tabular} &
  循环前缀(CP)-OFDM \\ 
\multirow{2}{*}[1.5ex]{\begin{tabular}[c]{@{}c@{}}更高传输\\ 速率\end{tabular}} &
  \begin{tabular}[c]{@{}c@{}}数据传输\\ 速率\end{tabular} &
  典型6Mbit/s &
  \begin{tabular}[c]{@{}c@{}}R14: 约30Mbit/s\\ R15: 约300Mbit/s\end{tabular} &
  \begin{tabular}[c]{@{}c@{}}与带宽有关,40MHz\\ 时R16单载波2层数据\\ 传输支持约400Mbit/s,\\ 多载波情况下更高\end{tabular} \\ 
 &
  调制方式 &
  64QAM &
  64QAM &
  256QAM \\ \bottomrule
\end{tabular}%
}
\end{table}

\begin{figure}[h] 
	\centering
	\captionsetup{font={small, stretch=1.312}}\includegraphics[width=1\columnwidth]{Fig1-3-intersection.pdf}
	\bicaption[基于车载信息物理融合的智慧全息路口]{基于车载信息物理融合的智慧全息路口}[Intelligent holographic intersection based vehicular cyber-physical fusion]{Intelligent holographic intersection based vehicular cyber-physical fusion}
	\label{fig 1-3}
\end{figure}

如图 \ref{fig 1-3} 所示,智慧全息路口是基于车载信息物理融合技术的智慧交通管理系统。它通过将城市道路上的全要素进行数字化还原,为各类智能交通系统应用提供数据支撑。智慧全息路口利用道路基础设施和智能网联汽车上搭载的激光雷达、毫米波雷达、摄像头等多源传感设备,对路口进行全方位感知和全要素采集。通过传感设备实时感知交通流量、车速、车道变化等数据,并结合高精度地图呈现路口数字化上帝视角,精准刻画路口上的每一条车道、每一个交通信息灯的状态,以及每一辆车的行为轨迹。在实现路口全要素数字化还原的过程中,采用车载边缘计算技术将异构感知数据在边缘节点进行融合处理,从而提高数据处理速度,降低数据传输成本。同时,利用目标检测、目标跟踪、行为分析等算法对感知数据进行预处理,进一步提高数据准确性和精度,为后续的交通管理、交通安全和交通规划等应用提供更可靠的数据基础。

智慧全息路口不仅可以实现路口全要素数字化还原,而且可以进一步作为车载信息物理融合系统的外在展示和数据内核,支撑各种智能交通系统的应用。例如,全息路口可以为公交优先通行、绿波通行、弱势道路参与者(Vulnerable Road User, VRU)感知等 ITS 应用提供强有力的支持。在公交优先通行方面,全息路口可以根据公交车的实时位置和行驶速度,并结合路口的交通状况,提前调整信号控制策略,使公交车获得更好的通行效率和服务。在绿波通行方面,全息路口可以通过感知路口的交通流量和车速等信息,实现路口绿灯时长的自适应调整,从而实现车辆在绿波通信路段上的高效通行。在 VRU 感知方面,全息路口可以利用摄像头等传感设备或V2P通信感知VRU(如行人、自行车、残疾人等)的存在和行动轨迹,提供实时的路口状态信息和预警信息,保障弱势道路参与者的通行便利和交通安全。

通过以上讨论可知,车载信息物理融合系统是实现各类ITS应用的基础。然而,在高异构、高动态、分布式的车联网中,实现车载信息物理融合系统以满足多元ITS应用需求仍然面临着诸多问题和挑战。因此,针对以上问题和挑战,需要进一步展开全面深入的研究。具体地,首先,异构车联网亟需服务架构融合创新,并设计车载信息物理融合质量指标。未来车联网是多服务范式并存的高异构移动网络,因此需要研究异构车联网融合服务架构,以最大化不同服务范式的协同效应,并支持VCPS的部署实现。同时,现有研究都没有对VCPS进行整体深入的评估。因此,需要设计基于多源异质感知信息融合的VCPS质量指标,并通过控制车辆感知行为与资源分配提升VCPS系统质量。其次,异构资源亟需协同优化。车联网中的通信和计算资源分布在不同的车辆和基础设施中,因此需要针对异构资源进行协同优化,支持VCPS中计算任务分布式处理,以进一步提升系统服务质量。再次,车载信息物理融合系统亟需质量-开销均衡优化。车载信息物理融合系统需要在保证实时性和准确性的同时,考虑资源开销和能耗问题。因此,需要研究质量-开销均衡的优化策略,以提高系统的资源利用率的同时降低能耗。最后,亟需实现原型系统以验证VCPS性能。通过在真实车联网环境中搭建原型系统,可以进一步验证车载信息物理融合系统的可行性和有效性,为其在实际应用中提供更可靠的支持和保障。

\section[\hspace{-2pt}国内外研究现状]{{\CJKfontspec{SimHei}\zihao{-3} \hspace{-8pt}国内外研究现状}}\label{section 1-3}

车载信息物理融合系统是实现各类智能交通系统应用的基础,其已成为国内外学术界研究的热点之一。本章节将对国内外相关研究工作进行梳理和总结,并从以下几个方面进行详细阐述:

\subsection[\hspace{-2pt}车联网服务架构研究与现状]{{\CJKfontspec{SimHei}\zihao{4} \hspace{-8pt}车联网服务架构研究与现状}}

随着智能交通系统应用的不断涌现,传统车联网架构已无法满足大规模、高可靠、低时延的需求,因此研究人员正致力于将软件定义网络(Software Defined Network, SDN)新范式应用于车联网中。在软件定义网络中,数据平面是网络中的实际数据传输部分,其负责处理网络流量的传输和转发,而控制平面是网络中的智能部分,负责决策数据包的转发路径和流量管理。SDN通过分离数据平面和控制平面,实现了高度灵活的数据调度策略和网络功能虚拟化(Network Functions Virtualization, NFV)。LIU等人\cite{liu2016cooperative}首次将SDN概念应用于车联网中,提出了软件定义车联网(Software Defined Vehicular Network, SDVN),并在此基础上提出了基于混合V2I/V2V通信的在线协同数据调度算法,以提高数据分发的性能。DAI等人\cite{dai2018cooperative}设计了基于SDN的异构车联网中具有时间约束的时态信息服务调度方案。LUO等人\cite{luo2018sdnmac}提出了基于SDN的媒体接入(Media Access Control, MAC)协议以提高车联网的通信性能。LIU等人\cite{liu2018coding}提出了基于SDN的服务架构,并结合车辆缓存和网络编码来提高带宽利用率。ZHANG等人\cite{zhang2022ac-sdvn}设计了解决SDVN中视频组播安全问题的安全访问控制协议,实现了多播视频请求车辆和RSU的身份认证。ZHAO等人\cite{zhao2022elite}提出了基于智能数字孪生技术的软件定义车联网分层路由方案,克服了SDVN架构中高动态拓扑局限性。LIN等人\cite{lin2023alps}研究了基于SDVN的自适应链路状态感知方案,能够在信标间隔内及时获取链路状态信息,减少数据包丢失。AHMED等人\cite{ahmed2023deep}提出了基于SDVN中车辆传感器负载均衡算法,并提出了数据包级入侵检测模型,可以跟踪并有效识别网络攻击。然而,现有大部分工作都仅是从数据分发、路由缓存、网络安全等方面展开了研究,缺乏对整体架构的深入分析。

移动边缘计算(Mobile Edge Computing, MEC)\cite{mao2017a}通过将计算、存储和网络资源靠近移动终端设备,提供更快速、更可靠的服务,同时减少网络流量消耗和服务延迟。越来越多的研究考虑将MEC范式应用在车联网环境中以提高系统实时性、可靠性和安全性。LIU等人\cite{liu2017a}首次将移动边缘计算融入车联网中,提出了车载边缘计算(Vehicular Edge Computing, VEC),并集成了不同类型的接入技术,以提供低延迟和高可靠性的通信。LANG等人\cite{lang2022cooperative}设计了基于区块链技术的协同计算卸载方案,以提高VEC资源的利用效率,并确保计算卸载的安全性。LIU等人\cite{liu2021fog}研究了端边云协同架构中的合作数据传播问题,并提出了基于Clique的算法来联合调度网络编码和数据分发。DAI等人\cite{dai2021edge}设计了基于自适应比特率多媒体流的VEC架构,其中边缘节点给以不同质量等级编码的文件块提供缓存和传输服务。ZHANG等人\cite{zhang2022digital}提出了车载边缘缓存技术,动态调整边缘节点和车辆的缓存能力以提高服务的可用性。LIU等人\cite{liu2020adaptive}提出了两层VEC架构,利用云、静态边缘节点和移动边缘节点来处理时延敏感性任务。LIAO等人\cite{liao2021learning}研究了空地一体的VEC任务卸载策略,其中车辆能够学习具有多维意图的长期策略。LIU等人\cite{liu2023mobility}提出了利用车辆计算资源来提高VEC场景下任务执行效率的计算任务卸载方案。LIU等人\cite{liu2023asynchronous}研究了VEC中任务卸载和资源管理的联合优化问题,并采用异步深度强化学习算法来寻找最优解。然而,上述研究缺乏考虑异构车联网中不同服务架构的协同效应。

\subsection[\hspace{-2pt}车载信息物理融合系统建模研究与现状]{{\CJKfontspec{SimHei}\zihao{4} \hspace{-8pt}车载信息物理融合系统建模研究与现状}}

越来越多研究人员聚焦于车载信息物理融合系统的预测、调度和控制技术,旨在有效提高 VCPS 系统的整体性能和可靠性。在预测技术方面,ZHANG等人 \cite{zhang2021a} 提出了基于 VCPS 架构的车辆速度曲线预测方法,其协同 VCPS 中的不同控制单元来完成速度曲线预测。ALBABA等人 \cite{albaba2021driver} 则结合深度Q网络(DQN)和层次博弈论,对高速公路驾驶场景中的驾驶员行为进行预测,其中k级推理被用来模拟驾驶员的决策过程。ZHANG等人 \cite{zhang2020data} 提出了变道行为预测模型和加速预测模型。在此基础上,对车辆状态进行预测,并通过动态路由算法实现车辆之间的协同合作,以优化资源利用率和降低能源消耗。ZHOU等人 \cite{zhou2021wide} 提出了基于宽-注意力和深度-组合模型用于交通流量预测。其中,宽-注意力模块和深度-组合模块分别用于提取全局关键特征和推广局部关键特征。在调度技术方面,LI等人 \cite{li2020cyber} 考虑了车辆移动性,并开发了基于物理比率-K干扰模型的广播方案,以确保通信的可靠性。LIAN等人 \cite{lian2021cyber} 提出了基于既定地图模型路径规划的调度方法,以优化路径利用效率。在控制技术方面,HU等人 \cite{hu2017cyber} 提出了基于车队头车状态的燃油最优控制器,以优化车辆速度和无级变速箱齿轮比。DAI等人 \cite{dai2016a} 提出了自主交叉路口控制机制,以确定车辆通过交叉路口的优先权。LV等人 \cite{lv2018driving} 提出了用于三种典型驾驶方式下控制车辆加速的自适应算法。上述研究集中于支持 VCPS 的不同技术,如轨迹预测、路径调度和车辆控制等,虽然促进了各种 ITS 应用的实施,但是均建立在高质量物理元素建模信息可用的假设基础上,并未对车载信息物理融合质量定量分析。

部分研究工作侧重于利用深度强化学习(Deep Reinforcement Learning, DRL)优化 VCPS 中车辆感知和信息融合。DONG等人 \cite{dong2020spatio} 提出了基于 DQN 的方法,融合本地环境信息并实现可靠的车道变更决策。ZHAO等人 \cite{zhao2020social} 设计了基于近似策略优化(Proximal Policy Optimization, PPO)的社会意识激励机制,以得出最佳的长期车辆感知策略。MIKA等人 \cite{mlika2022deep} 提出了基于深度确定性策略梯度(Deep Deterministic Policy Gradient, DDPG)的解决方案,通过调度资源块和广播覆盖来优化信息时效性。然而,上述算法不能直接应用于 VCPS 中的协同感知和异构信息融合,并且,当考虑到多辆车场景时,上述算法并不适用。另一方面,部分研究对 VCPS 中的信息质量进行了评估。LIU等人 \cite{liu2014temporal} 提出了用于 VCPS 中时态数据传播的调度算法,其在实时数据传播和及时信息感知之间取得了平衡。DAI等人 \cite{dai2019temporal} 提出了进化多目标算法,以提高信息质量和改善数据到达率。LIU等人 \cite{liu2014scheduling} 提出了两种在线算法,通过分析传播特性来调度不同一致性要求下的时态数据传播。RAGER等人 \cite{rager2017scalability} 开发了刻画真实网络随机性的框架,对随机数据负载进行建模以提高信息质量。YOON等人 \cite{yoon2021performance} 提出了车联网合作感知框架,考虑到通信损耗和车辆随机运动,以获得车辆的精确运动状态。上述研究主要聚焦于 VCPS 中数据及时性、准确性或一致性方面的信息质量评估。然而,这些研究仅考虑了同质数据项层面的质量评估,没有针对车载信息物理融合进行质量评估。

\subsection[\hspace{-2pt}车联网资源分配与任务卸载研究与现状]{{\CJKfontspec{SimHei}\zihao{4} \hspace{-8pt}车联网资源分配与任务卸载研究与现状}}

车联网中的资源分配一直是学术界的研究热点 \cite{noor-a-rahim2022a},大量研究人员针对车联网中通信资源分配进行了深入研究。HE等人 \cite{he2022meta} 设计了动态车联网资源管理框架,其采用马尔可夫决策过程(Markov Decision Process, MDP)和分层强化学习相结合的方法,可以显著提高资源管理性能。LU等人 \cite{lu2021user} 提出了基于用户行为的虚拟网络资源管理方法,以进一步优化车联网通信。PENG等人 \cite{peng2020deep} 提出了车联网资源管理方案,通过应用 DDPG 方法解决了多维资源优化问题,实现了资源快速分配,并满足了车联网服务质量(Quality of Service, QoS)要求。WEI等人 \cite{wei2022multi} 针对车联网云计算中的资源分配问题,从提供者和用户双重视角出发,提出了改进的 NSGA-II 算法来实现多目标优化。PENG等人 \cite{peng2021multi} 研究了无人机辅助车联网中的多维资源管理问题,并提出了基于多智能体深度确定性策略梯度(Multi-Agent Deep Deterministic Policy Gradient, MADDPG)的分布式优化方法,实现了车辆资源联合分配。为了进一步提高频谱利用率和支持更多车辆接入,部分研究将非正交多址接入(Non-Orthogonal Multiple Access, NOMA)技术融入车联网中。PATEL等人 \cite{patel2021performance} 评估了基于 NOMA 的车联网通信容量,其数值结果显示,NOMA 通信容量比传统的正交多址接入(Orthogonal Multiple Access, OMA)高出约20\%。ZHANG等人 \cite{zhang2021centralized} 利用基于图的匹配方法和非合作博弈(Non-Cooperative Game, NCG)分布式功率控制,为NOMA车联网开发了集中式两阶段资源分配策略。ZHU等人 \cite{zhu2021decentralized} 考虑随机任务到达和信道波动,提出了最优功率分配策略,以最大化长期的功率消耗和延迟。LIU等人 \cite{liu2019energy} 在基于 NOMA 的车联网中,提出了基于交替方向乘子法(Alternating Direction Method of Multipliers, ADMM)的功率分配算法。然而,上述研究主要是基于单边缘节点的情况,无法处理不同边缘节点之间的相互干扰情况。因此,仍然需要探索更加复杂的多边缘节点环境下的资源分配策略,以提高车联网的性能和可靠性。


随着车载边缘计算的发展,大量研究专注于 VEC 中的任务卸载和资源分配。其中,LIU等人\cite{liu2021rtds}提出了多周期任务卸载的实时分布式方法,通过评估 VEC 中的移动性感知通信模型、资源感知计算模型和截止时间感知奖励模型。SHANG等人\cite{shang2021deep}研究了节能的任务卸载,并开发了基于深度学习的算法来最小化能耗。为了最小化执行延迟、能源消耗和支付成本的加权和,LIU等人\cite{liu2022a}提出了结合 ADMM 和粒子群优化(Particle Swarm Optimization, PSO)的任务卸载算法。CHEN等人\cite{chen2020robust}设计了带有故障恢复功能的计算卸载方法,以减少能源消耗并缩短服务时间。为了实现超高可靠低时延通信(ultra-Reliable and Low-Latency Communication, uRLLC)服务需求下最大化吞吐量,PAN等人\cite{pan2022asynchronous}提出了基于异步联合 DRL 的计算卸载方案。ZHU等人\cite{zhu2022a}提出了用于智能反射面(Intelligent Reflecting Surface, IRS)辅助下的 VEC 的动态任务调度算法,优化有限资源分配并考虑了车辆的移动模式、传输条件和任务大小以及并发传输之间的相互干扰。此外,部分研究聚焦于采用多智能体强化学习(Multi-Agent Deep Reinforcement Learning, MADRL)算法的任务卸载和资源分配。ALAM等人\cite{alam2022multi}开发了基于 DRL 的多智能体匈牙利算法,用于 VEC 中的动态任务卸载以满足延迟、能耗和支付费用需求。ZHANG等人\cite{zhang2021adaptive}提出了基于 MADDPG 的边缘资源分配方法,在严格延迟约束下最小化车辆任务卸载成本。为了同时满足严格延迟要求和最小带宽消耗,HE等人\cite{he2021efficient}提出了用于车辆带宽分配的多智能体行动者-评论家(Multi-Agent Actor-Critic, MAAC)算法。然而,以上研究工作都没有考虑实时任务卸载和通信/计算资源分配的协同优化。

部分研究专注于VEC的联合通信和计算资源分配。CUI等人\cite{cui2021reinforcement}提出了多目标强化学习方法,通过协同通信和计算资源的分配来减少系统延迟。HAN等人\cite{han2020reliability}设计了基于动态规划(Dynamic Programming, DP)的资源分配方法,实现了耦合车辆通信和计算资源的可靠性计算。XU等人\cite{xu2021socially}采用契约理论为每个潜在的内容供应商和内容请求者分配通信和计算资源。少数研究者研究了联合任务卸载和资源分配。DAI等人\cite{dai2021asynchronous}提出了异步的DRL算法,实现了异构服务器数据驱动的任务卸载。此外,DAI等人 \cite{dai2022a}开发了概率计算卸载方法,根据边缘节点的计算分配概率进行计算卸载的独立调度。NIE等人\cite{nie2021semi}提出了在无人机辅助VEC中基于MADRL算法的联合优化资源分配和功率控制策略。然而,现有研究主要基于集中式调度,通信开销和调度复杂性较高,不适用于大规模的车联网。

\subsection[\hspace{-2pt}车载信息物理融合质量/开销优化研究与现状]{{\CJKfontspec{SimHei}\zihao{4} \hspace{-8pt}车载信息物理融合质量/开销优化研究与现状}}

近年来,许多研究人员致力于提高车载信息物理融合中的 QoS,以提升 ITS 应用的用户体验。其中,WANG等人\cite{wang2016offloading}提出了一种组合优化方法,旨在减少移动数据流量的同时满足 VCPS 中面向 QoS 的服务需求。JINDAL等人\cite{jindal2018sedative}提出了基于 SDN 和深度学习的 VCPS 网络流量控制方案,成功解决了网络流量管理问题。ZHU等人\cite{zhu2022joint}设计了基于双时间尺度 DRL 的方法,以优化基于车辆编队的 VCPS 中的车辆间距和通信效率,同时满足 V2I 通信的 QoS 要求。WANG等人\cite{wang2023a}提出了集群式车辆通信方法,通过公交车聚类和混合数据调度实现了从公交车到普通车辆的有效数据传播并满足了严格和个性化的 QoS 需求。此外,CHEN等人\cite{chen2021qos}致力于解决 IRS 辅助车联网中的频谱共享问题,通过优化车辆的发送功率、多用户检测矩阵、频谱重用以及 IRS 反射系数等参数,提高车联网通信的服务质量。LAI等人\cite{lai2017a}提出了基于 SDN 的流媒体传输方法,根据用户移动信息、播放缓冲区状态和当前网络信号强度向 SDN 控制器提供流媒体传输策略,以实现最小延迟和更好的 QoS。TIAN等人\cite{tian2022multiagent}则设计了基于 MADRL 的资源分配框架,以共同优化信道分配和功率控制,满足车联网中的异构 QoS 需求。同时,ZHANG等人\cite{zhang2020hierarchical}研究了 MEC 车联网中联合分配频谱、计算和存储资源问题,并利用 DDPG 解决该问题,以满足 ITS 应用的 QoS 要求。SODHRO等人\cite{sodhro2020ai}建立了可靠和延迟容忍的无线信道模型和多层边缘计算驱动的框架,有效提升了车联网服务质量。

另一方面,部分研究人员致力于降低 VCPS 中的各类开销。ZHAO等人\cite{zhao2021a}设计了基于 SDN 和无人机(Unmanned Aerial Vehicle, UAV)辅助的车辆计算卸载优化框架,其中采用了UAV辅助车辆计算成本优化算法以最小化系统成本。ZHANG等人\cite{zhang2019hybrid}提出了基于蚁群优化和三个变异算子的算法,用于优化具有灵活时间窗口的多目标车辆路径,以最小化行驶成本和车辆固定成本。NING等人\cite{ning2020when}则针对5G车联网中无线频谱有限的问题,构建了智能卸载框架,联合利用蜂窝频谱和未许可频谱来满足车辆需求,并在考虑时延限制的基础上使成本最小化。TAN等人\cite{tan2019twin}提出了基于人工智能(Artificial Intelligence, AI)的多时间尺度框架的联合通信、缓存和计算策略,其中考虑了车辆的移动性和硬服务截止期限约束,并实现了最大化网络成本效益。HUI等人\cite{hui2022collaboration}提出了协作自动驾驶方案,并通过联盟博弈机制来确定最佳车辆分簇,以最小化每个簇的计算成本和传输成本。虽然上述研究对 VCPS 系统中的开销进行了深入研究,但这些研究并未考虑车载信息物理融合系统构建的质量和开销。因此,需要进行对 VCPS 系统本身的评估与质量-开销均衡的深入研究。

\subsection[\hspace{-2pt}智能交通系统安全相关应用研究与现状]{{\CJKfontspec{SimHei}\zihao{4} \hspace{-8pt}智能交通系统安全相关应用研究与现状}}

随着城市化进程的加速和交通流量的不断增加,ITS 安全相关应用的部署可以大幅提高道路交通的安全性。因此,许多研究人员针对驾驶员状态监测、驾驶行为分析、交通监测等方面进行了研究。MUGABARIGIRA等人\cite{mugabarigira2023context}提出了基于车辆行为追踪和驾驶风险分析的导航系统,可提高城市道路上车辆的安全性。CHANG等人\cite{chang2018design}提出了基于可穿戴智能眼镜的疲劳驾驶检测系统,能够实时检测驾驶员的疲劳或嗜睡状态。DUTTA等人\cite{dutta2022design}提出了基于凸优化的鲁棒分布式状态估计系统,可保护连接车辆的传感器数据免受拒绝服务(Denial-of-Service, DoS)或虚假数据注入(False Data Injection, FDI)攻击。WANG等人\cite{wang2021deep}提出了基于深度学习加速器嵌入式平台的鲁棒雨滴检测系统,并利用检测结果自动控制汽车雨刷。SUN等人\cite{sun2022toward}提出了有效的交通估计系统,可通过与过往车辆通信并记录其出现情况来实现自动交通测量,为ITS提供关键信息。

部分研究工作从车辆控制、车辆编队控制、路口交通流控制等多个层面对 ITS 安全相关应用展开了深入分析。ZHANG等人\cite{zhang2021data}提出了分布式安全巡航控制系统,利用历史数据建立了车辆行为预测模型和动态驾驶系统模型,并设计了考虑合并行为概率的安全跟驰控制策略。ZHAO等人\cite{zhao2022resilient}提出了具有鲁棒性的车辆编队控制系统,并设计了在多重干扰和 DoS 攻击下恢复机制,降低 DoS 攻击对 VCPS 的不利影响。PAN等人\cite{pan2023privacy}设计了面向车联网的车队隐私保护集结控制系统,通过采样数据的动态加密和解密方案,使得车队之间的通信数据得以保密。LI等人\cite{li2021confidenitality}介绍了低延迟协作安全车辆编队数据传输系统,采用无线电信道相关性的协作密钥协商协议以保证数据传输的安全。KAMAL等人\cite{kamal2021control}提出了多智能体路口交通流控制系统,利用随机梯度方法计算交通信号灯持续时间。LIAN等人\cite{lian2021cyber}提出了基于交通控制的智能物流系统,并设计了改进A*路径规划算法实现主动调度。

作为典型 ITS 安全相关应用,车辆碰撞预警已引起广泛研究人员的关注。目前,大多数车辆碰撞预警系统都是基于超声波雷达或激光雷达等测距传感器的。SONG等人\cite{song2018real}提出了实时障碍物检测和状态分类方法,该方法融合了立体摄像头和毫米波雷达,并结合车辆运动模型,通过多个模块感知环境,能够准确快速地判断出\qthis{潜在危险}物体。WU等人\cite{wu2019series}提出了77GHz车辆碰撞预警雷达系统短程天线,该系统采用补丁阵列天线作为基本结构,并采用多层板设计技术使其尺寸更小。然而,这些方案都存在非视距(Non-Line-Of-Sight, NLOS)的问题,即在障碍物遮挡情况下基于视距(Line-Of-Sight, LOS)的方法不再适用。近年来,随着计算机视觉的发展,一些研究集中在基于摄像头实时视频流的碰撞检测上。WANG等人\cite{wang2016vision}提出了车辆制动行为检测方法,利用安装在挡风玻璃上的摄像头来捕获前车信息,以避免与前方车辆相撞。SONG等人\cite{song2018lane}提出了轻量级的基于立体视觉的车道检测和分类系统,以实现车辆的横向定位和前向碰撞预警。然而,基于计算机视觉的方法需要大量数据传输和密集计算,这使得系统的性能无法得到实时响应。另一方面,部分研究考虑了通过 V2X 通信实现碰撞预警。HAFNER等人\cite{hafner2013cooperative}提出了基于 V2V 通信的交叉路口车辆协同防撞分布式算法。GELBAR等人\cite{gelbal2017elastic}提出了基于 V2X 通信的车辆碰撞预警和避免系统。然而,无线通信中的传输时延和数据包丢失等内在特征是不可避免的,对于车辆碰撞预警系统也是不可忽视的。这使得在真实复杂车联网环境中实现实时和可靠的安全关键型服务变得更加困难。

\section[\hspace{-2pt}研究目标与研究内容]{{\CJKfontspec{SimHei}\zihao{-3} \hspace{-8pt}研究目标与研究内容}}\label{section 1-4}

\subsection[\hspace{-2pt}研究目标]{{\CJKfontspec{SimHei}\zihao{4} \hspace{-8pt}研究目标}}

本文针对车联网高动态物理环境、车联网分布式异构节点资源、智能交通系统多元应用需求,以及动态复杂车联网环境所带来的挑战,从架构融合与系统建模、协同资源优化、质量-开销均衡,以及原型系统实现四个方面对车载信息物理融合系统展开研究。本文研究目标概述如下:

\circled{1} 针对车联网高异构、高动态、高分布式等特征,提出融合软件定义网络和移动边缘计算的车联网分层服务架构,并实现视图质量的量化评估,是车载信息物理融合系统的架构基础与驱动核心。首先,结合软件定义网络、网络功能虚拟化和网络切片(Network Slicing, NS)等关键思想,提出车联网分层服务架构,以支持 VCPS 的部署与实现。其次,提出基于多类 M/G/1 优先队列的感知信息排队模型。进一步,针对边缘视图对于感知信息的时效性、完整性以及一致性需求,设计 VCPS 质量指标,并形式化定义视图质量优化问题。最后,提出基于差分奖励的多智能体强化学习视图质量优化策略,实现高效实时的边缘视图构建。

\circled{2} 针对车联网中异构节点资源、动态拓扑结构与无线通信干扰等特征,实现基于边缘协同的异构资源优化,是进一步优化 VCPS 服务质量的技术支撑。首先,面向 NOMA 车联网的车载边缘计算环境,考虑 V2I 通信中同一边缘内的干扰和不同边缘间的干扰,提出 V2I 传输模型,并考虑边缘协作提出任务卸载模型。其次,形式化定义协同资源优化问题,并将其分解为任务卸载与资源分配两个子问题。最后,提出基于博弈理论的多智能体强化学习算法的资源优化策略,基于多智能体强化学习实现任务卸载博弈的纳什均衡,并基于凸优化理论提出最优资源分配方案,实现最大化资源利用效率。

\circled{3} 针对多元智能交通系统应用对于视图质量/开销的差异性需求,实现车载信息物理融合质量-开销均衡,是实现高质量低成本车载信息物理融合的理论保障。首先,考虑视图中信息的及时性与一致性需求,建立车载信息物理融合质量模型。其次,考虑视图构建中感知信息的冗余度、感知开销与传输开销,建立车载信息物理融合开销模型。最后,提出基于多目标多智能体强化学习的质量与开销均衡策略,实现高质量低成本可扩展车载信息物理融合。

\circled{4} 针对动态复杂车联网环境中验证车载信息物理融合的需求,设计并实现基于车载信息物理融合的原型系统,是验证车载信息物理融合的必要手段。首先,提出基于车载信息物理融合系统优化的碰撞预警算法。其次,搭建基于 C-V2X 设备的硬件在环测试平台,实现硬件在环性能验证。最后,在真实车联网环境中,实现基于车载信息物理融合的超视距碰撞预警原型系统,进一步验证所提算法和系统模型的可行性和有效性。

\subsection[\hspace{-2pt}研究内容]{{\CJKfontspec{SimHei}\zihao{4} \hspace{-8pt}研究内容}}

本文致力于研究车载信息物理融合系统,主要研究内容及关系如图 \ref{fig 1-4} 所示。
首先,面向车联网高动态物理环境,融合不同的计算范式与服务架构,并实现有效的数据获取与建模评估是车载信息物理融合的架构基础与驱动核心。
因此,本文将首先研究如何设计融合软件定义网络和移动边缘计算的车联网分层服务架构,在此基础上,研究如何评估并提高车载边缘侧所构建的逻辑视图质量。
其次,面向车联网分布式异构节点资源,高效的任务调度与资源分配是车载信息物理融合的技术支撑。
因此,本文将研究如何实现异构资源协同优化,提高资源利用效率。
面向智能交通系统多元应用需求,实现车载信息物理融合质量-开销均衡是车载信息物理融合的理论保障。
因此,本文将进一步研究车载信息物理融合质量/开销模型及其均衡优化策略。
最后,面向动态复杂车联网环境,基于车载信息物理融合设计并实现具体系统原型是车载信息物理融合的验证手段。
因此,本文将更进一步设计及实现基于车载信息物理融合的超视距碰撞预警原型系统,实现理论与系统的相互促进和迭代。
本文主要研究内容概述如下:

\begin{figure}[h] 
	\centering
	\captionsetup{font={small, stretch=1.312}}\includegraphics[width=1\columnwidth]{Fig1-4-content.pdf}
	\bicaption[主要研究内容及关系]{主要研究内容及关系}[Main research content and relationship]{Main research content and relationship}
	\label{fig 1-4}
\end{figure}

\circled{1} 基于分层车联网架构的车载信息物理融合系统建模。
考虑车联网环境中的网络资源的高异构性、车联网物理环境分布式时变性、拓扑结构的高动态性,以及车辆节点感知能力差异性等关键特征,本文将研究融合软件定义网络和移动边缘计算的分层车联网服务架构。进一步,本文将重点研究基于分层服务架构的分布式感知与多源信息融合模型,考虑信息的多维需求,研究车载信息物理融合质量指标设计。在此基础上,研究基于差分奖励的多智能体强化学习(Multi-Agent Difference-Reward-based Deep Reinforcement Learning, MADR)算法的边缘视图优化策略。

\circled{2} 面向车载信息物理融合的通信与计算资源协同优化。
考虑车联网高动态环境与高异构分布式资源,本文将引入NOMA技术提升车联网频谱资源利用效率,并提出基于边缘协同的异构资源优化策略。本文将重点研究 V2I 传输与任务卸载模型,并在此基础上,研究基于博弈理论的多智能体深度强化学习(Multi-Agent Game-Theoretic Deep Reinforcement Learning, MAGT)算法的异构资源协同优化策略,研究基于凸优化理论的通信资源最优分配策略,并研究任务卸载势博弈模型的纳什均衡策略。

\circled{3} 面向车载信息物理融合的质量-开销均衡优化。
考虑智能交通系统中多元应用需求,本文将研究车联网中不同交通要素的视图质量与开销模型,并提出车载信息物理融合质量-开销均衡优化策略。本文将综合考虑视图的建模质量,包括信息的及时性与一致性,研究车载信息物理融合质量模型,并考虑视图的构建开销,包括信息冗余度、感知开销与传输开销,研究车载信息物理融合开销模型。在此基础上,研究基于多目标的多智能体深度强化学习(Multi-Agent Multi-Objective Deep Reinforcement Learning, MAMO)算法的车载信息物理融合质量-开销均衡优化策略。

\circled{4} 超视距碰撞预警原型系统设计与实现。
考虑动态复杂车联网环境,本文将研究基于车载信息物理融合系统的超视距碰撞预警原型系统设计与实现。具体地,本文将研究C-V2X应用层时延拟合模型和数据丢包检测机制,并研究基于车载信息物理融合系统优化的碰撞预警(Vehicular Cyber-Physical System Optimization based Collision Warning, VOCW)算法。在此基础上,研究基于 C-V2X 通信设备的硬件在环试验平台搭建方案,并研究在真实车联网环境中基于车载信息物理融合的超视距碰撞预警原型系统实现方案。

\section[\hspace{-2pt}论文的特色与创新之处]{{\CJKfontspec{SimHei}\zihao{-3} \hspace{-8pt}论文的特色与创新之处}}\label{section 1-6}

区别于目前仅专注于车联网通信协议、服务架构、资源分配和智能应用等方面的研究,本文旨在从实际需求出发,分析当前面临的挑战,并在车载信息物理融合系统的四个方面进行深入研究:架构基础与驱动核心、技术支撑、理论保障与验证手段。本文的具体特色在于:
a) 针对车联网高动态物理环境和信息感知的时效性与准确性需求,考虑到感知信息时变性、车辆节点移动性和感知能力差异性所带来的挑战,研究如何将基于SDN的集中控制和基于移动边缘计算的分布式调度有机结合,并在边缘侧建立有效的逻辑视图,为车载信息物理融合系统提供架构基础和驱动核心。
b) 针对车联网分布式异构节点资源,考虑节点异构资源的动态性、分布性和无线通信中边缘内和边缘间干扰所带来的挑战,研究如何实现边缘协同,最大化异构资源利用效率,为车载信息物理融合系统提供技术支撑。
c) 针对智能交通系统多元应用需求,考虑到车联网中不同交通要素视图质量和开销需求差异所带来的挑战,研究如何实现车载信息物理融合系统的质量-开销均衡,为车载信息物理融合系统提供理论保障。
d) 针对动态复杂车联网环境,考虑基于真实C-V2X通信设备部署和实现原型系统所带来的挑战,研究基于车载信息物理融合的超视距碰撞预警系统的原型设计和实现,为车载信息物理融合系统提供系统验证。
本文的主要创新点概述如下:

\circled{1} 提出融合软件定义网络与移动边缘计算的车联网分层服务架构,并定义边缘视图概念,率先设计视图评估指标并建立视图质量评估模型,提出分布式信息感知与多源信息融合的边缘视图构建机制:现有车联网服务架构相关研究主要关注于单一范式的实践应用,并不适用于具有大规模数据服务需求的下一代车联网场景,无法支撑车载信息物理融合系统等新兴智能交通系统应用。同时,现有研究重点关注于针对单一类型的时态数据建模与调度,难以面向车载信息物理融合系统形成有效的数据支撑。因此,本文首先综合考虑高移动数据节点、高动态网络拓扑、高异构通信资源、高分布式系统环境等车联网特征,设计基于 SDN 集中控制与基于 MEC 分布式服务有机结合的异构车联网架构。在此基础上,综合考虑感知信息的时效性、完整性与一致性,定义车联网边缘视图概念,建立针对视图质量的量化评估模型,并提出基于差分奖励的多智能体强化学习的边缘视图优化策略,实现车载边缘计算环境下的有效信息物理融合。

\circled{2} 提出基于边缘协同的异构资源协同优化策略,打破传统的单一资源优化模式:现有面向车联网资源优化策略的研究主要集中于单一资源(如通信、计算)的优化,难以满足车联网节点在不同任务中对异构资源的需求。因此,本文首先针对协同资源优化问题进行分解为任务卸载与通信资源分配两个子问题。进一步,提出基于博弈理论的多智能体深度强化学习的协同资源优化策略。具体地,将任务卸载子问题建模为势博弈模型,并证明其具有纳什均衡存在性与收敛性。最后,针对任务卸载博弈,提出基于多智能体深度强化学习的任务卸载策略。对于通信资源分配,提出基于凸优化的通信资源分配策略,实现最大化异构资源利用效率。

\circled{3} 定义车载信息物理融合系统质量与开销模型,提出基于多目标强化学习的优化策略,该策略注重实现 VCPS 质量最大化的同时同时满足 VCPS 开销最小化的要求:现有研究主要关注于基于车载信息物理融合系统的应用,而忽略了车载信息物理融合的质量与开销。因此,本文首先面向多元智能交通系统应用的差异性需求,针对车联网中不同要素建立视图模型。进一步,提出面向车联网中不同实体要素视图的质量/开销模型。最后,提出基于多目标多智能体深度强化学习的车载信息物理融合系统质量-开销均衡优化策略,以实现高质量、低成本和可扩展的车载信息物理融合。

\circled{4} 设计并实现面向车载信息物理融合的超视距碰撞预警原型系统,并在真实车联网环境下验证所提算法与系统模型:现有研究主要关注于基于仿真平台的实验验证,难以满足基于车载信息物理融合的实际 ITS 应用在真实车联网环境下的验证需求。因此,本文首先建立基于C-V2X的无线传输时延拟合模型。进一步,提出数据包丢失检测机制,并设计基于车载信息物理融合系统优化的碰撞预警算法。最后,搭建基于C-V2X设备的硬件在环试验平台,并在真实车联网环境中实现超视距碰撞预警系统原型,验证车载信息物理融合的可行性和有效性。

\section[\hspace{-2pt}论文的组织结构]{{\CJKfontspec{SimHei}\zihao{-3} \hspace{-8pt}论文的组织结构}}\label{section 1-7}
本文围绕车载信息物理融合系统相关问题展开了研究。
具体地,本文将结合车联网高异构、高动态、分布式特征与智能交通系统多元需求,从车联网的架构融合与系统建模、资源协同优化、质量-开销均衡,以及原型系统实现方面进行理论研究与技术创新。
本文共分为六个章节,详细内容安排如下:

第一章,绪论。首先,介绍了车载信息物理融合系统的研究背景和国内外相关研究现状。其次,阐述了本文的研究目标与详细内容。最后,总结了本文的组织结构。

第二章,基于分层车联网架构的车载信息物理融合系统建模。首先,设计了融合软件定义网络和移动边缘计算的分层服务架构,并提出了分布式感知与多源信息融合场景。在此基础上,设计了 Age of View 指标, 并形式化定义了车载信息物理融合质量最大化问题。其次,提出了基于差分奖励的多智能体深度强化学习的视图质量优化策略。最后,构建了实验仿真模型并验证了所提指标与算法的优越性。本章相关研究已经发表在2019年 IEEE Communications Magazine(中科院 SCI 1区),并已经投稿于 IEEE Transactions on Intelligent Transportation Systems(中科院 SCI 1区)。

第三章,面向车载信息物理融合的通信与计算资源协同优化。首先,提出了协同通信与计算卸载场景。其次,建立了V2I传输模型和任务卸载模型,在此基础上,形式化定义了协同资源优化问题。再次,提出了基于博弈理论的多智能体强化学习的资源优化策略。最后,建立了实验仿真模型并验证了所提算法的优越性。本章相关研究已经发表在2021年电子学报(CCF T1类)和2023年 Journal of Systems Architecture(中科院 SCI 2区)。

第四章,面向车载信息物理融合的质量-开销均衡优化。首先,提出了协同感知与V2I上传场景。其次,建立了VCPS系统质量和系统开销模型,在此基础上,形式化定义了最大化系统质量与最小化系统开销的双目标优化问题。再次,提出了基于多目标的多智能体深度强化学习的质量-开销均衡策略。最后,构建了实验仿真模型并验证了所提算法的优越性。本章相关研究已经投稿于 IEEE Transactions on Consumer Electronics(中科院 SCI 2区)。

第五章,超视距碰撞预警原型系统设计及实现。首先,提出了超视距碰撞预警场景。其次,设计了基于视图修正的车辆碰撞预警算法。再次,搭建了基于 C-V2X 设备的硬件在环试验平台。最后,在真实车联网环境中,实现了基于车载信息物理融合的超视距碰撞预警系统原型,验证了车载信息物理融合的可行性与有效性。本章相关研究已经发表在2020年 Mobile Networks and Applications(中科院SCI 3区)。

第六章,总结与展望。总结了全文研究内容,并讨论了后续研究计划。
% \include{contents/architecture}
% \include{contents/resource}
% \chapter[\hspace{0pt}面向车载信息物理融合的质量-开销均衡优化]{{\CJKfontspec{SimHei}\zihao{3}\hspace{-5pt}面向车载信息物理融合的质量-开销均衡优化}}
\removelofgap
\removelotgap
本章将研究面向车载信息物理融合的质量-开销均衡优化。
具体内容安排如下:
\ref{section 4-1} 节是本章的引言,介绍了车联网中车载信息物理融合系统的研究现状及存在的不足,同时阐述本章的主要贡献。
\ref{section 4-2} 节阐述了协同感知与V2I上传场景。
\ref{section 4-3} 节给出了系统模型的详细描述。
\ref{section 4-4} 节形式化定义了最大化VCPS质量并最小化VCPS开销的双目标优化问题。
\ref{section 4-5} 节设计了基于多目标的多智能体深度强化学习算法。
\ref{section 4-6} 节搭建了实验仿真模型并进行了性能验证。
\ref{section 4-7} 节对本章的研究工作进行总结。

\section[\hspace{-2pt}引言]{{\CJKfontspec{SimHei}\zihao{-3} \hspace{-8pt}引言}}\label{section 4-1}

新兴感知技术、无线通信和计算模式推动了现代新能源汽车和智能网联汽车的发展。现代汽车中装备了各种车载感知器,以增强车辆的环境感知能力 \cite{zhu2017overview}。另一方面,V2X通信\cite{chen2020a}的发展使车辆、路侧设备和云端之间的合作得以实现。同时,车载边缘计算\cite{dai2021edge}是很有前途的范式,可以实现计算密集型和延迟关键型的智能交通系统应用 \cite{zhao2022foundation}。这些进展都成为了开发车载信息物理融合系统的强大驱动力。具体来说,通过协同感知和上传,车联网中的物理实体,如车辆、行人和路侧设备等,可以在边缘节点上构建为相应的逻辑映射。

车载信息物理融合中的检测、预测、规划和控制技术被广泛研究。大量工作聚焦于检测技术,例如雨滴数量检测\cite{wang2021deep}和驾驶员疲劳检测\cite{chang2018design}。针对车辆状态预测方法,研究人员提出了混合速度曲线预测\cite{zhang2019a}、车辆跟踪\cite{iepure2021a}和加速预测\cite{zhang2020data}等。同时,部分研究工作提出了不同的调度方案,例如基于物理比率-K干扰模型的广播调度\cite{li2020cyber}和基于既定地图模型的路径规划\cite{lian2021cyber}。此外,部分研究集中在智能网联车辆的控制算法上,例如车辆加速控制\cite{lv2018driving}、交叉路口控制\cite{chang2021an}和电动汽车充电调度\cite{wi2013electric}。这些关于状态检测、轨迹预测、路径调度和车辆控制的研究促进了各种ITS应用的实施。然而,这些工作忽略了感知和上传开销,假设高质量可用信息可以在VEC中构建。少数研究考虑了VCPS中的信息质量,例如时效性\cite{liu2014temporal, dai2019temporal}和准确性\cite{rager2017scalability, yoon2021performance},但上述研究都没有考虑通过协同感知和上传,在VCPS中实现高质量低成本的信息物理融合。

本章旨在通过车辆协同感知与上传,构建基于车载信息物理融合的逻辑视图,并进一步在最大化车载信息物理融合质量和最小化视图构建开销方面寻求最佳平衡。然而,实现这一目标面临着以下主要挑战。首先,车联网中的信息高度动态,因此考虑感知频率、排队延迟和传输时延的协同效应,以确保信息的新鲜度和时效性是至关重要的。其次,物理信息是具有时空相关性的,不同车辆在不同的时间或空间范围内感应到的信息可能存在冗余或不一致性。因此,具有不同感知能力的车辆有望以分布式方式合作,以提高感知和通信资源的利用率。再次,物理信息在分布、更新频率和模式方面存在异质性,这给构建高质量视图带来很大挑战。最后,高质量的视图构建需要更高的感知和通信资源开销,这也是需要考虑的关键因素。综上所述,通过协同感知和上传,实现面向车载边缘计算的高质量、低开销视图具有重要意义,但也具有一定的挑战性。

本章致力于研究车载信息物理融合系统的质量-开销均衡优化问题,并通过协同感知与上传实现高质量、低开销的视图建模。本章的主要贡献如下:第一,提出了协同感知与V2I上传场景,考虑视图的及时性和一致性,设计了车载信息物理融合质量指标,并考虑边缘视图构建过程中信息冗余度、感知开销和传输开销,设计了车载信息物理融合开销指标。进一步,提出了双目标优化问题,在最大化VCPS质量的同时最小化VCPS开销。第二,提出了基于多目标的多智能体深度强化学习算法。具体地,在车辆和边缘节点中分别部署智能体,车辆动作空间包括感知决策、感知频率、上传优先级和传输功率分配,而边缘节点动作空间是V2I带宽分配策略。同时,设计了决斗评论家网络(Dueling Critic Network, DCN),其根据状态价值(State-Value, SV)和动作优势(Action-Advantage, AA)评估智能体动作。系统奖励是一维向量,其中包含VCPS质量和VCPS利润,并通过差分奖励信用分配得到车辆的个人奖励,进一步通过最小-最大归一化得到边缘节点的归一化奖励。第三,建立了基于现实世界车辆轨迹的仿真实验模型,并将MAMO与三种对比算法进行比较,包括随机分配、分布式深度确定性策略梯度\cite{barth2018distributed},以及多智能体分布式深度确定性策略梯度。此外,本文设计了两个指标,即单位开销质量(Quality Per Unit Cost, QPUC)和单位质量利润(Profit Per Unit Quality, PPUQ)用于定量衡量算法实现的均衡。仿真结果表明,与其他算法相比,MAMO在最大化QPUC和PPUQ方面更具优势。

\section[\hspace{-2pt}协同感知与 V2I 上传场景]{{\CJKfontspec{SimHei}\zihao{-3} \hspace{-8pt}协同感知与 V2I 上传场景}}\label{section 4-2}

本章节介绍了协同感知与V2I上传场景。如图\ref{fig 4-1}所示,车辆配备各种车载感知器,如超声波雷达、激光雷达、光学相机和毫米波雷达,可以对环境进行感知。通过车辆间协同地感知,可以获得多源信息,包括其他车辆、弱势道路参与者、停车场和路边基础设施的状态。这些信息可用于在边缘节点中建立视图模型,并进一步用于支撑各种ITS应用,如自动驾驶\cite{bai2022hybrid}、智慧路口控制系统\cite{hadjigeorgious2023real},以及全息城市交通流管理\cite{wang2023city}。逻辑视图需要融合车联网中物理实体的不同模式信息,以更好地反映实时物理车辆环境,从而提高ITS的性能。然而,构建高质量的逻辑视图可能需要更高的感知频率、更多的信息上传量以及更高的能量消耗。

\begin{figure}[h]
\centering
  \captionsetup{font={small, stretch=1.312}}\includegraphics[width=1\columnwidth]{Fig4-1-architerture.pdf}
  \bicaption[协同感知与 V2I 上传场景]{协同感知与 V2I 上传场景}[Cooperative sensing and V2I uploading scenario]{Cooperative sensing and V2I uploading scenario}
  \label{fig 4-1}
\end{figure} 

本系统的工作流程如下:首先,车辆感知并排队上传不同物理实体的实时状态。接着,边缘节点将V2I带宽分配给车辆,同时,车辆确定传输功率。物理实体的视图是基于从车辆收到的多源信息进行融合建立的。需要注意的是,在该系统中,多源信息是由车辆以不同的感知频率感应到的,因此上传时的新鲜度会不同。虽然增加感知频率可以提高新鲜度,但会增加排队延迟和能源消耗。此外,多个车辆可能感知到特定物理实体的信息,若由所有车辆上传,则可能会浪费通信资源。因此,为了提高资源利用率,需要有效而经济地分配通信资源。在此基础上,为了最大化面向车载边缘计算的视图的VCPS质量并最小化VCPS开销,必须量化衡量边缘节点构建的视图的质量和开销,并设计高效经济的协同感知和上传的调度机制。

\section[\hspace{-2pt}车载信息物理融合质量/开销模型]{{\CJKfontspec{SimHei}\zihao{-3} \hspace{-8pt}车载信息物理融合质量/开销模型}}\label{section 4-3}

\subsection[\hspace{-2pt}基本符号]{{\CJKfontspec{SimHei}\zihao{4} \hspace{-8pt}基本符号}}

本系统离散时间片的集合用$\mathbf{T}=\left\{1,\ldots,t,\ldots, T \right\}$表示。
多源信息集合用$\mathbf{D}$表示,其中信息$d \in \mathbf{D}$的特征是三元组$d=\left(\operatorname{type}_d, u_d, \left|d\right| \right)$,其中$\operatorname{type}_d$、$u_d$和$\left|d\right|$分别是信息类型、更新间隔和数据大小。
$\mathbf{V}$表示车辆的集合,每个车辆$v\in \mathbf{V}$的特征是三元组$v=\left (l_v^t, \mathbf{D}_v, \pi_v \right )$,其中$l_v^t$、$\mathbf{D}_v$和$\pi_v$分别是位置、感知的信息集和传输功率。
对于$d \in \mathbf{D}_v$,车辆$v$的感知开销(即能耗)用$\phi_{d, v}$表示。
用$\mathbf{E}$表示边缘节点的集合,其中每个边缘节点$e \in \mathbf{E}$的特征是$e=\left (l_e, g_e, b_e \right)$,其中$l_{e}$、$r_{e}$和$b_{e}$分别为位置、通信范围和带宽。
车辆$v$与边缘节点$e$之间的距离表示为$\operatorname{dis}_{v, e}^t \triangleq \operatorname{distance} \left (l_v^t, l_e \right ), \forall v \in \mathbf{V}, \forall e \in \mathbf{E}, \forall t \in \mathbf{T}$。
在时间$t$内处于边缘节点$e$的通信覆盖范围内的车辆集合表示为$\mathbf{V}_e^t=\left \{v \vert \operatorname{dis}_{v, e}^t \leq g_e, \forall v \in \mathbf{V} \right \}, \mathbf{V}_e^t \subseteq \mathbf{V}$。

感知决策指示器表示车辆$v$在时间$t$是否感知信息$d$,其用以下方式表示:
\begin{equation}
	c_{d, v}^t \in \{0, 1\}, \forall d \in \mathbf{D}_{v}, \forall v \in \mathbf{V}, \forall t \in \mathbf{T}
	\label{equ 4-1} 
\end{equation}
那么,车辆$v$在时间$t$的感应信息集合表示为 $\mathbf{D}_v^t = \{ d | c_{d, v}^{t} = 1, \forall d \in \mathbf{D}_v \}, \mathbf{D}_v^t \subseteq \mathbf{D}_v$。
对于任何信息$d \in \mathbf{D}_v^t$来说,信息类型都是不同的, 即$\operatorname{type}_{d^*} \neq \operatorname{type}_{d}, \forall d^* \in \mathbf{D}_v^t \setminus \left\{ d\right \}, \forall d \in \mathbf{D}_v^t$。
车辆$v$在时间$t$的信息$d$的感知频率用$\lambda_{d, v}^t$表示,其需要满足车辆$v$的感应能力要求。
\begin{equation}
	\lambda_{d, v}^{t} \in [\lambda_{d, v}^{\min} , \lambda_{d, v}^{\max} ], \ \forall d \in \mathbf{D}_v^t, \forall v \in \mathbf{V}, \forall t \in \mathbf{T}
\end{equation}
其中$\lambda_{d, v}^{\min}$和$\lambda_{d, v}^{\max}$分别是车辆$v$中信息${d}$的最小和最大感知频率。
车辆$v$中的信息$d$在时间$t$的上传优先级用$p_{d, v}^t$表示,不同信息的上传优先级需各不相同。
\begin{equation}
	{p}_{d^*, v}^t \neq {p}_{d, v}^t, \forall d^* \in \mathbf{D}_v^t \setminus \left\{ d\right \}, \forall d \in \mathbf{D}_v^t, \forall v \in \mathbf{V}, \forall t \in \mathbf{T}
\end{equation}
其中${p}_{d^*, v}^t$是信息$d^* \in \mathbf{D}_v^t$中的上传优先级。
车辆$v$在时间$t$的传输功率用$\pi_{v}^t$表示,其不能超过车辆$v$的功率容量。
\begin{equation}
	\pi_v^t \in \left[ 0 , \pi_v \right ], \forall v \in \mathbf{V}, \forall t \in \mathbf{T}
\end{equation}
边缘节点$e$在时间$t$为车辆$v$分配的V2I带宽用$b_{v, e}^t$表示,且其需要满足:
\begin{equation}
	b_{v, e}^t \in \left [0, b_e \right], \forall v \in \mathbf{V}_e^{t}, \forall e \in \mathbf{E}, \forall t \in \mathbf{T}
	\label{equ 4-5} 
\end{equation}
边缘节点$e$分配的V2I总带宽不能超过其容量$b_e$,即${\sum_{\forall v \in \mathbf{V}_e^{t}} b_{v, e}^t} \leq b_e, \forall t \in \mathbf{T}$。

本系统中物理实体的集合为 $\mathbf{I}^{\prime}$,其中$i^{\prime} \in \mathbf{I}^{\prime}$表示物理实体,如车辆、行人和路侧基础设施等。
$\mathbf{D}_{i^{\prime}}$是与实体$i^{\prime}$相关的信息集合,可以用$\mathbf{D}_{i^{\prime}}=\left\{d \mid y_{d, i^{\prime}} = 1, \forall d \in \mathbf{D} \right\}$, $\forall i^{\prime} \in \mathbf{I}^{\prime}$表示, 其中$y_{d, i^{\prime}}$是二进制数,表示信息$d$是否与实体$i^{\prime}$关联。
$\mathbf{D}_{i^{\prime}}$的大小用$|\mathbf{D}_{i^{\prime}}|$表示。
每个实体可能需要多个信息,即$|\mathbf{D}_{i^{\prime}}| = \sum_{\forall d \in \mathbf{D}}y_{d, i^{\prime}} \geq 1, \forall i^{\prime} \in \mathbf{I}^{\prime}$。
对于每个实体$i^{\prime} \in \mathbf{I}^{\prime}$,可能有一个视图$i$在边缘节点中建模。
用$\mathbf{I}$表示视图的集合,用$\mathbf{I}_e^{t}$表示时间为$t$时在边缘节点$e$中建模的视图集合。
因此,边缘节点$e$收到且被视图$i$需要的信息集合可以用$\mathbf{D}_{i, e}^t=\bigcup_{\forall v \in \mathbf{V}}\left(\mathbf{D}_{i^{\prime}} \cap \mathbf{D}_{v, e}^t\right), \forall i \in \mathbf{I}_e^{t}, \forall e \in \mathbf{E}$表示,且 $| \mathbf{D}_{i, e}^t |$是边缘节点$e$收到且被视图$i$需要的信息数量,其计算公式为$| \mathbf{D}_{i, e}^t | =  \sum_{\forall v \in \mathbf{V}} \sum_{\forall d \in \mathbf{D}_v} c_{d, v}^t  y_{d, i^{\prime}}$。

\subsection[\hspace{-2pt}协同感知模型]{{\CJKfontspec{SimHei}\zihao{4} \hspace{-8pt}协同感知模型}}
车辆协同感知是基于多类M/G/1优先级队列\cite{moltafet2020age}进行建模。
假设具有$\operatorname{type}_d$的信息的上传时间$\operatorname{\hat{g}}_{d, v, e}^t$遵循均值$\alpha_{d, v}^t$和方差$\beta_{d, v}^t$的一类一般分布。
那么,车辆$v$中的上传负载$\rho_{v}^{t}$由$ \rho_{v}^{t}=\sum_{\forall d \subseteq \mathbf{D}_v^t} \lambda_{d, v}^{t} \alpha_{d, v}^t$表示。
根据多类M/G/1优先级队列,需要满足$\rho_{v}^{t} < 1$才能达到队列的稳定状态。
信息$d$在时间$t$之前的到达时间用$\operatorname{a}_{d, v}^t$表示,其计算公式为:
\begin{equation}
    \operatorname{a}_{d, v}^t =  \frac{\left \lfloor t \lambda_{d, v}^t \right \rfloor }{\lambda_{d, v}^{t}} 
\end{equation}
在时间$t$之前,由$\operatorname{u}_{d, v}^t$表示的信息$d$的更新时间是通过下式计算:
\begin{equation}
    \operatorname{u}_{d, v}^t = \left \lfloor  \frac{\operatorname{a}_{d, v}^t}{u_d} \right \rfloor  u_d
\end{equation}
其中$u_d$是信息$d$的更新间隔时间。


在时间$t$,车辆$v$中比$d$有更高上传优先级的信息集合,用$\mathbf{D}_{d, v}^t = \{ d^* \mid p_{d^*, v}^{t} > p_{d, v}^{t} , \forall d^* \in \mathbf{D}_v^t \}$表示,其中$p_{d^*, v}^{t}$是信息$d^* \in \mathbf{D}_v^t$的上传优先级。
因此,信息$d$前面的上传负载(即$v$在时间$t$时要在$d$之前上传的信息数量)通过下方计算得出: 
\begin{equation}
	\rho_{d, v}^{t}=\sum_{\forall d^* \in \mathbf{D}_{d, v}^t} \lambda_{d^*, v}^t \alpha_{d^*, v}^t
\end{equation}
其中$\lambda_{d^*, v}^t$和$\alpha_{d^*, v}^t$分别为时间$t$内车辆$v$中信息$d^*$的感知频率和平均传输时间。
根据Pollaczek-Khintchine公式\cite{takine2001queue},车辆$v$中信息$d$的排队时间计算如下:
\begin{equation}
    \operatorname{q}_{d, v}^t= \frac{1} {1 - \rho_{d, v}^{t}} 
        \left[ \alpha_{d, v}^t + \frac{ \lambda_{d, v}^{t} \beta_{d, v}^t + \sum\limits_{\forall d^* \in \mathbf{D}_{d, v}^t} \lambda_{d^*, v}^t \beta_{d^*, v}^t }{2\left(1-\rho_{d, v}^{t} - \lambda_{d, v}^{t} \alpha_{d, v}^t\right)}\right] 
        - \alpha_{d, v}^t
\end{equation}

\subsection[\hspace{-2pt}V2I协同上传模型]{{\CJKfontspec{SimHei}\zihao{4} \hspace{-8pt}V2I协同上传模型}}

车辆间V2I协同上传是基于信道衰减分布和信噪比阈值来建模的。
车辆$v$和边缘节点$e$之间的V2I通信在时间$t$的信噪比通过公式\ref{equ 4-10}\cite{sadek2009distributed}计算得到。
\begin{equation}
    \operatorname{SNR}_{v, e}^{t}=\frac{1}{N_{0}} \left|h_{v, e}\right|^{2} \tau {\operatorname{dis}_{v, e}^{t}}^{-\varphi} {\pi}_v^t
    \label{equ 4-10}
\end{equation}
其中$N_{0}$为AWGN;$h_{v, e}$为信道衰减增益;$\tau$为取决于天线设计的常数;$\varphi$为路径损耗指数。
假设$\left|h_{v, e}\right|^{2}$遵循均值$\mu_{v, e}$和方差$\sigma_{v, e}$的一类分布,其表示方法为:
\begin{equation}
    \tilde{p}=\left\{\mathbb{P}: \mathbb{E}_{\mathbb{P}}\left[\left|h_{v, e}\right|^{2}\right]=\mu_{v, e}, \mathbb{E}_{\mathbb{P}}\left[\left|h_{v, e}\right|^{2}-\mu_{v, e}\right]^{2}=\sigma_{v, e}\right\}
\end{equation}
进一步,基于成功传输概率和可靠性阈值来衡量V2I传输可靠性。
\begin{equation}
    \inf_{\mathbb{P} \in \tilde{p}} \operatorname{Pr}_{[\mathbb{P}]}\left(\operatorname{SNR}_{v, e}^{t} \geq \operatorname{SNR}_{v, e}^{\operatorname{tgt}}\right) \geq \delta
\end{equation}
\noindent 其中$\operatorname{SNR}_{v, e}^{\operatorname{tgt}}$和$\delta$分别为目标SNR阈值和可靠性阈值。
由车辆$v$上传并由边缘节点$e$接收的信息集合用$\mathbf{D}_{v, e}^{t} = \bigcup_{\forall v \in \mathbf{V}_{e}^{t}} \mathbf{D}_{v}^{t}$表示。

根据香农理论,车辆$v$和边缘节点$e$之间在时间$t$的V2I通信的传输率用$\operatorname{z}_{v, e}^t$表示,其计算公式如下:
\begin{equation}
    \operatorname{z}_{v, e}^t=b_{v}^{t} \log _{2}\left(1+\mathrm{SNR}_{v, e}^{t}\right)
\end{equation}
假设车辆$v$被安排在时间$t$上传$d$,并且$d$将在一定的排队时间$\mathrm{\bar{q}}_{d, v}^t$后被传输。
然后,本章把车辆$v$开始传输$d$的时刻表示为$\mathrm{t}_{d, v}^t=t+\mathrm{q}_{d, v}^t$。
从$\mathrm{t}_{d, v}^t$到$\mathrm{t}_{d, v}^t + f$之间传输的数据量可由 $\int_{\mathrm{t}_{d, v}^t}^{\mathrm{t}_{d, v}^t+f} \mathrm{z}_{v, e}^t \mathrm{~d} t$ bits 得到,其中$f \in \mathbb{R}^{+}$和$\mathrm{z}_{i, e}^t$是时间$t$的传输速率。
如果在整个传输过程中可以传输的数据量大于信息$d$的大小,那么上传就会完成。
因此,从车辆$v$到边缘节点$e$传输信息$d$的时间,用$\operatorname{g}_{d, v, e}^t$表示,计算如下:
\begin{equation}
    \operatorname{g}_{d, v, e}^t=\inf _{j \in \mathbb{R}^+} \left \{ \int_{\operatorname{k}_{d, v}^t}^{\operatorname{k}_{d, v}^t + j} {\operatorname{z}_{v, e}^t} \operatorname{d}t \geq \left|d\right| \right \} 
\end{equation}
\noindent 其中$\operatorname{t}_{d, v}^t = t +\operatorname{q}_{d, v}^t$是车辆$v$开始传输信息$d$的时刻。

\section[\hspace{-2pt}质量-开销均衡问题定义]{{\CJKfontspec{SimHei}\zihao{-3} \hspace{-8pt}质量-开销均衡问题定义}}\label{section 4-4}

\subsection[\hspace{-2pt}VCPS质量]{{\CJKfontspec{SimHei}\zihao{4} \hspace{-8pt}VCPS质量}}

首先,由于视图是基于连续上传和时间变化的信息建模的,本章对信息$d$的及时性定义如下:
\begin{definition}
信息$d$在车辆$v$中的及时性$\theta_{d, v} \in \mathbb{Q}^{+}$被定义为更新和接收信息$d$之间的时间差。
\begin{equation}
    \theta_{d, v} = \operatorname{a}_{d, v}^t + \operatorname{q}_{d, v}^t + \operatorname{g}_{d, v, e}^t-\operatorname{u}_{d, v}^{t}, \forall d \in \mathbf{D}_v^t,\forall v \in \mathbf{V}
\end{equation}
\end{definition}
\begin{definition}
视图$i$的及时性 $\Theta_{i} \in \mathbb{Q}^{+}$定义为与物理实体$i^{\prime}$相关的信息的最大及时性之和。
	\begin{equation}
    	\Theta_{i} = \sum_{\forall v\in \mathbf{V}_{e}^{t}} \max_{\forall d \in \mathbf{D}_{i^{\prime}} \cap \mathbf{D}_v^t}\theta_{d, v}, \forall i \in \mathbf{I}_{e}^{t}, \forall e \in \mathbf{E}
    	\label{equ 4-16}
	\end{equation}
\end{definition}

其次,由于不同类型的信息有不同的感知频率和上传优先级,本章定义视图的一致性来衡量与同一物理实体相关的信息的一致性。
\begin{definition}
视图$i$的一致性$\Psi_{i} \in \mathbb{Q}^{+}$定义为信息更新时间差的最大值。
\begin{equation}
    \Psi_{i}=\max_{\forall d \in \mathbf{D}_{i, e}^{t}, \forall v \in \mathbf{V}_{e}^{t}} {\operatorname{u}_{d, v}^t} - \min_{\forall d \in \mathbf{D}_{i, e}^{t}, \forall v \in \mathbf{V}_{e}^{t}} {\operatorname{u}_{d, v}^t} , \forall i \in \mathbf{I}_{e}^{t}, \forall e \in \mathbf{E}
\end{equation}
\end{definition}

最后,本章给出了视图的质量的正式定义,其综合了视图的及时性和一致性。
\begin{definition}
视图质量$\operatorname{QV}_{i} \in (0, 1)$定义为视图$i$的归一化及时性和归一化一致性的加权平均和。
	\begin{equation}
	    \operatorname{QV}_{i} = w_1 (1 -\hat{\Theta_{i}}) + w_2 (1 - \hat{\Psi_{i}}), \forall i \in \mathbf{I}_{e}^t, \forall e \in \mathbf{E}
	\end{equation}
\end{definition}
\noindent 其中$\hat{\Theta_{i}} \in (0, 1)$和$\hat{\Psi_{i}} \in (0, 1)$分别表示归一化的及时性和归一化的一致性,这可以通过最小-最大归一化对及时性和一致性的范围进行重新调整至$(0, 1)$来获得。
$\hat{\Theta_{i}}$和$\hat{\Psi_{i}}$的加权系数分别用$w_1$和$w_2$表示,可以根据ITS应用的不同要求进行相应的调整,$w_1+w_2=1$。
$\operatorname{QV}_{i}$的值越高,说明构建的视图质量越高。
考虑车辆自动驾驶系统下,视图的时效性和一致性都是关键。一方面,汽车需要实时接收到周围车辆和行人的位置、速度和行为信息,以便能够做出准确的驾驶决策。如果视图数据延迟太高或过时,车辆可能无法及时识别出潜在的危险或变化情况,从而导致事故或冲突的发生。另一方面,一致性方面的重要性表现在视图数据的一致性建模。车辆需要确保从边缘节点接收到的视图数据是准确、完整且一致的,以便能够准确地理解周围环境并做出正确的决策。如果视图数据存在不一致或缺失,车辆可能会做出错误的判断,从而导致不安全的驾驶行为或导航错误。

进一步,基于视图质量定义车载信息物理融合质量如下:
\begin{definition}
VCPS质量$\mathscr{Q} \in (0, 1)$被定义为在调度期间$\mathbf{T}$的边缘节点中建模的每个视图的QV的平均值。
	\begin{equation}
		\mathscr{Q}=\frac{\sum_{\forall t \in \mathbf{T}} \sum_{\forall e \in \mathbf{E}} \sum_{\forall i \in \mathbf{I}_e^t} \operatorname{QV}_{i}}{\sum_{\forall t \in \mathbf{T}} \sum_{\forall e \in \mathbf{E}} |\mathbf{I}_e^t| }
	\end{equation}
\end{definition}

\subsection[\hspace{-2pt}VCPS开销]{{\CJKfontspec{SimHei}\zihao{4} \hspace{-8pt}VCPS开销}}

首先,由于同一物理实体的状态可能被多个车辆同时感应到,本章对信息$d$的冗余度定义如下:
\begin{definition}
信息$d$的冗余度$\xi_d \in \mathbb{N}$定义为车辆感应到同一类型$\operatorname{type}_d$的额外信息数量。
\begin{equation}
    \xi_d= \left | \mathbf{D}_{d, i, e} \right| - 1, \forall d \in \mathbf{D}_j, \forall i \in \mathbf{I}_{e}^{t}, \forall e \in \mathbf{E}
\end{equation}
\noindent 其中$\mathbf{D}_{d, i, e}$是边缘节点$e$收到且被视图$i$需要,且类型为$\operatorname{type}_d$的信息集合,其由$\mathbf{D}_{d, i, e}=\left\{ d^* \vert \operatorname{type}_{d^*} = \operatorname{type}_{d}, \forall d^* \in \mathbf{D}_{i, e}^t \right \}$表示。

\end{definition}
\begin{definition}
视图$i$的冗余度$\Xi_j \in \mathbb{N}$定义为视图$i$中的总冗余度。
	\begin{equation}
       \Xi_j =  \sum_{\forall d \in \mathbf{D}_{i^{\prime}}} \xi_d, \forall i \in \mathbf{I}_{e}^{t}, \forall e \in \mathbf{E}
       \label{equ 4-20}
    \end{equation}
\end{definition}

其次,信息感知和传输需要消耗车辆的能量,本章定义视图$i$的感知开销和传输开销如下:
\begin{definition}
视图$i$的感知开销$\Phi_{i} \in \mathbb{Q}^{+}$定义为视图$i$所需信息的总感知开销。
	\begin{equation}
        \Phi_{i} = \sum_{\forall v \in \mathbf{V}_{e}^{t}} \sum_{\forall d \in \mathbf{D}_{i^{\prime}} \cap \mathbf{D}_v^t}{\phi_{d, v}}, \forall i \in \mathbf{I}_{e}^t, \forall e \in \mathbf{E}
        \label{equ 4-21}
    \end{equation}
    其中$\phi_{d, v}$是信息$d$在车辆$v$中的感知开销。
\end{definition}
\begin{definition}
信息$d$在车辆$v$中的传输开销${\omega}_{d, v} \in \mathbb{Q}^{+}$定义为信息上传时消耗的传输功率。
\begin{equation}
    {\omega}_{d, v}= \pi_v^t \operatorname{g}_{d, v, e}^t, \forall d \in \mathbf{D}_v^t
\end{equation}
其中$\pi_v^t$和$\operatorname{g}_{d, v, e}^t$分别为传输功率和传输时间。
\end{definition}
\begin{definition}
视图$i$的传输开销$\Omega_{i} \in \mathbb{Q}^{+}$定义为视图$i$所需的信息总传输开销。
	\begin{equation}
        \Omega_{i} = \sum_{\forall v \in \mathbf{V}_{e}^{t}} \sum_{\forall d \in \mathbf{D}_{i^{\prime}} \cap \mathbf{D}_v^t} {\omega}_{d, v}, \forall i \in \mathbf{I}_{e}^t, \forall e \in \mathbf{E}
       	\label{equ 4-23}
    \end{equation}
\end{definition}

最后,给出视图开销的正式定义,其综合了冗余度、感知开销和传输开销。
\begin{definition}
视图的开销$\operatorname{CV}_{i} \in (0, 1)$定义为视图$i$的归一化冗余度、归一化感知开销和归一化传输开销的加权平均和。
	\begin{equation}
	    \operatorname{CV}_{i} = w_3  \hat{\Xi_{i}} +  w_4 \hat{\Phi_{i}} + w_5 \hat{\Omega_{i}}, \forall i \in \mathbf{I}_{e}^t, \forall e \in \mathbf{E}
	\end{equation}
\end{definition}
\noindent 其中 $\hat{\Xi_{i}}\in (0, 1)$、$\hat{\Phi_{i}} \in (0, 1)$和$\hat{\Omega_{i}} \in (0, 1)$ 分别表示视图$i$的归一化冗余度、归一化感知开销和归一化传输开销。
$\hat{\Xi_{i}}$、$\hat{\Phi_{i}}$和$\hat{\Omega_{i}}$ 的加权系数分别表示为 $w_3$、$w_4$和 $w_5$。
同样地,$w_3+w_4+w_5=1$。
进一步,VCPS开销定义如下:
\begin{definition}
VCPS 开销$\mathscr{C} \in (0, 1)$定义为$\mathbf{T}$调度期间边缘节点中每个视图模型的CV的平均值。
	\begin{equation}
		\mathscr{C}=\frac{\sum_{\forall t \in \mathbf{T}} \sum_{\forall e \in \mathbf{E}} \sum_{\forall i \in \mathbf{I}_e^t}  \operatorname{CV}_{i}}{\sum_{\forall t \in \mathbf{T}} \sum_{\forall e \in \mathbf{E}} |\mathbf{I}_e^t| }
	\end{equation}
\end{definition}

\subsection[\hspace{-2pt}双目标优化问题]{{\CJKfontspec{SimHei}\zihao{4} \hspace{-8pt}双目标优化问题}}

给定解决方案$( \mathbf{C}, \bf\Lambda, \mathbf{P}, \bf\Pi, \mathbf{B} )$,其中$\mathbf{C}$表示确定的感知信息决策,$\bf\Lambda$表示确定的感知频率。$\mathbf{P}$表示确定的上传优先级,$\bf\Pi$表示确定的传输功率,$\mathbf{B}$表示确定的V2I带宽分配,其中$c_{d, v}^t$、$\lambda_{d, v}^{t}$和$p_{d, v}^{t}$分别为时间$t$内车辆$v$的信息$d$的感知信息决策、感知频率和上传优先级,$\pi_v^t$和$b_v^t$分别为时间$t$内车辆$v$的传输功率和V2I带宽。
\begin{numcases}{}
	\mathbf{C} = \left \{ c_{d, v}^t \vert \forall d \in \mathbf{D}_{v}, \forall v \in \mathbf{V}, \forall t \in \mathbf{T} \right  \} \notag \\
	{\bf\Lambda} = \left \{ \lambda_{d, v}^{t} \vert \forall d \in \mathbf{D}_v^t  , \forall v \in \mathbf{V}, \forall t \in \mathbf{T} \right \} \notag \\ 
	\mathbf{P} = \left \{ p_{d, v}^{t} \vert \forall d \in \mathbf{D}_v^t  , \forall v \in \mathbf{V}, \forall t \in \mathbf{T}\right \}  \notag \\
	{\bf\Pi} = \left \{ \pi_v^t \vert \forall v \in \mathbf{V}, \forall t \in \mathbf{T} \right \} \notag \\
	\mathbf{B} = \left \{ b_v^t \vert \forall v \in \mathbf{V}, \forall t \in \mathbf{T}\right \}
\end{numcases}
本章提出了双目标优化问题,旨在同时实现VCPS质量的最大化和VCPS 开销的最小化::
\begin{align}
	\mathcal{P}4.1: & \max_{\mathbf{C}, \bf\Lambda, \mathbf{P}, \bf\Pi, \mathbf{B}} \mathscr{Q}, \min_{\mathbf{C}, \bf\Lambda, \mathbf{P}, \bf\Pi, \mathbf{B}} \mathscr{C} \notag \\
	\text { s.t. }
	& (\ref{equ 4-1}) \sim (\ref{equ 4-5}) \notag \\
    &\mathcal{C}4.1: \sum_{\forall d \subseteq \mathbf{D}_v^t} \lambda_{d, v}^{t} \mu_d<1,\ \forall v \in \mathbf{V}, \forall t \in \mathbf{T} \notag \\
    &\mathcal{C}4.2: \inf_{\mathbb{P} \in \tilde{p}} \operatorname{Pr}_{[\mathbb{P}]}\left(\operatorname{SNR}_{v, e}^{t} \geq \operatorname{SNR}_{v, e}^{\operatorname{tgt}}\right) \geq \delta, \forall v \in \mathbf{V}, \forall t \in \mathbf{T} \notag \\
    &\mathcal{C}4.3: {\sum_{\forall v \in \mathbf{V}_e^{t}}b_v^t} \leq b_e, \forall t \in \mathbf{T}
\end{align}
其中$\mathcal{C}4.1$保证队列稳定状态,$\mathcal{C}4.2$保证传输可靠性。
$\mathcal{C}4.3$要求边缘节点$e$分配的V2I带宽之和不能超过其容量$b_e$。
基于CV的定义,视图的利润定义如下:
\begin{definition}
视图的利润$\operatorname{PV}_{j} \in (0, 1)$定义为视图$i$的CV的补集。
	\begin{equation}
		\mathscr{P}= 1 - \operatorname{CV}_{i}
	\end{equation}
\end{definition}
\noindent 然后,本章将VCPS 利润定义如下:
\begin{definition}
VCPS 利润$\mathscr{P} \in (0, 1)$被定义为在调度期$\mathbf{T}$期间,边缘节点中每个视图模型的PV的平均值。
	\begin{equation}
		\mathscr{P}= \frac{\sum_{\forall t \in \mathbf{T}} \sum_{\forall e \in \mathbf{E}} \sum_{\forall i \in \mathbf{I}_e^t}   \operatorname{PV}_{j} }{\sum_{\forall t \in \mathbf{T}} \sum_{\forall e \in \mathbf{E}} |\mathbf{I}_e^t| }
	\end{equation}
\end{definition}
\noindent 因此,$\mathcal{P}4.1$问题可以改写如下:
\begin{align}
	\mathcal{P}4.2: & \max_{ \mathbf{C}, \bf\Lambda, \mathbf{P}, \bf\Pi, \mathbf{B} } \left (\mathscr{Q}, \mathscr{P} \right ) \notag \\
		\text { s.t. }
	&(\ref{equ 4-1}) \sim (\ref{equ 4-5}), \mathcal{C}4.1 \sim \mathcal{C}4.3
\end{align}

\section[\hspace{-2pt}基于多目标的多智能体强化学习算法设计]{{\CJKfontspec{SimHei}\zihao{-3} \hspace{-8pt}基于多目标的多智能体强化学习算法设计}}\label{section 4-5}

本章节提出了基于多目标的多智能体深度强化学习算法,其模型如图\ref{fig 4-2}所示,由$K$分布式行动者、学习器和经验回放缓存组成。
具体地,学习器由四个神经网络组成,即本地策略网络、本地评论家网络、目标策略网络和目标评论家网络。
其中车辆的本地策略网络、本地评论家网络、目标策略网络和目标评论家网络参数分别表示为 $\theta_{\mathbf{V}}^{\mu}$、$\theta_{\mathbf{V}}^{Q}$、 $\theta_{\mathbf{V}}^{\mu^{\prime}}$和$\theta_{\mathbf{V}}^{Q^{\prime}}$。
同样地,边缘节点的本地策略网络、本地评论家网络、目标策略网络和目标评论家网络参数分别表示为 $\theta_{\mathbf{E}}^{\mu}$、$\theta_{\mathbf{E}}^{Q}$、$\theta_{\mathbf{E}}^{\mu^{\prime}}$和$\theta_{\mathbf{E}}^{Q^{\prime}}$。
本地策略和本地评论家网络的参数是随机初始化的。
目标策略和目标评论家网络的参数被初始化为相应的本地网络。
然后,启动$K$分布式行动者,每个分布式行动者独立地与环境进行交互,并将交互经验存储到重放经验缓存。
分布式行动者由本地车辆策略网络和本地边缘策略网络组成,其分别用$\theta_{\mathbf{V}, k}^{\mu}$和$\theta_{\mathbf{E}, k}^{\mu}$表示,其网络参数是从学习器的本地策略网络复制而来的。
同时,初始化了最大存储容量为$|\mathcal{B}|$的经验回放缓存以存储重放经验。
基于多目标的多智能体深度强化学习的具体步骤见算法4.1,分布式行动者与环境的交互将持续到学习器的训练过程结束,其具体步骤见算法4.2。

\begin{figure}[h]
\centering
  \captionsetup{font={small, stretch=1.312}}\includegraphics[width=1\columnwidth]{Fig4-2-solution-model.pdf}
  \bicaption[基于多目标的多智能体深度强化学习模型]{基于多目标的多智能体深度强化学习模型}[Multi-agent multi-objective deep reinforcement learning model]{Multi-agent multi-objective deep reinforcement learning model}
  \label{fig 4-2}
\end{figure}


\SetKwInOut{KwIn}{输入}
\SetKwInOut{KwOut}{输出}

\begin{algorithm}[h]\small
\setstretch{1.245} %设置具有指定弹力的橡皮长度(原行宽的1.2倍)
\renewcommand{\algorithmcfname}{算法}
	\caption{基于多目标的多智能体深度强化学习}
	\KwIn{折扣因子 $\gamma$、批大小 $M$、回放经验缓存 $\mathcal{B}$、学习率$\alpha$和$\beta$、目标网络参数更新周期 $t_{\operatorname{tgt}}$、分布式行动者网络参数更新周期 $t_{\operatorname{act}}$、随机动作数量$N$}
	\KwOut{信息感知决策$\mathbf{C}_v^t$、信息感知频率决策$\lambda_{d, v}^{t}$、上传优先级决策$p_{d, v}^{t}$、传输功率$\pi_v^t$、V2I带宽分配$b_{v, e}^{t}$}
	初始化网络参数\\
	初始化经验回放缓存 $\mathcal{B}$\\
	启动 $K$ 分布式行动者并复制网络参数给行动者\\
	\For{\songti{迭代次数} $= 1$ \songti{到最大迭代次数}}{
		\For{\songti{时间片} $t = 1$ \songti{到} $T$}{
			从经验回放缓存$\mathcal{B}$随机采样$M$小批量\\
			通过目标评论家网络中DCN网络得到目标值\\
			基于分类分布的TD学习计算更新评论家网络\\
			更新本地策略和评论家网络\\
			\If{$t \mod t_{\operatorname{tgt}} = 0$}{
				更新目标网络\\
			}
			\If{$t \mod t_{\operatorname{act}} = 0$}{
				复制网络参数给分布式行动者\\
			}
		}
	}
\label{algorithm 4-1}
\end{algorithm}

\begin{algorithm}[t]\small
\setstretch{1.245} %设置具有指定弹力的橡皮长度(原行宽的1.2倍)
\renewcommand{\algorithmcfname}{算法}
	\caption{分布式行动者}
	\KwIn{车辆探索常数 $\epsilon_{v}$、边缘节点探索常数$\epsilon_{e}$、车辆本地观测 $\boldsymbol{o}_{v}^{t}$、边缘节点本地观测 $\boldsymbol{o}_{e}^{t}$}
	\KwOut{车辆动作$\boldsymbol{a}_{\mathbf{V}}^{t}$、边缘节点动作$\boldsymbol{a}_{e}^{t}$}
	\While{\songti{学习器没有结束}}{
		初始化随机过程 $\mathcal{N}$ 以进行探索\\
		生成随机权重 $\boldsymbol{w}^{t}$\\
		接收初始系统状态 $\boldsymbol{o}_{1}$\\
		\For{\songti{时间片} $t = 1$ \songti{到} $T$}{
			\For{\songti{车辆} $v=1$ \songti{到} $V$}{
				接收车辆本地观测 $\boldsymbol{o}_{v}^{t}$\\
				选择车辆动作 $\mu_{\mathbf{V}}\left(\boldsymbol{o}_{v}^{t} \mid \theta_{\mathbf{V}}^{\mu}\right)+\epsilon_{v} \mathcal{N}_{v}^{t}$\\
			}
			接收边缘节点本地观测 $\boldsymbol{o}_{e}^{t}$\\
			选择边缘节点动作 $\boldsymbol{a}_{e}^{t}=\mu_{\mathbf{E}}\left(\boldsymbol{o}_{e}^{t},  \boldsymbol{a}_{\boldsymbol{\mathbf{V}}}^{t} \mid \theta_{\mathbf{E}}^{\mu}\right)+\epsilon_{e} \mathcal{N}_{e}^{t}$\\
			接收奖励 $\boldsymbol{r}^{t}$ 和下一个系统状态 $\boldsymbol{o}^{t+1}$\\
			\For{\songti{车辆} $v=1$ \songti{到} $V$}{
				根据公式\ref{equ 4-40}计算车辆的差分奖励\\
			}
			根据公式\ref{equ 4-41}计算边缘节点的归一化奖励\\
			存储 $\left(\boldsymbol{o}^{t}, \boldsymbol{a}_{\mathbf{V}}^{t}, \boldsymbol{a}_{e}^{t}, \boldsymbol{r}_{\mathbf{V}}^{t}, \boldsymbol{r}_{e}^{t}, \boldsymbol{w}^{t}, \boldsymbol{o}^{t+1}\right)$ 到经验回放缓存 $\mathcal{B}$
		}
	}
\label{algorithm 4-2}
\end{algorithm}

\subsection[\hspace{-2pt}多智能体分布式策略执行]{{\CJKfontspec{SimHei}\zihao{4} \hspace{-8pt}多智能体分布式策略执行}}

在MAMO中,车辆和边缘节点分别通过本地策略网络分布式地决定动作。
车辆$v$在时间$t$上对系统状态的局部观测表示为:
	\begin{equation}
		\boldsymbol{o}_{v}^{t}=\left\{t, v, l_{v}^t, \mathbf{D}_{v}, \Phi_{v}, \mathbf{D}_{e}^{t}, \mathbf{D}_{\mathbf{I}_e^t}, \boldsymbol{w}^{t}\right\}
	\end{equation} 
\noindent 其中$t$为时间片索引;
$v$是车辆索引;$l_{v}^t$是车辆$v$的位置;
$\mathbf{D}_{v}$表示车辆$v$可以感知的信息集合;
$\Phi_{v}$代表$\mathbf{D}_{v}$中信息的感知开销;
$\mathbf{D}_{e}^{t}$ 代表$e$在时间$t$的边缘节点的缓存信息集;
$\mathbf{D}_{\mathbf{I}_e^t}$ 代表在时间$t$的边缘节点$e$中建模的视图所需的信息集合;
$\boldsymbol{w}^{t}$ 代表每个目标的权重向量,其在每次迭代中随机生成。
具体地,$\boldsymbol{w}^{t} = \begin{bmatrix}  w^{(1), t}  &  w^{(2), t} \end{bmatrix}$,其中$w^{(1), t} \in (0, 1)$和$w^{(2), t} \in (0, 1)$分别是VCPS质量和VCPS利润的权重,$\sum_{\forall i \in \{1, 2\}} w^{(j), t} = 1$。
另一方面,边缘节点$e$在时间$t$上对系统状态的局部观测表示为
\begin{equation}
	\boldsymbol{o}_{e}^{t}=\left\{t, e, \operatorname{\mathbf{Dis}}_{\mathbf{V}, e}^{t}, \mathbf{D}_{1}, \ldots, \mathbf{D}_{v}, \ldots, \mathbf{D}_{v}, \mathbf{D}_{e}^{t}, \mathbf{D}_{\mathbf{I}_e^t}, \boldsymbol{w}^{t} \right\}
\end{equation}
\noindent 其中$e$是边缘节点索引,$\operatorname{\mathbf{Dis}}_{\mathbf{V}, e}^{t}$代表车辆与边缘节点$e$之间的距离集合。
因此,系统在时间$t$的状态可以表示为$\boldsymbol{o}^{t}=\boldsymbol{o}_{e}^{t} \cup \boldsymbol{o}_{1}^{t} \cup \ldots \cup \boldsymbol{o}_{v}^{t} \cup \ldots \cup \boldsymbol{o}_{v}^{t}$。

车辆$v$的动作空间表示为:
\begin{equation}
	\boldsymbol{a}_{v}^{t} = \{ \mathbf{C}_v^t,  \{ \lambda_{d, v}^{t}, p_{d, v}^{t} \mid \forall d \in \mathbf{D}_{v}^t \} , \pi_v^t   \}
\end{equation}
其中,$\mathbf{C}_v^t$是感知决策;$\lambda_{d, v}^{t}$和$p_{d, v}^{t}$分别是信息$d$的感知频率和上传优先级,$\pi_v^t$是车辆$v$在时间$t$的传输功率。
车辆基于系统状态的本地观测,并通过本地车辆策略网络得到当前的动作。
\begin{equation}
	\boldsymbol{a}_{v}^{t}=\mu_{\mathbf{V}}\left(\boldsymbol{o}_{v}^{t} \mid \theta_{\mathbf{V}}^{\mu}\right)+\epsilon_{v} \mathcal{N}_{v}^{t}
\end{equation}
\noindent 其中,$\mathcal{N}_{v}^{t}$为探索噪音,以增加车辆动作的多样性,$\epsilon_{v}$为车辆$v$的探索常数。
车辆动作的集合被表示为 $\boldsymbol{a}_{\mathbf{V}}^{t} = \left\{\boldsymbol{a}_{v}^{t}\mid \forall v \in \mathbf{V}\right\}$。
另一方面,边缘节点$e$的动作空间表示为:
\begin{equation}
	\boldsymbol{a}_{e}^{t} = \{b_{v, e}^{t} \mid \forall v \in \mathbf{V}_{e}^{t}\}
\end{equation}
其中$b_{v, e}^t$是边缘节点$e$在时间$t$为车辆$v$分配的V2I带宽。
同样地,边缘节点$e$的动作可以由本地边缘策略网络根据系统状态以及车辆动作得到。
\begin{equation}
	\boldsymbol{a}_{e}^{t}=\mu_{\mathbf{E}}\left(\boldsymbol{o}_{e}^{t},  \boldsymbol{a}_{\boldsymbol{\mathbf{V}}}^{t} \mid \theta_{\mathbf{E}}^{\mu}\right)+\epsilon_{e} \mathcal{N}_{e}^{t}
\end{equation}
\noindent 其中$\mathcal{N}_{e}^{t}$和$\epsilon_{e}$分别为边缘节点$e$的探索噪声和探索常数。
此外,车辆和边缘节点的联合动作被表示为 $\boldsymbol{a}^{t}= \left\{\boldsymbol{a}_{e}^{t}, \boldsymbol{a}_{1}^{t}, \ldots, \boldsymbol{a}_{v}^{t}, \ldots, \boldsymbol{a}_{V}^{t}\right\}$。

环境通过执行联合动作获得系统奖励向量,其表示为:
	\begin{equation}
	\boldsymbol{r}^{t} = \begin{bmatrix}  r^{(1)}\left(\boldsymbol{a}_{\mathbf{V}}^{t},\boldsymbol{a}_{e}^{t} \mid \boldsymbol{o}^{t}\right)  &  r^{(2)}\left(\boldsymbol{a}_{\mathbf{V}}^{t},\boldsymbol{a}_{e}^{t} \mid \boldsymbol{o}^{t}\right) \end{bmatrix} ^{T}
	\end{equation}
	\noindent 其中 $r^{(1)}\left(\boldsymbol{a}_{\mathbf{V}}^{t},\boldsymbol{a}_{e}^{t} \mid \boldsymbol{o}^{t}\right)$ 和 $r^{(2)}\left(\boldsymbol{a}_{\mathbf{V}}^{t},\boldsymbol{a}_{e}^{t} \mid \boldsymbol{o}^{t}\right)$ 分别是两个目标(即实现的VCPS质量和VCPS 利润)的奖励,可以通过下式计算:  
	\begin{numcases}{}
			r^{(1)}\left(\boldsymbol{a}_{\mathbf{V}}^{t},\boldsymbol{a}_{e}^{t} \mid \boldsymbol{o}^{t}\right)=\frac{1}{\left|\mathbf{I}_e^t\right|} \sum_{\forall i \in \mathbf{I}_e^t}\operatorname{QV}_{i} \notag \\
			r^{(2)}\left(\boldsymbol{a}_{\mathbf{V}}^{t},\boldsymbol{a}_{e}^{t} \mid \boldsymbol{o}^{t}\right)=\frac{1}{\left|\mathbf{I}_e^t\right|} \sum_{\forall i \in \mathbf{I}_e^t} \operatorname{PV}_{j} 
	\end{numcases}
因此,车辆$v$在第$i$个目标中的奖励可以通过基于差分奖励的信用分配方案 \cite{foerster2018counterfactual} 得到,其为系统奖励和没有其动作所取得的奖励之间的差值,其表示为:
\begin{equation}
r_{v}^{(j), t}=r^{(j)}\left(\boldsymbol{a}_{\mathbf{V}}^{t},\boldsymbol{a}_{e}^{t} \mid \boldsymbol{o}^{t}\right)-r^{(j)}\left(\boldsymbol{a}_{\mathbf{V}-s}^{t},\boldsymbol{a}_{e}^{t} \mid \boldsymbol{o}^{t}\right), \forall i \in \{1, 2\}
\label{equ 4-40}
\end{equation}
\noindent 其中 $r^{(j)}\left(\boldsymbol{a}_{\mathbf{V}-s}^{t},\boldsymbol{a}_{e}^{t} \mid \boldsymbol{o}^{t}\right)$ 是在没有车辆$v$贡献的情况下实现的系统奖励,它可以通过设置车辆$v$的空动作集得到。
车辆$v$在时间$t$的奖励向量表示为$\boldsymbol{r}_{v}^{t} = \begin{bmatrix}  r_{v}^{(1), t}  &  r_{v}^{(2), t} \end{bmatrix} ^{T}$。
车辆的差分奖励集合表示为 $\boldsymbol{r}_{\mathbf{V}}^{t}=\{ \boldsymbol{r}_{v}^{t} \mid \forall v \in \mathbf{V}\}$。

另一方面,系统奖励通过最小-最大归一化进一步转化为边缘节点的归一化奖励。
边缘节点$e$在时间$t$的第$i$个目标中的奖励由以下方式计算:
\begin{equation}
	r_{e}^{(j), t}= \frac{r^{(j)}\left(\boldsymbol{a}_{\mathbf{V}}^{t},\boldsymbol{a}_{e}^{t} \mid \boldsymbol{o}^{t}\right) - \min \limits_{\forall {\boldsymbol{a}_{e}^{t}}^{\prime}} r^{(j)}\left(\boldsymbol{a}_{\mathbf{V}}^{t}, {\boldsymbol{a}_{e}^{t}}^{\prime} \mid \boldsymbol{o}^{t}\right)} {\max \limits_{\forall {\boldsymbol{a}_{e}^{t}}^{\prime}} r^{(j)}\left(\boldsymbol{a}_{\mathbf{V}}^{t}, {\boldsymbol{a}_{e}^{t}}^{\prime} \mid \boldsymbol{o}^{t}\right) - \min \limits_{\forall {\boldsymbol{a}_{e}^{t}}^{\prime}} r^{(j)}\left(\boldsymbol{a}_{\mathbf{V}}^{t}, {\boldsymbol{a}_{e}^{t}}^{\prime} \mid \boldsymbol{o}^{t}\right)}
\label{equ 4-41}
\end{equation}
\noindent 其中 $\min \limits_{\forall {\boldsymbol{a}_{e}^{t}}^{\prime}} r^{(j)} (\boldsymbol{a}_{\mathbf{V}}^{t}, {\boldsymbol{a}_{e}^{t}}^{\prime} \mid \boldsymbol{o}^{t})$ 和 $\max \limits_{\forall {\boldsymbol{a}_{e}^{t}}^{\prime}} r^{(j)}(\boldsymbol{a}_{\mathbf{V}}^{t}, {\boldsymbol{a}_{e}^{t}}^{\prime} \mid \boldsymbol{o}^{t})$ 分别是在相同的系统状态$\boldsymbol{o}^{t}$下,车辆动作$\boldsymbol{a}_{\mathbf{V}}^{t}$不变时,可实现的系统奖励最小值和最大值。
边缘节点$e$在时间$t$的奖励向量表示为 $\boldsymbol{r}_{e}^{t} = \begin{bmatrix}  r_{e}^{(1), t}  &  r_{e}^{(2), t} \end{bmatrix} ^{T}$。
交互经验包括当前系统状态$\boldsymbol{o}^{t}$、车辆动作$\boldsymbol{a}_{\mathbf{V}}^{t}$、边缘节点动作$\boldsymbol{a}_{e}^{t}$、车辆奖励$\boldsymbol{r}_{\mathbf{V}}^{t}$、边缘节点奖励$\boldsymbol{r}_{e}^{t}$、权重$\boldsymbol{w}^{t}$,以及下一时刻系统状态$\boldsymbol{o}^{t+1}$都存储到经验回放缓存$\mathcal{B}$。

\subsection[\hspace{-2pt}多目标策略评估]{{\CJKfontspec{SimHei}\zihao{4} \hspace{-8pt}多目标策略评估}}

本章节阐述了针对多目标的策略评估,具体地,提出了决斗评论家网络,根据状态的价值和动作的优势来评估智能体的动作。
在DCN中有两个全连接的网络,即动作优势网络和状态价值网络。
车辆和边缘节点的AA网络参数分别表示为 $\theta_{\mathbf{V}}^{\mathscr{A}}$ 和 $\theta_{\mathbf{E}}^{\mathscr{A}}$。
同样,车辆和边缘节点的SV网络的参数分别表示为 $\theta_{\mathbf{V}}^{\mathscr{V}}$ 和 $\theta_{\mathbf{E}}^{\mathscr{V}}$。
用$A_{\mathbf{V}}\left({o}_{v}^{m},  {a}_{v}^{m}, \boldsymbol{a}_{\boldsymbol{\mathbf{V}}-v}^{m}, \boldsymbol{w}^{m} \mid \theta_{\mathbf{V}}^{\mathscr{A}} \right)$表示车辆$v$中AA网络的输出标量, 其中 $\boldsymbol{a}_{\boldsymbol{\mathbf{V}}-v}^{m}$ 表示其他车辆动作。
同样地,以边缘节点$e$为输入的AA网络的输出标量表示为 $A_{\mathbf{E}}\left({o}_{e}^{m},  {a}_{e}^{m}, \boldsymbol{a}_{\boldsymbol{\mathbf{V}}}^{m}, \boldsymbol{w}^{m} \mid \theta_{\mathbf{E}}^{\mathscr{A}} \right)$, 其中 $\boldsymbol{a}_{\boldsymbol{\mathbf{V}}}^{m}$ 表示所有车辆动作。
车辆$v$的SV网络的输出标量表示为 $V_{\mathbf{V}}\left({o}_{v}^{m}, \boldsymbol{w}^{m} \mid \theta_{\mathbf{V}}^{\mathscr{V}} \right)$。
同样地,边缘节点$e$的SV网络的输出标量表示为 $V_{\mathbf{E}}\left({o}_{e}^{m}, \boldsymbol{w}^{m} \mid \theta_{\mathbf{E}}^{\mathscr{V}} \right)$。

多目标策略评估由三个步骤组成。
首先,AA网络基于观测、动作和权重输出智能体动作的优势。
其次,VS网络根据观测和权重,输出当前状态的价值。
最后,采用聚合模块,根据动作优势和状态价值,输出智能体动作的价值。
具体来说,在AA网络中随机生成$N$个动作并将智能体动作替换,以评估当前动作对于随机动作的平均优势。
用${a}_{v}^{m, n}$和${a}_{e}^{m, n}$分别表示车辆$v$和边缘节点$e$的第$n$个随机动作。
因此,车辆$v$和边缘节点$e$的第$n$个随机动作的优势可分别表示为 $A_{\mathbf{V}}\left({o}_{v}^{m},  {a}_{v}^{m, n}, \boldsymbol{a}_{\boldsymbol{\mathbf{V}}-v}^{m}, \boldsymbol{w}^{m} \mid \theta_{v}^{\mathscr{A}} \right)$ 和 $A_{\mathbf{E}}\left({o}_{e}^{m},  {a}_{e}^{m, n}, \boldsymbol{a}_{\boldsymbol{\mathbf{V}}}^{m}, \boldsymbol{w}^{m} \mid \theta_{\mathbf{E}}^{\mathscr{A}} \right)$。

进一步,通过评估智能体动作相对于随机动作的平均优势,对价值函数进行聚合。
因此,车辆$v\in\mathbf{V}$和边缘节点$e$的动作价值是通过下式计算: 
\begin{align}
    Q_{\mathbf{V}}\left({o}_{v}^{m}, {a}_{v}^{m}, \boldsymbol{a}_{\boldsymbol{\mathbf{V}}-v}^{m}, \boldsymbol{w}^{m} \mid \theta_{\mathbf{V}}^{Q} \right) &= V_{\mathbf{V}}\left({o}_{v}^{m}, \boldsymbol{w}^{m} \mid \theta_{\mathbf{V}}^{\mathscr{V}} \right) + A_{\mathbf{V}}\left({o}_{v}^{m},  {a}_{v}^{m}, \boldsymbol{a}_{\boldsymbol{\mathbf{V}}-v}^{m}, \boldsymbol{w}^{m} \mid \theta_{\mathbf{V}}^{\mathscr{A}} \right) \notag \\
    &- \frac{1}{N} \sum_{\forall n} A_{\mathbf{V}}\left({o}_{v}^{m},  {a}_{v}^{m, n}, \boldsymbol{a}_{\boldsymbol{\mathbf{V}}-v}^{m}, \boldsymbol{w}^{m} \mid \theta_{\mathbf{V}}^{\mathscr{A}} \right)
\end{align}
\begin{align}
    Q_{E}\left({o}_{e}^{m},  {a}_{e}^{m}, \boldsymbol{a}_{\boldsymbol{\mathbf{V}}}^{m}, \boldsymbol{w}^{m} \mid \theta_{\mathbf{E}}^{Q} \right) &= V_{\mathbf{E}}\left({o}_{e}^{m}, \boldsymbol{w}^{m} \mid \theta_{\mathbf{E}}^{\mathscr{V}} \right) + A_{\mathbf{E}}\left({o}_{e}^{m},  {a}_{e}^{m}, \boldsymbol{a}_{\boldsymbol{\mathbf{V}}}^{m}, \boldsymbol{w}^{m} \mid \theta_{\mathbf{E}}^{\mathscr{A}} \right) \notag \\
    &- \frac{1}{N} \sum_{\forall n} A_{\mathbf{E}}\left({o}_{e}^{m},  {a}_{e}^{m, n}, \boldsymbol{a}_{\boldsymbol{\mathbf{V}}}^{m}, \boldsymbol{w}^{m} \mid \theta_{\mathbf{E}}^{\mathscr{A}} \right)
\end{align}
其中,$\theta_{\mathbf{V}}^{Q}$ 和 $\theta_{\mathbf{V}}^{Q}$ 包含相应的AA和SV网络的参数。
\begin{align}
	\theta_{\mathbf{V}}^{Q} = (\theta_{\mathbf{V}}^{\mathscr{A}}, \theta_{\mathbf{V}}^{\mathscr{V}}), \theta_{\mathbf{V}}^{Q^{\prime}} = (\theta_{\mathbf{V}}^{\mathscr{A}^{\prime}}, \theta_{\mathbf{V}}^{\mathscr{V}^{\prime}}) \\
	\theta_{\mathbf{E}}^{Q} = (\theta_{\mathbf{E}}^{\mathscr{A}}, \theta_{\mathbf{E}}^{\mathscr{V}}), \theta_{\mathbf{E}}^{Q^{\prime}} = (\theta_{\mathbf{E}}^{\mathscr{A}^{\prime}}, \theta_{\mathbf{E}}^{\mathscr{V}^{\prime}})
\end{align}

\subsection[\hspace{-2pt}网络学习和更新]{{\CJKfontspec{SimHei}\zihao{4} \hspace{-8pt}网络学习和更新}}

从经验回放缓存$\mathcal{B}$中抽出$M$小批量,以训练车辆和边缘节点的策略和评论家网络,其中单个样本表示为 $\left(\boldsymbol{o}_{\mathbf{V}}^{m}, {o}_{e}^{m}, \boldsymbol{w}^{m}, \boldsymbol{a}_{\mathbf{V}}^{m}, {a}_{e}^{m}, \boldsymbol{r}_{\mathbf{V}}^{m}, \boldsymbol{r}_{e}^{m}, \boldsymbol{o}_{\mathbf{V}}^{m+1}, {o}_{e}^{m+1}, \boldsymbol{w}^{m+1}\right)$。
车辆$v$的目标值表示为:
\begin{equation}
	y_{v}^{m} = \boldsymbol{r}_{v}^{m} \boldsymbol{w}^{m} +\gamma Q_{\mathbf{V}}^{\prime}\left({o}_{v}^{m+1},  {a}_{v}^{m+1}, \boldsymbol{a}_{\boldsymbol{\mathbf{V}}-v}^{m+1}, \boldsymbol{w}^{m+1} \mid \theta_{\mathbf{V}}^{Q^{\prime}} \right)
\end{equation}
\noindent 其中 $Q_{\mathbf{V}}^{\prime}({o}_{v}^{m+1},  {a}_{v}^{m+1}, \boldsymbol{a}_{\boldsymbol{\mathbf{V}}-v}^{m+1}, \boldsymbol{w}^{m+1} \mid \theta_{\mathbf{V}}^{Q^{\prime}})$ 是目标车辆评论家网络产生的动作价值。
$\gamma$是折扣因子。
$\boldsymbol{a}_{\boldsymbol{\mathbf{V}}-v}^{m+1}$ 是没有车辆$v$的下一时刻车辆动作。
\begin{equation}
	\boldsymbol{a}_{\boldsymbol{\mathbf{V}}-v}^{m+1} = \{ {a}_{1}^{m+1}, \ldots, {a}_{s-1}^{m+1}, {a}_{s+1}^{m+1}, \ldots, {a}_{v}^{m+1} \}
\end{equation}
而 ${a}_{v}^{m+1}$ 是目标车辆策略网络根据对下一时刻系统状态的局部观测产生的车辆$v$的下一时刻动作。
\begin{equation}
	{a}_{v}^{m+1} = \mu_{\mathbf{V}}^{\prime}(\boldsymbol{o}_{v}^{m+1} \mid \theta_{\mathbf{V}}^{\mu^{\prime}})
\end{equation}
类似地,边缘节点$e$的目标值表示为:
\begin{equation}
	y_{e}^{m} = \boldsymbol{r}_{e}^{m} \boldsymbol{w}^{m} +\gamma Q_{\mathbf{E}}^{\prime}\left({o}_{e}^{m+1},  {a}_{e}^{m+1}, \boldsymbol{a}_{\boldsymbol{\mathbf{V}}}^{m+1}, \boldsymbol{w}^{m+1} \mid \theta_{\mathbf{E}}^{Q^{\prime}} \right)
\end{equation}
\noindent 其中 $Q_{\mathbf{E}}^{\prime}({o}_{e}^{m+1},  {a}_{e}^{m+1}, \boldsymbol{a}_{\boldsymbol{\mathbf{V}}}^{m+1}, \boldsymbol{w}^{m+1} \mid \theta_{\mathbf{E}}^{Q^{\prime}})$ 表示由目标边缘评论家网络产生的动作价值。
$\boldsymbol{a}_{\boldsymbol{\mathbf{V}}}^{m+1}$ 是下一时刻车辆动作。
${a}_{e}^{m+1}$表示下一时刻边缘节点动作,该动作可由目标边缘策略网络根据其对下一时刻系统状态的局部观测获得,即${a}_{e}^{m+1} = \mu_{\mathbf{E}}^{\prime}(\boldsymbol{o}_{e}^{m+1}, \boldsymbol{a}_{\mathbf{V}}^{m+1} \mid \theta_{\mathbf{E}}^{\mu^{\prime}})$。

车辆评论家网络和边缘评论家网络的损失函数是通过分类分布的时间差分(Temporal Difference, TD)学习得到的,其表示为:
\begin{equation}
	\mathcal{L}\left(\theta_{\mathbf{V}}^{Q}\right)=\frac{1}{M} \sum_{m} \frac{1}{S} \sum_{v} {Y_v^{m}}
\end{equation}
\begin{equation}
	\mathcal{L}\left(\theta_{\mathbf{E}}^{Q}\right)=\frac{1}{M} \sum_{m} {Y_e^{m}}
\end{equation}
\noindent 其中$Y_v^{m}$和$Y_e^{m}$分别是车辆$v$和边缘节点$e$的目标值和局部评论家网络产生的动作价值之差的平方。
\begin{equation}
	\begin{aligned}
		Y_v^{m} &= \left(y_{v}^{m}-Q_{\mathbf{V}}\left({o}_{v}^{m},  {a}_{v}^{m}, \boldsymbol{a}_{\boldsymbol{\mathbf{V}}-v}^{m}, \boldsymbol{w}^{m} \mid \theta_{\mathbf{V}}^{Q} \right)\right)^{2} \\
	\end{aligned}
\end{equation}
\begin{equation}
	\begin{aligned}
		Y_e^{m} &=\left(y_{e}^{m}-Q_{\mathbf{E}}\left({o}_{e}^{m},  {a}_{e}^{m}, \boldsymbol{a}_{\boldsymbol{\mathbf{V}}}^{m}, \boldsymbol{w}^{m} \mid \theta_{\mathbf{V}}^{Q} \right)\right)^{2} \\
	\end{aligned}
\end{equation}
车辆和边缘策略网络参数通过确定性的策略梯度进行更新。
\begin{equation}
	\nabla_{\theta_{\mathbf{V}}^{\mu}} \mathcal{J} (\theta_{\mathbf{V}}^{\mu}) \approx \frac{1}{M} \sum_{m} \frac{1}{S} \sum_{v} P_{v}^{m} 
\end{equation}
\begin{equation}
	\nabla_{\theta_{\mathbf{E}}^{\mu}} \mathcal{J} (\theta_{\mathbf{E}}^{\mu}) \approx \frac{1}{M} \sum_{m} P_{e}^{m} 
\end{equation}
\noindent 其中 
\begin{equation}
P_{v}^{m} = \nabla_{{a}_{v}^{m}} Q_{\mathbf{V}}\left({o}_{v}^{m}, {a}_{v}^{m}, \boldsymbol{a}_{\boldsymbol{\mathbf{V}}-v}^{m}, \boldsymbol{w}^{m} \mid \theta_{v}^{Q} \right) \nabla_{\theta_{\mathbf{V}}^{\mu}} \mu_{\mathbf{V}}\left({o}_{v}^{m} \mid \theta_{\mathbf{V}}^{\mu}\right)
\end{equation}
\begin{equation}
P_{e}^{m} = \nabla_{{a}_{e}^{m}} Q_{\mathbf{E}}\left({o}_{e}^{m}, {a}_{e}^{m}, \boldsymbol{a}_{\boldsymbol{\mathbf{V}}}^{m}, \boldsymbol{w}^{m} \mid \theta_{\mathbf{E}}^{Q} \right) \nabla_{\theta_{\mathbf{E}}^{\mu}} \mu_{\mathbf{E}}\left({o}_{e}^{m}, {\boldsymbol{a}}_{\boldsymbol{\mathbf{V}}}^{m} \mid \theta_{\mathbf{E}}^{\mu}\right)
\end{equation}

本地策略和评论家网络参数分别以$\alpha$和$\beta$的学习率更新。
特别地,车辆和边缘节点定期更新目标网络的参数,即当$t \mod t_{\operatorname{tgt}} = 0$, 其中 $t_{\operatorname{tgt}}$ 是目标网络的参数更新周期。
\begin{align}
	\theta_{\mathbf{V}}^{\mu^{\prime}} \leftarrow n_{\mathbf{V}} \theta_{\mathbf{V}}^{\mu}+(1-n_{\mathbf{V}}) \theta_{\mathbf{V}}^{\mu^{\prime}}, \theta_{\mathbf{V}}^{Q^{\prime}} \leftarrow n_{\mathbf{V}} \theta_{\mathbf{V}}^{Q}+(1-n_{\mathbf{V}}) \theta_{\mathbf{V}}^{Q^{\prime}}\\
	\theta_{\mathbf{E}}^{\mu^{\prime}} \leftarrow n_{\mathbf{E}} \theta_{\mathbf{E}}^{\mu}+(1-n_{\mathbf{E}}) \theta_{\mathbf{E}}^{\mu^{\prime}}, \theta_{\mathbf{E}}^{Q^{\prime}} \leftarrow n_{\mathbf{E}} \theta_{\mathbf{E}}^{Q}+(1-n_{\mathbf{E}})  \theta_{\mathbf{E}}^{Q^{\prime}}
\end{align}
\noindent 其中 $n_{\mathbf{V}} \ll 1$ 和 $n_{\mathbf{E}} \ll 1$。
同样,分布式行动者的策略网络参数也会定期更新,即当$t \mod t_{\operatorname{act}} = 0$,其中 $t_{\operatorname{act}}$ 是分布式行动者的策略网络的参数更新周期。
\begin{align}
	\theta_{\mathbf{V}, k}^{\mu} \leftarrow \theta^{{\mu}^{\prime}}_{\mathbf{V}}, \theta_{\mathbf{V}, k}^{Q} \leftarrow \theta_{\mathbf{V}}^{Q^{\prime}}, \forall k \in \{1, 2, \ldots, K\}\\
	\theta_{\mathbf{E}, k}^{\mu} \leftarrow \theta_{\mathbf{E}}^{\mu^{\prime}}, \theta_{\mathbf{E}, k}^{Q} \leftarrow \theta_{\mathbf{E}}^{Q^{\prime}}, \forall k \in \{1, 2, \ldots, K\}
\end{align}

\section[\hspace{-2pt}实验设置与结果分析]{{\CJKfontspec{SimHei}\zihao{-3} \hspace{-8pt}实验设置与结果分析}}\label{section 4-6}

\subsection[\hspace{-2pt}实验设置]{{\CJKfontspec{SimHei}\zihao{4} \hspace{-8pt}实验设置}}

本章节使用Python 3.9.13和TensorFlow 2.8.0来搭建仿真实验模型以评估所提MAMO方案的性能,其运行在配备AMD Ryzen 9 5950X 16核处理器@ 3.4 GHz,两个NVIDIA GeForce RTX 3090 GPU和64 GB内存的Ubuntu 20.04服务器上。
实验仿真参数设置如下:
V2I通信范围被设定为500 m。
传输功率被设定为100 mW。
AWGN和可靠性阈值分别设置为-90 dBm和0.9\cite{wang2019delay}。
V2I通信的信道衰减增益遵循高斯分布,其均值为2,方差为0.4\cite{sadek2009distributed}。
$\hat{\Theta_{i}}$、$\hat{\Psi_{i}}$、$\hat{\Xi_{i}}$、$\hat{\Phi_{i}}$和$\hat{\Omega_{i}}$的加权系数分别设置为0.6、0.4、0.2、0.4和0.4。

MAMO中策略和评论家网络的架构和超参数描述如下:
本地策略网络是有两层隐藏层的四层全连接神经网络,其中神经元的数量分别为256和128。
目标策略网络的结构与本地策略网络相同。
本地评论家网络是四层全连接神经网络,有两层隐藏层,其中神经元的数量分别为512和256。
目标评论家网络的结构与本地评论家网络相同。
折扣率、批大小和最大经验回放缓存大小分别为0.996、256和1$\times10^{6}$。
策略网络和评论家网络的学习率分别为1$\times10^{-4}$和1$\times10^{-4}$。

进一步,本章节实现了三个比较算法,其具体细节介绍如下:
\begin{itemize}
	\item \textbf{随机分配}: 随机选择动作来确定感知信息、感知频率、上传优先级、传输功率和V2I带宽分配。
	\item \textbf{分布式深度确定性策略梯度}\cite{barth2018distributed}: 在边缘节点实现了一个智能体,根据系统状态,集中式地确定感知信息、感知频率、上传优先级、传输功率和V2I带宽分配。VCPS质量和VCPS 利润权重分别设定为0.5和0.5。
	\item \textbf{多智能体分布式深度确定性策略梯度}: 其为D4PG的多智能体版本,并在车辆上分布式实现,根据对物理环境的局部观测决定感知信息、感知频率、上传优先级和传输功率,边缘节点决定V2I带宽分配。VCPS质量和VCPS 利润权重分别设为0.5和0.5。
\end{itemize}

为了评估算法在视图建模质量和有效性方面的表现,本章设计了以下两个新的指标。
\begin{itemize}
	\item \textbf{单位开销质量}:其定义为花费单位开销实现的VCPS质量,其计算公式为:
		\begin{equation}
			\operatorname{QPUC}=\frac{\sum_{\forall t \in \mathbf{T}} \sum_{\forall e \in \mathbf{E}} \sum_{\forall i \in \mathbf{I}_e^t} \mathrm{QV}_i}{\sum_{\forall t \in \mathbf{T}} \sum_{\forall e \in \mathbf{E}} \sum_{\forall i \in \mathbf{I}_e^t} \mathrm{CV}_i}
		\end{equation}
		其中$\mathrm{QV}_i$和$\mathrm{CV}_i$分别是视图$i$的质量和开销。
	\item \textbf{单位质量利润}:其定义为单位VCPS质量所实现的VCPS 利润,其计算公式为:
		\begin{equation}
		\operatorname{PPUQ}=\frac{\sum_{\forall t \in \mathbf{T}} \sum_{\forall e \in \mathbf{E}} \sum_{\forall i \in \mathbf{I}_e^t}\mathrm{PV}_i}{\sum_{\forall t \in \mathbf{T}} \sum_{\forall e \in \mathbf{E}} \sum_{\forall i \in \mathbf{I}_e^t} \mathrm{QV}_i}
		\end{equation}
		其中$\mathrm{PV}_i$和$\mathrm{CV}_i$分别是视图$i$的利润和开销。
\end{itemize}
QPUC越高表明它能在相同的开销下实现更高的VCPS质量,而PPUQ越高表明它能更有效地使用感知和通信资源。上述指标全面显示了算法在同时最大化VCPS质量和最小化VCPS 开销的性能。
本章进一步基于公式\ref{equ 4-16}、\ref{equ 4-20}、\ref{equ 4-21}和\ref{equ 4-23}设计了四个指标,分别是\textbf{平均及时性}(Average Timeliness, AT)、\textbf{平均冗余度}(Average Redundancy, AR)、\textbf{平均感知开销}(Average Sensing Cost, ASC)和\textbf{平均传输开销}(Average Transmission Cost, ATC)。 

\subsection[\hspace{-2pt}实验结果与分析]{{\CJKfontspec{SimHei}\zihao{4} \hspace{-8pt}实验结果与分析}}

\textbf{1) 算法收敛性:}图\ref{fig 4-3}比较了四种算法的收敛性。其中,图\ref{fig 4-3}(a)和\ref{fig 4-3}(b)分别展示了四种算法的QPUC和PPUQ表现。X轴表示迭代次数,Y轴表示达到的QPUC和PPUQ。QPUC和PPUQ越高,表明算法在VCPS质量和VCPS开销方面表现越好。MAMO在大约850次迭代后,达到了最高的QPUC(约13.6)和最高的PPUQ(约1.13)。相比之下,RA、D4PG和MAD4PG分别实现了约2.29、7.34和2.58的QPUC,并分别实现了约0.87、0.99和0.81的PPUQ。与RA、D4PG和MAD4PG相比,MAMO在QPUC方面分别实现了约494.1\%、85.5\%和428.8\%的提升,在PPUQ方面分别实现了约30.6\%、14.2\%和40.7\%的改善。值得注意的是,MAMO是唯一能够同时改善QPUC和PPUQ的方案。这显示了MAMO在同时实现QPUC和PPUQ最大化方面的优势。

\begin{figure}[h]
 \centering
 \captionsetup{font={small, stretch=1.312}}\includegraphics[width=1\columnwidth]{Fig4-3-different-algorithms.pdf}
 \bicaption[算法收敛性比较]{算法收敛性比较,其显示与RA、D4PG和MAD4PG相比,MAMO在收敛后(约850次迭代)达到了最高的QPUC和最高的PPUQ。(a)单位开销质量(b)单位质量利润}[Convergence comparison]{Convergence comparison, which shows MAMO achieves the highest QPUC and the highest PPUQ compared with RA, D4PG, and MAD4PG after convergence (around 850 iterations). (a) Quality per unit cost (b) Profit per unit quality}
 \label{fig 4-3}
\end{figure}

\begin{figure}[h]
 \centering
 \captionsetup{font={small, stretch=1.312}}\includegraphics[width=1\columnwidth]{Fig4-4-different-networks.pdf}
 \bicaption[隐藏层中不同神经元数量下MAMO性能比较]{隐藏层中不同神经元数量下MAMO性能比较。(a)单位开销质量(b)单位质量利润}[Performance comparison of MAMO under different numbers of neurons in the hidden layers]{Performance comparison of MAMO under different numbers of neurons in the hidden layers. (a) Quality per unit cost (b) Profit per unit quality}
 \label{fig 4-4}
\end{figure}

\textbf{2) 神经元数量的影响:}
图\ref{fig 4-4}比较了不同神经元数量下MAMO的性能。其中,X轴表示策略网络和评论家网络的两个隐藏层的神经元数量,分别设置为[64, 32] $\sim$ [1024, 512]和[128, 64] $\sim$ [2048, 1024]。如图\ref{fig 4-4}(a)所示,当策略网络和评论家网络的隐藏层的神经元数量设置为默认值(即[256, 128]和[512, 256])时,MAMO实现了最高的VCPS质量和利润。图\ref{fig 4-4}(b)比较了其他三个指标,包括AT、ASC和ATC。AT、ASC和ATC越低,说明在信息新鲜度、感知开销和传输开销方面表现越好。可以注意到,当每个隐藏层的神经元数量为默认设置时,MAMO在最小化AT、ASC和ATC方面表现最佳。

\begin{figure}[h]
 \centering
 \captionsetup{font={small, stretch=1.312}}\includegraphics[width=1\columnwidth]{Fig4-5-different-scenarios.pdf}
 \bicaption[不同交通场景下的性能比较]{不同交通场景下的性能比较。(a)单位开销质量(b)单位质量利润(c)平均感知开销(d)平均传输开销}[Performance comparison under different traffic scenarios]{Performance comparison under different traffic scenarios. (a) Quality per unit cost (b) Profit per unit quality (c) Average sensing cost (d) Average transmission cost}
 \label{fig 4-5}
\end{figure}

\textbf{3) 交通场景的影响:}
图\ref{fig 4-5}比较了四种算法在不同交通场景下的性能。X轴表示交通场景,不同场景在不同的时间和空间中提取了现实的车辆轨迹作为输入,分别为:1)2016年11月16日8:00至8:05,中国成都市青羊区1平方公里区域;2)同日23:00至23:05,同一区域;3)2016年11月27日8:00至8:05,中国西安碑林区1平方公里区域。图\ref{fig 4-5}(a)比较了四种算法的QPUC,MAMO在所有场景下都取得了最高的QPUC。图\ref{fig 4-5}(b)比较了四种算法的PPUQ,MAMO在所有情况下都表现最好。与RA、D4PG和MAD4PG相比,MAMO分别提高了589.0\%、106.7\%和514.8\%的QPUC,并分别提高了约41.6\%、23.6\%和45.7\%的PPUQ。图\ref{fig 4-5}(c)比较了四种算法的ASC。MAMO的ASC低于RA、D4PG和MAD4PG,说明MAMO可以实现车辆协同感知以降低感知开销。图\ref{fig 4-5}(d)比较了四种算法的ATC,在不同情况下,MAMO的ATC最低。

\begin{figure}[h]
 \centering
 \captionsetup{font={small, stretch=1.312}}\includegraphics[width=1\columnwidth]{Fig4-6-different-bandwidths.pdf}
 \bicaption[不同V2I带宽下的性能比较]{不同V2I带宽下的性能比较。(a)单位开销质量(b)单位质量利润(c)平均及时性(d)平均冗余度(e)平均感知开销(f)平均传输开销}[Performance comparison under different V2I bandwidths]{Performance comparison under different V2I bandwidths. (a) Quality per unit cost (b) Profit per unit quality (c) Average timeliness (d) Average redundancy (e) Average sensing cost (f) Average transmission cost}
 \label{fig 4-6}
\end{figure}

\begin{figure}[h]
 \centering
 \captionsetup{font={small, stretch=1.312}}\includegraphics[width=1\columnwidth]{Fig4-7-different-numbers.pdf}
 \bicaption[不同视图需求下的性能比较]{不同视图需求下的性能比较。(a)单位开销质量(b)单位质量利润(c)平均及时性(d)平均冗余度(e)平均感知开销(f)平均传输开销}[Performance comparison under different digit twin requirements]{Performance comparison under different digit twin requirements. (a) Quality per unit cost (b) Profit per unit quality (c) Average timeliness (d) Average redundancy (e) Average sensing cost (f) Average transmission cost}
 \label{fig 4-7}
\end{figure}

\textbf{4) V2I带宽的影响:}
图\ref{fig 4-6}比较了四种算法在不同V2I带宽下的性能。X轴表示V2I带宽,从1MHz增加到3MHz。较大的V2I带宽代表每辆车被分配的V2I带宽也随之增加。图\ref{fig 4-6}(a)比较了四种算法的QPUC。随着带宽的增加,MAMO的QPUC也相应增加。这是因为在带宽富余的场景中,MAMO中车辆之间的协同感知和上传更加有效。图\ref{fig 4-6}(b)显示了四种算法的PPUQ,可以进一步证明这一优势。如图\ref{fig 4-6}(b)所示,MAMO在不同的V2I带宽下实现了最高的PPUQ。特别地,与RA、D4PG和MAD4PG相比,MAMO分别提高了约453.3\%、131.4\%和437.6\%的QPUC,并使PPUQ提高了约33.0\%、18.3\%和48.4\%。图\ref{fig 4-6}(c)比较了四种算法的AT,MAMO实现了最低的AT。当带宽从2.5MHz增加到3MHz时,MAMO和D4PG的性能差距很小。这是因为随着带宽的增加,视图的及时性改善是有限的。图\ref{fig 4-6}(d)比较了四种算法的AR。AR越低意味着协同感知和上传的性能越好,MAMO实现了最低的AR。图\ref{fig 4-6}(e)和\ref{fig 4-6}(f)分别比较了四种算法的ASC和ATC。可以看出,当带宽增加时,这四种算法的ATC都会下降。原因是,当带宽增加时,信息上传时间减少,导致传输开销降低。MAMO的ASC和ATC在大多数情况下保持在最低水平。

\textbf{5) 视图需求的影响:}
图\ref{fig 4-7}比较了四种算法在不同视图需求下的性能,其中X轴表示视图所需信息的平均数量从3增加到7。视图所需信息的平均数越大,说明车辆的感知和上传工作负荷越大。图\ref{fig 4-7}(a)比较了四种算法的QPUC。随着平均所需信息数的增加,四种算法的QPUC也相应减少。然而,MAMO在所有情况下保持最高的QPUC。图\ref{fig 4-7}(b)比较了四种算法的PPUQ。正如预期的那样,MAMO在所有情况下都取得了最高的PPUQ。特别地,与RA、D4PG和MAD4PG相比,MAMO的QPUC分别高出458.7\%、130.6\%和426.2\%,PPUQ分别高出31.5\%、18.2\%和40.7\%。图\ref{fig 4-7}(c)比较了四种算法的AT。MAMO在AT方面取得了最佳性能。图\ref{fig 4-7}(d)比较了四种算法的AR,表明MAMO可以在所有情况下实现最低的AR。图\ref{fig 4-7}(e)和\ref{fig 4-7}(f)分别比较了四种算法的ASC和ATC。值得注意的是,当平均信息数增加时,四种算法的ASC和ATC都会增加。原因是视图需要的平均信息量增加,导致车辆感应和传输开销提高。

\section[\hspace{-2pt}本章小结]{{\CJKfontspec{SimHei}\zihao{-3} \hspace{-8pt}本章小结}}\label{section 4-7}

本章提出了协同感知与V2I上传场景,其中基于车辆协同感知与V2I协同上传构建逻辑视图。
具体地,基于多类M/G/1优先级队列构建了协同感知模型,并基于信道衰减分布和SNR阈值构建了V2I协同上传模型。
在此基础上,设计了两个指标QV和CV,以衡量在边缘节点建模的视图的质量和开销,并形式化定义了双目标优化问题,通过协同感知和上传,最大化VCPS质量的同时,最小化VCPS 开销。
进一步,提出了基于多目标的多智能体深度强化学习算法,其中采用了决斗评论家网络,根据状态价值和动作优势来评估智能体动作。
最后,进行了全面的性能评估,证明了所提MAMO算法的优越性。

% \include{contents/system}
% \include{contents/conclusion}
% \include{contents/yourFreeChoise}
\chapter[\hspace{0pt}绪\hskip\ccwd{}论]{{\CJKfontspec{SimHei}\zihao{3}\hspace{-5pt}绪\hskip\ccwd{}论}}
\section[\hspace{-2pt}课题背景与研究意义]{{\CJKfontspec{SimHei}\zihao{-3} \hspace{-8pt}课题背景与研究意义}}\label{section 1-1}
随着通信应用的快速发展,通信技术正经历着一场划时代的变革。
5G已成为新一代通信核心技术,
其高速、低时延的特性广泛应用于无人驾驶、智慧医疗、工业物联网等多个领域 \cite{10201911032}。其中,快速傅立叶变换(Fast Fourier Transform, FFT)
作为一种高效的信号处理算法,在现代5G通信系统中发挥着至关重要的作用\cite{1019620369.nh}。
5G,通信,OFDM,军事侦查,探测准确性

FFT算法是由离散傅立叶变换(Discreate Fourier Transform, DFT)演变而来,降低了算法的复杂度并且提高计算效率,
能实现信号在时域和频域之间的快速变换。
此外,FFT算法还适用于5G通行领域,能够在高速数据传输和多用户接入场景下有效降低运算延迟,提高系统吞吐率。
在现代通信系统中,正交分频复用(Orthogonal Frequency Division Multiplexing,OFDM)技术\cite{1023142698.nh}借助 FFT 进行解调,
使带宽资源分配更加灵活,并有效缓解多径干扰的影响。
为了满足5G对高数据率传输的需求,需要结合先进的FPGA硬件加速平台设计低延迟的FFT硬件架构,
从而为未来6G等更高标准的通信系统奠定坚实基础。在雷达应用场景中\cite{GLGL2025030701Q}\cite{JSJG202411001015},
FFT算法广泛应用于毫米波雷达和激光雷达系统中,能够实现高精度的目标检测。随着传感器灵明度提升,单位时间内会采样更多点数,
这将要求FFT处理器具备处理大点数的能力。
因此,在FPGA边缘计算领域中,低延迟的大点数FFT处理器在军事侦察系统、雷达信号处理系统领域中具有重要研究意义。

除此之外,FFT算法还适用于图像处理领域,随着医学成像、卫星遥感\cite{ZGTB201912003}\cite{DATE202009021}
领域数据量的爆发式增长,传统的1024点FFT处理器已无法满足需求,
高分辨率的亿级像素遥感影像需要高达16TB的存储空间以及单幅4K图像经FFT转换后存储空间激增至原始的2-3倍,因此
需要有足够多的存储空间和计算资源来处理大规模的图像数据。
除此之外,FPGA作为边缘加速设备,可以提高图像处理的效率同时为
开发人员提供硬件层面的可编程性。
为FFT加速器设计提供广泛的设计空间\cite{HKLD202302010},
其中包括RTL仿真、原型验证、板级验证和存储缓冲区设计等。基于上述描述的问题,
在FPGA平台上设计处理大数据量的FFT处理器,有利于快速原型验证并且突破传统FFT处理器的性能瓶颈。
dfas

现有主流的FFT硬件架构主要分为流水线架构和存储架构两大类:
流水线架构通过级联多个蝶形计算单元实现连续数据处理,其显著优势在于低延迟和高吞吐量特性\cite{1023593323.nh},特别适合实时信号处理场景。
该架构可通过配置不同规模的蝶形单元支持可变点数计算,但存在明显的资源利用率问题——当实际处理点数小于硬件设计容量时,
未被使用的蝶形单元将处于闲置状态,造成显著的硬件资源浪费\cite{5599896}。
这种架构的固有局限性在于其硬件结构与FFT点数深度绑定,缺乏运行时灵活性。

存储架构采用"计算-存储-反馈"的迭代机制,通过存储器实现数据重用\cite{liu2018high}。
其工作原理是将中间计算结果暂存于存储器,随后循环读取数据进行蝶形运算并写回存储,
直至完成所有FFT计算阶段。
这种架构的核心优势在于硬件资源的高度复用性,只需要固定的蝶形单元配合存储器即可处理任意点数FFT,
但存储架构存在显著的缺陷,由于各计算阶段存在严格的数据依赖关系,不同批次数据无法进行流水线处理,导致系统整体吞吐量受限。

因此,本文针对上述问题提出了两种新颖的 FFT 加速器架构。
第一种架构旨在融合存储架构和流水线架构的优势,通过自适应选择最佳计算策略,
不仅能够高速处理任意点数的FFT运算,同时还能实现硬件资源的最优利用。
第二种架构则是一种面向高基数、高并行度的大点数运算的FFT存储架构,旨在扩展传统存储架构在基数和并行度上的局限性,
可以灵活地适用不同规模的计算阵列,提高计算的并行度和效率。

% \begin{figure}[h]
% 	\centering
% 	\captionsetup{font={small, stretch=1.312}}
% \includegraphics[width=1\columnwidth]{Fig1-1-V2X.pdf}
% 	\bicaption[车联网演进方向]{车联网演进方向}[Evolution direction of the Internet of Vehicles]{Evolution direction of the Internet of Vehicles}
% 	\label{fig 1-1}
% \end{figure}

% 车联网是物联网(Internet of Things, IoT)技术在汽车领域的应用形式。早在2G/3G移动网络时代,车联网已应用于利用全球导航卫星系统(Global Navigation Satellite System, GNSS)的定位信息为车辆提供防盗和救援服务。如今,智能网联汽车(如宝马、比亚迪、福特、通用、蔚来以及特斯拉等)都支持空中下载(Over-the-Air, OTA)技术对车机系统进行在线更新。如图 \ref{fig 1-1} 所示,随着汽车朝着智能化、网联化、协同化方向发展,传统的面向信息服务的\qthis{车联网}已经转变为与万物互联互通的V2X(Vehicle-to-Everything)车联网。具体而言,V2X车联网是指多种通讯方式的融合,包括车辆间通讯(Vehicle-to-Vehicle, V2V)、车辆与行人通讯(Vehicle-to-Pedestrian, V2P)、车辆与基础设施通讯(Vehicle-to-Infrastructure, V2I)以及车辆与云端通讯(Vehicle-to-Cloud, V2C)。车联网利用实时数据分发,实现人、车、路等交通要素的协同配合,最终实现\qthis{聪明的车、智慧的路、协同的云}。此外,车联网还能促进基于单车智能的自动驾驶技术发展,通过车联网通信协助自动驾驶发现潜在危险,提升道路安全。随着我国车联网产业在政策规划、标准体系建设、关键技术研发、应用示范和基础设施建设等多方面的稳步发展,车联网的内涵和外延也在不断发展演进。依托快速落地的新型基础设施建设,车联网不仅广泛服务于智能网联汽车的辅助驾驶、自动驾驶等不同应用,还拓展服务于智慧矿山、智慧港口等企业生产环节以及智慧交通、智慧城市等社会治理领域\cite{zhong2021che}。

% \begin{figure}[h]
% 	\centering
% 	\captionsetup{font={small, stretch=1.312}}
% \includegraphics[width=0.95\columnwidth]{Fig1-2-V2X-evolution.pdf}
% 	\bicaption[3GPP C-V2X 标准演进]{3GPP C-V2X 标准演进}[3GPP C-V2X standard evolution]{3GPP C-V2X standard evolution}
% 	\label{fig 1-2}
% \end{figure}

% 在车联网通信标准方面,电气和电子工程师协会(Institute of Electrical and Electronics Engineers, IEEE)在2003年提出了专用短距通信技术(Dedicated Short-Range Communication, DSRC)。2010年,IEEE发布了名为无线接入车载环境(Wireless Access in Vehicular Environments, WAVE)的协议栈,其中包括IEEE 802.11p、IEEE 1609.1/.2/.3/.4协议族和SAE J2735消息集字典 \cite{wu2013vehicular}。同时,基于长期演进(Long-Term Evolution, LTE)的V2X通信已形成完善的技术标准体系和产业链\cite{chen2016lte}。此外,IMT-2020(5G)推进组成立了C-V2X工作组,加速基于5G的V2X通信的演进。如图 \ref{fig 1-2} 所示,国际标准组织第三代合作伙伴计划(The 3rd Generation Partnership Project, 3GPP)在2018年启动基于5G新空口(New Radio, NR)的V2X标准研究,并在2020年完成了Rel-16版本的5G NR-V2X标准\cite{saad2021advancements},Rel-17版本进一步优化了功率控制、资源调度等相关技术。5G 汽车协会(5G Automotive Association, 5GAA)、下一代移动通信网络(Next Generation Mobile Network, NGMN)联盟以及5G Americas对IEEE 802.11p和C-V2X进行了技术对比,表\ref{table 1_1}显示C-V2X在传输时延、范围、速率以及可靠性等方面具有显著优势。目前,我国LTE-V2X产业蓬勃发展,与DSRC技术路线之争取得了重大进展。我国已建成基于LTE-V2X技术的完备产业链,芯片、模组、车载终端(Onboard Unit, OBU)、路侧设备(Roadside Unit, RSU)等均已成熟且经过了\qthis{三跨}\qthis{四跨}\qthis{新四跨}以及大规模测试,满足了商用部署条件。

% \begin{table}[h]\small
% \setstretch{1.245} %设置具有指定弹力的橡皮长度(原行宽的1.2倍)
% \captionsetup{font={small, stretch=1.512}}
% \centering
% \bicaption[C-V2X和IEEE 802.11p技术对比]{C-V2X和IEEE 802.11p技术对比\cite{cheng2021feng}}[Technical comparisons of C-V2X and IEEE 802.11p]{Technical comparisons of C-V2X and IEEE 802.11p \cite{cheng2021feng}}
% \label{table 1_1}
% \resizebox{\columnwidth}{!}{%
% \begin{tabular}{@{}ccccc@{}}
% \toprule
% \begin{tabular}[c]{@{}c@{}}C-V2X\\ 技术优势\end{tabular} &
%  \begin{tabular}[c]{@{}c@{}}具体技术\\ 或性能\end{tabular} &
% IEEE 802.11p &
% \begin{tabular}[c]{@{}c@{}}LTE-V2X\\ (3GPP R14/R15)\end{tabular} &
% \begin{tabular}[c]{@{}c@{}}NR-V2X\\ (3GPP R16)\end{tabular} \\ \midrule
% 低时延 &
%   时延 &
%   不确定时延 &
%   \begin{tabular}[c]{@{}c@{}}R14: 20ms\\ R15: 10ms\end{tabular} &
%   3ms \\ 
% \begin{tabular}[c]{@{}c@{}}低时延/\\ 高可靠\end{tabular} &
%   \begin{tabular}[c]{@{}c@{}}资源分配\\ 机制\end{tabular} &
%   CSMA/CA &
%   \begin{tabular}[c]{@{}c@{}}支持感知+半持续\\ 调度和动态调度\end{tabular} &
%   \begin{tabular}[c]{@{}c@{}}支持感知+半持续\\ 调度和动态调度\end{tabular} \\ 
% \multirow{3}{*}[3.4ex]{高可靠} &
%   可靠性 &
%   不保证可靠性 &
%   \begin{tabular}[c]{@{}c@{}}R14: \textgreater{}90\%\\ R15: \textgreater{}95\%\end{tabular} &
%   支持99.999\% \\  
%  &
%   信道编码 &
%   卷积码 &
%   Turbo &
%   LDPC \\ 
%  &
%   重传机制 &
%   不支持 &
%   \begin{tabular}[c]{@{}c@{}}支持HARQ,\\ 固定2次传输\end{tabular} &
%   \begin{tabular}[c]{@{}c@{}}支持HARQ,\\ 传输次数灵活,\\ 最大支持32次传输\end{tabular} \\ 
% \multirow{2}{*}[0ex]{\begin{tabular}[c]{@{}c@{}}更远传输\\ 范围\end{tabular}} &
%   通信范围 &
%   100m &
%   \begin{tabular}[c]{@{}c@{}}R14: 320m\\ R15: 500m\end{tabular} &
%   1000m \\  
%  &
%   波形 &
%   OFDM &
%   \begin{tabular}[c]{@{}c@{}}单载波频分复用\\ (SC-FDM)\end{tabular} &
%   循环前缀(CP)-OFDM \\ 
% \multirow{2}{*}[1.5ex]{\begin{tabular}[c]{@{}c@{}}更高传输\\ 速率\end{tabular}} &
%   \begin{tabular}[c]{@{}c@{}}数据传输\\ 速率\end{tabular} &
%   典型6Mbit/s &
%   \begin{tabular}[c]{@{}c@{}}R14: 约30Mbit/s\\ R15: 约300Mbit/s\end{tabular} &
%   \begin{tabular}[c]{@{}c@{}}与带宽有关,40MHz\\ 时R16单载波2层数据\\ 传输支持约400Mbit/s,\\ 多载波情况下更高\end{tabular} \\ 
%  &
%   调制方式 &
%   64QAM &
%   64QAM &
%   256QAM \\ \bottomrule
% \end{tabular}%
% }
% \end{table}

% \begin{figure}[h] 
% 	\centering
% 	\captionsetup{font={small, stretch=1.312}}\includegraphics[width=1\columnwidth]{Fig1-3-intersection.pdf}
% 	\bicaption[基于车载信息物理融合的智慧全息路口]{基于车载信息物理融合的智慧全息路口}[Intelligent holographic intersection based vehicular cyber-physical fusion]{Intelligent holographic intersection based vehicular cyber-physical fusion}
% 	\label{fig 1-3}
% \end{figure}

% 如图 \ref{fig 1-3} 所示,智慧全息路口是基于车载信息物理融合技术的智慧交通管理系统。它通过将城市道路上的全要素进行数字化还原,为各类智能交通系统应用提供数据支撑。智慧全息路口利用道路基础设施和智能网联汽车上搭载的激光雷达、毫米波雷达、摄像头等多源传感设备,对路口进行全方位感知和全要素采集。通过传感设备实时感知交通流量、车速、车道变化等数据,并结合高精度地图呈现路口数字化上帝视角,精准刻画路口上的每一条车道、每一个交通信息灯的状态,以及每一辆车的行为轨迹。在实现路口全要素数字化还原的过程中,采用车载边缘计算技术将异构感知数据在边缘节点进行融合处理,从而提高数据处理速度,降低数据传输成本。同时,利用目标检测、目标跟踪、行为分析等算法对感知数据进行预处理,进一步提高数据准确性和精度,为后续的交通管理、交通安全和交通规划等应用提供更可靠的数据基础。

% 智慧全息路口不仅可以实现路口全要素数字化还原,而且可以进一步作为车载信息物理融合系统的外在展示和数据内核,支撑各种智能交通系统的应用。例如,全息路口可以为公交优先通行、绿波通行、弱势道路参与者(Vulnerable Road User, VRU)感知等 ITS 应用提供强有力的支持。在公交优先通行方面,全息路口可以根据公交车的实时位置和行驶速度,并结合路口的交通状况,提前调整信号控制策略,使公交车获得更好的通行效率和服务。在绿波通行方面,全息路口可以通过感知路口的交通流量和车速等信息,实现路口绿灯时长的自适应调整,从而实现车辆在绿波通信路段上的高效通行。在 VRU 感知方面,全息路口可以利用摄像头等传感设备或V2P通信感知VRU(如行人、自行车、残疾人等)的存在和行动轨迹,提供实时的路口状态信息和预警信息,保障弱势道路参与者的通行便利和交通安全。

% 通过以上讨论可知,车载信息物理融合系统是实现各类ITS应用的基础。然而,在高异构、高动态、分布式的车联网中,实现车载信息物理融合系统以满足多元ITS应用需求仍然面临着诸多问题和挑战。因此,针对以上问题和挑战,需要进一步展开全面深入的研究。具体地,首先,异构车联网亟需服务架构融合创新,并设计车载信息物理融合质量指标。未来车联网是多服务范式并存的高异构移动网络,因此需要研究异构车联网融合服务架构,以最大化不同服务范式的协同效应,并支持VCPS的部署实现。同时,现有研究都没有对VCPS进行整体深入的评估。因此,需要设计基于多源异质感知信息融合的VCPS质量指标,并通过控制车辆感知行为与资源分配提升VCPS系统质量。其次,异构资源亟需协同优化。车联网中的通信和计算资源分布在不同的车辆和基础设施中,因此需要针对异构资源进行协同优化,支持VCPS中计算任务分布式处理,以进一步提升系统服务质量。再次,车载信息物理融合系统亟需质量-开销均衡优化。车载信息物理融合系统需要在保证实时性和准确性的同时,考虑资源开销和能耗问题。因此,需要研究质量-开销均衡的优化策略,以提高系统的资源利用率的同时降低能耗。最后,亟需实现原型系统以验证VCPS性能。通过在真实车联网环境中搭建原型系统,可以进一步验证车载信息物理融合系统的可行性和有效性,为其在实际应用中提供更可靠的支持和保障。

% \section[\hspace{-2pt}研究动机]{{\CJKfontspec{SimHei}\zihao{-3} \hspace{-8pt}研究动机}}\label{section 1-3}

% 车载信息物理融合系统是实现各类智能交通系统应用的基础,其已成为国内外学术界研究的热点之一。本章节将对国内外相关研究工作进行梳理和总结,并从以下几个方面进行详细阐述:

% \subsection[\hspace{-2pt}车联网服务架构研究与现状]{{\CJKfontspec{SimHei}\zihao{4} \hspace{-8pt}车联网服务架构研究与现状}}

% 随着智能交通系统应用的不断涌现,传统车联网架构已无法满足大规模、高可靠、低时延的需求,因此研究人员正致力于将软件定义网络(Software Defined Network, SDN)新范式应用于车联网中。在软件定义网络中,数据平面是网络中的实际数据传输部分,其负责处理网络流量的传输和转发,而控制平面是网络中的智能部分,负责决策数据包的转发路径和流量管理。SDN通过分离数据平面和控制平面,实现了高度灵活的数据调度策略和网络功能虚拟化(Network Functions Virtualization, NFV)。LIU等人\cite{liu2016cooperative}首次将SDN概念应用于车联网中,提出了软件定义车联网(Software Defined Vehicular Network, SDVN),并在此基础上提出了基于混合V2I/V2V通信的在线协同数据调度算法,以提高数据分发的性能。DAI等人\cite{dai2018cooperative}设计了基于SDN的异构车联网中具有时间约束的时态信息服务调度方案。LUO等人\cite{luo2018sdnmac}提出了基于SDN的媒体接入(Media Access Control, MAC)协议以提高车联网的通信性能。LIU等人\cite{liu2018coding}提出了基于SDN的服务架构,并结合车辆缓存和网络编码来提高带宽利用率。ZHANG等人\cite{zhang2022ac-sdvn}设计了解决SDVN中视频组播安全问题的安全访问控制协议,实现了多播视频请求车辆和RSU的身份认证。ZHAO等人\cite{zhao2022elite}提出了基于智能数字孪生技术的软件定义车联网分层路由方案,克服了SDVN架构中高动态拓扑局限性。LIN等人\cite{lin2023alps}研究了基于SDVN的自适应链路状态感知方案,能够在信标间隔内及时获取链路状态信息,减少数据包丢失。AHMED等人\cite{ahmed2023deep}提出了基于SDVN中车辆传感器负载均衡算法,并提出了数据包级入侵检测模型,可以跟踪并有效识别网络攻击。然而,现有大部分工作都仅是从数据分发、路由缓存、网络安全等方面展开了研究,缺乏对整体架构的深入分析。

% 移动边缘计算(Mobile Edge Computing, MEC)\cite{mao2017a}通过将计算、存储和网络资源靠近移动终端设备,提供更快速、更可靠的服务,同时减少网络流量消耗和服务延迟。越来越多的研究考虑将MEC范式应用在车联网环境中以提高系统实时性、可靠性和安全性。LIU等人\cite{liu2017a}首次将移动边缘计算融入车联网中,提出了车载边缘计算(Vehicular Edge Computing, VEC),并集成了不同类型的接入技术,以提供低延迟和高可靠性的通信。LANG等人\cite{lang2022cooperative}设计了基于区块链技术的协同计算卸载方案,以提高VEC资源的利用效率,并确保计算卸载的安全性。LIU等人\cite{liu2021fog}研究了端边云协同架构中的合作数据传播问题,并提出了基于Clique的算法来联合调度网络编码和数据分发。DAI等人\cite{dai2021edge}设计了基于自适应比特率多媒体流的VEC架构,其中边缘节点给以不同质量等级编码的文件块提供缓存和传输服务。ZHANG等人\cite{zhang2022digital}提出了车载边缘缓存技术,动态调整边缘节点和车辆的缓存能力以提高服务的可用性。LIU等人\cite{liu2020adaptive}提出了两层VEC架构,利用云、静态边缘节点和移动边缘节点来处理时延敏感性任务。LIAO等人\cite{liao2021learning}研究了空地一体的VEC任务卸载策略,其中车辆能够学习具有多维意图的长期策略。LIU等人\cite{liu2023mobility}提出了利用车辆计算资源来提高VEC场景下任务执行效率的计算任务卸载方案。LIU等人\cite{liu2023asynchronous}研究了VEC中任务卸载和资源管理的联合优化问题,并采用异步深度强化学习算法来寻找最优解。然而,上述研究缺乏考虑异构车联网中不同服务架构的协同效应。

% \subsection[\hspace{-2pt}车载信息物理融合系统建模研究与现状]{{\CJKfontspec{SimHei}\zihao{4} \hspace{-8pt}车载信息物理融合系统建模研究与现状}}

% 越来越多研究人员聚焦于车载信息物理融合系统的预测、调度和控制技术,旨在有效提高 VCPS 系统的整体性能和可靠性。在预测技术方面,ZHANG等人 \cite{zhang2021a} 提出了基于 VCPS 架构的车辆速度曲线预测方法,其协同 VCPS 中的不同控制单元来完成速度曲线预测。ALBABA等人 \cite{albaba2021driver} 则结合深度Q网络(DQN)和层次博弈论,对高速公路驾驶场景中的驾驶员行为进行预测,其中k级推理被用来模拟驾驶员的决策过程。ZHANG等人 \cite{zhang2020data} 提出了变道行为预测模型和加速预测模型。在此基础上,对车辆状态进行预测,并通过动态路由算法实现车辆之间的协同合作,以优化资源利用率和降低能源消耗。ZHOU等人 \cite{zhou2021wide} 提出了基于宽-注意力和深度-组合模型用于交通流量预测。其中,宽-注意力模块和深度-组合模块分别用于提取全局关键特征和推广局部关键特征。在调度技术方面,LI等人 \cite{li2020cyber} 考虑了车辆移动性,并开发了基于物理比率-K干扰模型的广播方案,以确保通信的可靠性。LIAN等人 \cite{lian2021cyber} 提出了基于既定地图模型路径规划的调度方法,以优化路径利用效率。在控制技术方面,HU等人 \cite{hu2017cyber} 提出了基于车队头车状态的燃油最优控制器,以优化车辆速度和无级变速箱齿轮比。DAI等人 \cite{dai2016a} 提出了自主交叉路口控制机制,以确定车辆通过交叉路口的优先权。LV等人 \cite{lv2018driving} 提出了用于三种典型驾驶方式下控制车辆加速的自适应算法。上述研究集中于支持 VCPS 的不同技术,如轨迹预测、路径调度和车辆控制等,虽然促进了各种 ITS 应用的实施,但是均建立在高质量物理元素建模信息可用的假设基础上,并未对车载信息物理融合质量定量分析。

% 部分研究工作侧重于利用深度强化学习(Deep Reinforcement Learning, DRL)优化 VCPS 中车辆感知和信息融合。DONG等人 \cite{dong2020spatio} 提出了基于 DQN 的方法,融合本地环境信息并实现可靠的车道变更决策。ZHAO等人 \cite{zhao2020social} 设计了基于近似策略优化(Proximal Policy Optimization, PPO)的社会意识激励机制,以得出最佳的长期车辆感知策略。MIKA等人 \cite{mlika2022deep} 提出了基于深度确定性策略梯度(Deep Deterministic Policy Gradient, DDPG)的解决方案,通过调度资源块和广播覆盖来优化信息时效性。然而,上述算法不能直接应用于 VCPS 中的协同感知和异构信息融合,并且,当考虑到多辆车场景时,上述算法并不适用。另一方面,部分研究对 VCPS 中的信息质量进行了评估。LIU等人 \cite{liu2014temporal} 提出了用于 VCPS 中时态数据传播的调度算法,其在实时数据传播和及时信息感知之间取得了平衡。DAI等人 \cite{dai2019temporal} 提出了进化多目标算法,以提高信息质量和改善数据到达率。LIU等人 \cite{liu2014scheduling} 提出了两种在线算法,通过分析传播特性来调度不同一致性要求下的时态数据传播。RAGER等人 \cite{rager2017scalability} 开发了刻画真实网络随机性的框架,对随机数据负载进行建模以提高信息质量。YOON等人 \cite{yoon2021performance} 提出了车联网合作感知框架,考虑到通信损耗和车辆随机运动,以获得车辆的精确运动状态。上述研究主要聚焦于 VCPS 中数据及时性、准确性或一致性方面的信息质量评估。然而,这些研究仅考虑了同质数据项层面的质量评估,没有针对车载信息物理融合进行质量评估。

% \subsection[\hspace{-2pt}车联网资源分配与任务卸载研究与现状]{{\CJKfontspec{SimHei}\zihao{4} \hspace{-8pt}车联网资源分配与任务卸载研究与现状}}

% 车联网中的资源分配一直是学术界的研究热点 \cite{noor-a-rahim2022a},大量研究人员针对车联网中通信资源分配进行了深入研究。HE等人 \cite{he2022meta} 设计了动态车联网资源管理框架,其采用马尔可夫决策过程(Markov Decision Process, MDP)和分层强化学习相结合的方法,可以显著提高资源管理性能。LU等人 \cite{lu2021user} 提出了基于用户行为的虚拟网络资源管理方法,以进一步优化车联网通信。PENG等人 \cite{peng2020deep} 提出了车联网资源管理方案,通过应用 DDPG 方法解决了多维资源优化问题,实现了资源快速分配,并满足了车联网服务质量(Quality of Service, QoS)要求。WEI等人 \cite{wei2022multi} 针对车联网云计算中的资源分配问题,从提供者和用户双重视角出发,提出了改进的 NSGA-II 算法来实现多目标优化。PENG等人 \cite{peng2021multi} 研究了无人机辅助车联网中的多维资源管理问题,并提出了基于多智能体深度确定性策略梯度(Multi-Agent Deep Deterministic Policy Gradient, MADDPG)的分布式优化方法,实现了车辆资源联合分配。为了进一步提高频谱利用率和支持更多车辆接入,部分研究将非正交多址接入(Non-Orthogonal Multiple Access, NOMA)技术融入车联网中。PATEL等人 \cite{patel2021performance} 评估了基于 NOMA 的车联网通信容量,其数值结果显示,NOMA 通信容量比传统的正交多址接入(Orthogonal Multiple Access, OMA)高出约20\%。ZHANG等人 \cite{zhang2021centralized} 利用基于图的匹配方法和非合作博弈(Non-Cooperative Game, NCG)分布式功率控制,为NOMA车联网开发了集中式两阶段资源分配策略。ZHU等人 \cite{zhu2021decentralized} 考虑随机任务到达和信道波动,提出了最优功率分配策略,以最大化长期的功率消耗和延迟。LIU等人 \cite{liu2019energy} 在基于 NOMA 的车联网中,提出了基于交替方向乘子法(Alternating Direction Method of Multipliers, ADMM)的功率分配算法。然而,上述研究主要是基于单边缘节点的情况,无法处理不同边缘节点之间的相互干扰情况。因此,仍然需要探索更加复杂的多边缘节点环境下的资源分配策略,以提高车联网的性能和可靠性。


% 随着车载边缘计算的发展,大量研究专注于 VEC 中的任务卸载和资源分配。其中,LIU等人\cite{liu2021rtds}提出了多周期任务卸载的实时分布式方法,通过评估 VEC 中的移动性感知通信模型、资源感知计算模型和截止时间感知奖励模型。SHANG等人\cite{shang2021deep}研究了节能的任务卸载,并开发了基于深度学习的算法来最小化能耗。为了最小化执行延迟、能源消耗和支付成本的加权和,LIU等人\cite{liu2022a}提出了结合 ADMM 和粒子群优化(Particle Swarm Optimization, PSO)的任务卸载算法。CHEN等人\cite{chen2020robust}设计了带有故障恢复功能的计算卸载方法,以减少能源消耗并缩短服务时间。为了实现超高可靠低时延通信(ultra-Reliable and Low-Latency Communication, uRLLC)服务需求下最大化吞吐量,PAN等人\cite{pan2022asynchronous}提出了基于异步联合 DRL 的计算卸载方案。ZHU等人\cite{zhu2022a}提出了用于智能反射面(Intelligent Reflecting Surface, IRS)辅助下的 VEC 的动态任务调度算法,优化有限资源分配并考虑了车辆的移动模式、传输条件和任务大小以及并发传输之间的相互干扰。此外,部分研究聚焦于采用多智能体强化学习(Multi-Agent Deep Reinforcement Learning, MADRL)算法的任务卸载和资源分配。ALAM等人\cite{alam2022multi}开发了基于 DRL 的多智能体匈牙利算法,用于 VEC 中的动态任务卸载以满足延迟、能耗和支付费用需求。ZHANG等人\cite{zhang2021adaptive}提出了基于 MADDPG 的边缘资源分配方法,在严格延迟约束下最小化车辆任务卸载成本。为了同时满足严格延迟要求和最小带宽消耗,HE等人\cite{he2021efficient}提出了用于车辆带宽分配的多智能体行动者-评论家(Multi-Agent Actor-Critic, MAAC)算法。然而,以上研究工作都没有考虑实时任务卸载和通信/计算资源分配的协同优化。

% 部分研究专注于VEC的联合通信和计算资源分配。CUI等人\cite{cui2021reinforcement}提出了多目标强化学习方法,通过协同通信和计算资源的分配来减少系统延迟。HAN等人\cite{han2020reliability}设计了基于动态规划(Dynamic Programming, DP)的资源分配方法,实现了耦合车辆通信和计算资源的可靠性计算。XU等人\cite{xu2021socially}采用契约理论为每个潜在的内容供应商和内容请求者分配通信和计算资源。少数研究者研究了联合任务卸载和资源分配。DAI等人\cite{dai2021asynchronous}提出了异步的DRL算法,实现了异构服务器数据驱动的任务卸载。此外,DAI等人 \cite{dai2022a}开发了概率计算卸载方法,根据边缘节点的计算分配概率进行计算卸载的独立调度。NIE等人\cite{nie2021semi}提出了在无人机辅助VEC中基于MADRL算法的联合优化资源分配和功率控制策略。然而,现有研究主要基于集中式调度,通信开销和调度复杂性较高,不适用于大规模的车联网。

% \subsection[\hspace{-2pt}车载信息物理融合质量/开销优化研究与现状]{{\CJKfontspec{SimHei}\zihao{4} \hspace{-8pt}车载信息物理融合质量/开销优化研究与现状}}

% 近年来,许多研究人员致力于提高车载信息物理融合中的 QoS,以提升 ITS 应用的用户体验。其中,WANG等人\cite{wang2016offloading}提出了一种组合优化方法,旨在减少移动数据流量的同时满足 VCPS 中面向 QoS 的服务需求。JINDAL等人\cite{jindal2018sedative}提出了基于 SDN 和深度学习的 VCPS 网络流量控制方案,成功解决了网络流量管理问题。ZHU等人\cite{zhu2022joint}设计了基于双时间尺度 DRL 的方法,以优化基于车辆编队的 VCPS 中的车辆间距和通信效率,同时满足 V2I 通信的 QoS 要求。WANG等人\cite{wang2023a}提出了集群式车辆通信方法,通过公交车聚类和混合数据调度实现了从公交车到普通车辆的有效数据传播并满足了严格和个性化的 QoS 需求。此外,CHEN等人\cite{chen2021qos}致力于解决 IRS 辅助车联网中的频谱共享问题,通过优化车辆的发送功率、多用户检测矩阵、频谱重用以及 IRS 反射系数等参数,提高车联网通信的服务质量。LAI等人\cite{lai2017a}提出了基于 SDN 的流媒体传输方法,根据用户移动信息、播放缓冲区状态和当前网络信号强度向 SDN 控制器提供流媒体传输策略,以实现最小延迟和更好的 QoS。TIAN等人\cite{tian2022multiagent}则设计了基于 MADRL 的资源分配框架,以共同优化信道分配和功率控制,满足车联网中的异构 QoS 需求。同时,ZHANG等人\cite{zhang2020hierarchical}研究了 MEC 车联网中联合分配频谱、计算和存储资源问题,并利用 DDPG 解决该问题,以满足 ITS 应用的 QoS 要求。SODHRO等人\cite{sodhro2020ai}建立了可靠和延迟容忍的无线信道模型和多层边缘计算驱动的框架,有效提升了车联网服务质量。

% 另一方面,部分研究人员致力于降低 VCPS 中的各类开销。ZHAO等人\cite{zhao2021a}设计了基于 SDN 和无人机(Unmanned Aerial Vehicle, UAV)辅助的车辆计算卸载优化框架,其中采用了UAV辅助车辆计算成本优化算法以最小化系统成本。ZHANG等人\cite{zhang2019hybrid}提出了基于蚁群优化和三个变异算子的算法,用于优化具有灵活时间窗口的多目标车辆路径,以最小化行驶成本和车辆固定成本。NING等人\cite{ning2020when}则针对5G车联网中无线频谱有限的问题,构建了智能卸载框架,联合利用蜂窝频谱和未许可频谱来满足车辆需求,并在考虑时延限制的基础上使成本最小化。TAN等人\cite{tan2019twin}提出了基于人工智能(Artificial Intelligence, AI)的多时间尺度框架的联合通信、缓存和计算策略,其中考虑了车辆的移动性和硬服务截止期限约束,并实现了最大化网络成本效益。HUI等人\cite{hui2022collaboration}提出了协作自动驾驶方案,并通过联盟博弈机制来确定最佳车辆分簇,以最小化每个簇的计算成本和传输成本。虽然上述研究对 VCPS 系统中的开销进行了深入研究,但这些研究并未考虑车载信息物理融合系统构建的质量和开销。因此,需要进行对 VCPS 系统本身的评估与质量-开销均衡的深入研究。

% \subsection[\hspace{-2pt}智能交通系统安全相关应用研究与现状]{{\CJKfontspec{SimHei}\zihao{4} \hspace{-8pt}智能交通系统安全相关应用研究与现状}}

% 随着城市化进程的加速和交通流量的不断增加,ITS 安全相关应用的部署可以大幅提高道路交通的安全性。因此,许多研究人员针对驾驶员状态监测、驾驶行为分析、交通监测等方面进行了研究。MUGABARIGIRA等人\cite{mugabarigira2023context}提出了基于车辆行为追踪和驾驶风险分析的导航系统,可提高城市道路上车辆的安全性。CHANG等人\cite{chang2018design}提出了基于可穿戴智能眼镜的疲劳驾驶检测系统,能够实时检测驾驶员的疲劳或嗜睡状态。DUTTA等人\cite{dutta2022design}提出了基于凸优化的鲁棒分布式状态估计系统,可保护连接车辆的传感器数据免受拒绝服务(Denial-of-Service, DoS)或虚假数据注入(False Data Injection, FDI)攻击。WANG等人\cite{wang2021deep}提出了基于深度学习加速器嵌入式平台的鲁棒雨滴检测系统,并利用检测结果自动控制汽车雨刷。SUN等人\cite{sun2022toward}提出了有效的交通估计系统,可通过与过往车辆通信并记录其出现情况来实现自动交通测量,为ITS提供关键信息。

% 部分研究工作从车辆控制、车辆编队控制、路口交通流控制等多个层面对 ITS 安全相关应用展开了深入分析。ZHANG等人\cite{zhang2021data}提出了分布式安全巡航控制系统,利用历史数据建立了车辆行为预测模型和动态驾驶系统模型,并设计了考虑合并行为概率的安全跟驰控制策略。ZHAO等人\cite{zhao2022resilient}提出了具有鲁棒性的车辆编队控制系统,并设计了在多重干扰和 DoS 攻击下恢复机制,降低 DoS 攻击对 VCPS 的不利影响。PAN等人\cite{pan2023privacy}设计了面向车联网的车队隐私保护集结控制系统,通过采样数据的动态加密和解密方案,使得车队之间的通信数据得以保密。LI等人\cite{li2021confidenitality}介绍了低延迟协作安全车辆编队数据传输系统,采用无线电信道相关性的协作密钥协商协议以保证数据传输的安全。KAMAL等人\cite{kamal2021control}提出了多智能体路口交通流控制系统,利用随机梯度方法计算交通信号灯持续时间。LIAN等人\cite{lian2021cyber}提出了基于交通控制的智能物流系统,并设计了改进A*路径规划算法实现主动调度。

% 作为典型 ITS 安全相关应用,车辆碰撞预警已引起广泛研究人员的关注。目前,大多数车辆碰撞预警系统都是基于超声波雷达或激光雷达等测距传感器的。SONG等人\cite{song2018real}提出了实时障碍物检测和状态分类方法,该方法融合了立体摄像头和毫米波雷达,并结合车辆运动模型,通过多个模块感知环境,能够准确快速地判断出\qthis{潜在危险}物体。WU等人\cite{wu2019series}提出了77GHz车辆碰撞预警雷达系统短程天线,该系统采用补丁阵列天线作为基本结构,并采用多层板设计技术使其尺寸更小。然而,这些方案都存在非视距(Non-Line-Of-Sight, NLOS)的问题,即在障碍物遮挡情况下基于视距(Line-Of-Sight, LOS)的方法不再适用。近年来,随着计算机视觉的发展,一些研究集中在基于摄像头实时视频流的碰撞检测上。WANG等人\cite{wang2016vision}提出了车辆制动行为检测方法,利用安装在挡风玻璃上的摄像头来捕获前车信息,以避免与前方车辆相撞。SONG等人\cite{song2018lane}提出了轻量级的基于立体视觉的车道检测和分类系统,以实现车辆的横向定位和前向碰撞预警。然而,基于计算机视觉的方法需要大量数据传输和密集计算,这使得系统的性能无法得到实时响应。另一方面,部分研究考虑了通过 V2X 通信实现碰撞预警。HAFNER等人\cite{hafner2013cooperative}提出了基于 V2V 通信的交叉路口车辆协同防撞分布式算法。GELBAR等人\cite{gelbal2017elastic}提出了基于 V2X 通信的车辆碰撞预警和避免系统。然而,无线通信中的传输时延和数据包丢失等内在特征是不可避免的,对于车辆碰撞预警系统也是不可忽视的。这使得在真实复杂车联网环境中实现实时和可靠的安全关键型服务变得更加困难。

\section[\hspace{-2pt}国内外研究现状]{{\CJKfontspec{SimHei}\zihao{-3} \hspace{-8pt}国内外研究现状}}\label{section 1-2}

\subsection[\hspace{-2pt}FFT算法研究现状]{{\CJKfontspec{SimHei}\zihao{-3} \hspace{-8pt}}FFT算法研究现状}\label{section 1-2-2}

19世纪初,数学家J.B.Fourier首次提出了傅立叶级数的概念,利用三角函数的和来表示一般函数\cite{garrido2009pipelined}。
为了提高离散傅立叶变换的计算效率,快速傅立叶变换算法由此诞生,将原本$O(N^2)$计算复杂度降低至$O(N \log N)$,极大地简少了DFT的运算量。
目前,FFT算法有多种实现形式,其中主要包括了著名的Cooley-Tukey算法\cite{cooley1965algorithm},
Good-Thomas算法\cite{temperton1992generalized}以及Winograd算法\cite{winograd1978computing}。
从而有效减少加法与乘法次数,达到快速计算的目的。Good-Thomas算法又被称为互质因子算法(Prime-Factor Algorithm, PFA),
将长度为$N$的原始序列分解为两个互质整数$N_{1}$和$N_{2}$的乘积,并通过多维索引映射将DFT转换为嵌套的双层DFT计算,从而进一步减少乘法次数。
Winograd算法的核心思想是将DFT转化为多维线性卷积,在小规模应用场景下能有效降低复数乘法次数,但当计算大点数傅立叶变换时,会显著增加加法操作。
此外,Good-Thomas算法在实际应用中适用范围相对有限,难以覆盖一般情况。因此,在实际的应用场景中,依然广泛采用经典的Cooley-Tukey算法。

随着FFT理论的不断完善,研究人员持续探索并提出了多种新型傅里叶变换算法,为高速信号处理领域涉及的复杂问题提供了更多的解决方案。这
些算法主要分为非2次幂整数算法和2次幂整数算法两大类。
其中,基-6\cite{prakash1981new}和基-3\cite{dubois1978new}算法作为非2次幂整数算法的典型代表,
其核心思想是将原始长度为N的序列分解为若干个子序列长度乘积,
并通过多维映射与互质因子分解手段来实现准确计算。但这种非标准基算法会导致硬件实现的复杂度增加,并显著提高了旋转因子的存储开销。
此外,由Duhametl提出的分裂基算法\cite{duhamel1984split},也被广泛应用于非标准基计算中,通过将原始序列划分为
多个不同长度较小的子序列组合,从而降低算法复杂度。

另一方面,对于2次幂长度的序列,算法形式又分为基-4/8/16和更通用的基-$2^k$两大类。
基-4算法是在基-2算法的基础上衍生出来的改进版本,具体而言,每个蝶形单元一次处理四个点,利用旋转因子对称性和周期性特点,
将复数乘法次数从基-2算法的8次降低至3次。除此之外,Torkelson在1996年提出了基-$2^2$算法,同时结合了基-4算法和基-2算法的优点,
使得基-$2^2$算法计算复杂度与基-4算法相同,并且还保留了基-2硬件架构中简单的蝶形运算单元。
基于这种研究思路,高基数的硬件架构设计逐渐成为研究热点,比如基-$2^3$\cite{adiono200964}和基-$2^4$\cite{fan2006low}算法。
2011年,Garrido\cite{garrido2011pipelined}
提出了一种面向硬件架构的通用基-$2^k$ FFT算法,可以灵活地支持不同基数的硬件架构,并且保持流水线的计算高效性。

综上所述,FFT算法朝着高速计算、高吞吐率方向不断发展,经过学术界长达几十年的研究,通过采取高效的分解策略以及高次基数架构,FFT算法
整体计算量显著减少,在数字信号处理领域中具备广泛的实际应用价值。

\subsection[\hspace{-2pt}FFT加速器硬件研究现状]{{\CJKfontspec{SimHei}\zihao{-3} \hspace{-8pt}}FFT硬件架构研究现状}\label{section 1-2-1}

随着各行业对海量数据处理需求的快速增长,FFT 处理器在数字信号处理领域中的作用愈发重要。
随着物联网技术的快速迭代,传统的FFT实现平台在运算吞吐量、延迟以及量化精度等多方面面临严峻挑战。
为全面提升FFT处理器的综合性能,全球研究者不T致力于对 FFT 加速器实现平台进行优化。
结合先进集成电路技术的发展,目前主流的FFT硬件加速平台主要包括以下三种:

(1)通用处理器

CPU作为最常见的通用处理器,能够胜任各种复杂的日常计算任务。
然而,在处理大规模FFT计算时,CPU面临显著的性能瓶颈。

其根源在于冯·诺依曼架构的固有缺陷。尽管通过SIMD指令集(如Intel AVX-512
和多核并行(OpenMP)技术可在4096点双精度FFT中实现微秒级时延(Intel Xeon典型值4.8μs),
但其能效比表现堪忧。
更关键的是,通用处理器的串行指令流架构难以突破内存墙限制:FFT的跨步访问模式导致缓存命中率不足40%,
而DDR4内存带宽(约50GB/s)仅能满足中等规模(≤8192点)FFT需求。
此外,硬件资源的僵化分配使得动态算法切换需付出额外开销(如基-2/基-4混合算法时延波动达15%),
远不及FPGA硬件重构的实时适应性。虽然软件生态(FFTW/MKL库)的成熟性支撑了其在音频处理等非实时场景的应用,
但在5G URLLC(<100μs时延)和雷达信号处理(>1GHz带宽)等严苛场景中,CPU架构的并行粒度不足和确定性缺失问题日益凸显。

(2)专用集成电路


(3)现场可编程门阵列

现场可编程门阵列(Field Programmable Gate Array, FPGA)作为一种半定制集成电路,
凭借其高度可重构性和可编程性,在FFT加速器领域中展现出显著优势。
相比专用集成电路(ASIC),尽管二者在初期设计阶段存在一定相似性,
但FPGA具备更强的灵活性、更短的开发周期、更高的集成度,并提供更便捷的验证方式,使其在高速信号处理和实时计算任务中占据重要地位。
此外,FPGA内部包含了丰富的资源,包括逻辑资源、存储资源和DSP资源,能够支持大规模的FFT计算任务。
同时,FPGA 支持硬件层面的动态重构,可根据具体应用场景实时优化计算结构,从而提升资源利用率,使系统性能达到最优状态。
因此,研究人员选择FPGA作为FFT加速器的最佳开发平台。
目前,主流FFT加速器在处理4096点序列时,
计算时延通常在数十至数百微秒之间。
尽管赛灵思等FPGA厂商提供了定制化FFT IP核,在特定条件下可实现更快的运算速度,但在实际应用中需综合考虑具体场景。
由于定制化FFT软核不具备通用性,且无法修改底层硬件架构,研发人员仍需根据各自需求自行设计并实现满足应用要求的FFT加速器。


\subsection[\hspace{-2pt}研究现状总结]{{\CJKfontspec{SimHei}\zihao{-3} \hspace{-8pt}}研究现状总结}\label{section 1-2-3}

\section[\hspace{-2pt}本文的主要工作及创新]{{\CJKfontspec{SimHei}\zihao{-3} \hspace{-8pt}本文的主要工作及创新}}\label{section 1-3}


\section[\hspace{-2pt}论文的组织架构]{{\CJKfontspec{SimHei}\zihao{-3} \hspace{-8pt}论文的组织架构}}\label{section 1-4}

% \subsection[\hspace{-2pt}论文的主要工作及创新性]{{\CJKfontspec{SimHei}\zihao{4} \hspace{-8pt}论文的主要工作及创新性}}

% 本文针对车联网高动态物理环境、车联网分布式异构节点资源、智能交通系统多元应用需求,以及动态复杂车联网环境所带来的挑战,从架构融合与系统建模、协同资源优化、质量-开销均衡,以及原型系统实现四个方面对车载信息物理融合系统展开研究。本文研究目标概述如下:

% \circled{1} 针对车联网高异构、高动态、高分布式等特征,提出融合软件定义网络和移动边缘计算的车联网分层服务架构,并实现视图质量的量化评估,是车载信息物理融合系统的架构基础与驱动核心。首先,结合软件定义网络、网络功能虚拟化和网络切片(Network Slicing, NS)等关键思想,提出车联网分层服务架构,以支持 VCPS 的部署与实现。其次,提出基于多类 M/G/1 优先队列的感知信息排队模型。进一步,针对边缘视图对于感知信息的时效性、完整性以及一致性需求,设计 VCPS 质量指标,并形式化定义视图质量优化问题。最后,提出基于差分奖励的多智能体强化学习视图质量优化策略,实现高效实时的边缘视图构建。

% \circled{2} 针对车联网中异构节点资源、动态拓扑结构与无线通信干扰等特征,实现基于边缘协同的异构资源优化,是进一步优化 VCPS 服务质量的技术支撑。首先,面向 NOMA 车联网的车载边缘计算环境,考虑 V2I 通信中同一边缘内的干扰和不同边缘间的干扰,提出 V2I 传输模型,并考虑边缘协作提出任务卸载模型。其次,形式化定义协同资源优化问题,并将其分解为任务卸载与资源分配两个子问题。最后,提出基于博弈理论的多智能体强化学习算法的资源优化策略,基于多智能体强化学习实现任务卸载博弈的纳什均衡,并基于凸优化理论提出最优资源分配方案,实现最大化资源利用效率。

% \circled{3} 针对多元智能交通系统应用对于视图质量/开销的差异性需求,实现车载信息物理融合质量-开销均衡,是实现高质量低成本车载信息物理融合的理论保障。首先,考虑视图中信息的及时性与一致性需求,建立车载信息物理融合质量模型。其次,考虑视图构建中感知信息的冗余度、感知开销与传输开销,建立车载信息物理融合开销模型。最后,提出基于多目标多智能体强化学习的质量与开销均衡策略,实现高质量低成本可扩展车载信息物理融合。

% \circled{4} 针对动态复杂车联网环境中验证车载信息物理融合的需求,设计并实现基于车载信息物理融合的原型系统,是验证车载信息物理融合的必要手段。首先,提出基于车载信息物理融合系统优化的碰撞预警算法。其次,搭建基于 C-V2X 设备的硬件在环测试平台,实现硬件在环性能验证。最后,在真实车联网环境中,实现基于车载信息物理融合的超视距碰撞预警原型系统,进一步验证所提算法和系统模型的可行性和有效性。

% \subsection[\hspace{-2pt}论文的组织架构]{{\CJKfontspec{SimHei}\zihao{4} \hspace{-8pt}论文的组织架构}}

% 本文致力于研究车载信息物理融合系统,主要研究内容及关系如图 \ref{fig 1-4} 所示。
% 首先,面向车联网高动态物理环境,融合不同的计算范式与服务架构,并实现有效的数据获取与建模评估是车载信息物理融合的架构基础与驱动核心。
% 因此,本文将首先研究如何设计融合软件定义网络和移动边缘计算的车联网分层服务架构,在此基础上,研究如何评估并提高车载边缘侧所构建的逻辑视图质量。
% 其次,面向车联网分布式异构节点资源,高效的任务调度与资源分配是车载信息物理融合的技术支撑。
% 因此,本文将研究如何实现异构资源协同优化,提高资源利用效率。
% 面向智能交通系统多元应用需求,实现车载信息物理融合质量-开销均衡是车载信息物理融合的理论保障。
% 因此,本文将进一步研究车载信息物理融合质量/开销模型及其均衡优化策略。
% 最后,面向动态复杂车联网环境,基于车载信息物理融合设计并实现具体系统原型是车载信息物理融合的验证手段。
% 因此,本文将更进一步设计及实现基于车载信息物理融合的超视距碰撞预警原型系统,实现理论与系统的相互促进和迭代。
% 本文主要研究内容概述如下:

% \begin{figure}[h] 
% 	\centering
% 	\captionsetup{font={small, stretch=1.312}}\includegraphics[width=1\columnwidth]{Fig1-4-content.pdf}
% 	\bicaption[主要研究内容及关系]{主要研究内容及关系}[Main research content and relationship]{Main research content and relationship}
% 	\label{fig 1-4}
% \end{figure}

% \circled{1} 基于分层车联网架构的车载信息物理融合系统建模。
% 考虑车联网环境中的网络资源的高异构性、车联网物理环境分布式时变性、拓扑结构的高动态性,以及车辆节点感知能力差异性等关键特征,本文将研究融合软件定义网络和移动边缘计算的分层车联网服务架构。进一步,本文将重点研究基于分层服务架构的分布式感知与多源信息融合模型,考虑信息的多维需求,研究车载信息物理融合质量指标设计。在此基础上,研究基于差分奖励的多智能体强化学习(Multi-Agent Difference-Reward-based Deep Reinforcement Learning, MADR)算法的边缘视图优化策略。

% \circled{2} 面向车载信息物理融合的通信与计算资源协同优化。
% 考虑车联网高动态环境与高异构分布式资源,本文将引入NOMA技术提升车联网频谱资源利用效率,并提出基于边缘协同的异构资源优化策略。本文将重点研究 V2I 传输与任务卸载模型,并在此基础上,研究基于博弈理论的多智能体深度强化学习(Multi-Agent Game-Theoretic Deep Reinforcement Learning, MAGT)算法的异构资源协同优化策略,研究基于凸优化理论的通信资源最优分配策略,并研究任务卸载势博弈模型的纳什均衡策略。

% \circled{3} 面向车载信息物理融合的质量-开销均衡优化。
% 考虑智能交通系统中多元应用需求,本文将研究车联网中不同交通要素的视图质量与开销模型,并提出车载信息物理融合质量-开销均衡优化策略。本文将综合考虑视图的建模质量,包括信息的及时性与一致性,研究车载信息物理融合质量模型,并考虑视图的构建开销,包括信息冗余度、感知开销与传输开销,研究车载信息物理融合开销模型。在此基础上,研究基于多目标的多智能体深度强化学习(Multi-Agent Multi-Objective Deep Reinforcement Learning, MAMO)算法的车载信息物理融合质量-开销均衡优化策略。

% \circled{4} 超视距碰撞预警原型系统设计与实现。
% 考虑动态复杂车联网环境,本文将研究基于车载信息物理融合系统的超视距碰撞预警原型系统设计与实现。具体地,本文将研究C-V2X应用层时延拟合模型和数据丢包检测机制,并研究基于车载信息物理融合系统优化的碰撞预警(Vehicular Cyber-Physical System Optimization based Collision Warning, VOCW)算法。在此基础上,研究基于 C-V2X 通信设备的硬件在环试验平台搭建方案,并研究在真实车联网环境中基于车载信息物理融合的超视距碰撞预警原型系统实现方案。

% \section[\hspace{-2pt}本章小结]{{\CJKfontspec{SimHei}\zihao{-3} \hspace{-8pt}本章小结}}\label{section 1-5}
% % \section[\hspace{-2pt}论文的特色与创新之处]{{\CJKfontspec{SimHei}\zihao{-3} \hspace{-8pt}论文的特色与创新之处}}\label{section 1-6}

% % 区别于目前仅专注于车联网通信协议、服务架构、资源分配和智能应用等方面的研究,本文旨在从实际需求出发,分析当前面临的挑战,并在车载信息物理融合系统的四个方面进行深入研究:架构基础与驱动核心、技术支撑、理论保障与验证手段。本文的具体特色在于:
% % a) 针对车联网高动态物理环境和信息感知的时效性与准确性需求,考虑到感知信息时变性、车辆节点移动性和感知能力差异性所带来的挑战,研究如何将基于SDN的集中控制和基于移动边缘计算的分布式调度有机结合,并在边缘侧建立有效的逻辑视图,为车载信息物理融合系统提供架构基础和驱动核心。
% % b) 针对车联网分布式异构节点资源,考虑节点异构资源的动态性、分布性和无线通信中边缘内和边缘间干扰所带来的挑战,研究如何实现边缘协同,最大化异构资源利用效率,为车载信息物理融合系统提供技术支撑。
% % c) 针对智能交通系统多元应用需求,考虑到车联网中不同交通要素视图质量和开销需求差异所带来的挑战,研究如何实现车载信息物理融合系统的质量-开销均衡,为车载信息物理融合系统提供理论保障。
% % d) 针对动态复杂车联网环境,考虑基于真实C-V2X通信设备部署和实现原型系统所带来的挑战,研究基于车载信息物理融合的超视距碰撞预警系统的原型设计和实现,为车载信息物理融合系统提供系统验证。
% % 本文的主要创新点概述如下:

% % \circled{1} 提出融合软件定义网络与移动边缘计算的车联网分层服务架构,并定义边缘视图概念,率先设计视图评估指标并建立视图质量评估模型,提出分布式信息感知与多源信息融合的边缘视图构建机制:现有车联网服务架构相关研究主要关注于单一范式的实践应用,并不适用于具有大规模数据服务需求的下一代车联网场景,无法支撑车载信息物理融合系统等新兴智能交通系统应用。同时,现有研究重点关注于针对单一类型的时态数据建模与调度,难以面向车载信息物理融合系统形成有效的数据支撑。因此,本文首先综合考虑高移动数据节点、高动态网络拓扑、高异构通信资源、高分布式系统环境等车联网特征,设计基于 SDN 集中控制与基于 MEC 分布式服务有机结合的异构车联网架构。在此基础上,综合考虑感知信息的时效性、完整性与一致性,定义车联网边缘视图概念,建立针对视图质量的量化评估模型,并提出基于差分奖励的多智能体强化学习的边缘视图优化策略,实现车载边缘计算环境下的有效信息物理融合。

% % \circled{2} 提出基于边缘协同的异构资源协同优化策略,打破传统的单一资源优化模式:现有面向车联网资源优化策略的研究主要集中于单一资源(如通信、计算)的优化,难以满足车联网节点在不同任务中对异构资源的需求。因此,本文首先针对协同资源优化问题进行分解为任务卸载与通信资源分配两个子问题。进一步,提出基于博弈理论的多智能体深度强化学习的协同资源优化策略。具体地,将任务卸载子问题建模为势博弈模型,并证明其具有纳什均衡存在性与收敛性。最后,针对任务卸载博弈,提出基于多智能体深度强化学习的任务卸载策略。对于通信资源分配,提出基于凸优化的通信资源分配策略,实现最大化异构资源利用效率。

% % \circled{3} 定义车载信息物理融合系统质量与开销模型,提出基于多目标强化学习的优化策略,该策略注重实现 VCPS 质量最大化的同时同时满足 VCPS 开销最小化的要求:现有研究主要关注于基于车载信息物理融合系统的应用,而忽略了车载信息物理融合的质量与开销。因此,本文首先面向多元智能交通系统应用的差异性需求,针对车联网中不同要素建立视图模型。进一步,提出面向车联网中不同实体要素视图的质量/开销模型。最后,提出基于多目标多智能体深度强化学习的车载信息物理融合系统质量-开销均衡优化策略,以实现高质量、低成本和可扩展的车载信息物理融合。

% % \circled{4} 设计并实现面向车载信息物理融合的超视距碰撞预警原型系统,并在真实车联网环境下验证所提算法与系统模型:现有研究主要关注于基于仿真平台的实验验证,难以满足基于车载信息物理融合的实际 ITS 应用在真实车联网环境下的验证需求。因此,本文首先建立基于C-V2X的无线传输时延拟合模型。进一步,提出数据包丢失检测机制,并设计基于车载信息物理融合系统优化的碰撞预警算法。最后,搭建基于C-V2X设备的硬件在环试验平台,并在真实车联网环境中实现超视距碰撞预警系统原型,验证车载信息物理融合的可行性和有效性。

% % \section[\hspace{-2pt}论文的组织结构]{{\CJKfontspec{SimHei}\zihao{-3} \hspace{-8pt}论文的组织结构}}\label{section 1-7}
% % 本文围绕车载信息物理融合系统相关问题展开了研究。
% % 具体地,本文将结合车联网高异构、高动态、分布式特征与智能交通系统多元需求,从车联网的架构融合与系统建模、资源协同优化、质量-开销均衡,以及原型系统实现方面进行理论研究与技术创新。
% % 本文共分为六个章节,详细内容安排如下:

% % 第一章,绪论。首先,介绍了车载信息物理融合系统的研究背景和国内外相关研究现状。其次,阐述了本文的研究目标与详细内容。最后,总结了本文的组织结构。

% % 第二章,基于分层车联网架构的车载信息物理融合系统建模。首先,设计了融合软件定义网络和移动边缘计算的分层服务架构,并提出了分布式感知与多源信息融合场景。在此基础上,设计了 Age of View 指标, 并形式化定义了车载信息物理融合质量最大化问题。其次,提出了基于差分奖励的多智能体深度强化学习的视图质量优化策略。最后,构建了实验仿真模型并验证了所提指标与算法的优越性。本章相关研究已经发表在2019年 IEEE Communications Magazine(中科院 SCI 1区),并已经投稿于 IEEE Transactions on Intelligent Transportation Systems(中科院 SCI 1区)。

% % 第三章,面向车载信息物理融合的通信与计算资源协同优化。首先,提出了协同通信与计算卸载场景。其次,建立了V2I传输模型和任务卸载模型,在此基础上,形式化定义了协同资源优化问题。再次,提出了基于博弈理论的多智能体强化学习的资源优化策略。最后,建立了实验仿真模型并验证了所提算法的优越性。本章相关研究已经发表在2021年电子学报(CCF T1类)和2023年 Journal of Systems Architecture(中科院 SCI 2区)。

% % 第四章,面向车载信息物理融合的质量-开销均衡优化。首先,提出了协同感知与V2I上传场景。其次,建立了VCPS系统质量和系统开销模型,在此基础上,形式化定义了最大化系统质量与最小化系统开销的双目标优化问题。再次,提出了基于多目标的多智能体深度强化学习的质量-开销均衡策略。最后,构建了实验仿真模型并验证了所提算法的优越性。本章相关研究已经投稿于 IEEE Transactions on Consumer Electronics(中科院 SCI 2区)。

% % 第五章,超视距碰撞预警原型系统设计及实现。首先,提出了超视距碰撞预警场景。其次,设计了基于视图修正的车辆碰撞预警算法。再次,搭建了基于 C-V2X 设备的硬件在环试验平台。最后,在真实车联网环境中,实现了基于车载信息物理融合的超视距碰撞预警系统原型,验证了车载信息物理融合的可行性与有效性。本章相关研究已经发表在2020年 Mobile Networks and Applications(中科院SCI 3区)。

% % 第六章,总结与展望。总结了全文研究内容,并讨论了后续研究计划。
\chapter[\hspace{0pt}相关理论及技术]{{\CJKfontspec{SimHei}\zihao{3}\hspace{-5pt}相关理论及技术}}
\removelofgap
\removelotgap
本章将研究基于分层车联网架构的车载信息物理融合系统建模。
本章内容安排如下:
\ref{section 2-1} 节是本章的引言,介绍了车联网服务架构与车载信息物理融合系统质量指标的研究现状、目前研究的不足,以及本章的主要贡献。
% \ref{section 2-2} 节阐述了分层车联网架构设计。
% \ref{section 2-3} 节介绍了分布式感知与多源信息融合场景。
% \ref{section 2-4} 节设计了车载信息物理融合质量指标并形式化定义VCPS质量优化问题。
% \ref{section 2-5} 节提出了基于差分奖励的多智能体强化学习算法用以优化VCPS质量。
% \ref{section 2-6} 节搭建了仿真实验模型并进行了性能验证。
% \ref{section 2-7} 节总结了本章的研究工作。

\section[\hspace{-2pt}FFT硬件架构]
{{\CJKfontspec{SimHei}\zihao{-3} \hspace{-8pt}FFT硬件架构}}\label{section 2-1}

% 随着无线通信技术的蓬勃发展,车联网正逐步成为支持下一代智能交通系统的关键技术。随着 C-V2X 通信、大数据和人工智能的发展,汽车行业的下一场革命即将到来。回顾过去十余年手机的发展历程,可以看到手机已经从传统的通话和信息传递工具转变为具有社交、导航、多媒体娱乐等诸多功能的智能设备。类似于手机的发展,汽车不再仅是单纯的运输工具,而且将朝着智能化、网联化、协同化方向演进,成为支撑各种智能交通系统应用的关键。软件定义网络\cite{li2021zhi}和移动边缘计算\cite{liu2022fedcpf}新兴范式的涌现,为车联网提供了支持高密度车辆通信、海量数据传输、自适应计算卸载,以及逻辑集中控制等功能的解决方案。传感技术和车联网的最新进展也推动了车载信息物理融合系统的发展,并进一步推动下一代智能交通系统的实现。在VCPS中,交通灯信号、车辆位置、点云数据和监控视频等多源信息可以被车辆分布式感知并上传至边缘节点。边缘节点基于车辆感知信息进行融合,构建反映车联网中各元素的物理状态的逻辑映射,其被称为逻辑视图。因此,本章致力于提出一种新颖的分层车联网服务架构,以最大化软件定义网络和移动边缘计算范式的协同效应,并支撑实时、可靠的车载信息物理融合系统。在此基础上,考虑了车载分布式感知与多源信息融合场景,设计了车载信息物理融合质量指标与相应的调度算法,以最大化车载信息物理融合质量。

% 研究人员对车联网服务架构进行了深入研究。自2016年LIU等人\cite{liu2016cooperative}首次将SDN应用于车联网以来,大量研究人员围绕软件定义车联网进行了研究\cite{dai2018cooperative, luo2018sdnmac, liu2018coding, zhang2022ac-sdvn, zhao2022elite, lin2023alps, ahmed2023deep}。然而,现有的大部分工作仅仅是在软件定义车联网架构的基础上,从数据分发、路由缓存、数据安全等方面展开了研究,并没有对整体架构进行深入分析。另一方面,越来越多的研究在车联网环境中考虑将移动边缘计算范式应用于系统,以提高实时性、可靠性和安全性\cite{liu2017a, lang2022cooperative, liu2021fog, dai2021edge, zhang2022digital, liu2020adaptive, liao2021learning, liu2023mobility, liu2023asynchronous}。然而,上述研究并没有考虑在异构车联网中最大化不同服务架构的协同效应。研究人员围绕车联网中的数据传播 \cite{liu2021fog, singh2020intent}、信息缓存 \cite{zhang2022digital, dai2020deep, su2018an} 和任务卸载 \cite{shang2021deep, liao2021learning} 方面展开了深入研究。然而,现有研究工作都没有考虑分布式感知和多源信息融合的协同效应。部分研究人员对VCPS中的预测 \cite{zhang2019a, zhang2020data}、调度 \cite{li2020cyber, lian2021cyber} 和控制 \cite{dai2016a, hu2017cyber, lv2018driving}技术进行了大量的研究,并促进了各种ITS应用的实现。然而,上述研究都是基于边缘/云节点能够收集足够且可靠信息的基础上。部分研究聚焦于VCPS中的信息质量评估\cite{liu2014temporal, dai2019temporal, liu2014scheduling, rager2017scalability, yoon2021performance}。然而,大部分研究工作只评估了数据项层面的质量,而忽略了对多源信息融合的质量评估。部分研究专注于车联网中使用深度强化学习的车辆传感和信息融合\cite{dong2020spatio, zhao2020social, mlika2022deep},但并不适用于多车场景。少数研究将多智能体DRL应用于车联网中\cite{kumar2022multi, he2021efficient}。然而,这些解决方案都不能直接应用于车载信息物理融合系统中的分布式感知和多源信息融合。

% 基于以上分析,本章针对分层架构和质量指标进行了协同研究,在分层车联网架构基础上实现分布式感知和多源信息融合,并提高车载信息物理融合质量。本章的主要贡献如下:第一,提出了融合软件定义网络和移动边缘计算范式的车联网分层服务架构。该架构包含应用层、控制层、虚拟层和数据层,通过分离车联网中控制和数据平面实现了逻辑集中控制,并通过卸载边缘的网络、计算、存储资源实现了基于MEC的分布式服务。第二,提出了分布式感知与多源信息融合的场景,并建立了基于多类M/G/1优先级队列的分布式感知模型。在此基础上,设计了名为Age of View (AoV) 的车载信息物理融合质量指标,用于评估VCPS中多源信息的时效性、完整性和一致性。进一步,形式化定义了车载信息物理融合质量最大化问题。第三,提出了基于差分奖励的多智能体深度强化学习算法来最大化VCPS质量。具体地,车辆作为独立的智能体,其动作空间包含感知频率和上传优先级。然后,设计了基于差分奖励(Difference Reward, DR)的信用分配方案来评估各个车辆对视图构建的贡献,从而提高每个智能体动作的评估精度。同时,设计了基于车辆预测轨迹和视图需求的V2I带宽分配(V2I Bandwidth Allocation, VBA)方案。第四,基于现实世界的车辆轨迹,构建了仿真实验模型,并进行了全面的性能评估。具体地,实现了所提 MADR 算法和 4 种比较算法,其中包括随机分配(Random Allocation, RA)、深度确定性策略梯度\cite{mlika2022deep}、多智能体行动者-评论家\cite{he2021efficient}和采用VBA策略的多智能体行动者-评论家(Multi-Agent Actor-Critic with V2I Bandwidth Allocation, MAAC-VBA)。仿真结果表明,与DDPG、MAAC和MAAC-VBA相比,MADR 算法在提高VCPS质量方面分别高出约61.8\%、23.8\%、22.0\%和8.0\%,收敛速度分别加快了约6.8倍、1.4倍和1.3倍。

% \begin{figure}[h] 
% 	\centering
% 	\captionsetup{font={small, stretch=1.312}}\includegraphics[width=1\columnwidth]{Fig2-1-hierarchical-architecture.pdf}
% 	\bicaption[车联网分层服务架构]{车联网分层服务架构}[Hierarchical architecture for vehicular networks]{Hierarchical architecture for vehicular networks}
% 	\label{fig 2-1}
% \end{figure}
\subsection[\hspace{-2pt}流水型架构]{{\CJKfontspec{SimHei}\zihao{4} \hspace{-8pt}}流水型架构}



\subsection[\hspace{-2pt}存储架构]{{\CJKfontspec{SimHei}\zihao{4} \hspace{-8pt}}存储架构}



\section[\hspace{-2pt}FFT算法表示]{{\CJKfontspec{SimHei}\zihao{-3}\hspace{-8pt}FFT算法表示}}\label{section 2-2}
\subsection[\hspace{-2pt}库里-图基算法]{{\CJKfontspec{SimHei}\zihao{4} \hspace{-8pt}}库里-图基算法}
\subsection[\hspace{-2pt}二叉树表示方法]{{\CJKfontspec{SimHei}\zihao{4} \hspace{-8pt}}二叉树表示方法}
\subsection[\hspace{-2pt}下三角矩阵表示方法]{{\CJKfontspec{SimHei}\zihao{4} \hspace{-8pt}}下三角矩阵表示方法} 

\section[\hspace{-2pt}本章小结]{{\CJKfontspec{SimHei}\zihao{-3} \hspace{-8pt}本章小结}}\label{section 2-3}


% 为了改造和革新传统网络架构,研究人员提出了软件定义网络\cite{wang2020ji},其实现逻辑上的集中控制和网络功能快速迭代。目前,SDN 在云计算系统中的控制和管理已经显现出了巨大优势\cite{jain2013network}。其核心思想是通过解耦网络中的控制平面和数据平面来简化管理,加速网络系统的演进。在控制平面,网络中的控制功能集中于 SDN 控制器,并通过基于软件的方式实时修改和更新网络传输规则。在数据平面,网络节点(如交换机)将根据 SDN 控制器的决策转发数据包。然而,车联网的快速发展给传统的车联网架构带来了诸多挑战。例如,由于传统网络架构中网络控制和数据转发功能耦合,难以满足车联网中时变网络需求,并满足车联网实时性、可靠性和安全性等性能需求。SDN将网络控制和数据转发的功能解耦,实现网络资源的灵活配置和优化。具体地,SDN控制器位于云端,实现对车联网中所有的流量进行集中控制。此外,SDN的虚拟化技术可以将车联网中的物理资源虚拟化,使得网络资源管理更为高效和灵活。基于SDN技术,车联网可以实现更加精细化的管理和调度,提高网络的可靠性和性能,为智能交通系统应用提供更好的支持。考虑到车联网中动态网络拓扑、车辆高移动性和异质通信接口等特点,亟需基于 SDN 的框架来抽象资源,并在该系统中实现最佳服务调度。

% 另一方面,移动边缘计算能够在物联网时代为数十亿联网设备提供高可靠性和低延迟的信息服务 \cite{shi2016edge}。MEC通过将计算、网络、存储资源从云端卸载到终端用户附近,从而有效地缩短数据传输和响应时间,提高服务的可靠性和响应速度。与传统基于云的服务不同,MEC专注于支持高密度的设备连接和网络边缘的密集计算。毋庸置疑,车联网作为物联网中最具代表性的应用场景之一,有望从基于边缘的服务发展中获得巨大的收益。车联网不仅代表着车辆之间的连接,更重要的是,它还代表着行人、道路、基础设施等之间的协作。通过MEC技术,车联网能够实现实时数据采集、处理和传输,使车辆之间的协作更加高效和精确。同时,MEC还可以通过在车辆和设施之间构建更加紧密的联系,实现更加智能的交通控制,从而提高交通安全和效率。值得注意的是,5G技术的成熟和现代汽车在计算、存储和通信能力方面的快速发展,正强力驱动着MEC与车联网的结合 \cite{li2021che}。超可靠和低延迟的5G技术可以大幅提高数据传输和响应速度,进一步提升车联网的效率和可靠性。而现代汽车的智能化趋势也为移动边缘计算的应用提供了更为广泛的可能性。

% 本章提出了车联网分层架构,其各层是根据功能和逻辑划分的,并且结合了SDN和MEC的特点,旨在增强信息服务的可扩展性和可靠性,提高应用管理的敏捷性和灵活性,并为下一代 ITS 的实现奠定坚实的基础。如图 \ref{fig 2-1} 所示,该架构一共具有四层:应用层、控制层、虚拟层和数据层。具体地,应用层是车联网分层架构的最上层,主要包括各种面向业务需求的应用,如车联网中的安全认证、交通管理、数据管理等功能。划分应用层的依据是将车联网的具体应用功能进行分类,并在这一层实现对这些应用的管理和控制。控制层负责管理和控制网络资源,将车联网中的控制功能集中起来。在SDN的思想指导下,控制层采用集中控制的方式来实现逻辑上的集中控制。通过SDN控制器来管理车联网中的网络流量,实时修改和更新网络传输规则。划分控制层的依据是将网络的控制功能从数据转发功能中解耦,以实现网络的灵活配置和优化。虚拟层用于虚拟化和管理车联网中的网络、计算和存储资源。其目的是提供高效和灵活的资源管理,以满足车联网中不同应用的需求。虚拟层可以实现网络功能虚拟化,并为具有不同质量需求的服务实现网络切片。划分虚拟层的依据是将车联网中的物理资源抽象化,并提供虚拟资源的管理和分配。数据层位于架构的底层,负责存储和处理车联网产生的包括车辆信息、交通数据、传感器数据等各种数据。数据层的目标是为上层应用提供数据支持和服务。划分数据层的依据是将数据存储和处理功能与其他层的功能分离,使其可以独立进行优化和管理。本架构的设计整合了SDN和MEC范式,以最大限度地利用它们对车联网信息服务的协同效应。其主要目标包括:a)在高动态车联网环境中实现逻辑上的集中控制;b)在异构车联网环境中实现网络功能虚拟化,并为具有不同QoS要求的服务实现网络切片;c)协调基于云和边缘的服务,最大限度地利用车联网中的网络、计算、通信和存储资源。

% \subsection[\hspace{-2pt}基于移动边缘计算的车联网分布式服务]{{\CJKfontspec{SimHei}\zihao{4} \hspace{-8pt}基于移动边缘计算的车联网分布式服务}}

% 本架构的数据层由如LTE基站、RSU、Wi-Fi接入点(Access Point, AP)、5G小基站和车辆等数据节点组成。除了不同的无线通信接入能力,数据节点还具备一定的计算和存储能力。其中,一些节点被抽象为边缘节点用于提供分布式服务。移动和静态的数据节点可以根据调度服务的需要被动态地分配为边缘节点。特定车辆如公交车和出租车等也可作为移动边缘节点。边缘节点不仅可以根据SDN控制器的规则执行操作,还可以为本地服务实现某些智能,进一步提高服务质量和效率。同时,边缘节点对底层资源进行一定的聚合和抽象,并向虚拟层实时更新状态,从而有助于虚拟资源的管理。因此,SDN控制器可以更方便地进行服务卸载和负载均衡的调度,从而进一步提高整个系统的性能。此外,该架构还具有良好的灵活性和可扩展性。由于边缘节点具有不同的通信接口和计算能力,因此它们可以根据实际需求进行灵活的配置和组合。此外,随着新节点的不断加入,该架构可以随时进行扩展和升级,以适应未来的需求和挑战。

% \subsection[\hspace{-2pt}车联网中网络功能虚拟化和网络切片]{{\CJKfontspec{SimHei}\zihao{4} \hspace{-8pt}车联网中网络功能虚拟化和网络切片}}

% 尽管网络功能虚拟化和网络切片技术在5G网络中已经被广泛研究 \cite{zhu2021wang},但是考虑到车联网中底层资源的高度异构和分布,以及上层ITS应用的高度动态和差异化服务需求等特点,将上述技术应用到车联网中仍然面临巨大挑战。因此,本章专门设计了虚拟层,负责抽象车联网中的计算、网络和存储资源,以在物理基础设施之上提供更高层次的抽象,使得应用程序能够更方便地访问底层资源。然而,由于网络拓扑结构的高动态性、不同的无线通信接口的差异性,以及在数据层的节点之间不断产生、感知和共享的大量信息,构建并维护底层资源的准确逻辑视图是极具挑战的。为了解决这个问题,本章将部分数据节点(如公交车、出租车、5G小基站和RSU等)抽象为边缘节点。边缘节点能够提供基于本地计算、通信和数据资源的服务,并抽象和管理可用的本地资源。一方面,边缘节点可以作为资源的管理者和协调者,负责管理和分配本地的计算、通信和存储资源,为上层应用提供优质的服务。另一方面,边缘节点还可以作为数据的处理中心,对本地产生的数据进行处理和分析,从而降低数据的传输和处理延迟。此外,由于边缘节点本身具有一定的智能,可以对诸如视频流、激光雷达点云数据等进行预处理和分析,从而进一步降低数据的传输和处理压力,提高系统的效率和可靠性。通过上述方式,不仅降低了底层资源的动态性,也减少了上层资源虚拟化的工作量。此外,该分层架构有利于NFV和NS的垂直实施。例如,给定一组具有各自QoS要求的应用,可以根据边缘的分布式调度或SDN控制器的集中式调度,以不同方式对虚拟资源进行协调。

% \subsection[\hspace{-2pt}基于软件定义网络的逻辑集中控制]{{\CJKfontspec{SimHei}\zihao{4} \hspace{-8pt}基于软件定义网络的逻辑集中控制}}

% 在基于软件定义网络的车联网分层服务架构中,SDN控制器被部署在骨干网络中,并通过核心网络与云数据中心和互联网相连。与传统SDN组件类似,该控制器通过北向接口与上层应用进行通信,例如安全认证、交通管理和数据管理等。应用程序需要根据特定的需求使用相应的应用程序编程接口(Application Programming Interface, API)接口实现多维资源(例如计算、通信和存储资源)的分配、车辆行为控制、身份认证以及访问控制等功能。此外,SDN控制器通过南向接口与底层资源进行通信。需要指出的是,控制器不需要直接管理异构的物理资源。相反,通过直接使用虚拟层的资源抽象来获得虚拟资源的统一视图,从而促进SDN控制器的业务调度。虚拟层的资源抽象可以消除底层物理资源的复杂性,并为控制器提供更高的可靠性和性能。因此,通过分层架构的设计,SDN控制器不仅可以更好地管理车联网中的资源,而且可以提高车联网的可靠性和性能。为了更好地支持车联网的高度动态性和异构性,控制层还提供了一些额外的功能。例如,控制层还能够进行动态路由和流量调度,以应对车联网网络拓扑和负载的动态变化。上述功能的整合使得SDN控制器可以更好地适应车联网的复杂环境,并提供高质量的信息服务。由于SDN控制器集中控制网络资源,因此可以提高网络的安全性和可靠性。例如,控制器可以实现安全的访问控制,防止未经授权的用户访问网络资源。此外,控制器还可以实现流量监测和QoS保障,从而提高网络的可靠性和服务质量。上述服务实现的安全和可靠性的特性是车联网应用所必需的。因为上述应用涉及到交通安全和行车效率等关键问题,如果网络不稳定或者容易遭受攻击,则会对ITS应用的正常运行产生重大影响。


% \section[\hspace{-2pt}分布式感知与多源信息融合场景]{{\CJKfontspec{SimHei}\zihao{-3} \hspace{-8pt}分布式感知与多源信息融合场景}}\label{section 2-3}

% 为了实现逻辑集中控制和支持上层 ITS 应用,分层车联网架构中的 SDN 控制器需要准确、及时地构建包含系统全局知识的逻辑视图。因此,本章首先考虑面向车载边缘计算的协作感知和多源信息融合场景,如图 \ref{fig 2-2}(a) 所示,5G 基站和路侧设备(图中$e_1$$\sim$$e_5$)可作为边缘节点提供服务。车辆能够在无线电覆盖范围内通过 V2I 通信与边缘节点进行通信,并通过搭载的车载传感器(例如激光雷达、GPS 和车载摄像头)感知多源信息。显然,车联网中的物理信息具有高度的动态性和时空相关性。同时,搭载传感器的车辆具有异质能力和有限资源,车联网通信也具有间歇性和不可靠性。因此,亟需量身定制的指标来定量评估由边缘节点构建的逻辑视图的质量,从而有效地衡量 VCPS 的整体性能。

% \begin{figure}[h]
%      \centering
%      \captionsetup{font={small, stretch=1.312}}
%      \subfloat[][]{\includegraphics[width=0.7\columnwidth]{Fig2-2a-architerture.pdf}}
%      \subfloat[][]{\includegraphics[width=0.3\columnwidth]{Fig2-2b-procedures.pdf}}
%      \bicaption[系统场景]{系统场景。(a) 车载信息物理融合系统中分布式感知与多源信息融合 (b) 系统工作流程}[Scenario]{Scenario. (a) Distributed sensing and heterogeneous information fusion in VCPS (b) System workflow}
%      \label{fig 2-2}
% \end{figure}

% 本系统的工作流程如图\ref{fig 2-2}(b)所示,边缘节点$e_1$的逻辑视图构建包括以下三个步骤:步骤1(感知):每辆车根据其位置和感知能力感知到不同的信息。感知信息在每辆车上排队,以便上传到边缘节点。每辆车将决定这些信息的感知频率和上传优先级。由于多源信息是由车辆以不同的感知频率感知的,因此不同信息的到达时刻可能不同。同时,提高感知频率可以提高信息的新鲜度,但也会增加排队延迟。为了确定不同信息的上传优先级,必须综合考虑信息的数据量大小、V2I通信的连接性和视图需求。步骤2(上传):边缘节点将V2I带宽(即不同范围的非重叠频谱)分配给有上传任务的车辆,以便这些车辆能够同时上传他们的传感信息而不受干扰。由于边缘节点的带宽资源有限且车辆信道条件多变,分配的V2I带宽可能不足以支持及时上传数据。因此,在车辆准备上传急需的时效信息情况下,将更大的带宽分配给这些车辆,而不是分配给在更差的信道条件下的车辆(如离开V2I覆盖范围)。因此,可以通过对车辆和边缘节点之间的信噪比(Signal-to-Noise Ratio, SNR)建模来考虑了不同车辆的信道条件。同时,V2I传输速率由两个节点之间的距离和分配的带宽决定。步骤3(视图构建):边缘节点根据具体的ITS应用要求,将收到的物理信息映射到相应的逻辑元素上,从而构建逻辑视图。

% 此外,本章提供了一个例子来更好地说明上述观点。如图\ref{fig 2-2}(a)所示,在时间 $t$,边缘节点$e_1$构建了逻辑视图,并根据车辆$v_1$、$v_2$和$v_3$感知和上传的信息,在交叉路口启用了速度建议应用。一般而言,速度建议应用的目标是向正在接近交叉口的车辆提供最佳速度建议,使车辆可以顺利通过,从而达到最大化整体交通效率。假设车辆$v_2$和$v_3$都能感知交通灯信息,但感知的红灯剩余时间数值不一致,进一步导致信息不一致。同时,同一物理要素的状态可能会被多辆车同时感知到。在这种情况下,该消息只需要由其中一辆车在一定时间内上传以节省V2I带宽。只要物理要素在边缘节点以相同的质量水平建模,其就可以被应用于不同的应用,而不需要由不同的车辆重复上传。此外,数据包丢失可能导致物理车联网环境和视图之间的差距。例如,假设车辆$v_2$的位置更新数据包丢失,这会导致其真实位置与时间$t$视图上的位置之间存在明显的不一致。因此,定量评估边缘节点构建的视图的质量,并为协作感知和信息融合设计有效的调度机制,以最大限度地提高VCPS的整体质量是至关重要且具有挑战性的。

% \section[\hspace{-2pt}车载信息物理融合质量指标设计]{{\CJKfontspec{SimHei}\zihao{-3} \hspace{-8pt}车载信息物理融合质量指标设计}}\label{section 2-4}

% \subsection[\hspace{-2pt}基本符号]{{\CJKfontspec{SimHei}\zihao{4} \hspace{-8pt}基本符号}}

% 系统的离散时间片集合用$\mathbf{T}=\{1, \ldots, t, \ldots, T\}$表示。
% 多源信息的集合用$\mathbf{D}=\{1, \ldots, d, \ldots, D\}$表示,其中信息$d \in \mathbf{D}$可以用双元组$d=\left(\operatorname{type}_d, \left|d\right| \right)$表示,其中$\operatorname{type}_d$为类型,$\left|d\right|$为数据量。
% 车辆的集合用$\mathbf{V}=\{1, \ldots, v, \ldots, V\}$表示,其中车辆$v \in \mathbf{V}$的特征用三元组$v=\left (l_v^t, \mathbf{D}_v, \pi_v \right )$表示,其中$l_v^t$是车辆$v$在时间$t$的位置;$\mathbf{D}_v$是车辆$v$可以感知的信息集合,$\pi_v$是车辆$v$的传输功率。
% 边缘节点的集合用$\mathbf{E}=\{1, \ldots, e, \ldots, E\}$表示,其中边缘节点$e \in \mathbf{E}$的特征用三元组$e=\left (l_e, g_e, b_e \right)$表示,其中$l_e$是位置,$g_e$是通信范围,$b_e$是带宽。
% 在时间$t$,车辆$v$与边缘节点$e$的距离用$\operatorname{dis}_{v,e}^t$表示,且$\operatorname{dis}_{v,e}^t \triangleq \operatorname{distance} \left (l_v^t, l_e \right )$,其中$\operatorname{distance}\left(\cdot,\cdot\right)$是欧氏距离。

% 车辆$v$在时间$t$所感知的信息集合用$\mathbf{D}_v^t\subseteq \mathbf{D}_v$表示。
% 对于车辆在时间$t$感知到的信息类型,需要各不相同,即对于$\mathbf{D}_v^t$中的任意信息$d$,信息类型都是不同的,$\operatorname{type}_{d^*} \neq \operatorname{type}_{d}, \forall d^* \in \mathbf{D}_v^t \setminus \left\{ d\right \}, \forall d \in \mathbf{D}_v^t$。
% 车辆$v$在时间$t$对于信息$d$的感知频率用$\lambda_{d,v}^t$表示。
% 由于感知能力有限,车辆感知频率需满足$\lambda_{d,v}^{t} \in [\lambda_{d,v}^{\min} , \lambda_{d,v}^{\max} ], \forall d \in \mathbf{D}_v^t, \forall v \in \mathbf{V}, \forall t \in \mathbf{T}$, 其中 $\lambda_{d,v}^{\min}$ 和 $\lambda_{d,v}^{\max}$ 分别为车辆$v$对于类型为$\operatorname{type}_{d}$的信息的最低和最高感知频率。
% 车辆$v$中的信息$d$在时间$t$的上传优先级用$p_{d,v}^t$表示,且${p}_{d^*, v}^t \neq {p}_{d, v}^t, \forall d^* \in \mathbf{D}_v^t \setminus \left\{ d\right \}, \forall d \in \mathbf{D}_v^t, \forall v \in \mathbf{V}, \forall t \in \mathbf{T}$。
% 在时间$t$内处于边缘节点$e$的无线电覆盖范围内的车辆集合表示为$\mathbf{V}_e^t=\left \{v \vert \operatorname{dis}_{v,e}^t \leq g_e, \forall v \in \mathbf{V} \right \}, \mathbf{V}_e^t \subseteq \mathbf{V}, \forall e \in \mathbf{E}$。
% 边缘节点$e$在时间$t$为车辆$v$分配的V2I带宽用$b_{v, e}^t$表示,且$b_{v, e}^t \in \left [0,b_e \right], \forall v \in \mathbf{V}_e^{t}, \forall e \in \mathbf{E}, \forall t \in \mathbf{T}$。
% 边缘节点$e$分配的V2I带宽总和不能超过其带宽容量$b_e$,即${\sum_{\forall v \in \mathbf{V}_e^{t}}b_{v, e}^t} \leq b_e, \forall e \in \mathbf{E}, \forall t \in \mathbf{T}$。

% \subsection[\hspace{-2pt}系统模型]{{\CJKfontspec{SimHei}\zihao{4} \hspace{-8pt}系统模型}}
% 本系统分布式感知模型如图\ref{fig 2-3}所示。
% 基于多类M/G/1优先级队列(Multi-Class M/G/1 Priority Queue)\cite{qian2020minimizing},建立了车辆中的感知信息排队模型,并进一步得到了感知信息排队时间。
% 具体地,假定车辆$v$中具有相同类型$\operatorname{type}_d$的信息传输时间分布在每个时间片内保持稳定。
% 类型为$\operatorname{type}_d$的信息传输时间$\operatorname{\hat{g}}_{d, v, e}^t$遵循一类一般分布(General Distribution),其均值为$\alpha_{d, v}^t$,二阶矩和三阶矩分别为$\beta_{d, v}^t$和$\gamma_{d, v}^t$,那么该分布集合可以表示为:
% \begin{align}
% 	\mathbb{P}=\left\{\hat{\mathrm{g}}_{d, v, e}^t:\right. & \mathbb{E}\left[\hat{\mathrm{g}}_{d, v, e}^t\right]=\alpha_{d, v}^t, \notag \\
% 	& \mathbb{E}\left[\hat{\mathrm{g}}_{d, v, e}^t-\alpha_{d, v}^t\right]^2=\beta_{d, v}^t \notag \\
% 	& \mathbb{E}\left[\operatorname{\hat{g}}_{d, v, e}^t-\alpha_{d, v}^t\right]^{3}=\left.\gamma_{d, v}^t \right\}
% \end{align}
% 因此,上传负载 $\rho_{v}^{t}$ 可表示为:
% \begin{equation}
%     \rho_{v}^{t}=\sum_{\forall d \subseteq \mathbf{D}_v^t} \lambda_{d,v}^{t}  \alpha_{d, v}^t
% \end{equation}
% \noindent 其中,$\lambda_{d,v}^{t}$和$\alpha_{d, v}^t$分别为车辆$v$在时间$t$对于信息$d$的感知频率和传输时间均值。

% \begin{figure}[h]
% \centering
%   \captionsetup{font={small, stretch=1.312}}\includegraphics[width=0.8\columnwidth]{Fig2-3-cooperative-sensing.pdf}
%   \bicaption[分布式感知模型]{分布式感知模型}[Distributed sensing model]{Distributed sensing model}
%   \label{fig 2-3}
% \end{figure}

% 为了确保队列具有稳定状态,需要满足 $\rho_{v}^{t} < 1$。
% 到达间隔时间$\operatorname{a}_{d, v}^t$是指车辆$v$中两个相邻的具有相同类型$\operatorname{type}_d$的信息到达时间差,其计算公式为:
% \begin{equation}
%     \operatorname{a}_{d, v}^t=\frac{1}{\lambda_{d, v}^{t}}
% \end{equation}
% 在时间$t$内,车辆$v$中具有比信息$d$更高上传优先级的信息集合可表示为:
% \begin{equation}
% \mathbf{D}_{d, v}^t=\left\{d^* \mid p_{d^*, v}^t>p_{d, v}^t, \forall d^* \in \mathbf{D}_v^t\right\} 
% \end{equation}
% 其中$p_{d^*, v}$是信息$d^* \in \mathbf{D}_v^t$的上传优先级。
%   因此,信息$d$前面的上传负载(车辆$v$在时间$t$内要在$d$前面上传的数据量)表示为:
% \begin{equation}
% \rho_{d, v}^t=\sum_{\forall d^* \in \mathbf{D}_{d, v}^t} \lambda_{d^*, v}^t \alpha_{d^*, v}^t
% \end{equation}
% 其中$\lambda_{d^*, v}^t$和$\alpha_{d^*, v}^t$分别是时间$t$内车辆$v$中信息$d^*$的感知频率和平均传输时间。
% 车辆$v$中类型为$\operatorname{type}_d$的信息的排队时间用$\operatorname{q}_{d, v}^t$表示。
% 根据Pollaczek$-$Khintchine公式\cite{takine2001queue},平均排队时间$\operatorname{\bar{q}}_{d, v}^t$计算如下:
% \begin{equation}
%     \operatorname{\bar{q}}_{d, v}^t= \frac{1} {1 - \rho_{d, v}^{t}} 
%         \left[ \alpha_{d, v}^t + \frac{ \lambda_{d, v}^{t} \beta_{d, v}^t + \sum\limits_{\forall d^* \in \mathbf{D}_{d, v}^t} \lambda_{d^*,s}^t \beta_{d^*, v}^t }{2\left(1-\rho_{d, v}^{t} - \lambda_{d, v}^{t}  \alpha_{d, v}^t\right)}\right] 
%         - \alpha_{d, v}^t
% \label{equ 2-6}
% \end{equation}
% 在车辆$v$中类型为$\operatorname{type}_d$的信息排队时间方差由公式\ref{equ 2.7}得到,其中$\alpha_{d, v}^t$、$\beta_{d, v}^t$和$\gamma_{d, v}^t$分别是信息$d$传输时间的平均值、二阶矩和三阶矩。
% \begin{align}
% 	{Var}(\operatorname{q}_{d, v}^t) &= \frac{\beta_{d, v}^t}{(1- \rho_{d, v}^{t})^2} + \frac{\alpha_{d, v}^t \sum\limits_{\forall d^* \in D_{d, v}^t} \lambda_{d^*, v}^t \beta_{d^*, v}^t}{(1- \rho_{d, v}^{t})^3} + \frac{\lambda_{d, v}^{t} \gamma_{d, v}^t + \sum\limits_{\forall d^* \in D_{d, v}^t} \lambda_{d^*, v}^t \gamma_{d^*, v}^t}{3(1- \rho_{d, v}^{t})^2(1-\rho_{d, v}^{t} - \lambda_{d, v}^{t}  \alpha_{d, v}^t)} \notag \\ 
% 	& + \frac{(\lambda_{d, v}^{t} \beta_{d, v}^t + \sum\limits_{\forall d^* \in D_{d, v}^t} \lambda_{d^*, v}^t \beta_{d^*, v}^t) \sum\limits_{\forall d^* \in D_{d, v}^t} \lambda_{d^*, v}^t \beta_{d^*, v}^t }{2(1- \rho_{d, v}^{t})^3(1-\rho_{d, v}^{t} - \lambda_{d, v}^{t}  \alpha_{d, v}^t)} - \beta_{d, v}^t \notag \\
% 	& + \frac{(\lambda_{d, v}^{t} \beta_{d, v}^t + \sum\limits_{\forall d^* \in D_{d, v}^t} \lambda_{d^*, v}^t \beta_{d^*, v}^t)^2}{4(1- \rho_{d, v}^{t})^2(1-\rho_{d, v}^{t} - \lambda_{d, v}^{t}  \alpha_{d, v}^t)^2}
% \label{equ 2.7}
% \end{align}
% 根据切比雪夫不等式,有以下不等式:
% \begin{equation}
% 	\operatorname{Pr}(|\operatorname{q}_{d, v}^t - \operatorname{\bar{q}}_{d, v}^t| > j \sqrt{{Var}(\operatorname{q}_{d, v}^t)}) \leq \frac{1}{j^2}, j \in \mathbb{R}^{+}
% \end{equation}
% 因此,在99\%的置信度下,排队时间的上界可以通过下式得到:
% \begin{equation}
% 	\sup_{\operatorname{Pr}}{\operatorname{q}_{d, v}^t} \leq \operatorname{\bar{q}}_{d, v}^t + 10  \sqrt{{Var}(\operatorname{q}_{d, v}^t)}
% \end{equation}

% 为了进一步分析感知信息集合$D_v^t$中不同元素的平均排队时间和上传优先级之间的关系,公式\ref{equ 2-6}可以改写为:
% \begin{equation}
% \overline{\mathrm{q}}_{d, v}^t=\frac{\rho_{d, v}^t \alpha_{d, v}^t}{1-\rho_{d, v}^t}+\frac{\lambda_{d, v}^t \beta_{d, v}^t+\sum_{\forall d^* \in D_{d, v}^t}^t \lambda_{d^*, v}^t \beta_{d^*, v}^t}{2\left(1-\rho_{d, v}^t\right)\left(1-\rho_{d, v}^t-\lambda_{d, v}^t \alpha_{d, v}^t\right)}
% \end{equation}
% 假设有$n$种信息,信息${d^1}$具有最高的上传优先级,即$D_{d^1, v}^t = \emptyset$。
% 那么,信息${d^1}$的平均排队时间可以通过以下方式计算:
% \begin{equation}
% \operatorname{\bar{q}}_{d^{1}, v}^t=\frac{\lambda_{d^1, v}^t \beta_{d^1, v}^t}{2}
% \end{equation}
% 其中$\lambda_{d^1, v}^t$和$\beta_{d^1, v}^t$分别为信息$d^1$的感知频率和传输时间的二阶矩。
% 另一方面,信息${d^n}$的上传优先级最低。
% 由于要求$\rho_v^t < 1$以保证队列的稳定性和排队时间的有限性,可得到:
% \begin{equation}
% \rho_{d, v}^t=\sum_{\forall d^* \in D_{d, v}^t} \lambda_{d^*, v}^t \alpha_{d^*, v}^t<\sum_{\forall d \subseteq D_v^t} \lambda_{d, v}^t \alpha_{d, v}^t=\rho_v^t<1
% \end{equation}
% 同样地, $\rho_{d, v}^t+\lambda_{d, v}^t \alpha_{d, v}^t<1$。
% 当 $n$ 趋于无穷大时,由于$\lim _{n \rightarrow \infty}(1-\rho_{d^n, v}^t) \rightarrow 0$,类似地,$\lim _{n \rightarrow \infty}(1-\rho_{d^n, v}^t-\lambda_{d^n, v}^t \alpha_{d^n, v}^t) \rightarrow 0$,所以信息${d^n}$的平均排队时间由下式得到:
% \begin{equation}
% \begin{aligned}
% 	\lim _{n \rightarrow \infty}\left(\mathrm{\bar{q}}_{d^n, v}^t\right)&=\frac{\lambda_{d^n, v}^t \beta_{d^n, v}^t+\sum_{\forall d^* \in D_{d^n, v}^t} \lambda_{d^*, v}^t \beta_{d^*, v}^t}{2\left(1-\rho_{d^n, v}^t\right)\left(1-\rho_{d^n, v}^t-\lambda_{d^n, v}^t \alpha_{d^n, v}^t\right)} + \frac{\rho_{d^n, v^t}^t \alpha_{d^n, v}^t}{1-\rho_{d^n, v}^t}\rightarrow \infty
% \end{aligned}
% \end{equation}
% 其中,$\lambda_{d^n, v}^t$、$\alpha_{d^n, v}^t$和$\beta_{d^n, v}^t$分别为信息$d^n$的感知频率、传输时间平均值和传输时间二阶矩。

% 本章根据香农理论对通过V2I通信的数据上传进行建模。
% 在时间$t$,车辆$v$和边缘节点$e$的V2I通信的信噪比用$\operatorname{SNR}_{v, e}^{t}$表示,其计算方法如公式\ref{equ 2-7}\cite{sadek2009distributed}所示。
% \begin{equation}
%     \label{equ 2-7}
%     \operatorname{SNR}_{v, e}^{t}=\frac{1}{N_{0}}  \left|h_{v, e}\right|^{2} \zeta  {\operatorname{dis}_{v, e}^{t}}^{-\varphi} {\pi}_v
% \end{equation}
% 其中$N_{0}$为加性白高斯噪声(Additive White Gaussian Noise, AWGN);$h_{s, e}$为信道衰减增益;$\zeta$为取决于天线设计的常数;$\varphi$为路径损耗指数。
% 车辆$v$和边缘节点$e$之间在时间$t$的V2I传输率用$\operatorname{z}_{v, e}^t$表示,其计算如下: 
% \begin{equation}
%     \operatorname{z}_{v, e}^t=b_{v, e}^{t} \log _{2}\left(1+\mathrm{SNR}_{v, e}^{t}\right)
%     \label{equ 2-8}
% \end{equation}
% 其中$b_{v, e}^{t}$是分配给车辆$v$在时间$t$的带宽。
% 值得注意的是,给定车辆$v$的传输功率$\pi_s$,车辆$v$和边缘节点$e$之间在时间$t$的V2I通信的信噪比可以通过公式\ref{equ 2-7}得到,进一步可由公式\ref{equ 2-8}得到传输速率。
% 因此,信息$d$从车辆$v$到边缘节点$e$的传输时间用$\mathrm{w}_{d, v, e}^t$表示,其计算公式为:
% \begin{equation}
% 	\mathrm{w}_{d, v, e}^t=\frac{\left|d\right|}{\operatorname{z}_{v, e}^t}
% \end{equation}
% 成功传输需要在数据包传输过程中,接收到的信噪比高于某个阈值,其被称为 SNR Wall \cite{tandra2008snr},该阈值通过以下方式获得:
% \begin{equation}
% \mathrm{SNR}_{\text {wall }}=\frac{\sigma^{2}-1}{\sigma}
% \end{equation}
% 其中$\sigma=10^{\nu / 10}$,$\nu$是以dB衡量的参数,量化了噪声不确定性的大小。
% \begin{equation}
% 	\left(\nu^2 - 1\right) {N_0}={\pi_v} \nu
% \end{equation}
% 因此,表示信息$d$是否从车辆$v$成功传输到边缘节点$e$的成功传输指示器表示为:
% \begin{numcases}{\operatorname{c}_{d, v, e}^t=}
% 1, \forall {t^{*}} \in\left[t + \operatorname{\bar{q}}_{d, v}^t, t + \operatorname{\bar{q}}_{d, v}^t + \operatorname{w}_{d, v, e}^t\right], \operatorname{SNR}_{v, e}^{t^{*}}>\mathrm{SNR}_{\text {wall }} \notag \\
% 0, \exists {t^{*}} \in\left[t + \operatorname{\bar{q}}_{d, v}^t, t + \operatorname{\bar{q}}_{d, v}^t + \operatorname{w}_{d, v, e}^t\right], \operatorname{SNR}_{v, e}^{t^{*}} \leq \mathrm{SNR}_{\text {wall }}
% \end{numcases}
% 由车辆$v$传输并由边缘节点$e$接收的信息集合表示为 $\mathbf{D}_{v, e}^t = \{ d \mid \operatorname{c}_{d, v, e}^t = 1, \forall d \in \mathbf{D}_v \}, \mathbf{D}_{v, e}^t \subseteq \mathbf{D}_v^t, \forall v \in \mathbf{V}, \forall e \in \mathbf{E}$。

% \subsection[\hspace{-2pt}车载信息物理融合质量指标]{{\CJKfontspec{SimHei}\zihao{4} \hspace{-8pt}车载信息物理融合质量指标}}
% 系统中的视图集合用$\mathbf{I}$表示,视图$i \in \mathbf{I}$所需的信息集用$\mathbf{D}_{i}$表示,它是特定ITS应用所需的物理交通元素的映射,它表示为:
% \begin{equation}
% 	\mathbf{D}_{i} = \{d \mid y_{d, i} = 1, \forall d \in \mathbf{D} \}
% \end{equation}
% 视图$i$所需元素的数量用$|\mathbf{D}_{i}|$表示。
% 边缘节点$e$在时间$t$所需的视图集合用$\mathbf{I}_e^t \subseteq \mathbf{I}$表示。
% 因此,边缘节点$e$收到的并被视图$i$需要的信息集用下式表示:
% \begin{equation}
%     \mathbf{D}_{i, e}=\bigcup_{\forall i \in \mathbf{I}}\left(\mathbf{D}_i \cap \mathbf{D}_{v, e}^t\right), \forall v \in \mathbf{V}_e^t, \forall e \in \mathbf{E}
% \end{equation}
% 且$| \mathbf{D}_{i, e} |$是边缘节点$e$收到并被视图$i$需要的信息数量。
% 接下来,定义多源信息融合的三个特征,包括视图的时效性、完整性和一致性。

% 首先,多源信息是随时间变化的,信息的新鲜度对于视图质量至关重要。
% 因此,车辆$v$中的信息$d$的时效性定义如下:
% \begin{definition}
% 	车辆$v$的信息$d$的时效性 $\xi_{d,v} \in (0, +\infty)$被定义为信息$d$的间隔到达时间、排队时间和传输时间之和。
% 	\begin{equation}
%     	\xi_{d, v} = \operatorname{a}_{d, v}^t + \operatorname{q}_{d, v}^t + \operatorname{w}_{d, v, e}^t, \forall d \in \mathbf{D}_v^t, \forall v \in \mathbf{V}
% 	\end{equation}
% \end{definition}
% \noindent 其中 $\operatorname{a}_{d, v}^t$、$\operatorname{q}_{d, v}^t$ 和 $\operatorname{w}_{d, v, e}^t$ 分别为信息$d$的间隔到达时间、排队时间和传输时间。
% 进一步,视图的时效性定义如下:
% \begin{definition}
% 视图$i$的时效性 $\Xi_{i} \in (0,+\infty)$被定义为信息时效性总和。
% 	\begin{equation}
%     	\Xi_{i} = \sum_{\forall v \in \mathbf{V}} \sum_{\forall d \in \mathbf{D}_{i, e} \cap \mathbf{D}_v^t } \xi_{d, v}, \forall i \in \mathbf{I}_e^t, \forall e \in \mathbf{E}
% 	\end{equation}
% \end{definition}

% 其次,车联网具有包括车辆高移动性、网络资源有限性和无线通信不可靠的固有特性。
% 由于车辆和边缘节点之间的无线传输连接断开,或者传输过程中数据包的丢失,视图可能是不完整的。
% 因此,视图的完整性定义如下:
% \begin{definition}
% 	视图$i$的完整性$\Phi_{i} \in [0,1]$被定义为边缘节点$e$实际收到的信息数量与所需总量之比。
% 	\begin{equation}
% 	\Phi_{i}= {| \mathbf{D}_{i, e} |} \big/ {|D_{i} |}, \forall i \in \mathbf{I}_e^t, \forall e \in \mathbf{E}
% 	\end{equation}
% \end{definition}
% \noindent 其中$|\mathbf{D}_{i, e}|$是边缘节点$e$收到并被视图$i$需要的信息数量,$|\mathbf{D}_{i}|$是视图$i$需要的信息总数量。

% 再次,由于不同类型的信息有各自的感知频率和上传优先级,在构建视图时,需要使不同类型信息的版本尽可能接近。
% 因此,视图的一致性定义如下: 
% \begin{definition}
% 视图$i$的一致性$\Psi_{i} \in (0,+\infty)$被定义为信息接收时间与视图所需信息的平均接收时间之差的二次方和。
% \begin{equation}
% \Psi_{i}=\sum_{\forall v \in \mathbf{V}} \sum_{\forall d \in \mathbf{D}_{i, e} \cap \mathbf{D}_v^t} \left|\operatorname{q}_{d, v}^t + \operatorname{w}_{d, v, e}^t - \psi_{i} \right|^{2}, \forall i \in \mathbf{I}_e^t, \forall e \in \mathbf{E}
% \end{equation}
% \end{definition}
% \noindent 其中 $\psi_{i}$ 是视图$i$所需信息的平均接收时间,其可由下式得到:
% \begin{equation}
% 	\psi_{i} = \frac{1}{|D_{i, e}|} {\sum_{\forall v \in \mathbf{V}}\sum_{\forall d \in D_{i, e} \cap \mathbf{D}_v^t} \left( \operatorname{q}_{d, v}^t + \operatorname{w}_{d, v, e}^t\right) }, \forall i \in\mathbf{I}_e^t, \forall e \in \mathbf{E}
% \end{equation}

% 最后,本章给出了Age of View的正式定义,其综合了视图的时效性、完整性和一致性。
% \begin{definition}
% Age of View $\operatorname{AoV}_{i} \in (0, 1)$ 被定义为视图$i$的归一化时效性、完整性和一致性的加权平均值。
% 	\begin{equation}
% 	    \operatorname{AoV}_{i} = w_1  \hat{\Xi}_{i} + w_2  \hat{\Phi}_{i}+  w_3 \hat{\Psi}_{i}, \forall i \in \mathbf{I}_e^t, \forall e \in \mathbf{E}
% \end{equation}
% \end{definition}
% \noindent 其中,$\hat{\Xi}_{i} \in (0, 1)$、$\hat{\Phi}_{i} \in (0, 1)$和$\hat{\Psi}_{i} \in (0, 1)$分别表示视图$i$的归一化时效性、归一化完整性和归一化一致性。
% $\operatorname{AoV}_{i}$的值越低,说明构建的视图质量越高。
% 需要注意的是,由于视图的时效性、完整性和一致性的维度不同,为了形成AoV的统一表示,将它们归一化到$(0,1)$范围内,具体如下:
% \begin{numcases}{}
% \hat{\Xi}_{i} = {\Xi}_{i} \big/ \left( \delta_\xi | \mathbf{D}_{v, e} |   T \right) \notag \\ 
% \hat{\Phi}_{i} = 1 - {\Phi}_{i}  \notag \\
% \hat{\Psi}_{i} = {\Psi}_{i} \big/ \left( \delta_\psi  \max\limits_{\substack{\forall d \in \mathbf{D}_v \cap \mathbf{D}_v^t \\ \forall v \in \mathbf{V}}}{\left\{ \left|\operatorname{q}_{d, v}^t + \operatorname{g}_{d, v, e}^t - \psi_{i} \right|^{2}\right\}}   \right)
% \end{numcases}
% \noindent 其中$\delta_{\xi} \in(0,1)$和$\delta_\psi \in(0,1)$分别是时效性和一致性的数据比例系数,通过缩减时效性和一致性的理论最大值避免归一化结果将大部分数值集中在小范围内。
% $\hat{\Xi}_{i}$、$\hat{\Phi}_{i}$和$\hat{\Psi}_{i}$的加权系数分别用$w_1$、$w_2$和$w_3$表示,且$w_1+w_2+w_3=1$。
% 加权系数可以根据ITS应用的不同要求进行相应的调整。
% 例如,对于道路交叉口的速度咨询应用,车辆需要从边缘节点接收实时速度的指令,以便安全顺利地通过交叉口。
% 在这种情况下,时效性因素(例如,实时交通灯状态)与完整性因素(例如,行人在视图中被建模)相比,在视图建模中更为重要。

% 鉴于上述指标AoV是单独评估视图的质量,本章进一步在系统层面上定义VCPS的质量如下:
% \begin{definition}
% VCPS的质量$\Upsilon \in (0,1)$被定义为在调度期$\mathbf{T}$中边缘节点的每个视图$i$的AoV的补集平均值。
% \begin{equation}
% \Upsilon=\frac{\sum_{\forall t \in \mathbf{T}} \sum_{\forall e \in \mathbf{E}} \sum_{\forall i \in \mathbf{I}_e^t} \left(1 - \operatorname{AoV}_{i}\right)}{\sum_{\forall t \in \mathbf{T}} \sum_{\forall e \in \mathbf{E}} |\mathbf{I}_e^t| }
% \end{equation}
% \end{definition}

% \subsection[\hspace{-2pt}问题定义与分析]{{\CJKfontspec{SimHei}\zihao{4} \hspace{-8pt}问题定义与分析}}

% 给定解决方案$(\bf\Lambda, \mathbf{P}, \mathbf{B} )$,其中$\bf\Lambda$表示确定的感知频率,$\mathbf{P}$表示确定的上传优先级,$\mathbf{B}$表示确定的V2I带宽分配,它们分别表示为:
% \begin{numcases}{}
% {\bf\Lambda} = \left\{ \lambda_{d, v}^{t} \vert \forall d \in \mathbf{D}_v^t  , \forall v \in \mathbf{V}, \forall t \in \mathbf{T} \right\} \notag \\ 
% \mathbf{P} = \left \{ p_{d, v}^{t} \vert \forall d \in \mathbf{D}_v^t  , \forall v \in \mathbf{V}, \forall t \in \mathbf{T}\right \} \notag \\
% \mathbf{B} = \left \{ b_{v, e}^t \vert \forall v \in \mathbf{V}_e^t, \forall e \in \mathbf{E}, \forall t \in \mathbf{T}\right \}
% \end{numcases}
% \noindent 其中,$\lambda_{d,v}^{t}$ 表示车辆$v$在时间$t$对信息$d$的感知频率,$p_{d, v}^{t}$ 表示车辆$v$在时间$t$对信息$d$的上传优先级,$b_{v, e}^t$ 表示边缘节点$e$在时间$t$为车辆$v$分配的V2I 带宽。

% 本章旨在通过车辆间分布式感知与边缘节点的多源信息融合以构建边缘视图并进一步实现高质量车载信息物理融合。
% 本章的目标问题是通过确定所有车辆上不同信息感知频率、上传优先级,以及边缘节点对于通信覆盖范围内所有车辆进行V2I带宽分配,以最大限度地提高VCPS的质量。
% 因此,最大化VCPS质量问题形式化定义如下:
% \begin{align}
% 	\mathcal{P}2.1: &\max_{\mathbf\Lambda, \mathbf{P}, \mathbf{B}} \Upsilon \notag \\
% 	\text { s.t. }
%     \mathcal{C}2.1: & \lambda_{d,v}^{t} \in \left [\lambda_{d,v}^{\min} , \lambda_{d,v}^{\max} \right ], \forall d \in \mathbf{D}_v^t , \forall v \in \mathbf{V}, \forall t \in \mathbf{T} \notag \\
%      \mathcal{C}2.2: &{p}_{d^*, v}^t \neq {p}_{d, v}^t, \forall d^* \in \mathbf{D}_v^t \setminus \left\{ d\right \}, \forall d \in \mathbf{D}_v^t, \forall v \in \mathbf{V}, \forall t \in \mathbf{T} \notag \\
%     \mathcal{C}2.3: & b_{v, e}^t \in \left[ 0 , b_e \right ], \forall v \in \mathbf{V}_e^t, \forall e \in \mathbf{E}, \forall t \in \mathbf{T} \notag \\
%     \mathcal{C}2.4: & \sum_{\forall d \subseteq \mathbf{D}_v^t} \lambda_{d,v}^{t}  \alpha_{d, v}^t < 1,\ \forall v \in \mathbf{V}, \forall t \in \mathbf{T}  \notag \\
%     \mathcal{C}2.5: & {\sum_{\forall v \in \mathbf{V}_e^{t}}b_{v, e}^t} \leq b_e, \forall e \in \mathbf{E}, \forall t \in \mathbf{T}
% \end{align}
% 约束条件$\mathcal{C}2.1$要求车辆$v$中的信息$d$在时间$t$的感知频率应满足其感知能力的要求。
% $\mathcal{C}2.2$ 保证时间$t$内车辆$v$中信息$d$的上传优先级。
% $\mathcal{C}2.3$ 规定边缘节点$e$在时间$t$为车辆$v$分配的V2I带宽不能超过其带宽容量$b_e$。
% $\mathcal{C}2.4$保证在调度周期$\mathbf{T}$内队列稳定状态。
% $\mathcal{C}2.5$要求边缘节点$e$分配的V2I带宽之和不能超过其容量$b_e$。

% 本章进一步基于Weierstrass定理\cite{liu2020zui}分析问题$\mathcal{P}2.1$解的存在性。首先,为了方便分析,本章记问题解为$x = (\bf\Lambda, \mathbf{P}, \mathbf{B} )$,目标函数为$f(x) = -\Upsilon$,因此,问题$\mathcal{P}2.1$可以重写为:
% \begin{align}
% 	\mathcal{P}2.2: &\min_{x \in \mathbf{X}} f(x) \notag \\
% 	\text { s.t. }
%     \mathcal{C}2.1 &\sim \mathcal{C}2.5
% \end{align}
% \noindent 进一步,适当函数(一类重要的广义实值函数)的定义如下:
% \begin{definition}
% 	给定广义实值函数 $f$ 和非空集合 $\mathbf{X}$,如果存在 $x \in \mathbf{X}$ 使得 $f(x) < +\infty$,并且对于 $\forall x \in \mathbf{X} $,都有 $f(x) > -\infty$,那么称函数 $f$ 关于集合 $\mathbf{X}$ 是适当的。
% \end{definition}
% \noindent 显然,问题$\mathcal{P}2.2$中的目标函数是关于集合$\mathbf{X}$的适当函数,其定义域表示为:
% \begin{equation}
% \operatorname{\mathbf{dom}} f \triangleq \{x \in \mathbf{X}: f(x)<+\infty\}
% \end{equation}
% 显然,问题$\mathcal{P}2.2$的定义域是有界的。因为 $f$ 是适当的,即存在 $x_0 \in \mathbf{X}$ 使得 $f(x_0) < +\infty$。令 $\bar{\digamma}=f\left(x_0\right)$,则下水平集 $\mathbf{C}_{\bar{\digamma}}$ 是非空有界的。
% 假设下确界$\varrho \triangleq \inf_{x \in \mathbf{X}} f(x)=-\infty$,则存在点列$\left\{x^k\right\}_{k=1}^{\infty} \subset \mathbf{C}_{\bar{\digamma}}$,使得$\lim _{k \rightarrow \infty} f\left(x^k\right)=\varrho=-\infty$。
% 因为 $\mathbf{C}_{\bar{\digamma}}$ 的有界性,点列$\left\{x^k\right\}_{k=1}^{\infty}$一定存在聚点,记为$x^*$。根据上方图的闭性,可得$\left(x^*, \varrho\right) \in \mathbf{e p i} f $,即有$f(x^*) \leq \varrho = -\infty$,其与函数的适当性矛盾,故 $\varrho > -\infty$。
% 因此,问题$\mathcal{P}2.2$的最小值点集 $\{x \in \mathbf{X} \mid f(x) \leqslant f(y), \forall y \in \mathbf{X}\}$ 是非空且紧的,故问题$\mathcal{P}2.1$具有解存在性。


% \section[\hspace{-2pt}基于差分奖励的多智能体强化学习算法设计]{{\CJKfontspec{SimHei}\zihao{-3} \hspace{-8pt}基于差分奖励的多智能体强化学习算法设计}}\label{section 2-5}

% \subsection[\hspace{-2pt}算法模型]{{\CJKfontspec{SimHei}\zihao{4} \hspace{-8pt}算法模型}}
% 本章将详细介绍所提基于差分奖励的多智能体深度强化学习算法,其模型如图\ref{fig 2-4}所示,由$V$辆车、边缘节点$e$、VCPS环境和经验回放缓存组成。
% 首先,车辆$v$决定其动作$\boldsymbol{a}_{v}^{t}$,包括确定感知频率和上传优先级。
% 特别地,车辆$v$动作由行动者网络生成,其输入是对系统状态的局部观测$\boldsymbol{o}_{v}^{t}$。
% 车辆$v$的评论家网络评估由相应行动者网络产生的动作。
% 其次,边缘节点$e$根据车辆预测轨迹和视图需求决定其动作$\boldsymbol{a}_{e}^{t}$,即为通信覆盖范围内的车辆分配V2I带宽。
% 再次,环境根据动作$\{ \boldsymbol{a}_{1}^{t}, \ldots, \boldsymbol{a}_{v}^{t}, \ldots, \boldsymbol{a}_{V}^{t}, \boldsymbol{a}_{e}^{t}\}$ 获得系统奖励,即边缘节点$e$在时间$t$实现的VCPS质量。
% 并采用基于差分奖励\cite{foerster2018counterfactual}的信用分配,将系统奖励分为差分奖励$\{r_1^t, \ldots, r_{V}^t\}$,其中$r_v^t$被用来评估车辆$v$对视图构建的贡献。
% 最后,相关的交互经验包括当前系统状态、车辆动作、差分奖励和下一时刻系统状态,都存储在经验回放缓存中,并用来训练车辆的行动者和评论家网络。
% 算法模型的主要组成部分设计如下:

% 1) \textbf{系统状态}: 边缘节点定期广播其视图需求和缓存信息。在时间$t$内,车辆$v$的系统状态的本地观测被表示为:
% 	\begin{equation}
% 		\boldsymbol{o}_{v}^{t}=\left\{\mathbf{D}_{v}^{t}, \mathbf{D}_{e}^{t}, \mathbf{I}_e^t\right\}
% 	\end{equation} 
% 	\noindent 其中$\mathbf{D}_{v}^{t}$表示车辆$v$在时间$t$感知的信息集合;
% 	$\mathbf{D}_{e}^{t}$ 表示在时间$t$边缘节点$e$中的缓存信息集合,
% 	以及$\mathbf{I}_e^t$表示边缘节点$e$在时间$t$的边缘节点所需的视图集合。
% 	那么,时间$t$的系统状态可表示为:
% 	\begin{equation}
% 		\boldsymbol{o}^{t}=\left\{\mathbf{D}_{1}^{t}, \ldots, \mathbf{D}_{v}^{t}, \ldots, \mathbf{D}_{V}^{t}, \mathbf{D}_{e}^{t}, \mathbf{I}_{e}^{t}\right\}
% 	\end{equation}
	
% \begin{figure}[t]
% \centering
%   \captionsetup{font={small, stretch=1.312}}\includegraphics[width=1\columnwidth]{Fig2-4-solution-model.pdf}
%   \bicaption[基于差分奖励的多智能体深度强化学习模型]{基于差分奖励的多智能体深度强化学习模型}[Multi-agent difference-reward-based deep reinforcement learning model]{Multi-agent difference-reward-based deep reinforcement learning model}
%   \label{fig 2-4}
% \end{figure}

% 2) \textbf{动作空间}: 车辆$v$的动作空间由时间$t$的感知频率和传感信息的上传优先级组成,它被表示为:
% 	\begin{equation}
% 		\boldsymbol{a}_{v}^{t}=\{ \lambda_{d, v}^{t}, p_{d, v}^{t} \mid \forall d \in \mathbf{D}_{v}^t\}
% 	\end{equation}
% 	\noindent 其中$\lambda_{d, v}^{t}$和$p_{d, v}^{t}$分别是时间$t$内车辆$v$中信息$d$的感知频率和上传优先级。
% 	车辆动作的集合用$\boldsymbol{a}_{\mathbf{V}}^{t} = \left\{\boldsymbol{a}_{v}^{t}\mid \forall v \in \mathbf{V}\right\}$表示。
% 	边缘节点的动作是对车辆进行V2I带宽分配,其表示为:
% 	\begin{equation}
% 		\boldsymbol{a}_{e}^{t}=\{b_{v, e}^{t} \mid \forall v \in \mathbf{V}_{e}^{t}\}
% 	\end{equation}
% 	其中$b_{v, e}^t$是边缘节点$e$在时间$t$为车辆$v$分配的V2I带宽。
	
% 3) \textbf{系统奖励}: 在系统状态$\boldsymbol{o}^{t}$下,通过车辆动作$\boldsymbol{a}_{\mathbf{V}}^{t}$和边缘节点动作$\boldsymbol{a}_{e}^{t}$的系统奖励被定义为$t$时边缘节点$e$实现的VCPS质量,其计算公式为:
% 	\begin{equation}
% 		r\left(\boldsymbol{a}_{\mathbf{V}}^{t},\boldsymbol{a}_{e}^{t} \mid \boldsymbol{o}^{t}\right)=\frac{1}{\left|\mathbf{I}_e^t\right|} \sum_{\forall i \in \mathbf{I}_e^t}\left(1 -\operatorname{AoV}_{i} \right)
% 	\end{equation}
	
% 系统奖励展示了整个系统的综合表现,该表现来自于车辆和边缘节点的共同努力。
% 为了评估各车辆的贡献,需要将系统奖励分配给每个车辆作为个人奖励。
% 基于DR的信用分配方案是通过计算系统奖励与无该智能体动作所获奖励之间的差值来确定该智能体的个人奖励,可以更准确地评估每个智能体的行为,从而进一步提升所提出解决方案的性能。
% 因此,车辆$v$的差分奖励表示为:
% \begin{equation}
% r_{v}^{t}=r\left(\boldsymbol{a}_{\mathbf{V}}^{t},\boldsymbol{a}_{e}^{t} \mid \boldsymbol{o}^{t}\right)-r\left(\boldsymbol{a}_{\mathbf{V}-v}^{t},\boldsymbol{a}_{e}^{t} \mid \boldsymbol{o}^{t}\right)
% \end{equation}
% \noindent 其中 $r\left(\boldsymbol{a}_{\mathbf{V}-v}^{t},\boldsymbol{a}_{e}^{t} \mid \boldsymbol{o}^{t}\right)$是没有车辆$v$贡献的系统奖励,其可通过设置车辆$v$的空动作集得到。
% 车辆的差分奖励集合用$\boldsymbol{r}_{\mathbf{V}}^{t}=\{ r_{v}^{t} \mid \forall v \in \mathbf{V}\}$表示。

% \subsection[\hspace{-2pt}工作流程]{{\CJKfontspec{SimHei}\zihao{4} \hspace{-8pt}工作流程}}
% 本章节介绍基于差分奖励的多智能体强化学习算法的工作流程,其主要包括三个部分,即初始化、回放经验存储和训练,其详细步骤见算法2.1。

% 1) \textbf{初始化}: 首先,每辆车都作为智能体并由四个神经网络组成,即本地行动者网络、目标行动者网络、本地评论家网络和目标评论家网络。
% 车辆$v$的本地行动者和本地评价家网络的参数分别用$\theta_{v}^{\mu}$和$\theta_{v}^{Q}$来表示。
% 目标行动者和目标评论家网络的参数分别用$\theta_{v}^{\mu^{\prime}}$和$\theta_{v}^{Q^{\prime}}$表示。
% 其次,车辆的本地行动者和本地评价家网络的参数通过随机方式进行初始化。
% 目标行动者和目标评论家网络的参数初始化为与相应的本地网络一致。
% \begin{align}
% 	\theta_{v}^{\mu^{\prime}} \leftarrow \theta_{v}^{\mu}, \forall v \in \mathbf{V}\\
% 	\theta_{v}^{Q^{\prime}} \leftarrow \theta_{v}^{Q}, \forall v \in \mathbf{V}
% \end{align}
% 最后,初始化最大容量为$|\mathcal{B}|$的经验回放缓存以存储车辆与环境的交互经验。

% \SetKwInOut{KwIn}{输入}
% \SetKwInOut{KwOut}{输出}

% \begin{algorithm}[h]\small
% \setstretch{1.245} %设置具有指定弹力的橡皮长度(原行宽的1.2倍)
% \renewcommand{\algorithmcfname}{算法}
% 		\caption{基于差分奖励的多智能体深度强化学习}
% 		\KwIn{学习率$\alpha$和$\beta$、折扣因子$\tau$、经验回放缓存$\mathcal{B}$、批大小$M$、轨迹预测时间$H$}
% 		\KwOut{信息感知频率$\lambda_{d, v}^{t}$、上传优先级$p_{d, v}^{t}$、带宽分配$b_{v, e}^{t}$}
% 		初始化网络参数\\
% 		初始化经验回放缓存$\mathcal{B}$\\
%         \For{\songti{迭代次数} $= 1$ \songti{到最大迭代次数}}{
%             初始化随机过程 $\mathcal{N}$ 以进行探索 \\
%             接收初始系统状态 $\boldsymbol{o}_{1}$\\
%             \For{\songti{时间片} $t = 1$ \songti{到} $T$}{
%             	\For{\songti{车辆} $v=1$ \songti{到} $V$ }{
%             			接收本地观测值 $\boldsymbol{o}_{v}^{t}$ \\
%                     	选择动作 $\boldsymbol{a}_{v}^{t}=\boldsymbol{\mu}_{v}\left(\boldsymbol{o}_{v}^{t} \mid \theta_{v}^{\mu}\right)+\mathcal{N}_{t}$ \\
%             		得到所需信息 $\mathbf{D}_{v,\operatorname{R}}^{t}$\\
%         		通过基于EM方法利用历史相对距离来预测移动模式\\
%         		预测未来的轨迹 $\operatorname{Traj}_{v}^{t}$ \\
%         		计算平均距离$\operatorname{\bar{dis}}_{v, e}^{t}$
%             	}
%         	\For{\songti{车辆} $v=1$ \songti{到} $V$ }{
%         		通过VBA策略分配带宽 $b_{v, e}^{t}$ 给车辆 $s$\\}
%             	接收系统奖励 $r\left(\boldsymbol{a}_{\mathbf{V}}^{t},\boldsymbol{a}_{e}^{t} \mid \boldsymbol{o}^{t}\right)$ 和下一时刻系统状态 $\boldsymbol{o}^{t+1}$\\
%             	划分系统奖励为差分奖励$\boldsymbol{r}_{\mathbf{V}}^{t}$\\
%             	存储 $\left(\boldsymbol{o}^{t}, \boldsymbol{a}_{\mathbf{V}}^{t}, \boldsymbol{r}_{\mathbf{V}}^{t}, \boldsymbol{o}^{t+1}\right)$ 到经验回放缓存 $\mathcal{B}$
%             }
%             \For{\songti{车辆} $v=1$ \songti{到} $V$ }{
%             		从经验回放缓存$\mathcal{B}$随机采样 $M$ 最小批\\
%             		更新本地行动者和评论家网络参数\\
%             	}
%             	更新目标行动者和评论家网络参数
%        	}
% \label{algorithm 2-1}
% \end{algorithm}

% 2) \textbf{回放经验存储}:
% 在每次迭代的开始,初始化随机过程$\mathcal{N}$用于增加智能体探索。
% 车辆$v$在时间$t$的动作是由本地行动者网络基于系统状态的本地观察得到:
% \begin{equation}
% 	\boldsymbol{a}_{v}^{t}=\boldsymbol{\mu}_{\boldsymbol{v}}\left(\boldsymbol{o}_{v}^{t} \mid \theta_{v}^{\mu}\right)+\mathcal{N}_{t}
% \end{equation}
% \noindent 其中,$\mathcal{N}_{t}$是由随机过程$\mathcal{N}$得到的探索噪音,以增加车辆动作的多样性。

% 边缘节点$e$根据车辆预测轨迹和视图需求,通过VBA方案分配V2I带宽。
% 首先,边缘节点$e$根据车辆和边缘节点之间的历史距离,使用期望最大化(Expectation-Maximization, EM)方法\cite{hofmann2001unsupervised} 预测车辆的移动模式。
% 然后,根据基于EM的移动性预测模式,预测车辆$v$在未来$H$时间片的轨迹,用$\operatorname{Traj}_{v}^{t} = \{ \hat{l}_{v}^{t+1}, \dots, \hat{l}_{v}^{t+h}, \dots, \hat{l}_{v}^{t+H}\}$表示,其中$\hat{l}_{v}^{t+h}$是车辆$v$在时间$t+h$的预测位置。
% 因此,车辆在边缘节点之间的平均距离的计算公式如下:
% \begin{equation}
% 	\operatorname{\bar{dis}}_{v, e}^{t} = \frac{1}{H} {\sum_{\forall h \in [1, H]} \widehat{\operatorname{dis}}_{v, e}^{t+h}}
% \end{equation}
% 其中,$\widehat{\operatorname{dis}}_{v, e}^{t+h}$ 是车辆$v$预测位置与边缘节点的距离,即$\widehat{\operatorname{dis}}_{v, e}^{t+h}=\operatorname{distance}(\hat{l}_{v}^{t+h}, l_{e})$。

% 那么,由车辆$v$感知到的并被视图$i$在时间$t$所需的信息集表示为:
% \begin{equation}
% 	\mathbf{D}_{v, i}^{t} = \left\{ d \mid  d \in \mathbf{D}_{v}^t \cap  \mathbf{D}_i \right\}
% \end{equation}
% 因此,由车辆$v$感知并被边缘节点$e$上视图在时间$t$需要的信息集合表示为:
% \begin{equation}
% 	\mathbf{D}_{v, {\mathbf{I}_e^t}}^{t} = \{ d \mid  d \in \bigcup_{\forall v \in V_e^t} \mathbf{D}_{v, i}^{t}\}
% \end{equation}
% \noindent 该集合的大小记为$|\mathbf{D}_{v, {\mathbf{I}_e^t}}^{t}|$, 并可通过下式得到:
% \begin{equation}
% 	|\mathbf{D}_{v, {\mathbf{I}_e^t}}^{t}| = \sum_{\forall d \in \mathbf{D}_{v, {\mathbf{I}_e^t}}^{t}}|d|
% \end{equation}
% 最后,边缘节点$e$为车辆$v$分配的V2I带宽由下式计算:
% \begin{equation}
% 	b_{v, e}^{t} =\frac{b_{e}} {\omega+\operatorname{rank}_{v}}
% \end{equation}
% \noindent 其中$\omega$为常数,$\operatorname{rank}_{v}$为车辆$v$按$| \mathbf{D}_{v, {\mathbf{I}_e^t}}^{t}|$的序列降序并按$\operatorname{\bar{dis}}_{v, e}^{t}$的序列升序排列的序列名次。

% 在确定车辆和边缘节点的联合动作后,以实现的VCPS质量作为系统奖励$r\left(\boldsymbol{a}_{\mathbf{V}}^{t},\boldsymbol{a}_{e}^{t} \mid \boldsymbol{o}^{t}\right)$,并通过基于DR的信用分配方案进一步划分为差分奖励$\boldsymbol{r}_{\mathbf{V}}^{t}$。
% 最后,包括当前系统状态$\boldsymbol{o}^{t}$、车辆动作$\boldsymbol{a}_{\mathbf{V}}^{t}$、差分奖励$\boldsymbol{r}_{\mathbf{V}}^{t}$和下一时刻系统状态$\boldsymbol{o}^{t+1}$在内的交互经验被存储在经验回放缓存$\mathcal{B}$。

% 3) \textbf{训练}: 从经验回放缓存$\mathcal{B}$中随机抽取$M$样本的小批量,用于训练车辆中的行动者和评论家网络,其中单个样本用$(\boldsymbol{o}_{v}^{m}, \boldsymbol{a}_{\mathbf{V}}^{m}, \boldsymbol{r}_{\mathbf{V}}^{m}, \boldsymbol{o}_{v}^{m+1})$表示。
% 本地行动者网络和本地评论家网络的参数以学习率$\alpha$和$\beta$更新
% 车辆$v$的本地评论家网络的损失函数通过下式计算:
% \begin{equation}
% 	\mathcal{L}\left(\theta_{v}^{Q}\right)=\frac{1}{M} \Sigma_{m}\left(\eta_{m}-Q_{v}\left(\boldsymbol{o}_{v}^{m}, \boldsymbol{a}_{\mathbf{V}}^{m} \mid \theta_{v}^{Q}\right)\right)^{2}
% \end{equation}
% \noindent 其中,$\eta_{m}$是由目标评论家网络产生的目标值,$\eta_{m}=r_{v}^{m}+\tau Q_{v}^{\prime}(\boldsymbol{o}_{v}^{m+1}, \boldsymbol{a}_{\mathbf{V}}^{m+1} \mid \theta_{v}^{Q^{\prime}})$,$\tau$是奖励折扣因子。
% 车辆$v$在时间$m+1$的动作是由目标行动者网络根据对下一时刻系统状态的局部观察得到的,即$\boldsymbol{a}_{\mathbf{V}}^{m+1}=\mu_{v}^{\prime}(\boldsymbol{o}_{v}^{m+1} \mid \theta_{v}^{\mu^{\prime}})$。
% 车辆$v$的本地行动者网络的参数通过策略网络梯度更新。
% \begin{equation}
% 	\nabla_{\theta_{v}^{\mu}} \mathcal{J} \approx \frac{1}{M} \sum_{m} \nabla_{\boldsymbol{a}_{\mathbf{V}}^{m}} Q_{v}\left(\boldsymbol{o}_{v}^{m}, \boldsymbol{a}_{\mathbf{V}}^{m} \mid \theta_{v}^{Q}\right) \nabla_{\theta_{v}^{\mu}} \mu_{v}\left(\boldsymbol{o}_{v}^{m+1} \mid \theta_{v}^{\mu}\right)
% \end{equation}
% 最后,车辆更新目标网络的参数。
% \begin{align}
% 	\theta_{v}^{\mu^{\prime}} &\leftarrow n_{v} \theta_{v}^{\mu}+(1-n_{v})  \theta_{v}^{\mu^{\prime}}, \forall v \in \mathbf{V}\\
% 	\theta_{v}^{Q^{\prime}} &\leftarrow n_{v} \theta_{i}^{Q}+(1-n_{v})  \theta_{v}^{Q^{\prime}}, \forall v \in \mathbf{V}
% \end{align}
% \noindent 其中 $n_{v} \ll 1, \forall v \in \mathbf{V} $。

% \section[\hspace{-2pt}实验设置与结果分析]{{\CJKfontspec{SimHei}\zihao{-3} \hspace{-8pt}实验设置与结果分析}}\label{section 2-6}

% \subsection[\hspace{-2pt}实验设置]{{\CJKfontspec{SimHei}\zihao{4} \hspace{-8pt}实验设置}}
% 本章使用Python 3.9和PyTorch 1.11.0实现了仿真实验模型,以评估MADR的性能。
% 该仿真模型基于一台配备AMD Ryzen 9 5950X 16核处理器@3.4 GHz、两个NVIDIA GeForce RTX 3090图形处理单元和64 GB内存的Ubuntu 20.04服务器。
% 特别地,本章使用真实世界的车辆轨迹构建了三种交通场景,这些轨迹来自滴滴GAIA数据集,包括:1)中国成都市青羊区3平方公里区域,2016年11月16日8:00至8:05;2)同一区域,同日23:00至23:05;3)中国西安市碑林区3平方公里区域,2016年11月27日8:00至8:05。
% 车辆轨迹的具体分析包括车辆轨迹总数、车辆平均停留时间(Average Dwell Time,ADT)、停留时间方差(Variance of Dwell Time, VDT)、平均车辆数(Average Vehicle Number, AVN)、车辆数方差(Variance of Vehicle Number, VVN)、车辆平均速度(Average Vehicle Speed,AVS)和车辆速度方差(Variance of Vehicle Speed,VVS)的详细统计,其总结在表\ref{table 2-1}中。
% 图\ref{fig 2-5}显示了调度周期内车辆分布的热力图,以更好地展示不同场景下的交通特征。
% 比较图\ref{fig 2-5}(a)、图\ref{fig 2-5}(b)和图\ref{fig 2-5}(c),可以发现工作日高峰期(即2016年11月16日星期三8:00左右)的车辆密度远远高于同一地区的夜间(即同日23点左右),也比周末的高峰期(即2016年11月27日星期日8:00左右)高得多。
% 此外,可以观察到在图\ref{fig 2-5}(c)中车辆分布完全不同,因为车辆轨迹是从另一个城市提取的。

% 实验参数设置描述如下:
% 信息的数据大小均匀分布在[100 B, 1 MB]的范围内。
% 每辆车的。
% V2I通信的 AWGN 和路径损耗指数分别设置为-90 dBm和3 \cite{sadek2009distributed}。
% V2I通信的信道衰减增益遵循均值为2、方差为0.4的高斯分布。
% 边缘节点的带宽被设置为3 MHz \cite{wang2019delay}。
% 噪声的不确定性遵循[0,3] dB的均匀分布 \cite{tandra2008snr}。

% \begin{table}[h]\small
% \setstretch{1.245} %设置具有指定弹力的橡皮长度(原行宽的1.2倍)
% \captionsetup{font={small, stretch=1.512}}
% \centering
% \bicaption{不同场景的交通特征}{Traffic characteristics of each scenario}
% \resizebox{\columnwidth}{!}{%
% \begin{tabular}[t]{cccccccc}
% \toprule
% 场景&轨迹&ADT (s)&VDT&AVN&VVN&AVS (m/s)&VVS\\
% \midrule
% 1&718&198.3&123.8&474.6&11.6&5.22&2.61\\
% 2&359&173.7&124.1&207.9&3.93&7.30&3.16\\
% 3&206&145.5&114.7&99.9&7.65&8.06&3.70\\
% \bottomrule
% \end{tabular}}
% \label{table 2-1}
% \end{table}

% \begin{figure}[h]
% \centering
%   \captionsetup{font={small, stretch=1.312}}\includegraphics[width=1\columnwidth]{Fig2-5-heat-map.pdf}
%   \bicaption[不同场景下的车辆分布热力图]{不同场景下的车辆分布热力图。(a)场景1(b)场景2(c)场景3}[Heat map of the distribution of vehicles under different scenarios]{Heat map of the distribution of vehicles under different scenarios. (a) Scenario 1 (b) Scenario 2 (c) Scenario 3}
%   \label{fig 2-5}
% \end{figure} 

% MADR的架构和超参数描述如下:
% 本地行动者网络是四层全连接的神经网络,其中包含两个隐藏层,其神经元数量分别为64和32。
% 目标行动者网络结构与本地行动者网络相同。
% 本地评论家网络是四层全连接的神经网络,其中包含两个隐藏层,其神经元数量分别为128和64。
% 目标评论家网络的结构与本地评论家网络相同。
% 使用整流线性单元(Rectified Linear Unit, ReLU)作为激活函数,使用自适应矩估计(Adaptive Moment Estimation, Adam)优化器更新网络权重,本地行动者网络和本地评论家网络学习率均为 0.001,奖励折扣因子为 0.996。
% 经验回放缓存$|\mathcal{B}|$的大小为5,批大小为512。
% 此外,本章还实现了以下四种可比较的算法。

% \begin{itemize}
% 	\item \textbf{随机分配}: 在每个时间片中,随机选择动作,以确定感知频率、上传优先级和V2I带宽分配。
% 	\item \textbf{深度确定性策略梯度}\cite{mlika2022deep}: 在边缘节点实现智能体,根据系统状态以集中的方式确定感知频率、上传优先级和V2I带宽分配。同时,智能体接收系统奖励以评估其贡献。
% 	\item \textbf{多智能体行动者-评论家}\cite{he2021efficient}: 实现了车辆中的智能体,基于本地车联网环境观测来决定感知频率和上传优先级,以及边缘节点中的智能体来决定V2I带宽分配。每个智能体都接收相同的系统奖励以评估其贡献。
% 	\item \textbf{采用VBA策略的多智能体行动者-评论家}: 为了更好地分配V2I带宽,本章进一步设计了多智能体行动者-评论家算法的变体,其中边缘节点基于VBA策略来分配V2I带宽,其余部分与MAAC算法一致。
% \end{itemize}

% 此外,本章还设计了以下指标用于性能评估。
% \begin{itemize}
% 	\item \textbf{累积奖励} (Cumulative Reward, CR): 定义为调度期间的累积系统奖励, 其计算方法为:
% 		\begin{equation}
% 			\operatorname{CR} = \sum_{\forall t \in \mathbf{T}} r\left(\boldsymbol{a}_{v}^{t},\boldsymbol{a}_{e}^{t} \mid \boldsymbol{o}^{t}\right)
% 		\end{equation}
% 	\item \textbf{平均奖励的构成} (Composition of Average Reward, CAR): 定义为归一化的时效性、完整性和一致性在平均奖励中的百分比,其表示为:
% 		\begin{equation}
% 			\operatorname{CAR} \triangleq <\frac{3}{10}(1-\hat{\Xi}_{i}),\frac{4}{10}(1-\hat{\Phi}_{i}), \frac{3}{10}(1-\hat{\Psi}_{i})>
% 		\end{equation}
% 	\item \textbf{平均排队时间} (Average Queuing Time, AQT): 定义为感知信息的排队时间之和除以调度期$T$内的信息数量,其计算方法为:
% 		\begin{equation}
% 			\operatorname{AQT} =\sum_{\forall t \in \mathbf{T}} \left \{ \frac{\sum_{v \in \mathbf{V}} \sum_{\forall d \subseteq \mathbf{D}_{v}^t} \frac{\operatorname{q}_{d, v}^t}{|\mathbf{D}_{v}^t|} }{V} \right\} \bigg/ T
% 		\end{equation}
% 	\item \textbf{服务率} (Service Ratio, SR): 定义为满足完整性要求的视图的数量在调度期间$\mathbf{T}$所需的视图总数的占比,其计算方法是:
% 		\begin{equation}
% 			\operatorname{SR} = \frac{\sum_{\forall t \in \mathbf{T}}\sum_{\forall i \in \mathbf{I}_e^t} \mathds{1}\{\Phi_{i} \geq \Phi_{th}\}}{ \sum_{\forall t \in \mathbf{T}} |\mathbf{I}_e^t|}
% 		\end{equation}
% 	其中 $\Phi_{th}$ 是完整性阈值。
% \end{itemize}

% \subsection[\hspace{-2pt}实验结果与分析]{{\CJKfontspec{SimHei}\zihao{4} \hspace{-8pt}实验结果与分析}}

% \textbf{1) 算法收敛性:}
% \begin{figure}[b]
% \centering
%   \captionsetup{font={small, stretch=1.312}}\includegraphics[width=0.75\columnwidth]{Fig2-6-convergence.pdf}
%   \bicaption[算法收敛性比较]{算法收敛性比较}[Convergence comparison]{Convergence comparison}
%   \label{fig 2-6}
% \end{figure} 
% 图\ref{fig 2-6}比较了五种算法在收敛速度和CR值方面的表现。结果显示,本章提出的MADR算法收敛速度最快(约660次迭代),并获得了最高的CR值(约357)。相比之下,DDPG、MAAC和MAAC-VBA分别需要大约4500次、950次和870次迭代才能收敛,并分别达到约307、290和315的CR值。RA作为基线算法的CR值约为241。值得注意的是,与DDPG、MAAC和MAAC-VBA相比,MADR算法在CR值方面分别实现了大约16.3\%、23.1\%和13.3\%的增加,同时在收敛速度方面分别提升了大约6.8倍、1.4倍和1.3倍。这主要是因为MADR算法旨在维护车辆的稳定通信环境,从而使车辆中的行动者和评论家网络的训练更加有效。另外,由于MADR的动作空间较小,相比于DDPG,MADR更容易收敛,因为DDPG需要同时决定感知频率、上传优先级和V2I带宽分配。

% \begin{figure}[h]
%   \centering
%   \captionsetup{font={small, stretch=1.312}}\includegraphics[width=1\columnwidth]{Fig2-7-different-scenarios.pdf}
%   \bicaption[不同交通场景下的性能比较]{不同交通场景下的性能比较。(a)车载信息物理融合系统质量(b)平均 AoV(c)累积奖励(d)平均奖励的构成(e)平均排队时间(f)服务率}[Performance comparison under different traffic scenarios]{Performance comparison under different traffic scenarios. (a) Vehicular cyber-physical system quality (b) Average age of view (c) Cumulative reward (d) Composition of average reward (e) Average queuing time (f) Service ratio}
%   \label{fig 2-7}
% \end{figure}

% \textbf{2) 交通场景的影响:}
% 图\ref{fig 2-7}比较了五种算法在不同交通场景下的表现。图\ref{fig 2-7}(a) 显示了五种算法在VCPS质量方面的比较。如图所示,MADR算法在所有场景下都实现了最高的VCPS质量,比RA、DDPG、MAAC和MAAC-VBA分别平均提高了58.0\%、27.1\%、19.1\%和12.5\%的VCPS质量。图\ref{fig 2-7}(b) 显示了五种算法在平均AoV方面的比较。所有场景下,MADR都实现了最低的平均AoV。图\ref{fig 2-7}(c) 显示了五种算法在CR方面的比较。结果表明,MADR实现的CR高于RA、DDPG、MAAC和MAAC-VBA。在场景3下,MADR和MAAC-VBA的CR相似,原因是场景3中较低的车辆密度和较高的车辆动态性使得数据上传比场景1和2中更加困难。图\ref{fig 2-7}(d) 将平均奖励分解成时效性、完整性和一致性三个部分的比例,以显示五种算法在这些方面的表现。在场景3下,时效性和一致性都非常小,这主要是因为当视图不完整时,时效性和一致性的要求很难得到满足。图\ref{fig 2-7}(e) 和图\ref{fig 2-7}(f) 显示了五种算法在不同场景下的AQT和SR比较。结果表明,MADR实现了最低的AQT,并在所有场景下保持最高的SR。

% \begin{figure}[h]
%   \centering
%   \captionsetup{font={small, stretch=1.312}}\includegraphics[width=1\columnwidth]{Fig2-8-different-bandwidths.pdf}
%   \bicaption[不同V2I带宽下的性能比较]{不同V2I带宽下的性能比较。(a)车载信息物理融合系统质量(b)平均 AoV(c)累积奖励(d)平均奖励的构成(e)平均排队时间(f)服务率}[Performance comparison under different V2I bandwidths]{Performance comparison under different V2I bandwidths. (a) Vehicular cyber-physical system quality (b) Average age of view (c) Cumulative reward (d) Composition of average reward (e) Average queuing time (f) Service ratio}
%   \label{fig 2-8}
% \end{figure}

% \begin{figure}[h]
%   \centering
%   \captionsetup{font={small, stretch=1.312}}\includegraphics[width=1\columnwidth]{Fig2-9-different-view-sizes.pdf}
%   \bicaption[不同视图需求下的性能比较]{不同视图需求下的性能比较。(a)车载信息物理融合系统质量(b)平均 AoV(c)累积奖励(d)平均奖励的构成(e)平均排队时间(f)服务率}[Performance comparison under different requirements on views]{Performance comparison under different requirements on views. (a) Vehicular cyber-physical system quality (b) Average age of view (c) Cumulative reward (d) Composition of average reward (e) Average queuing time (f) Service ratio}
%   \label{fig 2-9}
% \end{figure}

% \textbf{3) V2I 带宽的影响:}
% 图\ref{fig 2-8}比较了不同V2I带宽下五种算法的性能。在这组实验中,边缘节点的V2I带宽从1 MHz增加到5 MHz,更大的带宽代表更多的信息可以通过V2I通信上传。图\ref{fig 2-8}(a) 显示了五种算法在VCPS质量方面的比较。随着带宽的增加,所有算法的VCPS质量都相应增加。在不同V2I带宽下,MADR的VCPS质量分别比RA、DDPG、MAAC和MAAC-VBA高出约72.9\%、28.3\%、17.8\%和9.3\%。图\ref{fig 2-8}(b) 显示了五种算法在平均AoV方面的比较。所有情况下,MADR实现了最低的平均AoV。图\ref{fig 2-8}(c) 显示了五种算法在CR方面的比较。当带宽增加时,所有五种算法的性能都有所提升。具体来说,相比于RA、DDPG、MAAC和MAAC-VBA,MADR在CR方面分别实现了75.1\%、29.4\%、22.7\%和10.6\%的提升。图\ref{fig 2-8}(d) 比较了五种算法在CAR方面的表现。MADR比其他四种算法表现更好,特别是在视图时效性和一致性方面。这是因为在有限的带宽下,所提出的方案中车辆之间的信息感知和上传的协作更加有效。图\ref{fig 2-8}(e) 显示了五种算法在AQT方面的比较。在不同的V2I带宽下,MADR的AQT保持最低,反映了MADR能够更有效地分配带宽。图\ref{fig 2-8}(f) 显示了五种算法在SR方面的比较。在所有情况下,MADR的SR都保持最高水平,进一步证明了MADR在利用有限带宽方面的优势。

% \textbf{4) 视图需求的影响:}
% 图\ref{fig 2-9}比较了五种算法在不同视图需求下的性能,其中ITS应用需求的视图平均大小从0.25倍增加到4倍,作为基准,1倍视图的平均大小约为6.46 MB。图\ref{fig 2-9}(a) 显示了五种算法在VCPS质量方面的比较。随着平均视图大小的增加,所有算法的性能都会变差。在不同的视图需求下,MADR在最大限度地提高VCPS质量方面分别比RA、DDPG、MAAC和MAAC-VBA高出约68.1\%、23.5\%、27.9\% 和4.9\%。图\ref{fig 2-9}(b)和图\ref{fig 2-9}(c) 比较了五种算法在平均AoV和CR方面的表现。当平均视图大小较小时,MADR中的平均AoV略低于MAAC和MAAC-VBA。MADR、MAAC和MAAC-VBA的CR相似,因为较小的数据量有较高的成功上传的概率。图\ref{fig 2-9}(d) 比较了五种算法在CAR方面的表现。当平均视图大小从0.25倍增加到0.5倍时,MADR和MAAC-VBA之间的性能差异较小,原因是当有足够的资源来满足较小的平均视图大小的要求时,算法的调度效果并不明显。图\ref{fig 2-9}(e) 和图\ref{fig 2-9}(f) 显示了五种算法在AQT和SR方面的比较。结果表明,MADR可以保持最低的AQT,同时在大多数情况下实现最高的SR。当平均视图大小为2倍时,MAAC-VBA实现了最低的AQT和最高的SR,这反映了所提出的VBA方案可以更有效地分配带宽。

% \section[\hspace{-2pt}本章小结]{{\CJKfontspec{SimHei}\zihao{-3} \hspace{-8pt}本章小结}}\label{section 2-7}

% 本章设计了包括应用层、控制层、虚拟层和数据层的车联网分层服务架构,以最大化软件定义网络和移动边缘计算范式的协同效应。
% 在此基础上,本章提出了分布式感知与多源信息融合场景,并考虑了车载信息物理融合中多源信息的时效性、完整性和一致性,设计了质量指标AoV用于评估边缘构建的逻辑视图。
% 形式化定义了最大化VCPS质量的问题,并设计了基于差分奖励的多智能体深度强化学习解决方案,其中车辆作为独立智能体,决定感知频率和上传优先级。
% 边缘节点基于车辆预测轨迹和视图需求,通过VBA策略分配V2I带宽。
% 并采用基于DR的信用分配方案,根据车辆差分奖励评估其对于视图构建的贡献。
% 通过仿真实验的全面性能评估表明,MADR算法比RA、DDPG、MAAC和MAAC-VBA在最大限度地提高VCPS质量方面分别高出约61.8\%、23.8\%、22.0\%和8.0\%,同时加快了收敛速度。

\chapter[\hspace{0pt}面向自适应的FFT加速器设计]{{\CJKfontspec{SimHei}\zihao{3}\hspace{-5pt}面向自适应的FFT加速器设计}}
\removelofgap
\removelotgap
本章将研究面向车载信息物理融合的质量-开销均衡优化。
具体内容安排如下:
\ref{section 3-1} 节是本章的引言,介绍了车联网中车载信息物理融合系统的研究现状及存在的不足,同时阐述本章的主要贡献。
% \ref{section 4-2} 节阐述了协同感知与V2I上传场景。
% \ref{section 4-3} 节给出了系统模型的详细描述。
% \ref{section 4-4} 节形式化定义了最大化VCPS质量并最小化VCPS开销的双目标优化问题。
% \ref{section 4-5} 节设计了基于多目标的多智能体深度强化学习算法。
% \ref{section 4-6} 节搭建了实验仿真模型并进行了性能验证。
% \ref{section 4-7} 节对本章的研究工作进行总结。
\section[\hspace{-2pt}现有FFT加速器存在的问题]{{\CJKfontspec{SimHei}\zihao{-3} \hspace{-8pt}现有FFT加速器存在的问题}}\label{section 3-1}

\section[\hspace{-2pt}加速器顶层设计方案]{{\CJKfontspec{SimHei}\zihao{-3} \hspace{-8pt}加速器顶层设计方案}}\label{section 3-2}


\section[\hspace{-2pt}加速器详细设计方案]{{\CJKfontspec{SimHei}\zihao{-3} \hspace{-8pt}加速器详细设计方案}}\label{section 3-3}

\subsection[\hspace{-2pt}基-$2^5$的MDC单元设计]{{\CJKfontspec{SimHei}\zihao{-3} \hspace{-8pt}基-$2^5$的MDC单元设计}}\label{section 3-3}


\subsection[\hspace{-2pt}数据重排策略]{{\CJKfontspec{SimHei}\zihao{-3} \hspace{-8pt}数据重排策略}}\label{section 3-3}

\subsection[\hspace{-2pt}无冲突的内存访问方案]{{\CJKfontspec{SimHei}\zihao{-3} \hspace{-8pt}无冲突的内存访问方案}}\label{section 3-3}

\section[\hspace{-2pt}本章小结]{{\CJKfontspec{SimHei}\zihao{-3} \hspace{-8pt}本章小结}}\label{section 3-4}
% 新兴感知技术、无线通信和计算模式推动了现代新能源汽车和智能网联汽车的发展。
% 现代汽车中装备了各种车载感知器,以增强车辆的环境感知能力 \cite{zhu2017overview}。
% 另一方面,V2X通信\cite{chen2020a}的发展使车辆、路侧设备和云端之间的合作得以实现。
% 同时,车载边缘计算\cite{dai2021edge}是很有前途的范式,
% 可以实现计算密集型和延迟关键型的智能交通系统应用 \cite{zhao2022foundation}。
% 这些进展都成为了开发车载信息物理融合系统的强大驱动力。
% 具体来说,通过协同感知和上传,车联网中的物理实体,如车辆、行人和路侧设备等,
% 可以在边缘节点上构建为相应的逻辑映射。

% 车载信息物理融合中的检测、预测、规划和控制技术被广泛研究。
% 大量工作聚焦于检测技术,例如雨滴数量检测\cite{wang2021deep}和驾驶员疲劳检测\cite{chang2018design}。
% 针对车辆状态预测方法,研究人员提出了混合速度曲线预测\cite{zhang2019a}、车辆跟踪\cite{iepure2021a}
% 和加速预测\cite{zhang2020data}等。同时,部分研究工作提出了不同的调度方案,
% 例如基于物理比率-K干扰模型的广播调度\cite{li2020cyber}和基于既定地图模型的路径规划\cite{lian2021cyber}。
% 此外,部分研究集中在智能网联车辆的控制算法上,例如车辆加速控制\cite{lv2018driving}、交叉路口控制\cite{chang2021an}
% 和电动汽车充电调度\cite{wi2013electric}。
% 这些关于状态检测、轨迹预测、路径调度和车辆控制的研究促进了各种ITS应用的实施。
% 然而,这些工作忽略了感知和上传开销,假设高质量可用信息可以在VEC中构建。
% 少数研究考虑了VCPS中的信息质量,例如时效性\cite{liu2014temporal, dai2019temporal}和准确性\cite{rager2017scalability, yoon2021performance},
% 但上述研究都没有考虑通过协同感知和上传,在VCPS中实现高质量低成本的信息物理融合。

% 本章旨在通过车辆协同感知与上传,构建基于车载信息物理融合的逻辑视图,
% 并进一步在最大化车载信息物理融合质量和最小化视图构建开销方面寻求最佳平衡。
% 然而,实现这一目标面临着以下主要挑战。
% 首先,车联网中的信息高度动态,因此考虑感知频率、排队延迟和传输时延的协同效应,以确保信息的新鲜度和时效性是至关重要的。
% 其次,物理信息是具有时空相关性的,不同车辆在不同的时间或空间范围内感应到的信息可能存在冗余或不一致性。
% 因此,具有不同感知能力的车辆有望以分布式方式合作,以提高感知和通信资源的利用率。
% 再次,物理信息在分布、更新频率和模式方面存在异质性,这给构建高质量视图带来很大挑战。
% 最后,高质量的视图构建需要更高的感知和通信资源开销,这也是需要考虑的关键因素。
% 综上所述,通过协同感知和上传,实现面向车载边缘计算的高质量、低开销视图具有重要意义,但也具有一定的挑战性。

% 本章致力于研究车载信息物理融合系统的质量-开销均衡优化问题,并通过协同感知与上传实现高质量、低开销的视图建模。
% 本章的主要贡献如下:第一,提出了协同感知与V2I上传场景,考虑视图的及时性和一致性,
% 设计了车载信息物理融合质量指标,并考虑边缘视图构建过程中信息冗余度、感知开销和传输开销,
% 设计了车载信息物理融合开销指标。
% 进一步,提出了双目标优化问题,在最大化VCPS质量的同时最小化VCPS开销
% 第二,提出了基于多目标的多智能体深度强化学习算法。
% 具体地,在车辆和边缘节点中分别部署智能体,车辆动作空间包括感知决策、感知频率、上传优先级和传输功率分配,
% 而边缘节点动作空间是V2I带宽分配策略。
% 同时,设计了决斗评论家网络(Dueling Critic Network, DCN),
% 其根据状态价值(State-Value, SV)和动作优势(Action-Advantage, AA)评估智能体动作。
% 系统奖励是一维向量,其中包含VCPS质量和VCPS利润,并通过差分奖励信用分配得到车辆的个人奖励,
% 进一步通过最小-最大归一化得到边缘节点的归一化奖励。
% 第三,建立了基于现实世界车辆轨迹的仿真实验模型,并将MAMO与三种对比算法进行比较,
% 包括随机分配、分布式深度确定性策略梯度\cite{barth2018distributed},
% 以及多智能体分布式深度确定性策略梯度。
% 此外,本文设计了两个指标,即单位开销质量(Quality Per Unit Cost, QPUC)和单位质量利润(Profit Per Unit Quality, PPUQ)
% 用于定量衡量算法实现的均衡。
% 仿真结果表明,与其他算法相比,MAMO在最大化QPUC和PPUQ方面更具优势。

% \section[\hspace{-2pt}协同感知与 V2I 上传场景]{{\CJKfontspec{SimHei}\zihao{-3} \hspace{-8pt}协同感知与 V2I 上传场景}}\label{section 4-2}

% 本章节介绍了协同感知与V2I上传场景。如图\ref{fig 4-1}所示,车辆配备各种车载感知器,如超声波雷达、激光雷达、光学相机和毫米波雷达,可以对环境进行感知。通过车辆间协同地感知,可以获得多源信息,包括其他车辆、弱势道路参与者、停车场和路边基础设施的状态。这些信息可用于在边缘节点中建立视图模型,并进一步用于支撑各种ITS应用,如自动驾驶\cite{bai2022hybrid}、智慧路口控制系统\cite{hadjigeorgious2023real},以及全息城市交通流管理\cite{wang2023city}。逻辑视图需要融合车联网中物理实体的不同模式信息,以更好地反映实时物理车辆环境,从而提高ITS的性能。然而,构建高质量的逻辑视图可能需要更高的感知频率、更多的信息上传量以及更高的能量消耗。

% \begin{figure}[h]
% \centering
%   \captionsetup{font={small, stretch=1.312}}\includegraphics[width=1\columnwidth]{Fig4-1-architerture.pdf}
%   \bicaption[协同感知与 V2I 上传场景]{协同感知与 V2I 上传场景}[Cooperative sensing and V2I uploading scenario]{Cooperative sensing and V2I uploading scenario}
%   \label{fig 4-1}
% \end{figure} 

% 本系统的工作流程如下:首先,车辆感知并排队上传不同物理实体的实时状态。接着,边缘节点将V2I带宽分配给车辆,同时,车辆确定传输功率。物理实体的视图是基于从车辆收到的多源信息进行融合建立的。需要注意的是,在该系统中,多源信息是由车辆以不同的感知频率感应到的,因此上传时的新鲜度会不同。虽然增加感知频率可以提高新鲜度,但会增加排队延迟和能源消耗。此外,多个车辆可能感知到特定物理实体的信息,若由所有车辆上传,则可能会浪费通信资源。因此,为了提高资源利用率,需要有效而经济地分配通信资源。在此基础上,为了最大化面向车载边缘计算的视图的VCPS质量并最小化VCPS开销,必须量化衡量边缘节点构建的视图的质量和开销,并设计高效经济的协同感知和上传的调度机制。

% \section[\hspace{-2pt}车载信息物理融合质量/开销模型]{{\CJKfontspec{SimHei}\zihao{-3} \hspace{-8pt}车载信息物理融合质量/开销模型}}\label{section 4-3}

% \subsection[\hspace{-2pt}基本符号]{{\CJKfontspec{SimHei}\zihao{4} \hspace{-8pt}基本符号}}

% 本系统离散时间片的集合用$\mathbf{T}=\left\{1,\ldots,t,\ldots, T \right\}$表示。
% 多源信息集合用$\mathbf{D}$表示,其中信息$d \in \mathbf{D}$的特征是三元组$d=\left(\operatorname{type}_d, u_d, \left|d\right| \right)$,其中$\operatorname{type}_d$、$u_d$和$\left|d\right|$分别是信息类型、更新间隔和数据大小。
% $\mathbf{V}$表示车辆的集合,每个车辆$v\in \mathbf{V}$的特征是三元组$v=\left (l_v^t, \mathbf{D}_v, \pi_v \right )$,其中$l_v^t$、$\mathbf{D}_v$和$\pi_v$分别是位置、感知的信息集和传输功率。
% 对于$d \in \mathbf{D}_v$,车辆$v$的感知开销(即能耗)用$\phi_{d, v}$表示。
% 用$\mathbf{E}$表示边缘节点的集合,其中每个边缘节点$e \in \mathbf{E}$的特征是$e=\left (l_e, g_e, b_e \right)$,其中$l_{e}$、$r_{e}$和$b_{e}$分别为位置、通信范围和带宽。
% 车辆$v$与边缘节点$e$之间的距离表示为$\operatorname{dis}_{v, e}^t \triangleq \operatorname{distance} \left (l_v^t, l_e \right ), \forall v \in \mathbf{V}, \forall e \in \mathbf{E}, \forall t \in \mathbf{T}$。
% 在时间$t$内处于边缘节点$e$的通信覆盖范围内的车辆集合表示为$\mathbf{V}_e^t=\left \{v \vert \operatorname{dis}_{v, e}^t \leq g_e, \forall v \in \mathbf{V} \right \}, \mathbf{V}_e^t \subseteq \mathbf{V}$。

% 感知决策指示器表示车辆$v$在时间$t$是否感知信息$d$,其用以下方式表示:
% \begin{equation}
% 	c_{d, v}^t \in \{0, 1\}, \forall d \in \mathbf{D}_{v}, \forall v \in \mathbf{V}, \forall t \in \mathbf{T}
% 	\label{equ 4-1} 
% \end{equation}
% 那么,车辆$v$在时间$t$的感应信息集合表示为 $\mathbf{D}_v^t = \{ d | c_{d, v}^{t} = 1, \forall d \in \mathbf{D}_v \}, \mathbf{D}_v^t \subseteq \mathbf{D}_v$。
% 对于任何信息$d \in \mathbf{D}_v^t$来说,信息类型都是不同的, 即$\operatorname{type}_{d^*} \neq \operatorname{type}_{d}, \forall d^* \in \mathbf{D}_v^t \setminus \left\{ d\right \}, \forall d \in \mathbf{D}_v^t$。
% 车辆$v$在时间$t$的信息$d$的感知频率用$\lambda_{d, v}^t$表示,其需要满足车辆$v$的感应能力要求。
% \begin{equation}
% 	\lambda_{d, v}^{t} \in [\lambda_{d, v}^{\min} , \lambda_{d, v}^{\max} ], \ \forall d \in \mathbf{D}_v^t, \forall v \in \mathbf{V}, \forall t \in \mathbf{T}
% \end{equation}
% 其中$\lambda_{d, v}^{\min}$和$\lambda_{d, v}^{\max}$分别是车辆$v$中信息${d}$的最小和最大感知频率。
% 车辆$v$中的信息$d$在时间$t$的上传优先级用$p_{d, v}^t$表示,不同信息的上传优先级需各不相同。
% \begin{equation}
% 	{p}_{d^*, v}^t \neq {p}_{d, v}^t, \forall d^* \in \mathbf{D}_v^t \setminus \left\{ d\right \}, \forall d \in \mathbf{D}_v^t, \forall v \in \mathbf{V}, \forall t \in \mathbf{T}
% \end{equation}
% 其中${p}_{d^*, v}^t$是信息$d^* \in \mathbf{D}_v^t$中的上传优先级。
% 车辆$v$在时间$t$的传输功率用$\pi_{v}^t$表示,其不能超过车辆$v$的功率容量。
% \begin{equation}
% 	\pi_v^t \in \left[ 0 , \pi_v \right ], \forall v \in \mathbf{V}, \forall t \in \mathbf{T}
% \end{equation}
% 边缘节点$e$在时间$t$为车辆$v$分配的V2I带宽用$b_{v, e}^t$表示,且其需要满足:
% \begin{equation}
% 	b_{v, e}^t \in \left [0, b_e \right], \forall v \in \mathbf{V}_e^{t}, \forall e \in \mathbf{E}, \forall t \in \mathbf{T}
% 	\label{equ 4-5} 
% \end{equation}
% 边缘节点$e$分配的V2I总带宽不能超过其容量$b_e$,即${\sum_{\forall v \in \mathbf{V}_e^{t}} b_{v, e}^t} \leq b_e, \forall t \in \mathbf{T}$。

% 本系统中物理实体的集合为 $\mathbf{I}^{\prime}$,其中$i^{\prime} \in \mathbf{I}^{\prime}$表示物理实体,如车辆、行人和路侧基础设施等。
% $\mathbf{D}_{i^{\prime}}$是与实体$i^{\prime}$相关的信息集合,可以用$\mathbf{D}_{i^{\prime}}=\left\{d \mid y_{d, i^{\prime}} = 1, \forall d \in \mathbf{D} \right\}$, $\forall i^{\prime} \in \mathbf{I}^{\prime}$表示, 其中$y_{d, i^{\prime}}$是二进制数,表示信息$d$是否与实体$i^{\prime}$关联。
% $\mathbf{D}_{i^{\prime}}$的大小用$|\mathbf{D}_{i^{\prime}}|$表示。
% 每个实体可能需要多个信息,即$|\mathbf{D}_{i^{\prime}}| = \sum_{\forall d \in \mathbf{D}}y_{d, i^{\prime}} \geq 1, \forall i^{\prime} \in \mathbf{I}^{\prime}$。
% 对于每个实体$i^{\prime} \in \mathbf{I}^{\prime}$,可能有一个视图$i$在边缘节点中建模。
% 用$\mathbf{I}$表示视图的集合,用$\mathbf{I}_e^{t}$表示时间为$t$时在边缘节点$e$中建模的视图集合。
% 因此,边缘节点$e$收到且被视图$i$需要的信息集合可以用$\mathbf{D}_{i, e}^t=\bigcup_{\forall v \in \mathbf{V}}\left(\mathbf{D}_{i^{\prime}} \cap \mathbf{D}_{v, e}^t\right), \forall i \in \mathbf{I}_e^{t}, \forall e \in \mathbf{E}$表示,且 $| \mathbf{D}_{i, e}^t |$是边缘节点$e$收到且被视图$i$需要的信息数量,其计算公式为$| \mathbf{D}_{i, e}^t | =  \sum_{\forall v \in \mathbf{V}} \sum_{\forall d \in \mathbf{D}_v} c_{d, v}^t  y_{d, i^{\prime}}$。

% \subsection[\hspace{-2pt}协同感知模型]{{\CJKfontspec{SimHei}\zihao{4} \hspace{-8pt}协同感知模型}}
% 车辆协同感知是基于多类M/G/1优先级队列\cite{moltafet2020age}进行建模。
% 假设具有$\operatorname{type}_d$的信息的上传时间$\operatorname{\hat{g}}_{d, v, e}^t$遵循均值$\alpha_{d, v}^t$和方差$\beta_{d, v}^t$的一类一般分布。
% 那么,车辆$v$中的上传负载$\rho_{v}^{t}$由$ \rho_{v}^{t}=\sum_{\forall d \subseteq \mathbf{D}_v^t} \lambda_{d, v}^{t} \alpha_{d, v}^t$表示。
% 根据多类M/G/1优先级队列,需要满足$\rho_{v}^{t} < 1$才能达到队列的稳定状态。
% 信息$d$在时间$t$之前的到达时间用$\operatorname{a}_{d, v}^t$表示,其计算公式为:
% \begin{equation}
%     \operatorname{a}_{d, v}^t =  \frac{\left \lfloor t \lambda_{d, v}^t \right \rfloor }{\lambda_{d, v}^{t}} 
% \end{equation}
% 在时间$t$之前,由$\operatorname{u}_{d, v}^t$表示的信息$d$的更新时间是通过下式计算:
% \begin{equation}
%     \operatorname{u}_{d, v}^t = \left \lfloor  \frac{\operatorname{a}_{d, v}^t}{u_d} \right \rfloor  u_d
% \end{equation}
% 其中$u_d$是信息$d$的更新间隔时间。


% 在时间$t$,车辆$v$中比$d$有更高上传优先级的信息集合,用$\mathbf{D}_{d, v}^t = \{ d^* \mid p_{d^*, v}^{t} > p_{d, v}^{t} , \forall d^* \in \mathbf{D}_v^t \}$表示,其中$p_{d^*, v}^{t}$是信息$d^* \in \mathbf{D}_v^t$的上传优先级。
% 因此,信息$d$前面的上传负载(即$v$在时间$t$时要在$d$之前上传的信息数量)通过下方计算得出: 
% \begin{equation}
% 	\rho_{d, v}^{t}=\sum_{\forall d^* \in \mathbf{D}_{d, v}^t} \lambda_{d^*, v}^t \alpha_{d^*, v}^t
% \end{equation}
% 其中$\lambda_{d^*, v}^t$和$\alpha_{d^*, v}^t$分别为时间$t$内车辆$v$中信息$d^*$的感知频率和平均传输时间。
% 根据Pollaczek-Khintchine公式\cite{takine2001queue},车辆$v$中信息$d$的排队时间计算如下:
% \begin{equation}
%     \operatorname{q}_{d, v}^t= \frac{1} {1 - \rho_{d, v}^{t}} 
%         \left[ \alpha_{d, v}^t + \frac{ \lambda_{d, v}^{t} \beta_{d, v}^t + \sum\limits_{\forall d^* \in \mathbf{D}_{d, v}^t} \lambda_{d^*, v}^t \beta_{d^*, v}^t }{2\left(1-\rho_{d, v}^{t} - \lambda_{d, v}^{t} \alpha_{d, v}^t\right)}\right] 
%         - \alpha_{d, v}^t
% \end{equation}

% \subsection[\hspace{-2pt}V2I协同上传模型]{{\CJKfontspec{SimHei}\zihao{4} \hspace{-8pt}V2I协同上传模型}}

% 车辆间V2I协同上传是基于信道衰减分布和信噪比阈值来建模的。
% 车辆$v$和边缘节点$e$之间的V2I通信在时间$t$的信噪比通过公式\ref{equ 4-10}\cite{sadek2009distributed}计算得到。
% \begin{equation}
%     \operatorname{SNR}_{v, e}^{t}=\frac{1}{N_{0}} \left|h_{v, e}\right|^{2} \tau {\operatorname{dis}_{v, e}^{t}}^{-\varphi} {\pi}_v^t
%     \label{equ 4-10}
% \end{equation}
% 其中$N_{0}$为AWGN;$h_{v, e}$为信道衰减增益;$\tau$为取决于天线设计的常数;$\varphi$为路径损耗指数。
% 假设$\left|h_{v, e}\right|^{2}$遵循均值$\mu_{v, e}$和方差$\sigma_{v, e}$的一类分布,其表示方法为:
% \begin{equation}
%     \tilde{p}=\left\{\mathbb{P}: \mathbb{E}_{\mathbb{P}}\left[\left|h_{v, e}\right|^{2}\right]=\mu_{v, e}, \mathbb{E}_{\mathbb{P}}\left[\left|h_{v, e}\right|^{2}-\mu_{v, e}\right]^{2}=\sigma_{v, e}\right\}
% \end{equation}
% 进一步,基于成功传输概率和可靠性阈值来衡量V2I传输可靠性。
% \begin{equation}
%     \inf_{\mathbb{P} \in \tilde{p}} \operatorname{Pr}_{[\mathbb{P}]}\left(\operatorname{SNR}_{v, e}^{t} \geq \operatorname{SNR}_{v, e}^{\operatorname{tgt}}\right) \geq \delta
% \end{equation}
% \noindent 其中$\operatorname{SNR}_{v, e}^{\operatorname{tgt}}$和$\delta$分别为目标SNR阈值和可靠性阈值。
% 由车辆$v$上传并由边缘节点$e$接收的信息集合用$\mathbf{D}_{v, e}^{t} = \bigcup_{\forall v \in \mathbf{V}_{e}^{t}} \mathbf{D}_{v}^{t}$表示。

% 根据香农理论,车辆$v$和边缘节点$e$之间在时间$t$的V2I通信的传输率用$\operatorname{z}_{v, e}^t$表示,其计算公式如下:
% \begin{equation}
%     \operatorname{z}_{v, e}^t=b_{v}^{t} \log _{2}\left(1+\mathrm{SNR}_{v, e}^{t}\right)
% \end{equation}
% 假设车辆$v$被安排在时间$t$上传$d$,并且$d$将在一定的排队时间$\mathrm{\bar{q}}_{d, v}^t$后被传输。
% 然后,本章把车辆$v$开始传输$d$的时刻表示为$\mathrm{t}_{d, v}^t=t+\mathrm{q}_{d, v}^t$。
% 从$\mathrm{t}_{d, v}^t$到$\mathrm{t}_{d, v}^t + f$之间传输的数据量可由 $\int_{\mathrm{t}_{d, v}^t}^{\mathrm{t}_{d, v}^t+f} \mathrm{z}_{v, e}^t \mathrm{~d} t$ bits 得到,其中$f \in \mathbb{R}^{+}$和$\mathrm{z}_{i, e}^t$是时间$t$的传输速率。
% 如果在整个传输过程中可以传输的数据量大于信息$d$的大小,那么上传就会完成。
% 因此,从车辆$v$到边缘节点$e$传输信息$d$的时间,用$\operatorname{g}_{d, v, e}^t$表示,计算如下:
% \begin{equation}
%     \operatorname{g}_{d, v, e}^t=\inf _{j \in \mathbb{R}^+} \left \{ \int_{\operatorname{k}_{d, v}^t}^{\operatorname{k}_{d, v}^t + j} {\operatorname{z}_{v, e}^t} \operatorname{d}t \geq \left|d\right| \right \} 
% \end{equation}
% \noindent 其中$\operatorname{t}_{d, v}^t = t +\operatorname{q}_{d, v}^t$是车辆$v$开始传输信息$d$的时刻。

% \section[\hspace{-2pt}质量-开销均衡问题定义]{{\CJKfontspec{SimHei}\zihao{-3} \hspace{-8pt}质量-开销均衡问题定义}}\label{section 4-4}

% \subsection[\hspace{-2pt}VCPS质量]{{\CJKfontspec{SimHei}\zihao{4} \hspace{-8pt}VCPS质量}}

% 首先,由于视图是基于连续上传和时间变化的信息建模的,本章对信息$d$的及时性定义如下:
% \begin{definition}
% 信息$d$在车辆$v$中的及时性$\theta_{d, v} \in \mathbb{Q}^{+}$被定义为更新和接收信息$d$之间的时间差。
% \begin{equation}
%     \theta_{d, v} = \operatorname{a}_{d, v}^t + \operatorname{q}_{d, v}^t + \operatorname{g}_{d, v, e}^t-\operatorname{u}_{d, v}^{t}, \forall d \in \mathbf{D}_v^t,\forall v \in \mathbf{V}
% \end{equation}
% \end{definition}
% \begin{definition}
% 视图$i$的及时性 $\Theta_{i} \in \mathbb{Q}^{+}$定义为与物理实体$i^{\prime}$相关的信息的最大及时性之和。
% 	\begin{equation}
%     	\Theta_{i} = \sum_{\forall v\in \mathbf{V}_{e}^{t}} \max_{\forall d \in \mathbf{D}_{i^{\prime}} \cap \mathbf{D}_v^t}\theta_{d, v}, \forall i \in \mathbf{I}_{e}^{t}, \forall e \in \mathbf{E}
%     	\label{equ 4-16}
% 	\end{equation}
% \end{definition}

% 其次,由于不同类型的信息有不同的感知频率和上传优先级,本章定义视图的一致性来衡量与同一物理实体相关的信息的一致性。
% \begin{definition}
% 视图$i$的一致性$\Psi_{i} \in \mathbb{Q}^{+}$定义为信息更新时间差的最大值。
% \begin{equation}
%     \Psi_{i}=\max_{\forall d \in \mathbf{D}_{i, e}^{t}, \forall v \in \mathbf{V}_{e}^{t}} {\operatorname{u}_{d, v}^t} - \min_{\forall d \in \mathbf{D}_{i, e}^{t}, \forall v \in \mathbf{V}_{e}^{t}} {\operatorname{u}_{d, v}^t} , \forall i \in \mathbf{I}_{e}^{t}, \forall e \in \mathbf{E}
% \end{equation}
% \end{definition}

% 最后,本章给出了视图的质量的正式定义,其综合了视图的及时性和一致性。
% \begin{definition}
% 视图质量$\operatorname{QV}_{i} \in (0, 1)$定义为视图$i$的归一化及时性和归一化一致性的加权平均和。
% 	\begin{equation}
% 	    \operatorname{QV}_{i} = w_1 (1 -\hat{\Theta_{i}}) + w_2 (1 - \hat{\Psi_{i}}), \forall i \in \mathbf{I}_{e}^t, \forall e \in \mathbf{E}
% 	\end{equation}
% \end{definition}
% \noindent 其中$\hat{\Theta_{i}} \in (0, 1)$和$\hat{\Psi_{i}} \in (0, 1)$分别表示归一化的及时性和归一化的一致性,这可以通过最小-最大归一化对及时性和一致性的范围进行重新调整至$(0, 1)$来获得。
% $\hat{\Theta_{i}}$和$\hat{\Psi_{i}}$的加权系数分别用$w_1$和$w_2$表示,可以根据ITS应用的不同要求进行相应的调整,$w_1+w_2=1$。
% $\operatorname{QV}_{i}$的值越高,说明构建的视图质量越高。
% 考虑车辆自动驾驶系统下,视图的时效性和一致性都是关键。一方面,汽车需要实时接收到周围车辆和行人的位置、速度和行为信息,以便能够做出准确的驾驶决策。如果视图数据延迟太高或过时,车辆可能无法及时识别出潜在的危险或变化情况,从而导致事故或冲突的发生。另一方面,一致性方面的重要性表现在视图数据的一致性建模。车辆需要确保从边缘节点接收到的视图数据是准确、完整且一致的,以便能够准确地理解周围环境并做出正确的决策。如果视图数据存在不一致或缺失,车辆可能会做出错误的判断,从而导致不安全的驾驶行为或导航错误。

% 进一步,基于视图质量定义车载信息物理融合质量如下:
% \begin{definition}
% VCPS质量$\mathscr{Q} \in (0, 1)$被定义为在调度期间$\mathbf{T}$的边缘节点中建模的每个视图的QV的平均值。
% 	\begin{equation}
% 		\mathscr{Q}=\frac{\sum_{\forall t \in \mathbf{T}} \sum_{\forall e \in \mathbf{E}} \sum_{\forall i \in \mathbf{I}_e^t} \operatorname{QV}_{i}}{\sum_{\forall t \in \mathbf{T}} \sum_{\forall e \in \mathbf{E}} |\mathbf{I}_e^t| }
% 	\end{equation}
% \end{definition}

% \subsection[\hspace{-2pt}VCPS开销]{{\CJKfontspec{SimHei}\zihao{4} \hspace{-8pt}VCPS开销}}

% 首先,由于同一物理实体的状态可能被多个车辆同时感应到,本章对信息$d$的冗余度定义如下:
% \begin{definition}
% 信息$d$的冗余度$\xi_d \in \mathbb{N}$定义为车辆感应到同一类型$\operatorname{type}_d$的额外信息数量。
% \begin{equation}
%     \xi_d= \left | \mathbf{D}_{d, i, e} \right| - 1, \forall d \in \mathbf{D}_j, \forall i \in \mathbf{I}_{e}^{t}, \forall e \in \mathbf{E}
% \end{equation}
% \noindent 其中$\mathbf{D}_{d, i, e}$是边缘节点$e$收到且被视图$i$需要,且类型为$\operatorname{type}_d$的信息集合,其由$\mathbf{D}_{d, i, e}=\left\{ d^* \vert \operatorname{type}_{d^*} = \operatorname{type}_{d}, \forall d^* \in \mathbf{D}_{i, e}^t \right \}$表示。

% \end{definition}
% \begin{definition}
% 视图$i$的冗余度$\Xi_j \in \mathbb{N}$定义为视图$i$中的总冗余度。
% 	\begin{equation}
%        \Xi_j =  \sum_{\forall d \in \mathbf{D}_{i^{\prime}}} \xi_d, \forall i \in \mathbf{I}_{e}^{t}, \forall e \in \mathbf{E}
%        \label{equ 4-20}
%     \end{equation}
% \end{definition}

% 其次,信息感知和传输需要消耗车辆的能量,本章定义视图$i$的感知开销和传输开销如下:
% \begin{definition}
% 视图$i$的感知开销$\Phi_{i} \in \mathbb{Q}^{+}$定义为视图$i$所需信息的总感知开销。
% 	\begin{equation}
%         \Phi_{i} = \sum_{\forall v \in \mathbf{V}_{e}^{t}} \sum_{\forall d \in \mathbf{D}_{i^{\prime}} \cap \mathbf{D}_v^t}{\phi_{d, v}}, \forall i \in \mathbf{I}_{e}^t, \forall e \in \mathbf{E}
%         \label{equ 4-21}
%     \end{equation}
%     其中$\phi_{d, v}$是信息$d$在车辆$v$中的感知开销。
% \end{definition}
% \begin{definition}
% 信息$d$在车辆$v$中的传输开销${\omega}_{d, v} \in \mathbb{Q}^{+}$定义为信息上传时消耗的传输功率。
% \begin{equation}
%     {\omega}_{d, v}= \pi_v^t \operatorname{g}_{d, v, e}^t, \forall d \in \mathbf{D}_v^t
% \end{equation}
% 其中$\pi_v^t$和$\operatorname{g}_{d, v, e}^t$分别为传输功率和传输时间。
% \end{definition}
% \begin{definition}
% 视图$i$的传输开销$\Omega_{i} \in \mathbb{Q}^{+}$定义为视图$i$所需的信息总传输开销。
% 	\begin{equation}
%         \Omega_{i} = \sum_{\forall v \in \mathbf{V}_{e}^{t}} \sum_{\forall d \in \mathbf{D}_{i^{\prime}} \cap \mathbf{D}_v^t} {\omega}_{d, v}, \forall i \in \mathbf{I}_{e}^t, \forall e \in \mathbf{E}
%        	\label{equ 4-23}
%     \end{equation}
% \end{definition}

% 最后,给出视图开销的正式定义,其综合了冗余度、感知开销和传输开销。
% \begin{definition}
% 视图的开销$\operatorname{CV}_{i} \in (0, 1)$定义为视图$i$的归一化冗余度、归一化感知开销和归一化传输开销的加权平均和。
% 	\begin{equation}
% 	    \operatorname{CV}_{i} = w_3  \hat{\Xi_{i}} +  w_4 \hat{\Phi_{i}} + w_5 \hat{\Omega_{i}}, \forall i \in \mathbf{I}_{e}^t, \forall e \in \mathbf{E}
% 	\end{equation}
% \end{definition}
% \noindent 其中 $\hat{\Xi_{i}}\in (0, 1)$、$\hat{\Phi_{i}} \in (0, 1)$和$\hat{\Omega_{i}} \in (0, 1)$ 分别表示视图$i$的归一化冗余度、归一化感知开销和归一化传输开销。
% $\hat{\Xi_{i}}$、$\hat{\Phi_{i}}$和$\hat{\Omega_{i}}$ 的加权系数分别表示为 $w_3$、$w_4$和 $w_5$。
% 同样地,$w_3+w_4+w_5=1$。
% 进一步,VCPS开销定义如下:
% \begin{definition}
% VCPS 开销$\mathscr{C} \in (0, 1)$定义为$\mathbf{T}$调度期间边缘节点中每个视图模型的CV的平均值。
% 	\begin{equation}
% 		\mathscr{C}=\frac{\sum_{\forall t \in \mathbf{T}} \sum_{\forall e \in \mathbf{E}} \sum_{\forall i \in \mathbf{I}_e^t}  \operatorname{CV}_{i}}{\sum_{\forall t \in \mathbf{T}} \sum_{\forall e \in \mathbf{E}} |\mathbf{I}_e^t| }
% 	\end{equation}
% \end{definition}

% \subsection[\hspace{-2pt}双目标优化问题]{{\CJKfontspec{SimHei}\zihao{4} \hspace{-8pt}双目标优化问题}}

% 给定解决方案$( \mathbf{C}, \bf\Lambda, \mathbf{P}, \bf\Pi, \mathbf{B} )$,其中$\mathbf{C}$表示确定的感知信息决策,$\bf\Lambda$表示确定的感知频率。$\mathbf{P}$表示确定的上传优先级,$\bf\Pi$表示确定的传输功率,$\mathbf{B}$表示确定的V2I带宽分配,其中$c_{d, v}^t$、$\lambda_{d, v}^{t}$和$p_{d, v}^{t}$分别为时间$t$内车辆$v$的信息$d$的感知信息决策、感知频率和上传优先级,$\pi_v^t$和$b_v^t$分别为时间$t$内车辆$v$的传输功率和V2I带宽。
% \begin{numcases}{}
% 	\mathbf{C} = \left \{ c_{d, v}^t \vert \forall d \in \mathbf{D}_{v}, \forall v \in \mathbf{V}, \forall t \in \mathbf{T} \right  \} \notag \\
% 	{\bf\Lambda} = \left \{ \lambda_{d, v}^{t} \vert \forall d \in \mathbf{D}_v^t  , \forall v \in \mathbf{V}, \forall t \in \mathbf{T} \right \} \notag \\ 
% 	\mathbf{P} = \left \{ p_{d, v}^{t} \vert \forall d \in \mathbf{D}_v^t  , \forall v \in \mathbf{V}, \forall t \in \mathbf{T}\right \}  \notag \\
% 	{\bf\Pi} = \left \{ \pi_v^t \vert \forall v \in \mathbf{V}, \forall t \in \mathbf{T} \right \} \notag \\
% 	\mathbf{B} = \left \{ b_v^t \vert \forall v \in \mathbf{V}, \forall t \in \mathbf{T}\right \}
% \end{numcases}
% 本章提出了双目标优化问题,旨在同时实现VCPS质量的最大化和VCPS 开销的最小化::
% \begin{align}
% 	\mathcal{P}4.1: & \max_{\mathbf{C}, \bf\Lambda, \mathbf{P}, \bf\Pi, \mathbf{B}} \mathscr{Q}, \min_{\mathbf{C}, \bf\Lambda, \mathbf{P}, \bf\Pi, \mathbf{B}} \mathscr{C} \notag \\
% 	\text { s.t. }
% 	& (\ref{equ 4-1}) \sim (\ref{equ 4-5}) \notag \\
%     &\mathcal{C}4.1: \sum_{\forall d \subseteq \mathbf{D}_v^t} \lambda_{d, v}^{t} \mu_d<1,\ \forall v \in \mathbf{V}, \forall t \in \mathbf{T} \notag \\
%     &\mathcal{C}4.2: \inf_{\mathbb{P} \in \tilde{p}} \operatorname{Pr}_{[\mathbb{P}]}\left(\operatorname{SNR}_{v, e}^{t} \geq \operatorname{SNR}_{v, e}^{\operatorname{tgt}}\right) \geq \delta, \forall v \in \mathbf{V}, \forall t \in \mathbf{T} \notag \\
%     &\mathcal{C}4.3: {\sum_{\forall v \in \mathbf{V}_e^{t}}b_v^t} \leq b_e, \forall t \in \mathbf{T}
% \end{align}
% 其中$\mathcal{C}4.1$保证队列稳定状态,$\mathcal{C}4.2$保证传输可靠性。
% $\mathcal{C}4.3$要求边缘节点$e$分配的V2I带宽之和不能超过其容量$b_e$。
% 基于CV的定义,视图的利润定义如下:
% \begin{definition}
% 视图的利润$\operatorname{PV}_{j} \in (0, 1)$定义为视图$i$的CV的补集。
% 	\begin{equation}
% 		\mathscr{P}= 1 - \operatorname{CV}_{i}
% 	\end{equation}
% \end{definition}
% \noindent 然后,本章将VCPS 利润定义如下:
% \begin{definition}
% VCPS 利润$\mathscr{P} \in (0, 1)$被定义为在调度期$\mathbf{T}$期间,边缘节点中每个视图模型的PV的平均值。
% 	\begin{equation}
% 		\mathscr{P}= \frac{\sum_{\forall t \in \mathbf{T}} \sum_{\forall e \in \mathbf{E}} \sum_{\forall i \in \mathbf{I}_e^t}   \operatorname{PV}_{j} }{\sum_{\forall t \in \mathbf{T}} \sum_{\forall e \in \mathbf{E}} |\mathbf{I}_e^t| }
% 	\end{equation}
% \end{definition}
% \noindent 因此,$\mathcal{P}4.1$问题可以改写如下:
% \begin{align}
% 	\mathcal{P}4.2: & \max_{ \mathbf{C}, \bf\Lambda, \mathbf{P}, \bf\Pi, \mathbf{B} } \left (\mathscr{Q}, \mathscr{P} \right ) \notag \\
% 		\text { s.t. }
% 	&(\ref{equ 4-1}) \sim (\ref{equ 4-5}), \mathcal{C}4.1 \sim \mathcal{C}4.3
% \end{align}

% \section[\hspace{-2pt}基于多目标的多智能体强化学习算法设计]{{\CJKfontspec{SimHei}\zihao{-3} \hspace{-8pt}基于多目标的多智能体强化学习算法设计}}\label{section 4-5}

% 本章节提出了基于多目标的多智能体深度强化学习算法,其模型如图\ref{fig 4-2}所示,由$K$分布式行动者、学习器和经验回放缓存组成。
% 具体地,学习器由四个神经网络组成,即本地策略网络、本地评论家网络、目标策略网络和目标评论家网络。
% 其中车辆的本地策略网络、本地评论家网络、目标策略网络和目标评论家网络参数分别表示为 $\theta_{\mathbf{V}}^{\mu}$、$\theta_{\mathbf{V}}^{Q}$、 $\theta_{\mathbf{V}}^{\mu^{\prime}}$和$\theta_{\mathbf{V}}^{Q^{\prime}}$。
% 同样地,边缘节点的本地策略网络、本地评论家网络、目标策略网络和目标评论家网络参数分别表示为 $\theta_{\mathbf{E}}^{\mu}$、$\theta_{\mathbf{E}}^{Q}$、$\theta_{\mathbf{E}}^{\mu^{\prime}}$和$\theta_{\mathbf{E}}^{Q^{\prime}}$。
% 本地策略和本地评论家网络的参数是随机初始化的。
% 目标策略和目标评论家网络的参数被初始化为相应的本地网络。
% 然后,启动$K$分布式行动者,每个分布式行动者独立地与环境进行交互,并将交互经验存储到重放经验缓存。
% 分布式行动者由本地车辆策略网络和本地边缘策略网络组成,其分别用$\theta_{\mathbf{V}, k}^{\mu}$和$\theta_{\mathbf{E}, k}^{\mu}$表示,其网络参数是从学习器的本地策略网络复制而来的。
% 同时,初始化了最大存储容量为$|\mathcal{B}|$的经验回放缓存以存储重放经验。
% 基于多目标的多智能体深度强化学习的具体步骤见算法4.1,分布式行动者与环境的交互将持续到学习器的训练过程结束,其具体步骤见算法4.2。

% \begin{figure}[h]
% \centering
%   \captionsetup{font={small, stretch=1.312}}\includegraphics[width=1\columnwidth]{Fig4-2-solution-model.pdf}
%   \bicaption[基于多目标的多智能体深度强化学习模型]{基于多目标的多智能体深度强化学习模型}[Multi-agent multi-objective deep reinforcement learning model]{Multi-agent multi-objective deep reinforcement learning model}
%   \label{fig 4-2}
% \end{figure}


% \SetKwInOut{KwIn}{输入}
% \SetKwInOut{KwOut}{输出}

% \begin{algorithm}[h]\small
% \setstretch{1.245} %设置具有指定弹力的橡皮长度(原行宽的1.2倍)
% \renewcommand{\algorithmcfname}{算法}
% 	\caption{基于多目标的多智能体深度强化学习}
% 	\KwIn{折扣因子 $\gamma$、批大小 $M$、回放经验缓存 $\mathcal{B}$、学习率$\alpha$和$\beta$、目标网络参数更新周期 $t_{\operatorname{tgt}}$、分布式行动者网络参数更新周期 $t_{\operatorname{act}}$、随机动作数量$N$}
% 	\KwOut{信息感知决策$\mathbf{C}_v^t$、信息感知频率决策$\lambda_{d, v}^{t}$、上传优先级决策$p_{d, v}^{t}$、传输功率$\pi_v^t$、V2I带宽分配$b_{v, e}^{t}$}
% 	初始化网络参数\\
% 	初始化经验回放缓存 $\mathcal{B}$\\
% 	启动 $K$ 分布式行动者并复制网络参数给行动者\\
% 	\For{\songti{迭代次数} $= 1$ \songti{到最大迭代次数}}{
% 		\For{\songti{时间片} $t = 1$ \songti{到} $T$}{
% 			从经验回放缓存$\mathcal{B}$随机采样$M$小批量\\
% 			通过目标评论家网络中DCN网络得到目标值\\
% 			基于分类分布的TD学习计算更新评论家网络\\
% 			更新本地策略和评论家网络\\
% 			\If{$t \mod t_{\operatorname{tgt}} = 0$}{
% 				更新目标网络\\
% 			}
% 			\If{$t \mod t_{\operatorname{act}} = 0$}{
% 				复制网络参数给分布式行动者\\
% 			}
% 		}
% 	}
% \label{algorithm 4-1}
% \end{algorithm}

% \begin{algorithm}[t]\small
% \setstretch{1.245} %设置具有指定弹力的橡皮长度(原行宽的1.2倍)
% \renewcommand{\algorithmcfname}{算法}
% 	\caption{分布式行动者}
% 	\KwIn{车辆探索常数 $\epsilon_{v}$、边缘节点探索常数$\epsilon_{e}$、车辆本地观测 $\boldsymbol{o}_{v}^{t}$、边缘节点本地观测 $\boldsymbol{o}_{e}^{t}$}
% 	\KwOut{车辆动作$\boldsymbol{a}_{\mathbf{V}}^{t}$、边缘节点动作$\boldsymbol{a}_{e}^{t}$}
% 	\While{\songti{学习器没有结束}}{
% 		初始化随机过程 $\mathcal{N}$ 以进行探索\\
% 		生成随机权重 $\boldsymbol{w}^{t}$\\
% 		接收初始系统状态 $\boldsymbol{o}_{1}$\\
% 		\For{\songti{时间片} $t = 1$ \songti{到} $T$}{
% 			\For{\songti{车辆} $v=1$ \songti{到} $V$}{
% 				接收车辆本地观测 $\boldsymbol{o}_{v}^{t}$\\
% 				选择车辆动作 $\mu_{\mathbf{V}}\left(\boldsymbol{o}_{v}^{t} \mid \theta_{\mathbf{V}}^{\mu}\right)+\epsilon_{v} \mathcal{N}_{v}^{t}$\\
% 			}
% 			接收边缘节点本地观测 $\boldsymbol{o}_{e}^{t}$\\
% 			选择边缘节点动作 $\boldsymbol{a}_{e}^{t}=\mu_{\mathbf{E}}\left(\boldsymbol{o}_{e}^{t},  \boldsymbol{a}_{\boldsymbol{\mathbf{V}}}^{t} \mid \theta_{\mathbf{E}}^{\mu}\right)+\epsilon_{e} \mathcal{N}_{e}^{t}$\\
% 			接收奖励 $\boldsymbol{r}^{t}$ 和下一个系统状态 $\boldsymbol{o}^{t+1}$\\
% 			\For{\songti{车辆} $v=1$ \songti{到} $V$}{
% 				根据公式\ref{equ 4-40}计算车辆的差分奖励\\
% 			}
% 			根据公式\ref{equ 4-41}计算边缘节点的归一化奖励\\
% 			存储 $\left(\boldsymbol{o}^{t}, \boldsymbol{a}_{\mathbf{V}}^{t}, \boldsymbol{a}_{e}^{t}, \boldsymbol{r}_{\mathbf{V}}^{t}, \boldsymbol{r}_{e}^{t}, \boldsymbol{w}^{t}, \boldsymbol{o}^{t+1}\right)$ 到经验回放缓存 $\mathcal{B}$
% 		}
% 	}
% \label{algorithm 4-2}
% \end{algorithm}

% \subsection[\hspace{-2pt}多智能体分布式策略执行]{{\CJKfontspec{SimHei}\zihao{4} \hspace{-8pt}多智能体分布式策略执行}}

% 在MAMO中,车辆和边缘节点分别通过本地策略网络分布式地决定动作。
% 车辆$v$在时间$t$上对系统状态的局部观测表示为:
% 	\begin{equation}
% 		\boldsymbol{o}_{v}^{t}=\left\{t, v, l_{v}^t, \mathbf{D}_{v}, \Phi_{v}, \mathbf{D}_{e}^{t}, \mathbf{D}_{\mathbf{I}_e^t}, \boldsymbol{w}^{t}\right\}
% 	\end{equation} 
% \noindent 其中$t$为时间片索引;
% $v$是车辆索引;$l_{v}^t$是车辆$v$的位置;
% $\mathbf{D}_{v}$表示车辆$v$可以感知的信息集合;
% $\Phi_{v}$代表$\mathbf{D}_{v}$中信息的感知开销;
% $\mathbf{D}_{e}^{t}$ 代表$e$在时间$t$的边缘节点的缓存信息集;
% $\mathbf{D}_{\mathbf{I}_e^t}$ 代表在时间$t$的边缘节点$e$中建模的视图所需的信息集合;
% $\boldsymbol{w}^{t}$ 代表每个目标的权重向量,其在每次迭代中随机生成。
% 具体地,$\boldsymbol{w}^{t} = \begin{bmatrix}  w^{(1), t}  &  w^{(2), t} \end{bmatrix}$,其中$w^{(1), t} \in (0, 1)$和$w^{(2), t} \in (0, 1)$分别是VCPS质量和VCPS利润的权重,$\sum_{\forall i \in \{1, 2\}} w^{(j), t} = 1$。
% 另一方面,边缘节点$e$在时间$t$上对系统状态的局部观测表示为
% \begin{equation}
% 	\boldsymbol{o}_{e}^{t}=\left\{t, e, \operatorname{\mathbf{Dis}}_{\mathbf{V}, e}^{t}, \mathbf{D}_{1}, \ldots, \mathbf{D}_{v}, \ldots, \mathbf{D}_{v}, \mathbf{D}_{e}^{t}, \mathbf{D}_{\mathbf{I}_e^t}, \boldsymbol{w}^{t} \right\}
% \end{equation}
% \noindent 其中$e$是边缘节点索引,$\operatorname{\mathbf{Dis}}_{\mathbf{V}, e}^{t}$代表车辆与边缘节点$e$之间的距离集合。
% 因此,系统在时间$t$的状态可以表示为$\boldsymbol{o}^{t}=\boldsymbol{o}_{e}^{t} \cup \boldsymbol{o}_{1}^{t} \cup \ldots \cup \boldsymbol{o}_{v}^{t} \cup \ldots \cup \boldsymbol{o}_{v}^{t}$。

% 车辆$v$的动作空间表示为:
% \begin{equation}
% 	\boldsymbol{a}_{v}^{t} = \{ \mathbf{C}_v^t,  \{ \lambda_{d, v}^{t}, p_{d, v}^{t} \mid \forall d \in \mathbf{D}_{v}^t \} , \pi_v^t   \}
% \end{equation}
% 其中,$\mathbf{C}_v^t$是感知决策;$\lambda_{d, v}^{t}$和$p_{d, v}^{t}$分别是信息$d$的感知频率和上传优先级,$\pi_v^t$是车辆$v$在时间$t$的传输功率。
% 车辆基于系统状态的本地观测,并通过本地车辆策略网络得到当前的动作。
% \begin{equation}
% 	\boldsymbol{a}_{v}^{t}=\mu_{\mathbf{V}}\left(\boldsymbol{o}_{v}^{t} \mid \theta_{\mathbf{V}}^{\mu}\right)+\epsilon_{v} \mathcal{N}_{v}^{t}
% \end{equation}
% \noindent 其中,$\mathcal{N}_{v}^{t}$为探索噪音,以增加车辆动作的多样性,$\epsilon_{v}$为车辆$v$的探索常数。
% 车辆动作的集合被表示为 $\boldsymbol{a}_{\mathbf{V}}^{t} = \left\{\boldsymbol{a}_{v}^{t}\mid \forall v \in \mathbf{V}\right\}$。
% 另一方面,边缘节点$e$的动作空间表示为:
% \begin{equation}
% 	\boldsymbol{a}_{e}^{t} = \{b_{v, e}^{t} \mid \forall v \in \mathbf{V}_{e}^{t}\}
% \end{equation}
% 其中$b_{v, e}^t$是边缘节点$e$在时间$t$为车辆$v$分配的V2I带宽。
% 同样地,边缘节点$e$的动作可以由本地边缘策略网络根据系统状态以及车辆动作得到。
% \begin{equation}
% 	\boldsymbol{a}_{e}^{t}=\mu_{\mathbf{E}}\left(\boldsymbol{o}_{e}^{t},  \boldsymbol{a}_{\boldsymbol{\mathbf{V}}}^{t} \mid \theta_{\mathbf{E}}^{\mu}\right)+\epsilon_{e} \mathcal{N}_{e}^{t}
% \end{equation}
% \noindent 其中$\mathcal{N}_{e}^{t}$和$\epsilon_{e}$分别为边缘节点$e$的探索噪声和探索常数。
% 此外,车辆和边缘节点的联合动作被表示为 $\boldsymbol{a}^{t}= \left\{\boldsymbol{a}_{e}^{t}, \boldsymbol{a}_{1}^{t}, \ldots, \boldsymbol{a}_{v}^{t}, \ldots, \boldsymbol{a}_{V}^{t}\right\}$。

% 环境通过执行联合动作获得系统奖励向量,其表示为:
% 	\begin{equation}
% 	\boldsymbol{r}^{t} = \begin{bmatrix}  r^{(1)}\left(\boldsymbol{a}_{\mathbf{V}}^{t},\boldsymbol{a}_{e}^{t} \mid \boldsymbol{o}^{t}\right)  &  r^{(2)}\left(\boldsymbol{a}_{\mathbf{V}}^{t},\boldsymbol{a}_{e}^{t} \mid \boldsymbol{o}^{t}\right) \end{bmatrix} ^{T}
% 	\end{equation}
% 	\noindent 其中 $r^{(1)}\left(\boldsymbol{a}_{\mathbf{V}}^{t},\boldsymbol{a}_{e}^{t} \mid \boldsymbol{o}^{t}\right)$ 和 $r^{(2)}\left(\boldsymbol{a}_{\mathbf{V}}^{t},\boldsymbol{a}_{e}^{t} \mid \boldsymbol{o}^{t}\right)$ 分别是两个目标(即实现的VCPS质量和VCPS 利润)的奖励,可以通过下式计算:  
% 	\begin{numcases}{}
% 			r^{(1)}\left(\boldsymbol{a}_{\mathbf{V}}^{t},\boldsymbol{a}_{e}^{t} \mid \boldsymbol{o}^{t}\right)=\frac{1}{\left|\mathbf{I}_e^t\right|} \sum_{\forall i \in \mathbf{I}_e^t}\operatorname{QV}_{i} \notag \\
% 			r^{(2)}\left(\boldsymbol{a}_{\mathbf{V}}^{t},\boldsymbol{a}_{e}^{t} \mid \boldsymbol{o}^{t}\right)=\frac{1}{\left|\mathbf{I}_e^t\right|} \sum_{\forall i \in \mathbf{I}_e^t} \operatorname{PV}_{j} 
% 	\end{numcases}
% 因此,车辆$v$在第$i$个目标中的奖励可以通过基于差分奖励的信用分配方案 \cite{foerster2018counterfactual} 得到,其为系统奖励和没有其动作所取得的奖励之间的差值,其表示为:
% \begin{equation}
% r_{v}^{(j), t}=r^{(j)}\left(\boldsymbol{a}_{\mathbf{V}}^{t},\boldsymbol{a}_{e}^{t} \mid \boldsymbol{o}^{t}\right)-r^{(j)}\left(\boldsymbol{a}_{\mathbf{V}-s}^{t},\boldsymbol{a}_{e}^{t} \mid \boldsymbol{o}^{t}\right), \forall i \in \{1, 2\}
% \label{equ 4-40}
% \end{equation}
% \noindent 其中 $r^{(j)}\left(\boldsymbol{a}_{\mathbf{V}-s}^{t},\boldsymbol{a}_{e}^{t} \mid \boldsymbol{o}^{t}\right)$ 是在没有车辆$v$贡献的情况下实现的系统奖励,它可以通过设置车辆$v$的空动作集得到。
% 车辆$v$在时间$t$的奖励向量表示为$\boldsymbol{r}_{v}^{t} = \begin{bmatrix}  r_{v}^{(1), t}  &  r_{v}^{(2), t} \end{bmatrix} ^{T}$。
% 车辆的差分奖励集合表示为 $\boldsymbol{r}_{\mathbf{V}}^{t}=\{ \boldsymbol{r}_{v}^{t} \mid \forall v \in \mathbf{V}\}$。

% 另一方面,系统奖励通过最小-最大归一化进一步转化为边缘节点的归一化奖励。
% 边缘节点$e$在时间$t$的第$i$个目标中的奖励由以下方式计算:
% \begin{equation}
% 	r_{e}^{(j), t}= \frac{r^{(j)}\left(\boldsymbol{a}_{\mathbf{V}}^{t},\boldsymbol{a}_{e}^{t} \mid \boldsymbol{o}^{t}\right) - \min \limits_{\forall {\boldsymbol{a}_{e}^{t}}^{\prime}} r^{(j)}\left(\boldsymbol{a}_{\mathbf{V}}^{t}, {\boldsymbol{a}_{e}^{t}}^{\prime} \mid \boldsymbol{o}^{t}\right)} {\max \limits_{\forall {\boldsymbol{a}_{e}^{t}}^{\prime}} r^{(j)}\left(\boldsymbol{a}_{\mathbf{V}}^{t}, {\boldsymbol{a}_{e}^{t}}^{\prime} \mid \boldsymbol{o}^{t}\right) - \min \limits_{\forall {\boldsymbol{a}_{e}^{t}}^{\prime}} r^{(j)}\left(\boldsymbol{a}_{\mathbf{V}}^{t}, {\boldsymbol{a}_{e}^{t}}^{\prime} \mid \boldsymbol{o}^{t}\right)}
% \label{equ 4-41}
% \end{equation}
% \noindent 其中 $\min \limits_{\forall {\boldsymbol{a}_{e}^{t}}^{\prime}} r^{(j)} (\boldsymbol{a}_{\mathbf{V}}^{t}, {\boldsymbol{a}_{e}^{t}}^{\prime} \mid \boldsymbol{o}^{t})$ 和 $\max \limits_{\forall {\boldsymbol{a}_{e}^{t}}^{\prime}} r^{(j)}(\boldsymbol{a}_{\mathbf{V}}^{t}, {\boldsymbol{a}_{e}^{t}}^{\prime} \mid \boldsymbol{o}^{t})$ 分别是在相同的系统状态$\boldsymbol{o}^{t}$下,车辆动作$\boldsymbol{a}_{\mathbf{V}}^{t}$不变时,可实现的系统奖励最小值和最大值。
% 边缘节点$e$在时间$t$的奖励向量表示为 $\boldsymbol{r}_{e}^{t} = \begin{bmatrix}  r_{e}^{(1), t}  &  r_{e}^{(2), t} \end{bmatrix} ^{T}$。
% 交互经验包括当前系统状态$\boldsymbol{o}^{t}$、车辆动作$\boldsymbol{a}_{\mathbf{V}}^{t}$、边缘节点动作$\boldsymbol{a}_{e}^{t}$、车辆奖励$\boldsymbol{r}_{\mathbf{V}}^{t}$、边缘节点奖励$\boldsymbol{r}_{e}^{t}$、权重$\boldsymbol{w}^{t}$,以及下一时刻系统状态$\boldsymbol{o}^{t+1}$都存储到经验回放缓存$\mathcal{B}$。

% \subsection[\hspace{-2pt}多目标策略评估]{{\CJKfontspec{SimHei}\zihao{4} \hspace{-8pt}多目标策略评估}}

% 本章节阐述了针对多目标的策略评估,具体地,提出了决斗评论家网络,根据状态的价值和动作的优势来评估智能体的动作。
% 在DCN中有两个全连接的网络,即动作优势网络和状态价值网络。
% 车辆和边缘节点的AA网络参数分别表示为 $\theta_{\mathbf{V}}^{\mathscr{A}}$ 和 $\theta_{\mathbf{E}}^{\mathscr{A}}$。
% 同样,车辆和边缘节点的SV网络的参数分别表示为 $\theta_{\mathbf{V}}^{\mathscr{V}}$ 和 $\theta_{\mathbf{E}}^{\mathscr{V}}$。
% 用$A_{\mathbf{V}}\left({o}_{v}^{m},  {a}_{v}^{m}, \boldsymbol{a}_{\boldsymbol{\mathbf{V}}-v}^{m}, \boldsymbol{w}^{m} \mid \theta_{\mathbf{V}}^{\mathscr{A}} \right)$表示车辆$v$中AA网络的输出标量, 其中 $\boldsymbol{a}_{\boldsymbol{\mathbf{V}}-v}^{m}$ 表示其他车辆动作。
% 同样地,以边缘节点$e$为输入的AA网络的输出标量表示为 $A_{\mathbf{E}}\left({o}_{e}^{m},  {a}_{e}^{m}, \boldsymbol{a}_{\boldsymbol{\mathbf{V}}}^{m}, \boldsymbol{w}^{m} \mid \theta_{\mathbf{E}}^{\mathscr{A}} \right)$, 其中 $\boldsymbol{a}_{\boldsymbol{\mathbf{V}}}^{m}$ 表示所有车辆动作。
% 车辆$v$的SV网络的输出标量表示为 $V_{\mathbf{V}}\left({o}_{v}^{m}, \boldsymbol{w}^{m} \mid \theta_{\mathbf{V}}^{\mathscr{V}} \right)$。
% 同样地,边缘节点$e$的SV网络的输出标量表示为 $V_{\mathbf{E}}\left({o}_{e}^{m}, \boldsymbol{w}^{m} \mid \theta_{\mathbf{E}}^{\mathscr{V}} \right)$。

% 多目标策略评估由三个步骤组成。
% 首先,AA网络基于观测、动作和权重输出智能体动作的优势。
% 其次,VS网络根据观测和权重,输出当前状态的价值。
% 最后,采用聚合模块,根据动作优势和状态价值,输出智能体动作的价值。
% 具体来说,在AA网络中随机生成$N$个动作并将智能体动作替换,以评估当前动作对于随机动作的平均优势。
% 用${a}_{v}^{m, n}$和${a}_{e}^{m, n}$分别表示车辆$v$和边缘节点$e$的第$n$个随机动作。
% 因此,车辆$v$和边缘节点$e$的第$n$个随机动作的优势可分别表示为 $A_{\mathbf{V}}\left({o}_{v}^{m},  {a}_{v}^{m, n}, \boldsymbol{a}_{\boldsymbol{\mathbf{V}}-v}^{m}, \boldsymbol{w}^{m} \mid \theta_{v}^{\mathscr{A}} \right)$ 和 $A_{\mathbf{E}}\left({o}_{e}^{m},  {a}_{e}^{m, n}, \boldsymbol{a}_{\boldsymbol{\mathbf{V}}}^{m}, \boldsymbol{w}^{m} \mid \theta_{\mathbf{E}}^{\mathscr{A}} \right)$。

% 进一步,通过评估智能体动作相对于随机动作的平均优势,对价值函数进行聚合。
% 因此,车辆$v\in\mathbf{V}$和边缘节点$e$的动作价值是通过下式计算: 
% \begin{align}
%     Q_{\mathbf{V}}\left({o}_{v}^{m}, {a}_{v}^{m}, \boldsymbol{a}_{\boldsymbol{\mathbf{V}}-v}^{m}, \boldsymbol{w}^{m} \mid \theta_{\mathbf{V}}^{Q} \right) &= V_{\mathbf{V}}\left({o}_{v}^{m}, \boldsymbol{w}^{m} \mid \theta_{\mathbf{V}}^{\mathscr{V}} \right) + A_{\mathbf{V}}\left({o}_{v}^{m},  {a}_{v}^{m}, \boldsymbol{a}_{\boldsymbol{\mathbf{V}}-v}^{m}, \boldsymbol{w}^{m} \mid \theta_{\mathbf{V}}^{\mathscr{A}} \right) \notag \\
%     &- \frac{1}{N} \sum_{\forall n} A_{\mathbf{V}}\left({o}_{v}^{m},  {a}_{v}^{m, n}, \boldsymbol{a}_{\boldsymbol{\mathbf{V}}-v}^{m}, \boldsymbol{w}^{m} \mid \theta_{\mathbf{V}}^{\mathscr{A}} \right)
% \end{align}
% \begin{align}
%     Q_{E}\left({o}_{e}^{m},  {a}_{e}^{m}, \boldsymbol{a}_{\boldsymbol{\mathbf{V}}}^{m}, \boldsymbol{w}^{m} \mid \theta_{\mathbf{E}}^{Q} \right) &= V_{\mathbf{E}}\left({o}_{e}^{m}, \boldsymbol{w}^{m} \mid \theta_{\mathbf{E}}^{\mathscr{V}} \right) + A_{\mathbf{E}}\left({o}_{e}^{m},  {a}_{e}^{m}, \boldsymbol{a}_{\boldsymbol{\mathbf{V}}}^{m}, \boldsymbol{w}^{m} \mid \theta_{\mathbf{E}}^{\mathscr{A}} \right) \notag \\
%     &- \frac{1}{N} \sum_{\forall n} A_{\mathbf{E}}\left({o}_{e}^{m},  {a}_{e}^{m, n}, \boldsymbol{a}_{\boldsymbol{\mathbf{V}}}^{m}, \boldsymbol{w}^{m} \mid \theta_{\mathbf{E}}^{\mathscr{A}} \right)
% \end{align}
% 其中,$\theta_{\mathbf{V}}^{Q}$ 和 $\theta_{\mathbf{V}}^{Q}$ 包含相应的AA和SV网络的参数。
% \begin{align}
% 	\theta_{\mathbf{V}}^{Q} = (\theta_{\mathbf{V}}^{\mathscr{A}}, \theta_{\mathbf{V}}^{\mathscr{V}}), \theta_{\mathbf{V}}^{Q^{\prime}} = (\theta_{\mathbf{V}}^{\mathscr{A}^{\prime}}, \theta_{\mathbf{V}}^{\mathscr{V}^{\prime}}) \\
% 	\theta_{\mathbf{E}}^{Q} = (\theta_{\mathbf{E}}^{\mathscr{A}}, \theta_{\mathbf{E}}^{\mathscr{V}}), \theta_{\mathbf{E}}^{Q^{\prime}} = (\theta_{\mathbf{E}}^{\mathscr{A}^{\prime}}, \theta_{\mathbf{E}}^{\mathscr{V}^{\prime}})
% \end{align}

% \subsection[\hspace{-2pt}网络学习和更新]{{\CJKfontspec{SimHei}\zihao{4} \hspace{-8pt}网络学习和更新}}

% 从经验回放缓存$\mathcal{B}$中抽出$M$小批量,以训练车辆和边缘节点的策略和评论家网络,其中单个样本表示为 $\left(\boldsymbol{o}_{\mathbf{V}}^{m}, {o}_{e}^{m}, \boldsymbol{w}^{m}, \boldsymbol{a}_{\mathbf{V}}^{m}, {a}_{e}^{m}, \boldsymbol{r}_{\mathbf{V}}^{m}, \boldsymbol{r}_{e}^{m}, \boldsymbol{o}_{\mathbf{V}}^{m+1}, {o}_{e}^{m+1}, \boldsymbol{w}^{m+1}\right)$。
% 车辆$v$的目标值表示为:
% \begin{equation}
% 	y_{v}^{m} = \boldsymbol{r}_{v}^{m} \boldsymbol{w}^{m} +\gamma Q_{\mathbf{V}}^{\prime}\left({o}_{v}^{m+1},  {a}_{v}^{m+1}, \boldsymbol{a}_{\boldsymbol{\mathbf{V}}-v}^{m+1}, \boldsymbol{w}^{m+1} \mid \theta_{\mathbf{V}}^{Q^{\prime}} \right)
% \end{equation}
% \noindent 其中 $Q_{\mathbf{V}}^{\prime}({o}_{v}^{m+1},  {a}_{v}^{m+1}, \boldsymbol{a}_{\boldsymbol{\mathbf{V}}-v}^{m+1}, \boldsymbol{w}^{m+1} \mid \theta_{\mathbf{V}}^{Q^{\prime}})$ 是目标车辆评论家网络产生的动作价值。
% $\gamma$是折扣因子。
% $\boldsymbol{a}_{\boldsymbol{\mathbf{V}}-v}^{m+1}$ 是没有车辆$v$的下一时刻车辆动作。
% \begin{equation}
% 	\boldsymbol{a}_{\boldsymbol{\mathbf{V}}-v}^{m+1} = \{ {a}_{1}^{m+1}, \ldots, {a}_{s-1}^{m+1}, {a}_{s+1}^{m+1}, \ldots, {a}_{v}^{m+1} \}
% \end{equation}
% 而 ${a}_{v}^{m+1}$ 是目标车辆策略网络根据对下一时刻系统状态的局部观测产生的车辆$v$的下一时刻动作。
% \begin{equation}
% 	{a}_{v}^{m+1} = \mu_{\mathbf{V}}^{\prime}(\boldsymbol{o}_{v}^{m+1} \mid \theta_{\mathbf{V}}^{\mu^{\prime}})
% \end{equation}
% 类似地,边缘节点$e$的目标值表示为:
% \begin{equation}
% 	y_{e}^{m} = \boldsymbol{r}_{e}^{m} \boldsymbol{w}^{m} +\gamma Q_{\mathbf{E}}^{\prime}\left({o}_{e}^{m+1},  {a}_{e}^{m+1}, \boldsymbol{a}_{\boldsymbol{\mathbf{V}}}^{m+1}, \boldsymbol{w}^{m+1} \mid \theta_{\mathbf{E}}^{Q^{\prime}} \right)
% \end{equation}
% \noindent 其中 $Q_{\mathbf{E}}^{\prime}({o}_{e}^{m+1},  {a}_{e}^{m+1}, \boldsymbol{a}_{\boldsymbol{\mathbf{V}}}^{m+1}, \boldsymbol{w}^{m+1} \mid \theta_{\mathbf{E}}^{Q^{\prime}})$ 表示由目标边缘评论家网络产生的动作价值。
% $\boldsymbol{a}_{\boldsymbol{\mathbf{V}}}^{m+1}$ 是下一时刻车辆动作。
% ${a}_{e}^{m+1}$表示下一时刻边缘节点动作,该动作可由目标边缘策略网络根据其对下一时刻系统状态的局部观测获得,即${a}_{e}^{m+1} = \mu_{\mathbf{E}}^{\prime}(\boldsymbol{o}_{e}^{m+1}, \boldsymbol{a}_{\mathbf{V}}^{m+1} \mid \theta_{\mathbf{E}}^{\mu^{\prime}})$。

% 车辆评论家网络和边缘评论家网络的损失函数是通过分类分布的时间差分(Temporal Difference, TD)学习得到的,其表示为:
% \begin{equation}
% 	\mathcal{L}\left(\theta_{\mathbf{V}}^{Q}\right)=\frac{1}{M} \sum_{m} \frac{1}{S} \sum_{v} {Y_v^{m}}
% \end{equation}
% \begin{equation}
% 	\mathcal{L}\left(\theta_{\mathbf{E}}^{Q}\right)=\frac{1}{M} \sum_{m} {Y_e^{m}}
% \end{equation}
% \noindent 其中$Y_v^{m}$和$Y_e^{m}$分别是车辆$v$和边缘节点$e$的目标值和局部评论家网络产生的动作价值之差的平方。
% \begin{equation}
% 	\begin{aligned}
% 		Y_v^{m} &= \left(y_{v}^{m}-Q_{\mathbf{V}}\left({o}_{v}^{m},  {a}_{v}^{m}, \boldsymbol{a}_{\boldsymbol{\mathbf{V}}-v}^{m}, \boldsymbol{w}^{m} \mid \theta_{\mathbf{V}}^{Q} \right)\right)^{2} \\
% 	\end{aligned}
% \end{equation}
% \begin{equation}
% 	\begin{aligned}
% 		Y_e^{m} &=\left(y_{e}^{m}-Q_{\mathbf{E}}\left({o}_{e}^{m},  {a}_{e}^{m}, \boldsymbol{a}_{\boldsymbol{\mathbf{V}}}^{m}, \boldsymbol{w}^{m} \mid \theta_{\mathbf{V}}^{Q} \right)\right)^{2} \\
% 	\end{aligned}
% \end{equation}
% 车辆和边缘策略网络参数通过确定性的策略梯度进行更新。
% \begin{equation}
% 	\nabla_{\theta_{\mathbf{V}}^{\mu}} \mathcal{J} (\theta_{\mathbf{V}}^{\mu}) \approx \frac{1}{M} \sum_{m} \frac{1}{S} \sum_{v} P_{v}^{m} 
% \end{equation}
% \begin{equation}
% 	\nabla_{\theta_{\mathbf{E}}^{\mu}} \mathcal{J} (\theta_{\mathbf{E}}^{\mu}) \approx \frac{1}{M} \sum_{m} P_{e}^{m} 
% \end{equation}
% \noindent 其中 
% \begin{equation}
% P_{v}^{m} = \nabla_{{a}_{v}^{m}} Q_{\mathbf{V}}\left({o}_{v}^{m}, {a}_{v}^{m}, \boldsymbol{a}_{\boldsymbol{\mathbf{V}}-v}^{m}, \boldsymbol{w}^{m} \mid \theta_{v}^{Q} \right) \nabla_{\theta_{\mathbf{V}}^{\mu}} \mu_{\mathbf{V}}\left({o}_{v}^{m} \mid \theta_{\mathbf{V}}^{\mu}\right)
% \end{equation}
% \begin{equation}
% P_{e}^{m} = \nabla_{{a}_{e}^{m}} Q_{\mathbf{E}}\left({o}_{e}^{m}, {a}_{e}^{m}, \boldsymbol{a}_{\boldsymbol{\mathbf{V}}}^{m}, \boldsymbol{w}^{m} \mid \theta_{\mathbf{E}}^{Q} \right) \nabla_{\theta_{\mathbf{E}}^{\mu}} \mu_{\mathbf{E}}\left({o}_{e}^{m}, {\boldsymbol{a}}_{\boldsymbol{\mathbf{V}}}^{m} \mid \theta_{\mathbf{E}}^{\mu}\right)
% \end{equation}

% 本地策略和评论家网络参数分别以$\alpha$和$\beta$的学习率更新。
% 特别地,车辆和边缘节点定期更新目标网络的参数,即当$t \mod t_{\operatorname{tgt}} = 0$, 其中 $t_{\operatorname{tgt}}$ 是目标网络的参数更新周期。
% \begin{align}
% 	\theta_{\mathbf{V}}^{\mu^{\prime}} \leftarrow n_{\mathbf{V}} \theta_{\mathbf{V}}^{\mu}+(1-n_{\mathbf{V}}) \theta_{\mathbf{V}}^{\mu^{\prime}}, \theta_{\mathbf{V}}^{Q^{\prime}} \leftarrow n_{\mathbf{V}} \theta_{\mathbf{V}}^{Q}+(1-n_{\mathbf{V}}) \theta_{\mathbf{V}}^{Q^{\prime}}\\
% 	\theta_{\mathbf{E}}^{\mu^{\prime}} \leftarrow n_{\mathbf{E}} \theta_{\mathbf{E}}^{\mu}+(1-n_{\mathbf{E}}) \theta_{\mathbf{E}}^{\mu^{\prime}}, \theta_{\mathbf{E}}^{Q^{\prime}} \leftarrow n_{\mathbf{E}} \theta_{\mathbf{E}}^{Q}+(1-n_{\mathbf{E}})  \theta_{\mathbf{E}}^{Q^{\prime}}
% \end{align}
% \noindent 其中 $n_{\mathbf{V}} \ll 1$ 和 $n_{\mathbf{E}} \ll 1$。
% 同样,分布式行动者的策略网络参数也会定期更新,即当$t \mod t_{\operatorname{act}} = 0$,其中 $t_{\operatorname{act}}$ 是分布式行动者的策略网络的参数更新周期。
% \begin{align}
% 	\theta_{\mathbf{V}, k}^{\mu} \leftarrow \theta^{{\mu}^{\prime}}_{\mathbf{V}}, \theta_{\mathbf{V}, k}^{Q} \leftarrow \theta_{\mathbf{V}}^{Q^{\prime}}, \forall k \in \{1, 2, \ldots, K\}\\
% 	\theta_{\mathbf{E}, k}^{\mu} \leftarrow \theta_{\mathbf{E}}^{\mu^{\prime}}, \theta_{\mathbf{E}, k}^{Q} \leftarrow \theta_{\mathbf{E}}^{Q^{\prime}}, \forall k \in \{1, 2, \ldots, K\}
% \end{align}

% \section[\hspace{-2pt}实验设置与结果分析]{{\CJKfontspec{SimHei}\zihao{-3} \hspace{-8pt}实验设置与结果分析}}\label{section 4-6}

% \subsection[\hspace{-2pt}实验设置]{{\CJKfontspec{SimHei}\zihao{4} \hspace{-8pt}实验设置}}

% 本章节使用Python 3.9.13和TensorFlow 2.8.0来搭建仿真实验模型以评估所提MAMO方案的性能,其运行在配备AMD Ryzen 9 5950X 16核处理器@ 3.4 GHz,两个NVIDIA GeForce RTX 3090 GPU和64 GB内存的Ubuntu 20.04服务器上。
% 实验仿真参数设置如下:
% V2I通信范围被设定为500 m。
% 传输功率被设定为100 mW。
% AWGN和可靠性阈值分别设置为-90 dBm和0.9\cite{wang2019delay}。
% V2I通信的信道衰减增益遵循高斯分布,其均值为2,方差为0.4\cite{sadek2009distributed}。
% $\hat{\Theta_{i}}$、$\hat{\Psi_{i}}$、$\hat{\Xi_{i}}$、$\hat{\Phi_{i}}$和$\hat{\Omega_{i}}$的加权系数分别设置为0.6、0.4、0.2、0.4和0.4。

% MAMO中策略和评论家网络的架构和超参数描述如下:
% 本地策略网络是有两层隐藏层的四层全连接神经网络,其中神经元的数量分别为256和128。
% 目标策略网络的结构与本地策略网络相同。
% 本地评论家网络是四层全连接神经网络,有两层隐藏层,其中神经元的数量分别为512和256。
% 目标评论家网络的结构与本地评论家网络相同。
% 折扣率、批大小和最大经验回放缓存大小分别为0.996、256和1$\times10^{6}$。
% 策略网络和评论家网络的学习率分别为1$\times10^{-4}$和1$\times10^{-4}$。

% 进一步,本章节实现了三个比较算法,其具体细节介绍如下:
% \begin{itemize}
% 	\item \textbf{随机分配}: 随机选择动作来确定感知信息、感知频率、上传优先级、传输功率和V2I带宽分配。
% 	\item \textbf{分布式深度确定性策略梯度}\cite{barth2018distributed}: 在边缘节点实现了一个智能体,根据系统状态,集中式地确定感知信息、感知频率、上传优先级、传输功率和V2I带宽分配。VCPS质量和VCPS 利润权重分别设定为0.5和0.5。
% 	\item \textbf{多智能体分布式深度确定性策略梯度}: 其为D4PG的多智能体版本,并在车辆上分布式实现,根据对物理环境的局部观测决定感知信息、感知频率、上传优先级和传输功率,边缘节点决定V2I带宽分配。VCPS质量和VCPS 利润权重分别设为0.5和0.5。
% \end{itemize}

% 为了评估算法在视图建模质量和有效性方面的表现,本章设计了以下两个新的指标。
% \begin{itemize}
% 	\item \textbf{单位开销质量}:其定义为花费单位开销实现的VCPS质量,其计算公式为:
% 		\begin{equation}
% 			\operatorname{QPUC}=\frac{\sum_{\forall t \in \mathbf{T}} \sum_{\forall e \in \mathbf{E}} \sum_{\forall i \in \mathbf{I}_e^t} \mathrm{QV}_i}{\sum_{\forall t \in \mathbf{T}} \sum_{\forall e \in \mathbf{E}} \sum_{\forall i \in \mathbf{I}_e^t} \mathrm{CV}_i}
% 		\end{equation}
% 		其中$\mathrm{QV}_i$和$\mathrm{CV}_i$分别是视图$i$的质量和开销。
% 	\item \textbf{单位质量利润}:其定义为单位VCPS质量所实现的VCPS 利润,其计算公式为:
% 		\begin{equation}
% 		\operatorname{PPUQ}=\frac{\sum_{\forall t \in \mathbf{T}} \sum_{\forall e \in \mathbf{E}} \sum_{\forall i \in \mathbf{I}_e^t}\mathrm{PV}_i}{\sum_{\forall t \in \mathbf{T}} \sum_{\forall e \in \mathbf{E}} \sum_{\forall i \in \mathbf{I}_e^t} \mathrm{QV}_i}
% 		\end{equation}
% 		其中$\mathrm{PV}_i$和$\mathrm{CV}_i$分别是视图$i$的利润和开销。
% \end{itemize}
% QPUC越高表明它能在相同的开销下实现更高的VCPS质量,而PPUQ越高表明它能更有效地使用感知和通信资源。上述指标全面显示了算法在同时最大化VCPS质量和最小化VCPS 开销的性能。
% 本章进一步基于公式\ref{equ 4-16}、\ref{equ 4-20}、\ref{equ 4-21}和\ref{equ 4-23}设计了四个指标,分别是\textbf{平均及时性}(Average Timeliness, AT)、\textbf{平均冗余度}(Average Redundancy, AR)、\textbf{平均感知开销}(Average Sensing Cost, ASC)和\textbf{平均传输开销}(Average Transmission Cost, ATC)。 

% \subsection[\hspace{-2pt}实验结果与分析]{{\CJKfontspec{SimHei}\zihao{4} \hspace{-8pt}实验结果与分析}}

% \textbf{1) 算法收敛性:}图\ref{fig 4-3}比较了四种算法的收敛性。其中,图\ref{fig 4-3}(a)和\ref{fig 4-3}(b)分别展示了四种算法的QPUC和PPUQ表现。X轴表示迭代次数,Y轴表示达到的QPUC和PPUQ。QPUC和PPUQ越高,表明算法在VCPS质量和VCPS开销方面表现越好。MAMO在大约850次迭代后,达到了最高的QPUC(约13.6)和最高的PPUQ(约1.13)。相比之下,RA、D4PG和MAD4PG分别实现了约2.29、7.34和2.58的QPUC,并分别实现了约0.87、0.99和0.81的PPUQ。与RA、D4PG和MAD4PG相比,MAMO在QPUC方面分别实现了约494.1\%、85.5\%和428.8\%的提升,在PPUQ方面分别实现了约30.6\%、14.2\%和40.7\%的改善。值得注意的是,MAMO是唯一能够同时改善QPUC和PPUQ的方案。这显示了MAMO在同时实现QPUC和PPUQ最大化方面的优势。

% \begin{figure}[h]
%  \centering
%  \captionsetup{font={small, stretch=1.312}}\includegraphics[width=1\columnwidth]{Fig4-3-different-algorithms.pdf}
%  \bicaption[算法收敛性比较]{算法收敛性比较,其显示与RA、D4PG和MAD4PG相比,MAMO在收敛后(约850次迭代)达到了最高的QPUC和最高的PPUQ。(a)单位开销质量(b)单位质量利润}[Convergence comparison]{Convergence comparison, which shows MAMO achieves the highest QPUC and the highest PPUQ compared with RA, D4PG, and MAD4PG after convergence (around 850 iterations). (a) Quality per unit cost (b) Profit per unit quality}
%  \label{fig 4-3}
% \end{figure}

% \begin{figure}[h]
%  \centering
%  \captionsetup{font={small, stretch=1.312}}\includegraphics[width=1\columnwidth]{Fig4-4-different-networks.pdf}
%  \bicaption[隐藏层中不同神经元数量下MAMO性能比较]{隐藏层中不同神经元数量下MAMO性能比较。(a)单位开销质量(b)单位质量利润}[Performance comparison of MAMO under different numbers of neurons in the hidden layers]{Performance comparison of MAMO under different numbers of neurons in the hidden layers. (a) Quality per unit cost (b) Profit per unit quality}
%  \label{fig 4-4}
% \end{figure}

% \textbf{2) 神经元数量的影响:}
% 图\ref{fig 4-4}比较了不同神经元数量下MAMO的性能。其中,X轴表示策略网络和评论家网络的两个隐藏层的神经元数量,分别设置为[64, 32] $\sim$ [1024, 512]和[128, 64] $\sim$ [2048, 1024]。如图\ref{fig 4-4}(a)所示,当策略网络和评论家网络的隐藏层的神经元数量设置为默认值(即[256, 128]和[512, 256])时,MAMO实现了最高的VCPS质量和利润。图\ref{fig 4-4}(b)比较了其他三个指标,包括AT、ASC和ATC。AT、ASC和ATC越低,说明在信息新鲜度、感知开销和传输开销方面表现越好。可以注意到,当每个隐藏层的神经元数量为默认设置时,MAMO在最小化AT、ASC和ATC方面表现最佳。

% \begin{figure}[h]
%  \centering
%  \captionsetup{font={small, stretch=1.312}}\includegraphics[width=1\columnwidth]{Fig4-5-different-scenarios.pdf}
%  \bicaption[不同交通场景下的性能比较]{不同交通场景下的性能比较。(a)单位开销质量(b)单位质量利润(c)平均感知开销(d)平均传输开销}[Performance comparison under different traffic scenarios]{Performance comparison under different traffic scenarios. (a) Quality per unit cost (b) Profit per unit quality (c) Average sensing cost (d) Average transmission cost}
%  \label{fig 4-5}
% \end{figure}

% \textbf{3) 交通场景的影响:}
% 图\ref{fig 4-5}比较了四种算法在不同交通场景下的性能。X轴表示交通场景,不同场景在不同的时间和空间中提取了现实的车辆轨迹作为输入,分别为:1)2016年11月16日8:00至8:05,中国成都市青羊区1平方公里区域;2)同日23:00至23:05,同一区域;3)2016年11月27日8:00至8:05,中国西安碑林区1平方公里区域。图\ref{fig 4-5}(a)比较了四种算法的QPUC,MAMO在所有场景下都取得了最高的QPUC。图\ref{fig 4-5}(b)比较了四种算法的PPUQ,MAMO在所有情况下都表现最好。与RA、D4PG和MAD4PG相比,MAMO分别提高了589.0\%、106.7\%和514.8\%的QPUC,并分别提高了约41.6\%、23.6\%和45.7\%的PPUQ。图\ref{fig 4-5}(c)比较了四种算法的ASC。MAMO的ASC低于RA、D4PG和MAD4PG,说明MAMO可以实现车辆协同感知以降低感知开销。图\ref{fig 4-5}(d)比较了四种算法的ATC,在不同情况下,MAMO的ATC最低。

% \begin{figure}[h]
%  \centering
%  \captionsetup{font={small, stretch=1.312}}\includegraphics[width=1\columnwidth]{Fig4-6-different-bandwidths.pdf}
%  \bicaption[不同V2I带宽下的性能比较]{不同V2I带宽下的性能比较。(a)单位开销质量(b)单位质量利润(c)平均及时性(d)平均冗余度(e)平均感知开销(f)平均传输开销}[Performance comparison under different V2I bandwidths]{Performance comparison under different V2I bandwidths. (a) Quality per unit cost (b) Profit per unit quality (c) Average timeliness (d) Average redundancy (e) Average sensing cost (f) Average transmission cost}
%  \label{fig 4-6}
% \end{figure}

% \begin{figure}[h]
%  \centering
%  \captionsetup{font={small, stretch=1.312}}\includegraphics[width=1\columnwidth]{Fig4-7-different-numbers.pdf}
%  \bicaption[不同视图需求下的性能比较]{不同视图需求下的性能比较。(a)单位开销质量(b)单位质量利润(c)平均及时性(d)平均冗余度(e)平均感知开销(f)平均传输开销}[Performance comparison under different digit twin requirements]{Performance comparison under different digit twin requirements. (a) Quality per unit cost (b) Profit per unit quality (c) Average timeliness (d) Average redundancy (e) Average sensing cost (f) Average transmission cost}
%  \label{fig 4-7}
% \end{figure}

% \textbf{4) V2I带宽的影响:}
% 图\ref{fig 4-6}比较了四种算法在不同V2I带宽下的性能。X轴表示V2I带宽,从1MHz增加到3MHz。较大的V2I带宽代表每辆车被分配的V2I带宽也随之增加。图\ref{fig 4-6}(a)比较了四种算法的QPUC。随着带宽的增加,MAMO的QPUC也相应增加。这是因为在带宽富余的场景中,MAMO中车辆之间的协同感知和上传更加有效。图\ref{fig 4-6}(b)显示了四种算法的PPUQ,可以进一步证明这一优势。如图\ref{fig 4-6}(b)所示,MAMO在不同的V2I带宽下实现了最高的PPUQ。特别地,与RA、D4PG和MAD4PG相比,MAMO分别提高了约453.3\%、131.4\%和437.6\%的QPUC,并使PPUQ提高了约33.0\%、18.3\%和48.4\%。图\ref{fig 4-6}(c)比较了四种算法的AT,MAMO实现了最低的AT。当带宽从2.5MHz增加到3MHz时,MAMO和D4PG的性能差距很小。这是因为随着带宽的增加,视图的及时性改善是有限的。图\ref{fig 4-6}(d)比较了四种算法的AR。AR越低意味着协同感知和上传的性能越好,MAMO实现了最低的AR。图\ref{fig 4-6}(e)和\ref{fig 4-6}(f)分别比较了四种算法的ASC和ATC。可以看出,当带宽增加时,这四种算法的ATC都会下降。原因是,当带宽增加时,信息上传时间减少,导致传输开销降低。MAMO的ASC和ATC在大多数情况下保持在最低水平。

% \textbf{5) 视图需求的影响:}
% 图\ref{fig 4-7}比较了四种算法在不同视图需求下的性能,其中X轴表示视图所需信息的平均数量从3增加到7。视图所需信息的平均数越大,说明车辆的感知和上传工作负荷越大。图\ref{fig 4-7}(a)比较了四种算法的QPUC。随着平均所需信息数的增加,四种算法的QPUC也相应减少。然而,MAMO在所有情况下保持最高的QPUC。图\ref{fig 4-7}(b)比较了四种算法的PPUQ。正如预期的那样,MAMO在所有情况下都取得了最高的PPUQ。特别地,与RA、D4PG和MAD4PG相比,MAMO的QPUC分别高出458.7\%、130.6\%和426.2\%,PPUQ分别高出31.5\%、18.2\%和40.7\%。图\ref{fig 4-7}(c)比较了四种算法的AT。MAMO在AT方面取得了最佳性能。图\ref{fig 4-7}(d)比较了四种算法的AR,表明MAMO可以在所有情况下实现最低的AR。图\ref{fig 4-7}(e)和\ref{fig 4-7}(f)分别比较了四种算法的ASC和ATC。值得注意的是,当平均信息数增加时,四种算法的ASC和ATC都会增加。原因是视图需要的平均信息量增加,导致车辆感应和传输开销提高。

% \section[\hspace{-2pt}本章小结]{{\CJKfontspec{SimHei}\zihao{-3} \hspace{-8pt}本章小结}}\label{section 4-7}

% 本章提出了协同感知与V2I上传场景,其中基于车辆协同感知与V2I协同上传构建逻辑视图。
% 具体地,基于多类M/G/1优先级队列构建了协同感知模型,并基于信道衰减分布和SNR阈值构建了V2I协同上传模型。
% 在此基础上,设计了两个指标QV和CV,以衡量在边缘节点建模的视图的质量和开销,并形式化定义了双目标优化问题,通过协同感知和上传,最大化VCPS质量的同时,最小化VCPS 开销。
% 进一步,提出了基于多目标的多智能体深度强化学习算法,其中采用了决斗评论家网络,根据状态价值和动作优势来评估智能体动作。
% 最后,进行了全面的性能评估,证明了所提MAMO算法的优越性。


\chapter[\hspace{0pt}基于存储架构的高并行度高基数FFT加速器设计]{{\CJKfontspec{SimHei}\zihao{3}\hspace{-5pt}面向自适应的FFT加速器设计}}
\removelofgap
\removelotgap
本章将研究面向车载信息物理融合的质量-开销均衡优化。
具体内容安排如下:
\ref{section 3-1} 节是本章的引言,介绍了车联网中车载信息物理融合系统的研究现状及存在的不足,同时阐述本章的主要贡献。
% \ref{section 4-2} 节阐述了协同感知与V2I上传场景。
% \ref{section 4-3} 节给出了系统模型的详细描述。
% \ref{section 4-4} 节形式化定义了最大化VCPS质量并最小化VCPS开销的双目标优化问题。
% \ref{section 4-5} 节设计了基于多目标的多智能体深度强化学习算法。
% \ref{section 4-6} 节搭建了实验仿真模型并进行了性能验证。
% \ref{section 4-7} 节对本章的研究工作进行总结。
\section[\hspace{-2pt}基于存储架构的FFT加速器存在的问题]{{\CJKfontspec{SimHei}\zihao{-3} \hspace{-8pt}基于存储架构的FFT加速器存在的问题}}\label{section 4-1}

\section[\hspace{-2pt}位维度变换表示方法]{{\CJKfontspec{SimHei}\zihao{-3} \hspace{-8pt}位维度变换表示}}\label{section 4-2}
\subsection[\hspace{-2pt}数学理论分析]{{\CJKfontspec{SimHei}\zihao{-3} \hspace{-8pt}数学理论分析}}
\subsection[\hspace{-2pt}对应硬件架构映射关系]{{\CJKfontspec{SimHei}\zihao{-3} \hspace{-8pt}基-$2^5$的MDC单元设计}}

\section[\hspace{-2pt}高基数的FFT存储架构顶层设计]{{\CJKfontspec{SimHei}\zihao{-3} \hspace{-8pt}高基数的FFT存储架构顶层设计}}\label{section 4-3}

\section[\hspace{-2pt}高基数的FFT存储架构详细设计]{{\CJKfontspec{SimHei}\zihao{-3} \hspace{-8pt}高基数的FFT存储架构详细设计}}\label{section 4-4}
\subsection[\hspace{-2pt}串型-串型变换的详细分析]{{\CJKfontspec{SimHei}\zihao{-3} \hspace{-8pt}}串型-串型变换的详细分析}

\subsection[\hspace{-2pt}变换电路的配置方案分析]{{\CJKfontspec{SimHei}\zihao{-3} \hspace{-8pt}}变换电路的配置方案分析}

\subsection[\hspace{-2pt}最优硬资源推导]{{\CJKfontspec{SimHei}\zihao{-3} \hspace{-8pt}}最优硬件资源推导}

\subsection[\hspace{-2pt}高并行度高基数的电路变换设计]{{\CJKfontspec{SimHei}\zihao{-3} \hspace{-8pt}高并行度高基数的电路变换设计}}

\section[\hspace{-2pt}旋转因子设计]{{\CJKfontspec{SimHei}\zihao{-3} \hspace{-8pt}旋转因子设计}}\label{section 4-5}
\subsection[\hspace{-2pt}旋转因子分解策略]{{\CJKfontspec{SimHei}\zihao{-3} \hspace{-8pt}}旋转因子分解策略}

\subsection[\hspace{-2pt}旋转因子存储器配置]{{\CJKfontspec{SimHei}\zihao{-3} \hspace{-8pt}}旋转因子存储器配置}

\section[\hspace{-2pt}本章小结]{{\CJKfontspec{SimHei}\zihao{-3} \hspace{-8pt}}本章小结}\label{section 4-6}

% 新兴感知技术、无线通信和计算模式推动了现代新能源汽车和智能网联汽车的发展。
% 现代汽车中装备了各种车载感知器,以增强车辆的环境感知能力 \cite{zhu2017overview}。
% 另一方面,V2X通信\cite{chen2020a}的发展使车辆、路侧设备和云端之间的合作得以实现。
% 同时,车载边缘计算\cite{dai2021edge}是很有前途的范式,
% 可以实现计算密集型和延迟关键型的智能交通系统应用 \cite{zhao2022foundation}。
% 这些进展都成为了开发车载信息物理融合系统的强大驱动力。
% 具体来说,通过协同感知和上传,车联网中的物理实体,如车辆、行人和路侧设备等,
% 可以在边缘节点上构建为相应的逻辑映射。

% 车载信息物理融合中的检测、预测、规划和控制技术被广泛研究。
% 大量工作聚焦于检测技术,例如雨滴数量检测\cite{wang2021deep}和驾驶员疲劳检测\cite{chang2018design}。
% 针对车辆状态预测方法,研究人员提出了混合速度曲线预测\cite{zhang2019a}、车辆跟踪\cite{iepure2021a}
% 和加速预测\cite{zhang2020data}等。同时,部分研究工作提出了不同的调度方案,
% 例如基于物理比率-K干扰模型的广播调度\cite{li2020cyber}和基于既定地图模型的路径规划\cite{lian2021cyber}。
% 此外,部分研究集中在智能网联车辆的控制算法上,例如车辆加速控制\cite{lv2018driving}、交叉路口控制\cite{chang2021an}
% 和电动汽车充电调度\cite{wi2013electric}。
% 这些关于状态检测、轨迹预测、路径调度和车辆控制的研究促进了各种ITS应用的实施。
% 然而,这些工作忽略了感知和上传开销,假设高质量可用信息可以在VEC中构建。
% 少数研究考虑了VCPS中的信息质量,例如时效性\cite{liu2014temporal, dai2019temporal}和准确性\cite{rager2017scalability, yoon2021performance},
% 但上述研究都没有考虑通过协同感知和上传,在VCPS中实现高质量低成本的信息物理融合。

% 本章旨在通过车辆协同感知与上传,构建基于车载信息物理融合的逻辑视图,
% 并进一步在最大化车载信息物理融合质量和最小化视图构建开销方面寻求最佳平衡。
% 然而,实现这一目标面临着以下主要挑战。
% 首先,车联网中的信息高度动态,因此考虑感知频率、排队延迟和传输时延的协同效应,以确保信息的新鲜度和时效性是至关重要的。
% 其次,物理信息是具有时空相关性的,不同车辆在不同的时间或空间范围内感应到的信息可能存在冗余或不一致性。
% 因此,具有不同感知能力的车辆有望以分布式方式合作,以提高感知和通信资源的利用率。
% 再次,物理信息在分布、更新频率和模式方面存在异质性,这给构建高质量视图带来很大挑战。
% 最后,高质量的视图构建需要更高的感知和通信资源开销,这也是需要考虑的关键因素。
% 综上所述,通过协同感知和上传,实现面向车载边缘计算的高质量、低开销视图具有重要意义,但也具有一定的挑战性。

% 本章致力于研究车载信息物理融合系统的质量-开销均衡优化问题,并通过协同感知与上传实现高质量、低开销的视图建模。
% 本章的主要贡献如下:第一,提出了协同感知与V2I上传场景,考虑视图的及时性和一致性,
% 设计了车载信息物理融合质量指标,并考虑边缘视图构建过程中信息冗余度、感知开销和传输开销,
% 设计了车载信息物理融合开销指标。
% 进一步,提出了双目标优化问题,在最大化VCPS质量的同时最小化VCPS开销
% 第二,提出了基于多目标的多智能体深度强化学习算法。
% 具体地,在车辆和边缘节点中分别部署智能体,车辆动作空间包括感知决策、感知频率、上传优先级和传输功率分配,
% 而边缘节点动作空间是V2I带宽分配策略。
% 同时,设计了决斗评论家网络(Dueling Critic Network, DCN),
% 其根据状态价值(State-Value, SV)和动作优势(Action-Advantage, AA)评估智能体动作。
% 系统奖励是一维向量,其中包含VCPS质量和VCPS利润,并通过差分奖励信用分配得到车辆的个人奖励,
% 进一步通过最小-最大归一化得到边缘节点的归一化奖励。
% 第三,建立了基于现实世界车辆轨迹的仿真实验模型,并将MAMO与三种对比算法进行比较,
% 包括随机分配、分布式深度确定性策略梯度\cite{barth2018distributed},
% 以及多智能体分布式深度确定性策略梯度。
% 此外,本文设计了两个指标,即单位开销质量(Quality Per Unit Cost, QPUC)和单位质量利润(Profit Per Unit Quality, PPUQ)
% 用于定量衡量算法实现的均衡。
% 仿真结果表明,与其他算法相比,MAMO在最大化QPUC和PPUQ方面更具优势。

% \section[\hspace{-2pt}协同感知与 V2I 上传场景]{{\CJKfontspec{SimHei}\zihao{-3} \hspace{-8pt}协同感知与 V2I 上传场景}}\label{section 4-2}

% 本章节介绍了协同感知与V2I上传场景。如图\ref{fig 4-1}所示,车辆配备各种车载感知器,如超声波雷达、激光雷达、光学相机和毫米波雷达,可以对环境进行感知。通过车辆间协同地感知,可以获得多源信息,包括其他车辆、弱势道路参与者、停车场和路边基础设施的状态。这些信息可用于在边缘节点中建立视图模型,并进一步用于支撑各种ITS应用,如自动驾驶\cite{bai2022hybrid}、智慧路口控制系统\cite{hadjigeorgious2023real},以及全息城市交通流管理\cite{wang2023city}。逻辑视图需要融合车联网中物理实体的不同模式信息,以更好地反映实时物理车辆环境,从而提高ITS的性能。然而,构建高质量的逻辑视图可能需要更高的感知频率、更多的信息上传量以及更高的能量消耗。

% \begin{figure}[h]
% \centering
%   \captionsetup{font={small, stretch=1.312}}\includegraphics[width=1\columnwidth]{Fig4-1-architerture.pdf}
%   \bicaption[协同感知与 V2I 上传场景]{协同感知与 V2I 上传场景}[Cooperative sensing and V2I uploading scenario]{Cooperative sensing and V2I uploading scenario}
%   \label{fig 4-1}
% \end{figure} 

% 本系统的工作流程如下:首先,车辆感知并排队上传不同物理实体的实时状态。接着,边缘节点将V2I带宽分配给车辆,同时,车辆确定传输功率。物理实体的视图是基于从车辆收到的多源信息进行融合建立的。需要注意的是,在该系统中,多源信息是由车辆以不同的感知频率感应到的,因此上传时的新鲜度会不同。虽然增加感知频率可以提高新鲜度,但会增加排队延迟和能源消耗。此外,多个车辆可能感知到特定物理实体的信息,若由所有车辆上传,则可能会浪费通信资源。因此,为了提高资源利用率,需要有效而经济地分配通信资源。在此基础上,为了最大化面向车载边缘计算的视图的VCPS质量并最小化VCPS开销,必须量化衡量边缘节点构建的视图的质量和开销,并设计高效经济的协同感知和上传的调度机制。

% \section[\hspace{-2pt}车载信息物理融合质量/开销模型]{{\CJKfontspec{SimHei}\zihao{-3} \hspace{-8pt}车载信息物理融合质量/开销模型}}\label{section 4-3}

% \subsection[\hspace{-2pt}基本符号]{{\CJKfontspec{SimHei}\zihao{4} \hspace{-8pt}基本符号}}

% 本系统离散时间片的集合用$\mathbf{T}=\left\{1,\ldots,t,\ldots, T \right\}$表示。
% 多源信息集合用$\mathbf{D}$表示,其中信息$d \in \mathbf{D}$的特征是三元组$d=\left(\operatorname{type}_d, u_d, \left|d\right| \right)$,其中$\operatorname{type}_d$、$u_d$和$\left|d\right|$分别是信息类型、更新间隔和数据大小。
% $\mathbf{V}$表示车辆的集合,每个车辆$v\in \mathbf{V}$的特征是三元组$v=\left (l_v^t, \mathbf{D}_v, \pi_v \right )$,其中$l_v^t$、$\mathbf{D}_v$和$\pi_v$分别是位置、感知的信息集和传输功率。
% 对于$d \in \mathbf{D}_v$,车辆$v$的感知开销(即能耗)用$\phi_{d, v}$表示。
% 用$\mathbf{E}$表示边缘节点的集合,其中每个边缘节点$e \in \mathbf{E}$的特征是$e=\left (l_e, g_e, b_e \right)$,其中$l_{e}$、$r_{e}$和$b_{e}$分别为位置、通信范围和带宽。
% 车辆$v$与边缘节点$e$之间的距离表示为$\operatorname{dis}_{v, e}^t \triangleq \operatorname{distance} \left (l_v^t, l_e \right ), \forall v \in \mathbf{V}, \forall e \in \mathbf{E}, \forall t \in \mathbf{T}$。
% 在时间$t$内处于边缘节点$e$的通信覆盖范围内的车辆集合表示为$\mathbf{V}_e^t=\left \{v \vert \operatorname{dis}_{v, e}^t \leq g_e, \forall v \in \mathbf{V} \right \}, \mathbf{V}_e^t \subseteq \mathbf{V}$。

% 感知决策指示器表示车辆$v$在时间$t$是否感知信息$d$,其用以下方式表示:
% \begin{equation}
% 	c_{d, v}^t \in \{0, 1\}, \forall d \in \mathbf{D}_{v}, \forall v \in \mathbf{V}, \forall t \in \mathbf{T}
% 	\label{equ 4-1} 
% \end{equation}
% 那么,车辆$v$在时间$t$的感应信息集合表示为 $\mathbf{D}_v^t = \{ d | c_{d, v}^{t} = 1, \forall d \in \mathbf{D}_v \}, \mathbf{D}_v^t \subseteq \mathbf{D}_v$。
% 对于任何信息$d \in \mathbf{D}_v^t$来说,信息类型都是不同的, 即$\operatorname{type}_{d^*} \neq \operatorname{type}_{d}, \forall d^* \in \mathbf{D}_v^t \setminus \left\{ d\right \}, \forall d \in \mathbf{D}_v^t$。
% 车辆$v$在时间$t$的信息$d$的感知频率用$\lambda_{d, v}^t$表示,其需要满足车辆$v$的感应能力要求。
% \begin{equation}
% 	\lambda_{d, v}^{t} \in [\lambda_{d, v}^{\min} , \lambda_{d, v}^{\max} ], \ \forall d \in \mathbf{D}_v^t, \forall v \in \mathbf{V}, \forall t \in \mathbf{T}
% \end{equation}
% 其中$\lambda_{d, v}^{\min}$和$\lambda_{d, v}^{\max}$分别是车辆$v$中信息${d}$的最小和最大感知频率。
% 车辆$v$中的信息$d$在时间$t$的上传优先级用$p_{d, v}^t$表示,不同信息的上传优先级需各不相同。
% \begin{equation}
% 	{p}_{d^*, v}^t \neq {p}_{d, v}^t, \forall d^* \in \mathbf{D}_v^t \setminus \left\{ d\right \}, \forall d \in \mathbf{D}_v^t, \forall v \in \mathbf{V}, \forall t \in \mathbf{T}
% \end{equation}
% 其中${p}_{d^*, v}^t$是信息$d^* \in \mathbf{D}_v^t$中的上传优先级。
% 车辆$v$在时间$t$的传输功率用$\pi_{v}^t$表示,其不能超过车辆$v$的功率容量。
% \begin{equation}
% 	\pi_v^t \in \left[ 0 , \pi_v \right ], \forall v \in \mathbf{V}, \forall t \in \mathbf{T}
% \end{equation}
% 边缘节点$e$在时间$t$为车辆$v$分配的V2I带宽用$b_{v, e}^t$表示,且其需要满足:
% \begin{equation}
% 	b_{v, e}^t \in \left [0, b_e \right], \forall v \in \mathbf{V}_e^{t}, \forall e \in \mathbf{E}, \forall t \in \mathbf{T}
% 	\label{equ 4-5} 
% \end{equation}
% 边缘节点$e$分配的V2I总带宽不能超过其容量$b_e$,即${\sum_{\forall v \in \mathbf{V}_e^{t}} b_{v, e}^t} \leq b_e, \forall t \in \mathbf{T}$。

% 本系统中物理实体的集合为 $\mathbf{I}^{\prime}$,其中$i^{\prime} \in \mathbf{I}^{\prime}$表示物理实体,如车辆、行人和路侧基础设施等。
% $\mathbf{D}_{i^{\prime}}$是与实体$i^{\prime}$相关的信息集合,可以用$\mathbf{D}_{i^{\prime}}=\left\{d \mid y_{d, i^{\prime}} = 1, \forall d \in \mathbf{D} \right\}$, $\forall i^{\prime} \in \mathbf{I}^{\prime}$表示, 其中$y_{d, i^{\prime}}$是二进制数,表示信息$d$是否与实体$i^{\prime}$关联。
% $\mathbf{D}_{i^{\prime}}$的大小用$|\mathbf{D}_{i^{\prime}}|$表示。
% 每个实体可能需要多个信息,即$|\mathbf{D}_{i^{\prime}}| = \sum_{\forall d \in \mathbf{D}}y_{d, i^{\prime}} \geq 1, \forall i^{\prime} \in \mathbf{I}^{\prime}$。
% 对于每个实体$i^{\prime} \in \mathbf{I}^{\prime}$,可能有一个视图$i$在边缘节点中建模。
% 用$\mathbf{I}$表示视图的集合,用$\mathbf{I}_e^{t}$表示时间为$t$时在边缘节点$e$中建模的视图集合。
% 因此,边缘节点$e$收到且被视图$i$需要的信息集合可以用$\mathbf{D}_{i, e}^t=\bigcup_{\forall v \in \mathbf{V}}\left(\mathbf{D}_{i^{\prime}} \cap \mathbf{D}_{v, e}^t\right), \forall i \in \mathbf{I}_e^{t}, \forall e \in \mathbf{E}$表示,且 $| \mathbf{D}_{i, e}^t |$是边缘节点$e$收到且被视图$i$需要的信息数量,其计算公式为$| \mathbf{D}_{i, e}^t | =  \sum_{\forall v \in \mathbf{V}} \sum_{\forall d \in \mathbf{D}_v} c_{d, v}^t  y_{d, i^{\prime}}$。

% \subsection[\hspace{-2pt}协同感知模型]{{\CJKfontspec{SimHei}\zihao{4} \hspace{-8pt}协同感知模型}}
% 车辆协同感知是基于多类M/G/1优先级队列\cite{moltafet2020age}进行建模。
% 假设具有$\operatorname{type}_d$的信息的上传时间$\operatorname{\hat{g}}_{d, v, e}^t$遵循均值$\alpha_{d, v}^t$和方差$\beta_{d, v}^t$的一类一般分布。
% 那么,车辆$v$中的上传负载$\rho_{v}^{t}$由$ \rho_{v}^{t}=\sum_{\forall d \subseteq \mathbf{D}_v^t} \lambda_{d, v}^{t} \alpha_{d, v}^t$表示。
% 根据多类M/G/1优先级队列,需要满足$\rho_{v}^{t} < 1$才能达到队列的稳定状态。
% 信息$d$在时间$t$之前的到达时间用$\operatorname{a}_{d, v}^t$表示,其计算公式为:
% \begin{equation}
%     \operatorname{a}_{d, v}^t =  \frac{\left \lfloor t \lambda_{d, v}^t \right \rfloor }{\lambda_{d, v}^{t}} 
% \end{equation}
% 在时间$t$之前,由$\operatorname{u}_{d, v}^t$表示的信息$d$的更新时间是通过下式计算:
% \begin{equation}
%     \operatorname{u}_{d, v}^t = \left \lfloor  \frac{\operatorname{a}_{d, v}^t}{u_d} \right \rfloor  u_d
% \end{equation}
% 其中$u_d$是信息$d$的更新间隔时间。


% 在时间$t$,车辆$v$中比$d$有更高上传优先级的信息集合,用$\mathbf{D}_{d, v}^t = \{ d^* \mid p_{d^*, v}^{t} > p_{d, v}^{t} , \forall d^* \in \mathbf{D}_v^t \}$表示,其中$p_{d^*, v}^{t}$是信息$d^* \in \mathbf{D}_v^t$的上传优先级。
% 因此,信息$d$前面的上传负载(即$v$在时间$t$时要在$d$之前上传的信息数量)通过下方计算得出: 
% \begin{equation}
% 	\rho_{d, v}^{t}=\sum_{\forall d^* \in \mathbf{D}_{d, v}^t} \lambda_{d^*, v}^t \alpha_{d^*, v}^t
% \end{equation}
% 其中$\lambda_{d^*, v}^t$和$\alpha_{d^*, v}^t$分别为时间$t$内车辆$v$中信息$d^*$的感知频率和平均传输时间。
% 根据Pollaczek-Khintchine公式\cite{takine2001queue},车辆$v$中信息$d$的排队时间计算如下:
% \begin{equation}
%     \operatorname{q}_{d, v}^t= \frac{1} {1 - \rho_{d, v}^{t}} 
%         \left[ \alpha_{d, v}^t + \frac{ \lambda_{d, v}^{t} \beta_{d, v}^t + \sum\limits_{\forall d^* \in \mathbf{D}_{d, v}^t} \lambda_{d^*, v}^t \beta_{d^*, v}^t }{2\left(1-\rho_{d, v}^{t} - \lambda_{d, v}^{t} \alpha_{d, v}^t\right)}\right] 
%         - \alpha_{d, v}^t
% \end{equation}

% \subsection[\hspace{-2pt}V2I协同上传模型]{{\CJKfontspec{SimHei}\zihao{4} \hspace{-8pt}V2I协同上传模型}}

% 车辆间V2I协同上传是基于信道衰减分布和信噪比阈值来建模的。
% 车辆$v$和边缘节点$e$之间的V2I通信在时间$t$的信噪比通过公式\ref{equ 4-10}\cite{sadek2009distributed}计算得到。
% \begin{equation}
%     \operatorname{SNR}_{v, e}^{t}=\frac{1}{N_{0}} \left|h_{v, e}\right|^{2} \tau {\operatorname{dis}_{v, e}^{t}}^{-\varphi} {\pi}_v^t
%     \label{equ 4-10}
% \end{equation}
% 其中$N_{0}$为AWGN;$h_{v, e}$为信道衰减增益;$\tau$为取决于天线设计的常数;$\varphi$为路径损耗指数。
% 假设$\left|h_{v, e}\right|^{2}$遵循均值$\mu_{v, e}$和方差$\sigma_{v, e}$的一类分布,其表示方法为:
% \begin{equation}
%     \tilde{p}=\left\{\mathbb{P}: \mathbb{E}_{\mathbb{P}}\left[\left|h_{v, e}\right|^{2}\right]=\mu_{v, e}, \mathbb{E}_{\mathbb{P}}\left[\left|h_{v, e}\right|^{2}-\mu_{v, e}\right]^{2}=\sigma_{v, e}\right\}
% \end{equation}
% 进一步,基于成功传输概率和可靠性阈值来衡量V2I传输可靠性。
% \begin{equation}
%     \inf_{\mathbb{P} \in \tilde{p}} \operatorname{Pr}_{[\mathbb{P}]}\left(\operatorname{SNR}_{v, e}^{t} \geq \operatorname{SNR}_{v, e}^{\operatorname{tgt}}\right) \geq \delta
% \end{equation}
% \noindent 其中$\operatorname{SNR}_{v, e}^{\operatorname{tgt}}$和$\delta$分别为目标SNR阈值和可靠性阈值。
% 由车辆$v$上传并由边缘节点$e$接收的信息集合用$\mathbf{D}_{v, e}^{t} = \bigcup_{\forall v \in \mathbf{V}_{e}^{t}} \mathbf{D}_{v}^{t}$表示。

% 根据香农理论,车辆$v$和边缘节点$e$之间在时间$t$的V2I通信的传输率用$\operatorname{z}_{v, e}^t$表示,其计算公式如下:
% \begin{equation}
%     \operatorname{z}_{v, e}^t=b_{v}^{t} \log _{2}\left(1+\mathrm{SNR}_{v, e}^{t}\right)
% \end{equation}
% 假设车辆$v$被安排在时间$t$上传$d$,并且$d$将在一定的排队时间$\mathrm{\bar{q}}_{d, v}^t$后被传输。
% 然后,本章把车辆$v$开始传输$d$的时刻表示为$\mathrm{t}_{d, v}^t=t+\mathrm{q}_{d, v}^t$。
% 从$\mathrm{t}_{d, v}^t$到$\mathrm{t}_{d, v}^t + f$之间传输的数据量可由 $\int_{\mathrm{t}_{d, v}^t}^{\mathrm{t}_{d, v}^t+f} \mathrm{z}_{v, e}^t \mathrm{~d} t$ bits 得到,其中$f \in \mathbb{R}^{+}$和$\mathrm{z}_{i, e}^t$是时间$t$的传输速率。
% 如果在整个传输过程中可以传输的数据量大于信息$d$的大小,那么上传就会完成。
% 因此,从车辆$v$到边缘节点$e$传输信息$d$的时间,用$\operatorname{g}_{d, v, e}^t$表示,计算如下:
% \begin{equation}
%     \operatorname{g}_{d, v, e}^t=\inf _{j \in \mathbb{R}^+} \left \{ \int_{\operatorname{k}_{d, v}^t}^{\operatorname{k}_{d, v}^t + j} {\operatorname{z}_{v, e}^t} \operatorname{d}t \geq \left|d\right| \right \} 
% \end{equation}
% \noindent 其中$\operatorname{t}_{d, v}^t = t +\operatorname{q}_{d, v}^t$是车辆$v$开始传输信息$d$的时刻。

% \section[\hspace{-2pt}质量-开销均衡问题定义]{{\CJKfontspec{SimHei}\zihao{-3} \hspace{-8pt}质量-开销均衡问题定义}}\label{section 4-4}

% \subsection[\hspace{-2pt}VCPS质量]{{\CJKfontspec{SimHei}\zihao{4} \hspace{-8pt}VCPS质量}}

% 首先,由于视图是基于连续上传和时间变化的信息建模的,本章对信息$d$的及时性定义如下:
% \begin{definition}
% 信息$d$在车辆$v$中的及时性$\theta_{d, v} \in \mathbb{Q}^{+}$被定义为更新和接收信息$d$之间的时间差。
% \begin{equation}
%     \theta_{d, v} = \operatorname{a}_{d, v}^t + \operatorname{q}_{d, v}^t + \operatorname{g}_{d, v, e}^t-\operatorname{u}_{d, v}^{t}, \forall d \in \mathbf{D}_v^t,\forall v \in \mathbf{V}
% \end{equation}
% \end{definition}
% \begin{definition}
% 视图$i$的及时性 $\Theta_{i} \in \mathbb{Q}^{+}$定义为与物理实体$i^{\prime}$相关的信息的最大及时性之和。
% 	\begin{equation}
%     	\Theta_{i} = \sum_{\forall v\in \mathbf{V}_{e}^{t}} \max_{\forall d \in \mathbf{D}_{i^{\prime}} \cap \mathbf{D}_v^t}\theta_{d, v}, \forall i \in \mathbf{I}_{e}^{t}, \forall e \in \mathbf{E}
%     	\label{equ 4-16}
% 	\end{equation}
% \end{definition}

% 其次,由于不同类型的信息有不同的感知频率和上传优先级,本章定义视图的一致性来衡量与同一物理实体相关的信息的一致性。
% \begin{definition}
% 视图$i$的一致性$\Psi_{i} \in \mathbb{Q}^{+}$定义为信息更新时间差的最大值。
% \begin{equation}
%     \Psi_{i}=\max_{\forall d \in \mathbf{D}_{i, e}^{t}, \forall v \in \mathbf{V}_{e}^{t}} {\operatorname{u}_{d, v}^t} - \min_{\forall d \in \mathbf{D}_{i, e}^{t}, \forall v \in \mathbf{V}_{e}^{t}} {\operatorname{u}_{d, v}^t} , \forall i \in \mathbf{I}_{e}^{t}, \forall e \in \mathbf{E}
% \end{equation}
% \end{definition}

% 最后,本章给出了视图的质量的正式定义,其综合了视图的及时性和一致性。
% \begin{definition}
% 视图质量$\operatorname{QV}_{i} \in (0, 1)$定义为视图$i$的归一化及时性和归一化一致性的加权平均和。
% 	\begin{equation}
% 	    \operatorname{QV}_{i} = w_1 (1 -\hat{\Theta_{i}}) + w_2 (1 - \hat{\Psi_{i}}), \forall i \in \mathbf{I}_{e}^t, \forall e \in \mathbf{E}
% 	\end{equation}
% \end{definition}
% \noindent 其中$\hat{\Theta_{i}} \in (0, 1)$和$\hat{\Psi_{i}} \in (0, 1)$分别表示归一化的及时性和归一化的一致性,这可以通过最小-最大归一化对及时性和一致性的范围进行重新调整至$(0, 1)$来获得。
% $\hat{\Theta_{i}}$和$\hat{\Psi_{i}}$的加权系数分别用$w_1$和$w_2$表示,可以根据ITS应用的不同要求进行相应的调整,$w_1+w_2=1$。
% $\operatorname{QV}_{i}$的值越高,说明构建的视图质量越高。
% 考虑车辆自动驾驶系统下,视图的时效性和一致性都是关键。一方面,汽车需要实时接收到周围车辆和行人的位置、速度和行为信息,以便能够做出准确的驾驶决策。如果视图数据延迟太高或过时,车辆可能无法及时识别出潜在的危险或变化情况,从而导致事故或冲突的发生。另一方面,一致性方面的重要性表现在视图数据的一致性建模。车辆需要确保从边缘节点接收到的视图数据是准确、完整且一致的,以便能够准确地理解周围环境并做出正确的决策。如果视图数据存在不一致或缺失,车辆可能会做出错误的判断,从而导致不安全的驾驶行为或导航错误。

% 进一步,基于视图质量定义车载信息物理融合质量如下:
% \begin{definition}
% VCPS质量$\mathscr{Q} \in (0, 1)$被定义为在调度期间$\mathbf{T}$的边缘节点中建模的每个视图的QV的平均值。
% 	\begin{equation}
% 		\mathscr{Q}=\frac{\sum_{\forall t \in \mathbf{T}} \sum_{\forall e \in \mathbf{E}} \sum_{\forall i \in \mathbf{I}_e^t} \operatorname{QV}_{i}}{\sum_{\forall t \in \mathbf{T}} \sum_{\forall e \in \mathbf{E}} |\mathbf{I}_e^t| }
% 	\end{equation}
% \end{definition}

% \subsection[\hspace{-2pt}VCPS开销]{{\CJKfontspec{SimHei}\zihao{4} \hspace{-8pt}VCPS开销}}

% 首先,由于同一物理实体的状态可能被多个车辆同时感应到,本章对信息$d$的冗余度定义如下:
% \begin{definition}
% 信息$d$的冗余度$\xi_d \in \mathbb{N}$定义为车辆感应到同一类型$\operatorname{type}_d$的额外信息数量。
% \begin{equation}
%     \xi_d= \left | \mathbf{D}_{d, i, e} \right| - 1, \forall d \in \mathbf{D}_j, \forall i \in \mathbf{I}_{e}^{t}, \forall e \in \mathbf{E}
% \end{equation}
% \noindent 其中$\mathbf{D}_{d, i, e}$是边缘节点$e$收到且被视图$i$需要,且类型为$\operatorname{type}_d$的信息集合,其由$\mathbf{D}_{d, i, e}=\left\{ d^* \vert \operatorname{type}_{d^*} = \operatorname{type}_{d}, \forall d^* \in \mathbf{D}_{i, e}^t \right \}$表示。

% \end{definition}
% \begin{definition}
% 视图$i$的冗余度$\Xi_j \in \mathbb{N}$定义为视图$i$中的总冗余度。
% 	\begin{equation}
%        \Xi_j =  \sum_{\forall d \in \mathbf{D}_{i^{\prime}}} \xi_d, \forall i \in \mathbf{I}_{e}^{t}, \forall e \in \mathbf{E}
%        \label{equ 4-20}
%     \end{equation}
% \end{definition}

% 其次,信息感知和传输需要消耗车辆的能量,本章定义视图$i$的感知开销和传输开销如下:
% \begin{definition}
% 视图$i$的感知开销$\Phi_{i} \in \mathbb{Q}^{+}$定义为视图$i$所需信息的总感知开销。
% 	\begin{equation}
%         \Phi_{i} = \sum_{\forall v \in \mathbf{V}_{e}^{t}} \sum_{\forall d \in \mathbf{D}_{i^{\prime}} \cap \mathbf{D}_v^t}{\phi_{d, v}}, \forall i \in \mathbf{I}_{e}^t, \forall e \in \mathbf{E}
%         \label{equ 4-21}
%     \end{equation}
%     其中$\phi_{d, v}$是信息$d$在车辆$v$中的感知开销。
% \end{definition}
% \begin{definition}
% 信息$d$在车辆$v$中的传输开销${\omega}_{d, v} \in \mathbb{Q}^{+}$定义为信息上传时消耗的传输功率。
% \begin{equation}
%     {\omega}_{d, v}= \pi_v^t \operatorname{g}_{d, v, e}^t, \forall d \in \mathbf{D}_v^t
% \end{equation}
% 其中$\pi_v^t$和$\operatorname{g}_{d, v, e}^t$分别为传输功率和传输时间。
% \end{definition}
% \begin{definition}
% 视图$i$的传输开销$\Omega_{i} \in \mathbb{Q}^{+}$定义为视图$i$所需的信息总传输开销。
% 	\begin{equation}
%         \Omega_{i} = \sum_{\forall v \in \mathbf{V}_{e}^{t}} \sum_{\forall d \in \mathbf{D}_{i^{\prime}} \cap \mathbf{D}_v^t} {\omega}_{d, v}, \forall i \in \mathbf{I}_{e}^t, \forall e \in \mathbf{E}
%        	\label{equ 4-23}
%     \end{equation}
% \end{definition}

% 最后,给出视图开销的正式定义,其综合了冗余度、感知开销和传输开销。
% \begin{definition}
% 视图的开销$\operatorname{CV}_{i} \in (0, 1)$定义为视图$i$的归一化冗余度、归一化感知开销和归一化传输开销的加权平均和。
% 	\begin{equation}
% 	    \operatorname{CV}_{i} = w_3  \hat{\Xi_{i}} +  w_4 \hat{\Phi_{i}} + w_5 \hat{\Omega_{i}}, \forall i \in \mathbf{I}_{e}^t, \forall e \in \mathbf{E}
% 	\end{equation}
% \end{definition}
% \noindent 其中 $\hat{\Xi_{i}}\in (0, 1)$、$\hat{\Phi_{i}} \in (0, 1)$和$\hat{\Omega_{i}} \in (0, 1)$ 分别表示视图$i$的归一化冗余度、归一化感知开销和归一化传输开销。
% $\hat{\Xi_{i}}$、$\hat{\Phi_{i}}$和$\hat{\Omega_{i}}$ 的加权系数分别表示为 $w_3$、$w_4$和 $w_5$。
% 同样地,$w_3+w_4+w_5=1$。
% 进一步,VCPS开销定义如下:
% \begin{definition}
% VCPS 开销$\mathscr{C} \in (0, 1)$定义为$\mathbf{T}$调度期间边缘节点中每个视图模型的CV的平均值。
% 	\begin{equation}
% 		\mathscr{C}=\frac{\sum_{\forall t \in \mathbf{T}} \sum_{\forall e \in \mathbf{E}} \sum_{\forall i \in \mathbf{I}_e^t}  \operatorname{CV}_{i}}{\sum_{\forall t \in \mathbf{T}} \sum_{\forall e \in \mathbf{E}} |\mathbf{I}_e^t| }
% 	\end{equation}
% \end{definition}

% \subsection[\hspace{-2pt}双目标优化问题]{{\CJKfontspec{SimHei}\zihao{4} \hspace{-8pt}双目标优化问题}}

% 给定解决方案$( \mathbf{C}, \bf\Lambda, \mathbf{P}, \bf\Pi, \mathbf{B} )$,其中$\mathbf{C}$表示确定的感知信息决策,$\bf\Lambda$表示确定的感知频率。$\mathbf{P}$表示确定的上传优先级,$\bf\Pi$表示确定的传输功率,$\mathbf{B}$表示确定的V2I带宽分配,其中$c_{d, v}^t$、$\lambda_{d, v}^{t}$和$p_{d, v}^{t}$分别为时间$t$内车辆$v$的信息$d$的感知信息决策、感知频率和上传优先级,$\pi_v^t$和$b_v^t$分别为时间$t$内车辆$v$的传输功率和V2I带宽。
% \begin{numcases}{}
% 	\mathbf{C} = \left \{ c_{d, v}^t \vert \forall d \in \mathbf{D}_{v}, \forall v \in \mathbf{V}, \forall t \in \mathbf{T} \right  \} \notag \\
% 	{\bf\Lambda} = \left \{ \lambda_{d, v}^{t} \vert \forall d \in \mathbf{D}_v^t  , \forall v \in \mathbf{V}, \forall t \in \mathbf{T} \right \} \notag \\ 
% 	\mathbf{P} = \left \{ p_{d, v}^{t} \vert \forall d \in \mathbf{D}_v^t  , \forall v \in \mathbf{V}, \forall t \in \mathbf{T}\right \}  \notag \\
% 	{\bf\Pi} = \left \{ \pi_v^t \vert \forall v \in \mathbf{V}, \forall t \in \mathbf{T} \right \} \notag \\
% 	\mathbf{B} = \left \{ b_v^t \vert \forall v \in \mathbf{V}, \forall t \in \mathbf{T}\right \}
% \end{numcases}
% 本章提出了双目标优化问题,旨在同时实现VCPS质量的最大化和VCPS 开销的最小化::
% \begin{align}
% 	\mathcal{P}4.1: & \max_{\mathbf{C}, \bf\Lambda, \mathbf{P}, \bf\Pi, \mathbf{B}} \mathscr{Q}, \min_{\mathbf{C}, \bf\Lambda, \mathbf{P}, \bf\Pi, \mathbf{B}} \mathscr{C} \notag \\
% 	\text { s.t. }
% 	& (\ref{equ 4-1}) \sim (\ref{equ 4-5}) \notag \\
%     &\mathcal{C}4.1: \sum_{\forall d \subseteq \mathbf{D}_v^t} \lambda_{d, v}^{t} \mu_d<1,\ \forall v \in \mathbf{V}, \forall t \in \mathbf{T} \notag \\
%     &\mathcal{C}4.2: \inf_{\mathbb{P} \in \tilde{p}} \operatorname{Pr}_{[\mathbb{P}]}\left(\operatorname{SNR}_{v, e}^{t} \geq \operatorname{SNR}_{v, e}^{\operatorname{tgt}}\right) \geq \delta, \forall v \in \mathbf{V}, \forall t \in \mathbf{T} \notag \\
%     &\mathcal{C}4.3: {\sum_{\forall v \in \mathbf{V}_e^{t}}b_v^t} \leq b_e, \forall t \in \mathbf{T}
% \end{align}
% 其中$\mathcal{C}4.1$保证队列稳定状态,$\mathcal{C}4.2$保证传输可靠性。
% $\mathcal{C}4.3$要求边缘节点$e$分配的V2I带宽之和不能超过其容量$b_e$。
% 基于CV的定义,视图的利润定义如下:
% \begin{definition}
% 视图的利润$\operatorname{PV}_{j} \in (0, 1)$定义为视图$i$的CV的补集。
% 	\begin{equation}
% 		\mathscr{P}= 1 - \operatorname{CV}_{i}
% 	\end{equation}
% \end{definition}
% \noindent 然后,本章将VCPS 利润定义如下:
% \begin{definition}
% VCPS 利润$\mathscr{P} \in (0, 1)$被定义为在调度期$\mathbf{T}$期间,边缘节点中每个视图模型的PV的平均值。
% 	\begin{equation}
% 		\mathscr{P}= \frac{\sum_{\forall t \in \mathbf{T}} \sum_{\forall e \in \mathbf{E}} \sum_{\forall i \in \mathbf{I}_e^t}   \operatorname{PV}_{j} }{\sum_{\forall t \in \mathbf{T}} \sum_{\forall e \in \mathbf{E}} |\mathbf{I}_e^t| }
% 	\end{equation}
% \end{definition}
% \noindent 因此,$\mathcal{P}4.1$问题可以改写如下:
% \begin{align}
% 	\mathcal{P}4.2: & \max_{ \mathbf{C}, \bf\Lambda, \mathbf{P}, \bf\Pi, \mathbf{B} } \left (\mathscr{Q}, \mathscr{P} \right ) \notag \\
% 		\text { s.t. }
% 	&(\ref{equ 4-1}) \sim (\ref{equ 4-5}), \mathcal{C}4.1 \sim \mathcal{C}4.3
% \end{align}

% \section[\hspace{-2pt}基于多目标的多智能体强化学习算法设计]{{\CJKfontspec{SimHei}\zihao{-3} \hspace{-8pt}基于多目标的多智能体强化学习算法设计}}\label{section 4-5}

% 本章节提出了基于多目标的多智能体深度强化学习算法,其模型如图\ref{fig 4-2}所示,由$K$分布式行动者、学习器和经验回放缓存组成。
% 具体地,学习器由四个神经网络组成,即本地策略网络、本地评论家网络、目标策略网络和目标评论家网络。
% 其中车辆的本地策略网络、本地评论家网络、目标策略网络和目标评论家网络参数分别表示为 $\theta_{\mathbf{V}}^{\mu}$、$\theta_{\mathbf{V}}^{Q}$、 $\theta_{\mathbf{V}}^{\mu^{\prime}}$和$\theta_{\mathbf{V}}^{Q^{\prime}}$。
% 同样地,边缘节点的本地策略网络、本地评论家网络、目标策略网络和目标评论家网络参数分别表示为 $\theta_{\mathbf{E}}^{\mu}$、$\theta_{\mathbf{E}}^{Q}$、$\theta_{\mathbf{E}}^{\mu^{\prime}}$和$\theta_{\mathbf{E}}^{Q^{\prime}}$。
% 本地策略和本地评论家网络的参数是随机初始化的。
% 目标策略和目标评论家网络的参数被初始化为相应的本地网络。
% 然后,启动$K$分布式行动者,每个分布式行动者独立地与环境进行交互,并将交互经验存储到重放经验缓存。
% 分布式行动者由本地车辆策略网络和本地边缘策略网络组成,其分别用$\theta_{\mathbf{V}, k}^{\mu}$和$\theta_{\mathbf{E}, k}^{\mu}$表示,其网络参数是从学习器的本地策略网络复制而来的。
% 同时,初始化了最大存储容量为$|\mathcal{B}|$的经验回放缓存以存储重放经验。
% 基于多目标的多智能体深度强化学习的具体步骤见算法4.1,分布式行动者与环境的交互将持续到学习器的训练过程结束,其具体步骤见算法4.2。

% \begin{figure}[h]
% \centering
%   \captionsetup{font={small, stretch=1.312}}\includegraphics[width=1\columnwidth]{Fig4-2-solution-model.pdf}
%   \bicaption[基于多目标的多智能体深度强化学习模型]{基于多目标的多智能体深度强化学习模型}[Multi-agent multi-objective deep reinforcement learning model]{Multi-agent multi-objective deep reinforcement learning model}
%   \label{fig 4-2}
% \end{figure}


% \SetKwInOut{KwIn}{输入}
% \SetKwInOut{KwOut}{输出}

% \begin{algorithm}[h]\small
% \setstretch{1.245} %设置具有指定弹力的橡皮长度(原行宽的1.2倍)
% \renewcommand{\algorithmcfname}{算法}
% 	\caption{基于多目标的多智能体深度强化学习}
% 	\KwIn{折扣因子 $\gamma$、批大小 $M$、回放经验缓存 $\mathcal{B}$、学习率$\alpha$和$\beta$、目标网络参数更新周期 $t_{\operatorname{tgt}}$、分布式行动者网络参数更新周期 $t_{\operatorname{act}}$、随机动作数量$N$}
% 	\KwOut{信息感知决策$\mathbf{C}_v^t$、信息感知频率决策$\lambda_{d, v}^{t}$、上传优先级决策$p_{d, v}^{t}$、传输功率$\pi_v^t$、V2I带宽分配$b_{v, e}^{t}$}
% 	初始化网络参数\\
% 	初始化经验回放缓存 $\mathcal{B}$\\
% 	启动 $K$ 分布式行动者并复制网络参数给行动者\\
% 	\For{\songti{迭代次数} $= 1$ \songti{到最大迭代次数}}{
% 		\For{\songti{时间片} $t = 1$ \songti{到} $T$}{
% 			从经验回放缓存$\mathcal{B}$随机采样$M$小批量\\
% 			通过目标评论家网络中DCN网络得到目标值\\
% 			基于分类分布的TD学习计算更新评论家网络\\
% 			更新本地策略和评论家网络\\
% 			\If{$t \mod t_{\operatorname{tgt}} = 0$}{
% 				更新目标网络\\
% 			}
% 			\If{$t \mod t_{\operatorname{act}} = 0$}{
% 				复制网络参数给分布式行动者\\
% 			}
% 		}
% 	}
% \label{algorithm 4-1}
% \end{algorithm}

% \begin{algorithm}[t]\small
% \setstretch{1.245} %设置具有指定弹力的橡皮长度(原行宽的1.2倍)
% \renewcommand{\algorithmcfname}{算法}
% 	\caption{分布式行动者}
% 	\KwIn{车辆探索常数 $\epsilon_{v}$、边缘节点探索常数$\epsilon_{e}$、车辆本地观测 $\boldsymbol{o}_{v}^{t}$、边缘节点本地观测 $\boldsymbol{o}_{e}^{t}$}
% 	\KwOut{车辆动作$\boldsymbol{a}_{\mathbf{V}}^{t}$、边缘节点动作$\boldsymbol{a}_{e}^{t}$}
% 	\While{\songti{学习器没有结束}}{
% 		初始化随机过程 $\mathcal{N}$ 以进行探索\\
% 		生成随机权重 $\boldsymbol{w}^{t}$\\
% 		接收初始系统状态 $\boldsymbol{o}_{1}$\\
% 		\For{\songti{时间片} $t = 1$ \songti{到} $T$}{
% 			\For{\songti{车辆} $v=1$ \songti{到} $V$}{
% 				接收车辆本地观测 $\boldsymbol{o}_{v}^{t}$\\
% 				选择车辆动作 $\mu_{\mathbf{V}}\left(\boldsymbol{o}_{v}^{t} \mid \theta_{\mathbf{V}}^{\mu}\right)+\epsilon_{v} \mathcal{N}_{v}^{t}$\\
% 			}
% 			接收边缘节点本地观测 $\boldsymbol{o}_{e}^{t}$\\
% 			选择边缘节点动作 $\boldsymbol{a}_{e}^{t}=\mu_{\mathbf{E}}\left(\boldsymbol{o}_{e}^{t},  \boldsymbol{a}_{\boldsymbol{\mathbf{V}}}^{t} \mid \theta_{\mathbf{E}}^{\mu}\right)+\epsilon_{e} \mathcal{N}_{e}^{t}$\\
% 			接收奖励 $\boldsymbol{r}^{t}$ 和下一个系统状态 $\boldsymbol{o}^{t+1}$\\
% 			\For{\songti{车辆} $v=1$ \songti{到} $V$}{
% 				根据公式\ref{equ 4-40}计算车辆的差分奖励\\
% 			}
% 			根据公式\ref{equ 4-41}计算边缘节点的归一化奖励\\
% 			存储 $\left(\boldsymbol{o}^{t}, \boldsymbol{a}_{\mathbf{V}}^{t}, \boldsymbol{a}_{e}^{t}, \boldsymbol{r}_{\mathbf{V}}^{t}, \boldsymbol{r}_{e}^{t}, \boldsymbol{w}^{t}, \boldsymbol{o}^{t+1}\right)$ 到经验回放缓存 $\mathcal{B}$
% 		}
% 	}
% \label{algorithm 4-2}
% \end{algorithm}

% \subsection[\hspace{-2pt}多智能体分布式策略执行]{{\CJKfontspec{SimHei}\zihao{4} \hspace{-8pt}多智能体分布式策略执行}}

% 在MAMO中,车辆和边缘节点分别通过本地策略网络分布式地决定动作。
% 车辆$v$在时间$t$上对系统状态的局部观测表示为:
% 	\begin{equation}
% 		\boldsymbol{o}_{v}^{t}=\left\{t, v, l_{v}^t, \mathbf{D}_{v}, \Phi_{v}, \mathbf{D}_{e}^{t}, \mathbf{D}_{\mathbf{I}_e^t}, \boldsymbol{w}^{t}\right\}
% 	\end{equation} 
% \noindent 其中$t$为时间片索引;
% $v$是车辆索引;$l_{v}^t$是车辆$v$的位置;
% $\mathbf{D}_{v}$表示车辆$v$可以感知的信息集合;
% $\Phi_{v}$代表$\mathbf{D}_{v}$中信息的感知开销;
% $\mathbf{D}_{e}^{t}$ 代表$e$在时间$t$的边缘节点的缓存信息集;
% $\mathbf{D}_{\mathbf{I}_e^t}$ 代表在时间$t$的边缘节点$e$中建模的视图所需的信息集合;
% $\boldsymbol{w}^{t}$ 代表每个目标的权重向量,其在每次迭代中随机生成。
% 具体地,$\boldsymbol{w}^{t} = \begin{bmatrix}  w^{(1), t}  &  w^{(2), t} \end{bmatrix}$,其中$w^{(1), t} \in (0, 1)$和$w^{(2), t} \in (0, 1)$分别是VCPS质量和VCPS利润的权重,$\sum_{\forall i \in \{1, 2\}} w^{(j), t} = 1$。
% 另一方面,边缘节点$e$在时间$t$上对系统状态的局部观测表示为
% \begin{equation}
% 	\boldsymbol{o}_{e}^{t}=\left\{t, e, \operatorname{\mathbf{Dis}}_{\mathbf{V}, e}^{t}, \mathbf{D}_{1}, \ldots, \mathbf{D}_{v}, \ldots, \mathbf{D}_{v}, \mathbf{D}_{e}^{t}, \mathbf{D}_{\mathbf{I}_e^t}, \boldsymbol{w}^{t} \right\}
% \end{equation}
% \noindent 其中$e$是边缘节点索引,$\operatorname{\mathbf{Dis}}_{\mathbf{V}, e}^{t}$代表车辆与边缘节点$e$之间的距离集合。
% 因此,系统在时间$t$的状态可以表示为$\boldsymbol{o}^{t}=\boldsymbol{o}_{e}^{t} \cup \boldsymbol{o}_{1}^{t} \cup \ldots \cup \boldsymbol{o}_{v}^{t} \cup \ldots \cup \boldsymbol{o}_{v}^{t}$。

% 车辆$v$的动作空间表示为:
% \begin{equation}
% 	\boldsymbol{a}_{v}^{t} = \{ \mathbf{C}_v^t,  \{ \lambda_{d, v}^{t}, p_{d, v}^{t} \mid \forall d \in \mathbf{D}_{v}^t \} , \pi_v^t   \}
% \end{equation}
% 其中,$\mathbf{C}_v^t$是感知决策;$\lambda_{d, v}^{t}$和$p_{d, v}^{t}$分别是信息$d$的感知频率和上传优先级,$\pi_v^t$是车辆$v$在时间$t$的传输功率。
% 车辆基于系统状态的本地观测,并通过本地车辆策略网络得到当前的动作。
% \begin{equation}
% 	\boldsymbol{a}_{v}^{t}=\mu_{\mathbf{V}}\left(\boldsymbol{o}_{v}^{t} \mid \theta_{\mathbf{V}}^{\mu}\right)+\epsilon_{v} \mathcal{N}_{v}^{t}
% \end{equation}
% \noindent 其中,$\mathcal{N}_{v}^{t}$为探索噪音,以增加车辆动作的多样性,$\epsilon_{v}$为车辆$v$的探索常数。
% 车辆动作的集合被表示为 $\boldsymbol{a}_{\mathbf{V}}^{t} = \left\{\boldsymbol{a}_{v}^{t}\mid \forall v \in \mathbf{V}\right\}$。
% 另一方面,边缘节点$e$的动作空间表示为:
% \begin{equation}
% 	\boldsymbol{a}_{e}^{t} = \{b_{v, e}^{t} \mid \forall v \in \mathbf{V}_{e}^{t}\}
% \end{equation}
% 其中$b_{v, e}^t$是边缘节点$e$在时间$t$为车辆$v$分配的V2I带宽。
% 同样地,边缘节点$e$的动作可以由本地边缘策略网络根据系统状态以及车辆动作得到。
% \begin{equation}
% 	\boldsymbol{a}_{e}^{t}=\mu_{\mathbf{E}}\left(\boldsymbol{o}_{e}^{t},  \boldsymbol{a}_{\boldsymbol{\mathbf{V}}}^{t} \mid \theta_{\mathbf{E}}^{\mu}\right)+\epsilon_{e} \mathcal{N}_{e}^{t}
% \end{equation}
% \noindent 其中$\mathcal{N}_{e}^{t}$和$\epsilon_{e}$分别为边缘节点$e$的探索噪声和探索常数。
% 此外,车辆和边缘节点的联合动作被表示为 $\boldsymbol{a}^{t}= \left\{\boldsymbol{a}_{e}^{t}, \boldsymbol{a}_{1}^{t}, \ldots, \boldsymbol{a}_{v}^{t}, \ldots, \boldsymbol{a}_{V}^{t}\right\}$。

% 环境通过执行联合动作获得系统奖励向量,其表示为:
% 	\begin{equation}
% 	\boldsymbol{r}^{t} = \begin{bmatrix}  r^{(1)}\left(\boldsymbol{a}_{\mathbf{V}}^{t},\boldsymbol{a}_{e}^{t} \mid \boldsymbol{o}^{t}\right)  &  r^{(2)}\left(\boldsymbol{a}_{\mathbf{V}}^{t},\boldsymbol{a}_{e}^{t} \mid \boldsymbol{o}^{t}\right) \end{bmatrix} ^{T}
% 	\end{equation}
% 	\noindent 其中 $r^{(1)}\left(\boldsymbol{a}_{\mathbf{V}}^{t},\boldsymbol{a}_{e}^{t} \mid \boldsymbol{o}^{t}\right)$ 和 $r^{(2)}\left(\boldsymbol{a}_{\mathbf{V}}^{t},\boldsymbol{a}_{e}^{t} \mid \boldsymbol{o}^{t}\right)$ 分别是两个目标(即实现的VCPS质量和VCPS 利润)的奖励,可以通过下式计算:  
% 	\begin{numcases}{}
% 			r^{(1)}\left(\boldsymbol{a}_{\mathbf{V}}^{t},\boldsymbol{a}_{e}^{t} \mid \boldsymbol{o}^{t}\right)=\frac{1}{\left|\mathbf{I}_e^t\right|} \sum_{\forall i \in \mathbf{I}_e^t}\operatorname{QV}_{i} \notag \\
% 			r^{(2)}\left(\boldsymbol{a}_{\mathbf{V}}^{t},\boldsymbol{a}_{e}^{t} \mid \boldsymbol{o}^{t}\right)=\frac{1}{\left|\mathbf{I}_e^t\right|} \sum_{\forall i \in \mathbf{I}_e^t} \operatorname{PV}_{j} 
% 	\end{numcases}
% 因此,车辆$v$在第$i$个目标中的奖励可以通过基于差分奖励的信用分配方案 \cite{foerster2018counterfactual} 得到,其为系统奖励和没有其动作所取得的奖励之间的差值,其表示为:
% \begin{equation}
% r_{v}^{(j), t}=r^{(j)}\left(\boldsymbol{a}_{\mathbf{V}}^{t},\boldsymbol{a}_{e}^{t} \mid \boldsymbol{o}^{t}\right)-r^{(j)}\left(\boldsymbol{a}_{\mathbf{V}-s}^{t},\boldsymbol{a}_{e}^{t} \mid \boldsymbol{o}^{t}\right), \forall i \in \{1, 2\}
% \label{equ 4-40}
% \end{equation}
% \noindent 其中 $r^{(j)}\left(\boldsymbol{a}_{\mathbf{V}-s}^{t},\boldsymbol{a}_{e}^{t} \mid \boldsymbol{o}^{t}\right)$ 是在没有车辆$v$贡献的情况下实现的系统奖励,它可以通过设置车辆$v$的空动作集得到。
% 车辆$v$在时间$t$的奖励向量表示为$\boldsymbol{r}_{v}^{t} = \begin{bmatrix}  r_{v}^{(1), t}  &  r_{v}^{(2), t} \end{bmatrix} ^{T}$。
% 车辆的差分奖励集合表示为 $\boldsymbol{r}_{\mathbf{V}}^{t}=\{ \boldsymbol{r}_{v}^{t} \mid \forall v \in \mathbf{V}\}$。

% 另一方面,系统奖励通过最小-最大归一化进一步转化为边缘节点的归一化奖励。
% 边缘节点$e$在时间$t$的第$i$个目标中的奖励由以下方式计算:
% \begin{equation}
% 	r_{e}^{(j), t}= \frac{r^{(j)}\left(\boldsymbol{a}_{\mathbf{V}}^{t},\boldsymbol{a}_{e}^{t} \mid \boldsymbol{o}^{t}\right) - \min \limits_{\forall {\boldsymbol{a}_{e}^{t}}^{\prime}} r^{(j)}\left(\boldsymbol{a}_{\mathbf{V}}^{t}, {\boldsymbol{a}_{e}^{t}}^{\prime} \mid \boldsymbol{o}^{t}\right)} {\max \limits_{\forall {\boldsymbol{a}_{e}^{t}}^{\prime}} r^{(j)}\left(\boldsymbol{a}_{\mathbf{V}}^{t}, {\boldsymbol{a}_{e}^{t}}^{\prime} \mid \boldsymbol{o}^{t}\right) - \min \limits_{\forall {\boldsymbol{a}_{e}^{t}}^{\prime}} r^{(j)}\left(\boldsymbol{a}_{\mathbf{V}}^{t}, {\boldsymbol{a}_{e}^{t}}^{\prime} \mid \boldsymbol{o}^{t}\right)}
% \label{equ 4-41}
% \end{equation}
% \noindent 其中 $\min \limits_{\forall {\boldsymbol{a}_{e}^{t}}^{\prime}} r^{(j)} (\boldsymbol{a}_{\mathbf{V}}^{t}, {\boldsymbol{a}_{e}^{t}}^{\prime} \mid \boldsymbol{o}^{t})$ 和 $\max \limits_{\forall {\boldsymbol{a}_{e}^{t}}^{\prime}} r^{(j)}(\boldsymbol{a}_{\mathbf{V}}^{t}, {\boldsymbol{a}_{e}^{t}}^{\prime} \mid \boldsymbol{o}^{t})$ 分别是在相同的系统状态$\boldsymbol{o}^{t}$下,车辆动作$\boldsymbol{a}_{\mathbf{V}}^{t}$不变时,可实现的系统奖励最小值和最大值。
% 边缘节点$e$在时间$t$的奖励向量表示为 $\boldsymbol{r}_{e}^{t} = \begin{bmatrix}  r_{e}^{(1), t}  &  r_{e}^{(2), t} \end{bmatrix} ^{T}$。
% 交互经验包括当前系统状态$\boldsymbol{o}^{t}$、车辆动作$\boldsymbol{a}_{\mathbf{V}}^{t}$、边缘节点动作$\boldsymbol{a}_{e}^{t}$、车辆奖励$\boldsymbol{r}_{\mathbf{V}}^{t}$、边缘节点奖励$\boldsymbol{r}_{e}^{t}$、权重$\boldsymbol{w}^{t}$,以及下一时刻系统状态$\boldsymbol{o}^{t+1}$都存储到经验回放缓存$\mathcal{B}$。

% \subsection[\hspace{-2pt}多目标策略评估]{{\CJKfontspec{SimHei}\zihao{4} \hspace{-8pt}多目标策略评估}}

% 本章节阐述了针对多目标的策略评估,具体地,提出了决斗评论家网络,根据状态的价值和动作的优势来评估智能体的动作。
% 在DCN中有两个全连接的网络,即动作优势网络和状态价值网络。
% 车辆和边缘节点的AA网络参数分别表示为 $\theta_{\mathbf{V}}^{\mathscr{A}}$ 和 $\theta_{\mathbf{E}}^{\mathscr{A}}$。
% 同样,车辆和边缘节点的SV网络的参数分别表示为 $\theta_{\mathbf{V}}^{\mathscr{V}}$ 和 $\theta_{\mathbf{E}}^{\mathscr{V}}$。
% 用$A_{\mathbf{V}}\left({o}_{v}^{m},  {a}_{v}^{m}, \boldsymbol{a}_{\boldsymbol{\mathbf{V}}-v}^{m}, \boldsymbol{w}^{m} \mid \theta_{\mathbf{V}}^{\mathscr{A}} \right)$表示车辆$v$中AA网络的输出标量, 其中 $\boldsymbol{a}_{\boldsymbol{\mathbf{V}}-v}^{m}$ 表示其他车辆动作。
% 同样地,以边缘节点$e$为输入的AA网络的输出标量表示为 $A_{\mathbf{E}}\left({o}_{e}^{m},  {a}_{e}^{m}, \boldsymbol{a}_{\boldsymbol{\mathbf{V}}}^{m}, \boldsymbol{w}^{m} \mid \theta_{\mathbf{E}}^{\mathscr{A}} \right)$, 其中 $\boldsymbol{a}_{\boldsymbol{\mathbf{V}}}^{m}$ 表示所有车辆动作。
% 车辆$v$的SV网络的输出标量表示为 $V_{\mathbf{V}}\left({o}_{v}^{m}, \boldsymbol{w}^{m} \mid \theta_{\mathbf{V}}^{\mathscr{V}} \right)$。
% 同样地,边缘节点$e$的SV网络的输出标量表示为 $V_{\mathbf{E}}\left({o}_{e}^{m}, \boldsymbol{w}^{m} \mid \theta_{\mathbf{E}}^{\mathscr{V}} \right)$。

% 多目标策略评估由三个步骤组成。
% 首先,AA网络基于观测、动作和权重输出智能体动作的优势。
% 其次,VS网络根据观测和权重,输出当前状态的价值。
% 最后,采用聚合模块,根据动作优势和状态价值,输出智能体动作的价值。
% 具体来说,在AA网络中随机生成$N$个动作并将智能体动作替换,以评估当前动作对于随机动作的平均优势。
% 用${a}_{v}^{m, n}$和${a}_{e}^{m, n}$分别表示车辆$v$和边缘节点$e$的第$n$个随机动作。
% 因此,车辆$v$和边缘节点$e$的第$n$个随机动作的优势可分别表示为 $A_{\mathbf{V}}\left({o}_{v}^{m},  {a}_{v}^{m, n}, \boldsymbol{a}_{\boldsymbol{\mathbf{V}}-v}^{m}, \boldsymbol{w}^{m} \mid \theta_{v}^{\mathscr{A}} \right)$ 和 $A_{\mathbf{E}}\left({o}_{e}^{m},  {a}_{e}^{m, n}, \boldsymbol{a}_{\boldsymbol{\mathbf{V}}}^{m}, \boldsymbol{w}^{m} \mid \theta_{\mathbf{E}}^{\mathscr{A}} \right)$。

% 进一步,通过评估智能体动作相对于随机动作的平均优势,对价值函数进行聚合。
% 因此,车辆$v\in\mathbf{V}$和边缘节点$e$的动作价值是通过下式计算: 
% \begin{align}
%     Q_{\mathbf{V}}\left({o}_{v}^{m}, {a}_{v}^{m}, \boldsymbol{a}_{\boldsymbol{\mathbf{V}}-v}^{m}, \boldsymbol{w}^{m} \mid \theta_{\mathbf{V}}^{Q} \right) &= V_{\mathbf{V}}\left({o}_{v}^{m}, \boldsymbol{w}^{m} \mid \theta_{\mathbf{V}}^{\mathscr{V}} \right) + A_{\mathbf{V}}\left({o}_{v}^{m},  {a}_{v}^{m}, \boldsymbol{a}_{\boldsymbol{\mathbf{V}}-v}^{m}, \boldsymbol{w}^{m} \mid \theta_{\mathbf{V}}^{\mathscr{A}} \right) \notag \\
%     &- \frac{1}{N} \sum_{\forall n} A_{\mathbf{V}}\left({o}_{v}^{m},  {a}_{v}^{m, n}, \boldsymbol{a}_{\boldsymbol{\mathbf{V}}-v}^{m}, \boldsymbol{w}^{m} \mid \theta_{\mathbf{V}}^{\mathscr{A}} \right)
% \end{align}
% \begin{align}
%     Q_{E}\left({o}_{e}^{m},  {a}_{e}^{m}, \boldsymbol{a}_{\boldsymbol{\mathbf{V}}}^{m}, \boldsymbol{w}^{m} \mid \theta_{\mathbf{E}}^{Q} \right) &= V_{\mathbf{E}}\left({o}_{e}^{m}, \boldsymbol{w}^{m} \mid \theta_{\mathbf{E}}^{\mathscr{V}} \right) + A_{\mathbf{E}}\left({o}_{e}^{m},  {a}_{e}^{m}, \boldsymbol{a}_{\boldsymbol{\mathbf{V}}}^{m}, \boldsymbol{w}^{m} \mid \theta_{\mathbf{E}}^{\mathscr{A}} \right) \notag \\
%     &- \frac{1}{N} \sum_{\forall n} A_{\mathbf{E}}\left({o}_{e}^{m},  {a}_{e}^{m, n}, \boldsymbol{a}_{\boldsymbol{\mathbf{V}}}^{m}, \boldsymbol{w}^{m} \mid \theta_{\mathbf{E}}^{\mathscr{A}} \right)
% \end{align}
% 其中,$\theta_{\mathbf{V}}^{Q}$ 和 $\theta_{\mathbf{V}}^{Q}$ 包含相应的AA和SV网络的参数。
% \begin{align}
% 	\theta_{\mathbf{V}}^{Q} = (\theta_{\mathbf{V}}^{\mathscr{A}}, \theta_{\mathbf{V}}^{\mathscr{V}}), \theta_{\mathbf{V}}^{Q^{\prime}} = (\theta_{\mathbf{V}}^{\mathscr{A}^{\prime}}, \theta_{\mathbf{V}}^{\mathscr{V}^{\prime}}) \\
% 	\theta_{\mathbf{E}}^{Q} = (\theta_{\mathbf{E}}^{\mathscr{A}}, \theta_{\mathbf{E}}^{\mathscr{V}}), \theta_{\mathbf{E}}^{Q^{\prime}} = (\theta_{\mathbf{E}}^{\mathscr{A}^{\prime}}, \theta_{\mathbf{E}}^{\mathscr{V}^{\prime}})
% \end{align}

% \subsection[\hspace{-2pt}网络学习和更新]{{\CJKfontspec{SimHei}\zihao{4} \hspace{-8pt}网络学习和更新}}

% 从经验回放缓存$\mathcal{B}$中抽出$M$小批量,以训练车辆和边缘节点的策略和评论家网络,其中单个样本表示为 $\left(\boldsymbol{o}_{\mathbf{V}}^{m}, {o}_{e}^{m}, \boldsymbol{w}^{m}, \boldsymbol{a}_{\mathbf{V}}^{m}, {a}_{e}^{m}, \boldsymbol{r}_{\mathbf{V}}^{m}, \boldsymbol{r}_{e}^{m}, \boldsymbol{o}_{\mathbf{V}}^{m+1}, {o}_{e}^{m+1}, \boldsymbol{w}^{m+1}\right)$。
% 车辆$v$的目标值表示为:
% \begin{equation}
% 	y_{v}^{m} = \boldsymbol{r}_{v}^{m} \boldsymbol{w}^{m} +\gamma Q_{\mathbf{V}}^{\prime}\left({o}_{v}^{m+1},  {a}_{v}^{m+1}, \boldsymbol{a}_{\boldsymbol{\mathbf{V}}-v}^{m+1}, \boldsymbol{w}^{m+1} \mid \theta_{\mathbf{V}}^{Q^{\prime}} \right)
% \end{equation}
% \noindent 其中 $Q_{\mathbf{V}}^{\prime}({o}_{v}^{m+1},  {a}_{v}^{m+1}, \boldsymbol{a}_{\boldsymbol{\mathbf{V}}-v}^{m+1}, \boldsymbol{w}^{m+1} \mid \theta_{\mathbf{V}}^{Q^{\prime}})$ 是目标车辆评论家网络产生的动作价值。
% $\gamma$是折扣因子。
% $\boldsymbol{a}_{\boldsymbol{\mathbf{V}}-v}^{m+1}$ 是没有车辆$v$的下一时刻车辆动作。
% \begin{equation}
% 	\boldsymbol{a}_{\boldsymbol{\mathbf{V}}-v}^{m+1} = \{ {a}_{1}^{m+1}, \ldots, {a}_{s-1}^{m+1}, {a}_{s+1}^{m+1}, \ldots, {a}_{v}^{m+1} \}
% \end{equation}
% 而 ${a}_{v}^{m+1}$ 是目标车辆策略网络根据对下一时刻系统状态的局部观测产生的车辆$v$的下一时刻动作。
% \begin{equation}
% 	{a}_{v}^{m+1} = \mu_{\mathbf{V}}^{\prime}(\boldsymbol{o}_{v}^{m+1} \mid \theta_{\mathbf{V}}^{\mu^{\prime}})
% \end{equation}
% 类似地,边缘节点$e$的目标值表示为:
% \begin{equation}
% 	y_{e}^{m} = \boldsymbol{r}_{e}^{m} \boldsymbol{w}^{m} +\gamma Q_{\mathbf{E}}^{\prime}\left({o}_{e}^{m+1},  {a}_{e}^{m+1}, \boldsymbol{a}_{\boldsymbol{\mathbf{V}}}^{m+1}, \boldsymbol{w}^{m+1} \mid \theta_{\mathbf{E}}^{Q^{\prime}} \right)
% \end{equation}
% \noindent 其中 $Q_{\mathbf{E}}^{\prime}({o}_{e}^{m+1},  {a}_{e}^{m+1}, \boldsymbol{a}_{\boldsymbol{\mathbf{V}}}^{m+1}, \boldsymbol{w}^{m+1} \mid \theta_{\mathbf{E}}^{Q^{\prime}})$ 表示由目标边缘评论家网络产生的动作价值。
% $\boldsymbol{a}_{\boldsymbol{\mathbf{V}}}^{m+1}$ 是下一时刻车辆动作。
% ${a}_{e}^{m+1}$表示下一时刻边缘节点动作,该动作可由目标边缘策略网络根据其对下一时刻系统状态的局部观测获得,即${a}_{e}^{m+1} = \mu_{\mathbf{E}}^{\prime}(\boldsymbol{o}_{e}^{m+1}, \boldsymbol{a}_{\mathbf{V}}^{m+1} \mid \theta_{\mathbf{E}}^{\mu^{\prime}})$。

% 车辆评论家网络和边缘评论家网络的损失函数是通过分类分布的时间差分(Temporal Difference, TD)学习得到的,其表示为:
% \begin{equation}
% 	\mathcal{L}\left(\theta_{\mathbf{V}}^{Q}\right)=\frac{1}{M} \sum_{m} \frac{1}{S} \sum_{v} {Y_v^{m}}
% \end{equation}
% \begin{equation}
% 	\mathcal{L}\left(\theta_{\mathbf{E}}^{Q}\right)=\frac{1}{M} \sum_{m} {Y_e^{m}}
% \end{equation}
% \noindent 其中$Y_v^{m}$和$Y_e^{m}$分别是车辆$v$和边缘节点$e$的目标值和局部评论家网络产生的动作价值之差的平方。
% \begin{equation}
% 	\begin{aligned}
% 		Y_v^{m} &= \left(y_{v}^{m}-Q_{\mathbf{V}}\left({o}_{v}^{m},  {a}_{v}^{m}, \boldsymbol{a}_{\boldsymbol{\mathbf{V}}-v}^{m}, \boldsymbol{w}^{m} \mid \theta_{\mathbf{V}}^{Q} \right)\right)^{2} \\
% 	\end{aligned}
% \end{equation}
% \begin{equation}
% 	\begin{aligned}
% 		Y_e^{m} &=\left(y_{e}^{m}-Q_{\mathbf{E}}\left({o}_{e}^{m},  {a}_{e}^{m}, \boldsymbol{a}_{\boldsymbol{\mathbf{V}}}^{m}, \boldsymbol{w}^{m} \mid \theta_{\mathbf{V}}^{Q} \right)\right)^{2} \\
% 	\end{aligned}
% \end{equation}
% 车辆和边缘策略网络参数通过确定性的策略梯度进行更新。
% \begin{equation}
% 	\nabla_{\theta_{\mathbf{V}}^{\mu}} \mathcal{J} (\theta_{\mathbf{V}}^{\mu}) \approx \frac{1}{M} \sum_{m} \frac{1}{S} \sum_{v} P_{v}^{m} 
% \end{equation}
% \begin{equation}
% 	\nabla_{\theta_{\mathbf{E}}^{\mu}} \mathcal{J} (\theta_{\mathbf{E}}^{\mu}) \approx \frac{1}{M} \sum_{m} P_{e}^{m} 
% \end{equation}
% \noindent 其中 
% \begin{equation}
% P_{v}^{m} = \nabla_{{a}_{v}^{m}} Q_{\mathbf{V}}\left({o}_{v}^{m}, {a}_{v}^{m}, \boldsymbol{a}_{\boldsymbol{\mathbf{V}}-v}^{m}, \boldsymbol{w}^{m} \mid \theta_{v}^{Q} \right) \nabla_{\theta_{\mathbf{V}}^{\mu}} \mu_{\mathbf{V}}\left({o}_{v}^{m} \mid \theta_{\mathbf{V}}^{\mu}\right)
% \end{equation}
% \begin{equation}
% P_{e}^{m} = \nabla_{{a}_{e}^{m}} Q_{\mathbf{E}}\left({o}_{e}^{m}, {a}_{e}^{m}, \boldsymbol{a}_{\boldsymbol{\mathbf{V}}}^{m}, \boldsymbol{w}^{m} \mid \theta_{\mathbf{E}}^{Q} \right) \nabla_{\theta_{\mathbf{E}}^{\mu}} \mu_{\mathbf{E}}\left({o}_{e}^{m}, {\boldsymbol{a}}_{\boldsymbol{\mathbf{V}}}^{m} \mid \theta_{\mathbf{E}}^{\mu}\right)
% \end{equation}

% 本地策略和评论家网络参数分别以$\alpha$和$\beta$的学习率更新。
% 特别地,车辆和边缘节点定期更新目标网络的参数,即当$t \mod t_{\operatorname{tgt}} = 0$, 其中 $t_{\operatorname{tgt}}$ 是目标网络的参数更新周期。
% \begin{align}
% 	\theta_{\mathbf{V}}^{\mu^{\prime}} \leftarrow n_{\mathbf{V}} \theta_{\mathbf{V}}^{\mu}+(1-n_{\mathbf{V}}) \theta_{\mathbf{V}}^{\mu^{\prime}}, \theta_{\mathbf{V}}^{Q^{\prime}} \leftarrow n_{\mathbf{V}} \theta_{\mathbf{V}}^{Q}+(1-n_{\mathbf{V}}) \theta_{\mathbf{V}}^{Q^{\prime}}\\
% 	\theta_{\mathbf{E}}^{\mu^{\prime}} \leftarrow n_{\mathbf{E}} \theta_{\mathbf{E}}^{\mu}+(1-n_{\mathbf{E}}) \theta_{\mathbf{E}}^{\mu^{\prime}}, \theta_{\mathbf{E}}^{Q^{\prime}} \leftarrow n_{\mathbf{E}} \theta_{\mathbf{E}}^{Q}+(1-n_{\mathbf{E}})  \theta_{\mathbf{E}}^{Q^{\prime}}
% \end{align}
% \noindent 其中 $n_{\mathbf{V}} \ll 1$ 和 $n_{\mathbf{E}} \ll 1$。
% 同样,分布式行动者的策略网络参数也会定期更新,即当$t \mod t_{\operatorname{act}} = 0$,其中 $t_{\operatorname{act}}$ 是分布式行动者的策略网络的参数更新周期。
% \begin{align}
% 	\theta_{\mathbf{V}, k}^{\mu} \leftarrow \theta^{{\mu}^{\prime}}_{\mathbf{V}}, \theta_{\mathbf{V}, k}^{Q} \leftarrow \theta_{\mathbf{V}}^{Q^{\prime}}, \forall k \in \{1, 2, \ldots, K\}\\
% 	\theta_{\mathbf{E}, k}^{\mu} \leftarrow \theta_{\mathbf{E}}^{\mu^{\prime}}, \theta_{\mathbf{E}, k}^{Q} \leftarrow \theta_{\mathbf{E}}^{Q^{\prime}}, \forall k \in \{1, 2, \ldots, K\}
% \end{align}

% \section[\hspace{-2pt}实验设置与结果分析]{{\CJKfontspec{SimHei}\zihao{-3} \hspace{-8pt}实验设置与结果分析}}\label{section 4-6}

% \subsection[\hspace{-2pt}实验设置]{{\CJKfontspec{SimHei}\zihao{4} \hspace{-8pt}实验设置}}

% 本章节使用Python 3.9.13和TensorFlow 2.8.0来搭建仿真实验模型以评估所提MAMO方案的性能,其运行在配备AMD Ryzen 9 5950X 16核处理器@ 3.4 GHz,两个NVIDIA GeForce RTX 3090 GPU和64 GB内存的Ubuntu 20.04服务器上。
% 实验仿真参数设置如下:
% V2I通信范围被设定为500 m。
% 传输功率被设定为100 mW。
% AWGN和可靠性阈值分别设置为-90 dBm和0.9\cite{wang2019delay}。
% V2I通信的信道衰减增益遵循高斯分布,其均值为2,方差为0.4\cite{sadek2009distributed}。
% $\hat{\Theta_{i}}$、$\hat{\Psi_{i}}$、$\hat{\Xi_{i}}$、$\hat{\Phi_{i}}$和$\hat{\Omega_{i}}$的加权系数分别设置为0.6、0.4、0.2、0.4和0.4。

% MAMO中策略和评论家网络的架构和超参数描述如下:
% 本地策略网络是有两层隐藏层的四层全连接神经网络,其中神经元的数量分别为256和128。
% 目标策略网络的结构与本地策略网络相同。
% 本地评论家网络是四层全连接神经网络,有两层隐藏层,其中神经元的数量分别为512和256。
% 目标评论家网络的结构与本地评论家网络相同。
% 折扣率、批大小和最大经验回放缓存大小分别为0.996、256和1$\times10^{6}$。
% 策略网络和评论家网络的学习率分别为1$\times10^{-4}$和1$\times10^{-4}$。

% 进一步,本章节实现了三个比较算法,其具体细节介绍如下:
% \begin{itemize}
% 	\item \textbf{随机分配}: 随机选择动作来确定感知信息、感知频率、上传优先级、传输功率和V2I带宽分配。
% 	\item \textbf{分布式深度确定性策略梯度}\cite{barth2018distributed}: 在边缘节点实现了一个智能体,根据系统状态,集中式地确定感知信息、感知频率、上传优先级、传输功率和V2I带宽分配。VCPS质量和VCPS 利润权重分别设定为0.5和0.5。
% 	\item \textbf{多智能体分布式深度确定性策略梯度}: 其为D4PG的多智能体版本,并在车辆上分布式实现,根据对物理环境的局部观测决定感知信息、感知频率、上传优先级和传输功率,边缘节点决定V2I带宽分配。VCPS质量和VCPS 利润权重分别设为0.5和0.5。
% \end{itemize}

% 为了评估算法在视图建模质量和有效性方面的表现,本章设计了以下两个新的指标。
% \begin{itemize}
% 	\item \textbf{单位开销质量}:其定义为花费单位开销实现的VCPS质量,其计算公式为:
% 		\begin{equation}
% 			\operatorname{QPUC}=\frac{\sum_{\forall t \in \mathbf{T}} \sum_{\forall e \in \mathbf{E}} \sum_{\forall i \in \mathbf{I}_e^t} \mathrm{QV}_i}{\sum_{\forall t \in \mathbf{T}} \sum_{\forall e \in \mathbf{E}} \sum_{\forall i \in \mathbf{I}_e^t} \mathrm{CV}_i}
% 		\end{equation}
% 		其中$\mathrm{QV}_i$和$\mathrm{CV}_i$分别是视图$i$的质量和开销。
% 	\item \textbf{单位质量利润}:其定义为单位VCPS质量所实现的VCPS 利润,其计算公式为:
% 		\begin{equation}
% 		\operatorname{PPUQ}=\frac{\sum_{\forall t \in \mathbf{T}} \sum_{\forall e \in \mathbf{E}} \sum_{\forall i \in \mathbf{I}_e^t}\mathrm{PV}_i}{\sum_{\forall t \in \mathbf{T}} \sum_{\forall e \in \mathbf{E}} \sum_{\forall i \in \mathbf{I}_e^t} \mathrm{QV}_i}
% 		\end{equation}
% 		其中$\mathrm{PV}_i$和$\mathrm{CV}_i$分别是视图$i$的利润和开销。
% \end{itemize}
% QPUC越高表明它能在相同的开销下实现更高的VCPS质量,而PPUQ越高表明它能更有效地使用感知和通信资源。上述指标全面显示了算法在同时最大化VCPS质量和最小化VCPS 开销的性能。
% 本章进一步基于公式\ref{equ 4-16}、\ref{equ 4-20}、\ref{equ 4-21}和\ref{equ 4-23}设计了四个指标,分别是\textbf{平均及时性}(Average Timeliness, AT)、\textbf{平均冗余度}(Average Redundancy, AR)、\textbf{平均感知开销}(Average Sensing Cost, ASC)和\textbf{平均传输开销}(Average Transmission Cost, ATC)。 

% \subsection[\hspace{-2pt}实验结果与分析]{{\CJKfontspec{SimHei}\zihao{4} \hspace{-8pt}实验结果与分析}}

% \textbf{1) 算法收敛性:}图\ref{fig 4-3}比较了四种算法的收敛性。其中,图\ref{fig 4-3}(a)和\ref{fig 4-3}(b)分别展示了四种算法的QPUC和PPUQ表现。X轴表示迭代次数,Y轴表示达到的QPUC和PPUQ。QPUC和PPUQ越高,表明算法在VCPS质量和VCPS开销方面表现越好。MAMO在大约850次迭代后,达到了最高的QPUC(约13.6)和最高的PPUQ(约1.13)。相比之下,RA、D4PG和MAD4PG分别实现了约2.29、7.34和2.58的QPUC,并分别实现了约0.87、0.99和0.81的PPUQ。与RA、D4PG和MAD4PG相比,MAMO在QPUC方面分别实现了约494.1\%、85.5\%和428.8\%的提升,在PPUQ方面分别实现了约30.6\%、14.2\%和40.7\%的改善。值得注意的是,MAMO是唯一能够同时改善QPUC和PPUQ的方案。这显示了MAMO在同时实现QPUC和PPUQ最大化方面的优势。

% \begin{figure}[h]
%  \centering
%  \captionsetup{font={small, stretch=1.312}}\includegraphics[width=1\columnwidth]{Fig4-3-different-algorithms.pdf}
%  \bicaption[算法收敛性比较]{算法收敛性比较,其显示与RA、D4PG和MAD4PG相比,MAMO在收敛后(约850次迭代)达到了最高的QPUC和最高的PPUQ。(a)单位开销质量(b)单位质量利润}[Convergence comparison]{Convergence comparison, which shows MAMO achieves the highest QPUC and the highest PPUQ compared with RA, D4PG, and MAD4PG after convergence (around 850 iterations). (a) Quality per unit cost (b) Profit per unit quality}
%  \label{fig 4-3}
% \end{figure}

% \begin{figure}[h]
%  \centering
%  \captionsetup{font={small, stretch=1.312}}\includegraphics[width=1\columnwidth]{Fig4-4-different-networks.pdf}
%  \bicaption[隐藏层中不同神经元数量下MAMO性能比较]{隐藏层中不同神经元数量下MAMO性能比较。(a)单位开销质量(b)单位质量利润}[Performance comparison of MAMO under different numbers of neurons in the hidden layers]{Performance comparison of MAMO under different numbers of neurons in the hidden layers. (a) Quality per unit cost (b) Profit per unit quality}
%  \label{fig 4-4}
% \end{figure}

% \textbf{2) 神经元数量的影响:}
% 图\ref{fig 4-4}比较了不同神经元数量下MAMO的性能。其中,X轴表示策略网络和评论家网络的两个隐藏层的神经元数量,分别设置为[64, 32] $\sim$ [1024, 512]和[128, 64] $\sim$ [2048, 1024]。如图\ref{fig 4-4}(a)所示,当策略网络和评论家网络的隐藏层的神经元数量设置为默认值(即[256, 128]和[512, 256])时,MAMO实现了最高的VCPS质量和利润。图\ref{fig 4-4}(b)比较了其他三个指标,包括AT、ASC和ATC。AT、ASC和ATC越低,说明在信息新鲜度、感知开销和传输开销方面表现越好。可以注意到,当每个隐藏层的神经元数量为默认设置时,MAMO在最小化AT、ASC和ATC方面表现最佳。

% \begin{figure}[h]
%  \centering
%  \captionsetup{font={small, stretch=1.312}}\includegraphics[width=1\columnwidth]{Fig4-5-different-scenarios.pdf}
%  \bicaption[不同交通场景下的性能比较]{不同交通场景下的性能比较。(a)单位开销质量(b)单位质量利润(c)平均感知开销(d)平均传输开销}[Performance comparison under different traffic scenarios]{Performance comparison under different traffic scenarios. (a) Quality per unit cost (b) Profit per unit quality (c) Average sensing cost (d) Average transmission cost}
%  \label{fig 4-5}
% \end{figure}

% \textbf{3) 交通场景的影响:}
% 图\ref{fig 4-5}比较了四种算法在不同交通场景下的性能。X轴表示交通场景,不同场景在不同的时间和空间中提取了现实的车辆轨迹作为输入,分别为:1)2016年11月16日8:00至8:05,中国成都市青羊区1平方公里区域;2)同日23:00至23:05,同一区域;3)2016年11月27日8:00至8:05,中国西安碑林区1平方公里区域。图\ref{fig 4-5}(a)比较了四种算法的QPUC,MAMO在所有场景下都取得了最高的QPUC。图\ref{fig 4-5}(b)比较了四种算法的PPUQ,MAMO在所有情况下都表现最好。与RA、D4PG和MAD4PG相比,MAMO分别提高了589.0\%、106.7\%和514.8\%的QPUC,并分别提高了约41.6\%、23.6\%和45.7\%的PPUQ。图\ref{fig 4-5}(c)比较了四种算法的ASC。MAMO的ASC低于RA、D4PG和MAD4PG,说明MAMO可以实现车辆协同感知以降低感知开销。图\ref{fig 4-5}(d)比较了四种算法的ATC,在不同情况下,MAMO的ATC最低。

% \begin{figure}[h]
%  \centering
%  \captionsetup{font={small, stretch=1.312}}\includegraphics[width=1\columnwidth]{Fig4-6-different-bandwidths.pdf}
%  \bicaption[不同V2I带宽下的性能比较]{不同V2I带宽下的性能比较。(a)单位开销质量(b)单位质量利润(c)平均及时性(d)平均冗余度(e)平均感知开销(f)平均传输开销}[Performance comparison under different V2I bandwidths]{Performance comparison under different V2I bandwidths. (a) Quality per unit cost (b) Profit per unit quality (c) Average timeliness (d) Average redundancy (e) Average sensing cost (f) Average transmission cost}
%  \label{fig 4-6}
% \end{figure}

% \begin{figure}[h]
%  \centering
%  \captionsetup{font={small, stretch=1.312}}\includegraphics[width=1\columnwidth]{Fig4-7-different-numbers.pdf}
%  \bicaption[不同视图需求下的性能比较]{不同视图需求下的性能比较。(a)单位开销质量(b)单位质量利润(c)平均及时性(d)平均冗余度(e)平均感知开销(f)平均传输开销}[Performance comparison under different digit twin requirements]{Performance comparison under different digit twin requirements. (a) Quality per unit cost (b) Profit per unit quality (c) Average timeliness (d) Average redundancy (e) Average sensing cost (f) Average transmission cost}
%  \label{fig 4-7}
% \end{figure}

% \textbf{4) V2I带宽的影响:}
% 图\ref{fig 4-6}比较了四种算法在不同V2I带宽下的性能。X轴表示V2I带宽,从1MHz增加到3MHz。较大的V2I带宽代表每辆车被分配的V2I带宽也随之增加。图\ref{fig 4-6}(a)比较了四种算法的QPUC。随着带宽的增加,MAMO的QPUC也相应增加。这是因为在带宽富余的场景中,MAMO中车辆之间的协同感知和上传更加有效。图\ref{fig 4-6}(b)显示了四种算法的PPUQ,可以进一步证明这一优势。如图\ref{fig 4-6}(b)所示,MAMO在不同的V2I带宽下实现了最高的PPUQ。特别地,与RA、D4PG和MAD4PG相比,MAMO分别提高了约453.3\%、131.4\%和437.6\%的QPUC,并使PPUQ提高了约33.0\%、18.3\%和48.4\%。图\ref{fig 4-6}(c)比较了四种算法的AT,MAMO实现了最低的AT。当带宽从2.5MHz增加到3MHz时,MAMO和D4PG的性能差距很小。这是因为随着带宽的增加,视图的及时性改善是有限的。图\ref{fig 4-6}(d)比较了四种算法的AR。AR越低意味着协同感知和上传的性能越好,MAMO实现了最低的AR。图\ref{fig 4-6}(e)和\ref{fig 4-6}(f)分别比较了四种算法的ASC和ATC。可以看出,当带宽增加时,这四种算法的ATC都会下降。原因是,当带宽增加时,信息上传时间减少,导致传输开销降低。MAMO的ASC和ATC在大多数情况下保持在最低水平。

% \textbf{5) 视图需求的影响:}
% 图\ref{fig 4-7}比较了四种算法在不同视图需求下的性能,其中X轴表示视图所需信息的平均数量从3增加到7。视图所需信息的平均数越大,说明车辆的感知和上传工作负荷越大。图\ref{fig 4-7}(a)比较了四种算法的QPUC。随着平均所需信息数的增加,四种算法的QPUC也相应减少。然而,MAMO在所有情况下保持最高的QPUC。图\ref{fig 4-7}(b)比较了四种算法的PPUQ。正如预期的那样,MAMO在所有情况下都取得了最高的PPUQ。特别地,与RA、D4PG和MAD4PG相比,MAMO的QPUC分别高出458.7\%、130.6\%和426.2\%,PPUQ分别高出31.5\%、18.2\%和40.7\%。图\ref{fig 4-7}(c)比较了四种算法的AT。MAMO在AT方面取得了最佳性能。图\ref{fig 4-7}(d)比较了四种算法的AR,表明MAMO可以在所有情况下实现最低的AR。图\ref{fig 4-7}(e)和\ref{fig 4-7}(f)分别比较了四种算法的ASC和ATC。值得注意的是,当平均信息数增加时,四种算法的ASC和ATC都会增加。原因是视图需要的平均信息量增加,导致车辆感应和传输开销提高。

% \section[\hspace{-2pt}本章小结]{{\CJKfontspec{SimHei}\zihao{-3} \hspace{-8pt}本章小结}}\label{section 4-7}

% 本章提出了协同感知与V2I上传场景,其中基于车辆协同感知与V2I协同上传构建逻辑视图。
% 具体地,基于多类M/G/1优先级队列构建了协同感知模型,并基于信道衰减分布和SNR阈值构建了V2I协同上传模型。
% 在此基础上,设计了两个指标QV和CV,以衡量在边缘节点建模的视图的质量和开销,并形式化定义了双目标优化问题,通过协同感知和上传,最大化VCPS质量的同时,最小化VCPS 开销。
% 进一步,提出了基于多目标的多智能体深度强化学习算法,其中采用了决斗评论家网络,根据状态价值和动作优势来评估智能体动作。
% 最后,进行了全面的性能评估,证明了所提MAMO算法的优越性。

\chapter[\hspace{0pt}实验与验证]{{\CJKfontspec{SimHei}\zihao{3}\hspace{-5pt}实验平台与结果分析}}
\removelofgap
\removelotgap
本章将研究面向车载信息物理融合的质量-开销均衡优化。
具体内容安排如下:
\ref{section 3-1} 节是本章的引言,介绍了车联网中车载信息物理融合系统的研究现状及存在的不足,同时阐述本章的主要贡献。
% \ref{section 4-2} 节阐述了协同感知与V2I上传场景。
% \ref{section 4-3} 节给出了系统模型的详细描述。
% \ref{section 4-4} 节形式化定义了最大化VCPS质量并最小化VCPS开销的双目标优化问题。
% \ref{section 4-5} 节设计了基于多目标的多智能体深度强化学习算法。
% \ref{section 4-6} 节搭建了实验仿真模型并进行了性能验证。
% \ref{section 4-7} 节对本章的研究工作进行总结。
\section[\hspace{-2pt}实验平台和实验方案介绍]{{\CJKfontspec{SimHei}\zihao{-3} \hspace{-8pt}实验平台和实验方案介绍}}\label{section 5-1}
\subsection[\hspace{-2pt}实验平台介绍]{{\CJKfontspec{SimHei}\zihao{-3} \hspace{-8pt}实验平台介绍}}

\subsection[\hspace{-2pt}实验方案介绍]{{\CJKfontspec{SimHei}\zihao{-3} \hspace{-8pt}实验方案介绍}}

\section[\hspace{-2pt}RTL级功能级仿真]{{\CJKfontspec{SimHei}\zihao{-3} \hspace{-8pt}RTL级功能级仿真}}\label{section 5-2}
\subsection[\hspace{-2pt}串行-串行电路变换仿真]{{\CJKfontspec{SimHei}\zihao{-3} \hspace{-8pt}串型-串型电路变换仿真}}
\subsection[\hspace{-2pt}并行-并行电路变换仿真]{{\CJKfontspec{SimHei}\zihao{-3} \hspace{-8pt}并行-并行电路变换仿真}}
\subsection[\hspace{-2pt}串行-并行电路变换仿真]{{\CJKfontspec{SimHei}\zihao{-3} \hspace{-8pt}串行-并行电路变换仿真}}
\subsection[\hspace{-2pt}MDC单元仿真]{{\CJKfontspec{SimHei}\zihao{-3} \hspace{-8pt}MDC单元仿真}}
\subsection[\hspace{-2pt}控制模块仿真]{{\CJKfontspec{SimHei}\zihao{-3} \hspace{-8pt}控制模块仿真}}


\section[\hspace{-2pt}加速器性能能分析]{{\CJKfontspec{SimHei}\zihao{-3} \hspace{-8pt}加速器性能分析}}\label{section 5-3}
\subsection[\hspace{-2pt}架构层级对比]{{\CJKfontspec{SimHei}\zihao{-3} \hspace{-8pt}架构层级对比}}
\subsection[\hspace{-2pt}性能分析]{{\CJKfontspec{SimHei}\zihao{-3} \hspace{-8pt}性能分析}}
\subsection[\hspace{-2pt}资源使用情况]{{\CJKfontspec{SimHei}\zihao{-3} \hspace{-8pt}资源使用情况}}

\section[\hspace{-2pt}计算结果与误差分析]{{\CJKfontspec{SimHei}\zihao{-3} \hspace{-8pt}计算结果与误差分析}}\label{section 5-4}
\subsection[\hspace{-2pt}板级计算]{{\CJKfontspec{SimHei}\zihao{-3} \hspace{-8pt}板级计算}}
\subsection[\hspace{-2pt}误差分析]{{\CJKfontspec{SimHei}\zihao{-3} \hspace{-8pt}误差分析}}
\section[\hspace{-2pt}本章小结]{{\CJKfontspec{SimHei}\zihao{-3} \hspace{-8pt}本章小结}}\label{section 5-5}

% 新兴感知技术、无线通信和计算模式推动了现代新能源汽车和智能网联汽车的发展。
% 现代汽车中装备了各种车载感知器,以增强车辆的环境感知能力 \cite{zhu2017overview}。
% 另一方面,V2X通信\cite{chen2020a}的发展使车辆、路侧设备和云端之间的合作得以实现。
% 同时,车载边缘计算\cite{dai2021edge}是很有前途的范式,
% 可以实现计算密集型和延迟关键型的智能交通系统应用 \cite{zhao2022foundation}。
% 这些进展都成为了开发车载信息物理融合系统的强大驱动力。
% 具体来说,通过协同感知和上传,车联网中的物理实体,如车辆、行人和路侧设备等,
% 可以在边缘节点上构建为相应的逻辑映射。

% 车载信息物理融合中的检测、预测、规划和控制技术被广泛研究。
% 大量工作聚焦于检测技术,例如雨滴数量检测\cite{wang2021deep}和驾驶员疲劳检测\cite{chang2018design}。
% 针对车辆状态预测方法,研究人员提出了混合速度曲线预测\cite{zhang2019a}、车辆跟踪\cite{iepure2021a}
% 和加速预测\cite{zhang2020data}等。同时,部分研究工作提出了不同的调度方案,
% 例如基于物理比率-K干扰模型的广播调度\cite{li2020cyber}和基于既定地图模型的路径规划\cite{lian2021cyber}。
% 此外,部分研究集中在智能网联车辆的控制算法上,例如车辆加速控制\cite{lv2018driving}、交叉路口控制\cite{chang2021an}
% 和电动汽车充电调度\cite{wi2013electric}。
% 这些关于状态检测、轨迹预测、路径调度和车辆控制的研究促进了各种ITS应用的实施。
% 然而,这些工作忽略了感知和上传开销,假设高质量可用信息可以在VEC中构建。
% 少数研究考虑了VCPS中的信息质量,例如时效性\cite{liu2014temporal, dai2019temporal}和准确性\cite{rager2017scalability, yoon2021performance},
% 但上述研究都没有考虑通过协同感知和上传,在VCPS中实现高质量低成本的信息物理融合。

% 本章旨在通过车辆协同感知与上传,构建基于车载信息物理融合的逻辑视图,
% 并进一步在最大化车载信息物理融合质量和最小化视图构建开销方面寻求最佳平衡。
% 然而,实现这一目标面临着以下主要挑战。
% 首先,车联网中的信息高度动态,因此考虑感知频率、排队延迟和传输时延的协同效应,以确保信息的新鲜度和时效性是至关重要的。
% 其次,物理信息是具有时空相关性的,不同车辆在不同的时间或空间范围内感应到的信息可能存在冗余或不一致性。
% 因此,具有不同感知能力的车辆有望以分布式方式合作,以提高感知和通信资源的利用率。
% 再次,物理信息在分布、更新频率和模式方面存在异质性,这给构建高质量视图带来很大挑战。
% 最后,高质量的视图构建需要更高的感知和通信资源开销,这也是需要考虑的关键因素。
% 综上所述,通过协同感知和上传,实现面向车载边缘计算的高质量、低开销视图具有重要意义,但也具有一定的挑战性。

% 本章致力于研究车载信息物理融合系统的质量-开销均衡优化问题,并通过协同感知与上传实现高质量、低开销的视图建模。
% 本章的主要贡献如下:第一,提出了协同感知与V2I上传场景,考虑视图的及时性和一致性,
% 设计了车载信息物理融合质量指标,并考虑边缘视图构建过程中信息冗余度、感知开销和传输开销,
% 设计了车载信息物理融合开销指标。
% 进一步,提出了双目标优化问题,在最大化VCPS质量的同时最小化VCPS开销
% 第二,提出了基于多目标的多智能体深度强化学习算法。
% 具体地,在车辆和边缘节点中分别部署智能体,车辆动作空间包括感知决策、感知频率、上传优先级和传输功率分配,
% 而边缘节点动作空间是V2I带宽分配策略。
% 同时,设计了决斗评论家网络(Dueling Critic Network, DCN),
% 其根据状态价值(State-Value, SV)和动作优势(Action-Advantage, AA)评估智能体动作。
% 系统奖励是一维向量,其中包含VCPS质量和VCPS利润,并通过差分奖励信用分配得到车辆的个人奖励,
% 进一步通过最小-最大归一化得到边缘节点的归一化奖励。
% 第三,建立了基于现实世界车辆轨迹的仿真实验模型,并将MAMO与三种对比算法进行比较,
% 包括随机分配、分布式深度确定性策略梯度\cite{barth2018distributed},
% 以及多智能体分布式深度确定性策略梯度。
% 此外,本文设计了两个指标,即单位开销质量(Quality Per Unit Cost, QPUC)和单位质量利润(Profit Per Unit Quality, PPUQ)
% 用于定量衡量算法实现的均衡。
% 仿真结果表明,与其他算法相比,MAMO在最大化QPUC和PPUQ方面更具优势。

% \section[\hspace{-2pt}协同感知与 V2I 上传场景]{{\CJKfontspec{SimHei}\zihao{-3} \hspace{-8pt}协同感知与 V2I 上传场景}}\label{section 4-2}

% 本章节介绍了协同感知与V2I上传场景。如图\ref{fig 4-1}所示,车辆配备各种车载感知器,如超声波雷达、激光雷达、光学相机和毫米波雷达,可以对环境进行感知。通过车辆间协同地感知,可以获得多源信息,包括其他车辆、弱势道路参与者、停车场和路边基础设施的状态。这些信息可用于在边缘节点中建立视图模型,并进一步用于支撑各种ITS应用,如自动驾驶\cite{bai2022hybrid}、智慧路口控制系统\cite{hadjigeorgious2023real},以及全息城市交通流管理\cite{wang2023city}。逻辑视图需要融合车联网中物理实体的不同模式信息,以更好地反映实时物理车辆环境,从而提高ITS的性能。然而,构建高质量的逻辑视图可能需要更高的感知频率、更多的信息上传量以及更高的能量消耗。

% \begin{figure}[h]
% \centering
%   \captionsetup{font={small, stretch=1.312}}\includegraphics[width=1\columnwidth]{Fig4-1-architerture.pdf}
%   \bicaption[协同感知与 V2I 上传场景]{协同感知与 V2I 上传场景}[Cooperative sensing and V2I uploading scenario]{Cooperative sensing and V2I uploading scenario}
%   \label{fig 4-1}
% \end{figure} 

% 本系统的工作流程如下:首先,车辆感知并排队上传不同物理实体的实时状态。接着,边缘节点将V2I带宽分配给车辆,同时,车辆确定传输功率。物理实体的视图是基于从车辆收到的多源信息进行融合建立的。需要注意的是,在该系统中,多源信息是由车辆以不同的感知频率感应到的,因此上传时的新鲜度会不同。虽然增加感知频率可以提高新鲜度,但会增加排队延迟和能源消耗。此外,多个车辆可能感知到特定物理实体的信息,若由所有车辆上传,则可能会浪费通信资源。因此,为了提高资源利用率,需要有效而经济地分配通信资源。在此基础上,为了最大化面向车载边缘计算的视图的VCPS质量并最小化VCPS开销,必须量化衡量边缘节点构建的视图的质量和开销,并设计高效经济的协同感知和上传的调度机制。

% \section[\hspace{-2pt}车载信息物理融合质量/开销模型]{{\CJKfontspec{SimHei}\zihao{-3} \hspace{-8pt}车载信息物理融合质量/开销模型}}\label{section 4-3}

% \subsection[\hspace{-2pt}基本符号]{{\CJKfontspec{SimHei}\zihao{4} \hspace{-8pt}基本符号}}

% 本系统离散时间片的集合用$\mathbf{T}=\left\{1,\ldots,t,\ldots, T \right\}$表示。
% 多源信息集合用$\mathbf{D}$表示,其中信息$d \in \mathbf{D}$的特征是三元组$d=\left(\operatorname{type}_d, u_d, \left|d\right| \right)$,其中$\operatorname{type}_d$、$u_d$和$\left|d\right|$分别是信息类型、更新间隔和数据大小。
% $\mathbf{V}$表示车辆的集合,每个车辆$v\in \mathbf{V}$的特征是三元组$v=\left (l_v^t, \mathbf{D}_v, \pi_v \right )$,其中$l_v^t$、$\mathbf{D}_v$和$\pi_v$分别是位置、感知的信息集和传输功率。
% 对于$d \in \mathbf{D}_v$,车辆$v$的感知开销(即能耗)用$\phi_{d, v}$表示。
% 用$\mathbf{E}$表示边缘节点的集合,其中每个边缘节点$e \in \mathbf{E}$的特征是$e=\left (l_e, g_e, b_e \right)$,其中$l_{e}$、$r_{e}$和$b_{e}$分别为位置、通信范围和带宽。
% 车辆$v$与边缘节点$e$之间的距离表示为$\operatorname{dis}_{v, e}^t \triangleq \operatorname{distance} \left (l_v^t, l_e \right ), \forall v \in \mathbf{V}, \forall e \in \mathbf{E}, \forall t \in \mathbf{T}$。
% 在时间$t$内处于边缘节点$e$的通信覆盖范围内的车辆集合表示为$\mathbf{V}_e^t=\left \{v \vert \operatorname{dis}_{v, e}^t \leq g_e, \forall v \in \mathbf{V} \right \}, \mathbf{V}_e^t \subseteq \mathbf{V}$。

% 感知决策指示器表示车辆$v$在时间$t$是否感知信息$d$,其用以下方式表示:
% \begin{equation}
% 	c_{d, v}^t \in \{0, 1\}, \forall d \in \mathbf{D}_{v}, \forall v \in \mathbf{V}, \forall t \in \mathbf{T}
% 	\label{equ 4-1} 
% \end{equation}
% 那么,车辆$v$在时间$t$的感应信息集合表示为 $\mathbf{D}_v^t = \{ d | c_{d, v}^{t} = 1, \forall d \in \mathbf{D}_v \}, \mathbf{D}_v^t \subseteq \mathbf{D}_v$。
% 对于任何信息$d \in \mathbf{D}_v^t$来说,信息类型都是不同的, 即$\operatorname{type}_{d^*} \neq \operatorname{type}_{d}, \forall d^* \in \mathbf{D}_v^t \setminus \left\{ d\right \}, \forall d \in \mathbf{D}_v^t$。
% 车辆$v$在时间$t$的信息$d$的感知频率用$\lambda_{d, v}^t$表示,其需要满足车辆$v$的感应能力要求。
% \begin{equation}
% 	\lambda_{d, v}^{t} \in [\lambda_{d, v}^{\min} , \lambda_{d, v}^{\max} ], \ \forall d \in \mathbf{D}_v^t, \forall v \in \mathbf{V}, \forall t \in \mathbf{T}
% \end{equation}
% 其中$\lambda_{d, v}^{\min}$和$\lambda_{d, v}^{\max}$分别是车辆$v$中信息${d}$的最小和最大感知频率。
% 车辆$v$中的信息$d$在时间$t$的上传优先级用$p_{d, v}^t$表示,不同信息的上传优先级需各不相同。
% \begin{equation}
% 	{p}_{d^*, v}^t \neq {p}_{d, v}^t, \forall d^* \in \mathbf{D}_v^t \setminus \left\{ d\right \}, \forall d \in \mathbf{D}_v^t, \forall v \in \mathbf{V}, \forall t \in \mathbf{T}
% \end{equation}
% 其中${p}_{d^*, v}^t$是信息$d^* \in \mathbf{D}_v^t$中的上传优先级。
% 车辆$v$在时间$t$的传输功率用$\pi_{v}^t$表示,其不能超过车辆$v$的功率容量。
% \begin{equation}
% 	\pi_v^t \in \left[ 0 , \pi_v \right ], \forall v \in \mathbf{V}, \forall t \in \mathbf{T}
% \end{equation}
% 边缘节点$e$在时间$t$为车辆$v$分配的V2I带宽用$b_{v, e}^t$表示,且其需要满足:
% \begin{equation}
% 	b_{v, e}^t \in \left [0, b_e \right], \forall v \in \mathbf{V}_e^{t}, \forall e \in \mathbf{E}, \forall t \in \mathbf{T}
% 	\label{equ 4-5} 
% \end{equation}
% 边缘节点$e$分配的V2I总带宽不能超过其容量$b_e$,即${\sum_{\forall v \in \mathbf{V}_e^{t}} b_{v, e}^t} \leq b_e, \forall t \in \mathbf{T}$。

% 本系统中物理实体的集合为 $\mathbf{I}^{\prime}$,其中$i^{\prime} \in \mathbf{I}^{\prime}$表示物理实体,如车辆、行人和路侧基础设施等。
% $\mathbf{D}_{i^{\prime}}$是与实体$i^{\prime}$相关的信息集合,可以用$\mathbf{D}_{i^{\prime}}=\left\{d \mid y_{d, i^{\prime}} = 1, \forall d \in \mathbf{D} \right\}$, $\forall i^{\prime} \in \mathbf{I}^{\prime}$表示, 其中$y_{d, i^{\prime}}$是二进制数,表示信息$d$是否与实体$i^{\prime}$关联。
% $\mathbf{D}_{i^{\prime}}$的大小用$|\mathbf{D}_{i^{\prime}}|$表示。
% 每个实体可能需要多个信息,即$|\mathbf{D}_{i^{\prime}}| = \sum_{\forall d \in \mathbf{D}}y_{d, i^{\prime}} \geq 1, \forall i^{\prime} \in \mathbf{I}^{\prime}$。
% 对于每个实体$i^{\prime} \in \mathbf{I}^{\prime}$,可能有一个视图$i$在边缘节点中建模。
% 用$\mathbf{I}$表示视图的集合,用$\mathbf{I}_e^{t}$表示时间为$t$时在边缘节点$e$中建模的视图集合。
% 因此,边缘节点$e$收到且被视图$i$需要的信息集合可以用$\mathbf{D}_{i, e}^t=\bigcup_{\forall v \in \mathbf{V}}\left(\mathbf{D}_{i^{\prime}} \cap \mathbf{D}_{v, e}^t\right), \forall i \in \mathbf{I}_e^{t}, \forall e \in \mathbf{E}$表示,且 $| \mathbf{D}_{i, e}^t |$是边缘节点$e$收到且被视图$i$需要的信息数量,其计算公式为$| \mathbf{D}_{i, e}^t | =  \sum_{\forall v \in \mathbf{V}} \sum_{\forall d \in \mathbf{D}_v} c_{d, v}^t  y_{d, i^{\prime}}$。

% \subsection[\hspace{-2pt}协同感知模型]{{\CJKfontspec{SimHei}\zihao{4} \hspace{-8pt}协同感知模型}}
% 车辆协同感知是基于多类M/G/1优先级队列\cite{moltafet2020age}进行建模。
% 假设具有$\operatorname{type}_d$的信息的上传时间$\operatorname{\hat{g}}_{d, v, e}^t$遵循均值$\alpha_{d, v}^t$和方差$\beta_{d, v}^t$的一类一般分布。
% 那么,车辆$v$中的上传负载$\rho_{v}^{t}$由$ \rho_{v}^{t}=\sum_{\forall d \subseteq \mathbf{D}_v^t} \lambda_{d, v}^{t} \alpha_{d, v}^t$表示。
% 根据多类M/G/1优先级队列,需要满足$\rho_{v}^{t} < 1$才能达到队列的稳定状态。
% 信息$d$在时间$t$之前的到达时间用$\operatorname{a}_{d, v}^t$表示,其计算公式为:
% \begin{equation}
%     \operatorname{a}_{d, v}^t =  \frac{\left \lfloor t \lambda_{d, v}^t \right \rfloor }{\lambda_{d, v}^{t}} 
% \end{equation}
% 在时间$t$之前,由$\operatorname{u}_{d, v}^t$表示的信息$d$的更新时间是通过下式计算:
% \begin{equation}
%     \operatorname{u}_{d, v}^t = \left \lfloor  \frac{\operatorname{a}_{d, v}^t}{u_d} \right \rfloor  u_d
% \end{equation}
% 其中$u_d$是信息$d$的更新间隔时间。


% 在时间$t$,车辆$v$中比$d$有更高上传优先级的信息集合,用$\mathbf{D}_{d, v}^t = \{ d^* \mid p_{d^*, v}^{t} > p_{d, v}^{t} , \forall d^* \in \mathbf{D}_v^t \}$表示,其中$p_{d^*, v}^{t}$是信息$d^* \in \mathbf{D}_v^t$的上传优先级。
% 因此,信息$d$前面的上传负载(即$v$在时间$t$时要在$d$之前上传的信息数量)通过下方计算得出: 
% \begin{equation}
% 	\rho_{d, v}^{t}=\sum_{\forall d^* \in \mathbf{D}_{d, v}^t} \lambda_{d^*, v}^t \alpha_{d^*, v}^t
% \end{equation}
% 其中$\lambda_{d^*, v}^t$和$\alpha_{d^*, v}^t$分别为时间$t$内车辆$v$中信息$d^*$的感知频率和平均传输时间。
% 根据Pollaczek-Khintchine公式\cite{takine2001queue},车辆$v$中信息$d$的排队时间计算如下:
% \begin{equation}
%     \operatorname{q}_{d, v}^t= \frac{1} {1 - \rho_{d, v}^{t}} 
%         \left[ \alpha_{d, v}^t + \frac{ \lambda_{d, v}^{t} \beta_{d, v}^t + \sum\limits_{\forall d^* \in \mathbf{D}_{d, v}^t} \lambda_{d^*, v}^t \beta_{d^*, v}^t }{2\left(1-\rho_{d, v}^{t} - \lambda_{d, v}^{t} \alpha_{d, v}^t\right)}\right] 
%         - \alpha_{d, v}^t
% \end{equation}

% \subsection[\hspace{-2pt}V2I协同上传模型]{{\CJKfontspec{SimHei}\zihao{4} \hspace{-8pt}V2I协同上传模型}}

% 车辆间V2I协同上传是基于信道衰减分布和信噪比阈值来建模的。
% 车辆$v$和边缘节点$e$之间的V2I通信在时间$t$的信噪比通过公式\ref{equ 4-10}\cite{sadek2009distributed}计算得到。
% \begin{equation}
%     \operatorname{SNR}_{v, e}^{t}=\frac{1}{N_{0}} \left|h_{v, e}\right|^{2} \tau {\operatorname{dis}_{v, e}^{t}}^{-\varphi} {\pi}_v^t
%     \label{equ 4-10}
% \end{equation}
% 其中$N_{0}$为AWGN;$h_{v, e}$为信道衰减增益;$\tau$为取决于天线设计的常数;$\varphi$为路径损耗指数。
% 假设$\left|h_{v, e}\right|^{2}$遵循均值$\mu_{v, e}$和方差$\sigma_{v, e}$的一类分布,其表示方法为:
% \begin{equation}
%     \tilde{p}=\left\{\mathbb{P}: \mathbb{E}_{\mathbb{P}}\left[\left|h_{v, e}\right|^{2}\right]=\mu_{v, e}, \mathbb{E}_{\mathbb{P}}\left[\left|h_{v, e}\right|^{2}-\mu_{v, e}\right]^{2}=\sigma_{v, e}\right\}
% \end{equation}
% 进一步,基于成功传输概率和可靠性阈值来衡量V2I传输可靠性。
% \begin{equation}
%     \inf_{\mathbb{P} \in \tilde{p}} \operatorname{Pr}_{[\mathbb{P}]}\left(\operatorname{SNR}_{v, e}^{t} \geq \operatorname{SNR}_{v, e}^{\operatorname{tgt}}\right) \geq \delta
% \end{equation}
% \noindent 其中$\operatorname{SNR}_{v, e}^{\operatorname{tgt}}$和$\delta$分别为目标SNR阈值和可靠性阈值。
% 由车辆$v$上传并由边缘节点$e$接收的信息集合用$\mathbf{D}_{v, e}^{t} = \bigcup_{\forall v \in \mathbf{V}_{e}^{t}} \mathbf{D}_{v}^{t}$表示。

% 根据香农理论,车辆$v$和边缘节点$e$之间在时间$t$的V2I通信的传输率用$\operatorname{z}_{v, e}^t$表示,其计算公式如下:
% \begin{equation}
%     \operatorname{z}_{v, e}^t=b_{v}^{t} \log _{2}\left(1+\mathrm{SNR}_{v, e}^{t}\right)
% \end{equation}
% 假设车辆$v$被安排在时间$t$上传$d$,并且$d$将在一定的排队时间$\mathrm{\bar{q}}_{d, v}^t$后被传输。
% 然后,本章把车辆$v$开始传输$d$的时刻表示为$\mathrm{t}_{d, v}^t=t+\mathrm{q}_{d, v}^t$。
% 从$\mathrm{t}_{d, v}^t$到$\mathrm{t}_{d, v}^t + f$之间传输的数据量可由 $\int_{\mathrm{t}_{d, v}^t}^{\mathrm{t}_{d, v}^t+f} \mathrm{z}_{v, e}^t \mathrm{~d} t$ bits 得到,其中$f \in \mathbb{R}^{+}$和$\mathrm{z}_{i, e}^t$是时间$t$的传输速率。
% 如果在整个传输过程中可以传输的数据量大于信息$d$的大小,那么上传就会完成。
% 因此,从车辆$v$到边缘节点$e$传输信息$d$的时间,用$\operatorname{g}_{d, v, e}^t$表示,计算如下:
% \begin{equation}
%     \operatorname{g}_{d, v, e}^t=\inf _{j \in \mathbb{R}^+} \left \{ \int_{\operatorname{k}_{d, v}^t}^{\operatorname{k}_{d, v}^t + j} {\operatorname{z}_{v, e}^t} \operatorname{d}t \geq \left|d\right| \right \} 
% \end{equation}
% \noindent 其中$\operatorname{t}_{d, v}^t = t +\operatorname{q}_{d, v}^t$是车辆$v$开始传输信息$d$的时刻。

% \section[\hspace{-2pt}质量-开销均衡问题定义]{{\CJKfontspec{SimHei}\zihao{-3} \hspace{-8pt}质量-开销均衡问题定义}}\label{section 4-4}

% \subsection[\hspace{-2pt}VCPS质量]{{\CJKfontspec{SimHei}\zihao{4} \hspace{-8pt}VCPS质量}}

% 首先,由于视图是基于连续上传和时间变化的信息建模的,本章对信息$d$的及时性定义如下:
% \begin{definition}
% 信息$d$在车辆$v$中的及时性$\theta_{d, v} \in \mathbb{Q}^{+}$被定义为更新和接收信息$d$之间的时间差。
% \begin{equation}
%     \theta_{d, v} = \operatorname{a}_{d, v}^t + \operatorname{q}_{d, v}^t + \operatorname{g}_{d, v, e}^t-\operatorname{u}_{d, v}^{t}, \forall d \in \mathbf{D}_v^t,\forall v \in \mathbf{V}
% \end{equation}
% \end{definition}
% \begin{definition}
% 视图$i$的及时性 $\Theta_{i} \in \mathbb{Q}^{+}$定义为与物理实体$i^{\prime}$相关的信息的最大及时性之和。
% 	\begin{equation}
%     	\Theta_{i} = \sum_{\forall v\in \mathbf{V}_{e}^{t}} \max_{\forall d \in \mathbf{D}_{i^{\prime}} \cap \mathbf{D}_v^t}\theta_{d, v}, \forall i \in \mathbf{I}_{e}^{t}, \forall e \in \mathbf{E}
%     	\label{equ 4-16}
% 	\end{equation}
% \end{definition}

% 其次,由于不同类型的信息有不同的感知频率和上传优先级,本章定义视图的一致性来衡量与同一物理实体相关的信息的一致性。
% \begin{definition}
% 视图$i$的一致性$\Psi_{i} \in \mathbb{Q}^{+}$定义为信息更新时间差的最大值。
% \begin{equation}
%     \Psi_{i}=\max_{\forall d \in \mathbf{D}_{i, e}^{t}, \forall v \in \mathbf{V}_{e}^{t}} {\operatorname{u}_{d, v}^t} - \min_{\forall d \in \mathbf{D}_{i, e}^{t}, \forall v \in \mathbf{V}_{e}^{t}} {\operatorname{u}_{d, v}^t} , \forall i \in \mathbf{I}_{e}^{t}, \forall e \in \mathbf{E}
% \end{equation}
% \end{definition}

% 最后,本章给出了视图的质量的正式定义,其综合了视图的及时性和一致性。
% \begin{definition}
% 视图质量$\operatorname{QV}_{i} \in (0, 1)$定义为视图$i$的归一化及时性和归一化一致性的加权平均和。
% 	\begin{equation}
% 	    \operatorname{QV}_{i} = w_1 (1 -\hat{\Theta_{i}}) + w_2 (1 - \hat{\Psi_{i}}), \forall i \in \mathbf{I}_{e}^t, \forall e \in \mathbf{E}
% 	\end{equation}
% \end{definition}
% \noindent 其中$\hat{\Theta_{i}} \in (0, 1)$和$\hat{\Psi_{i}} \in (0, 1)$分别表示归一化的及时性和归一化的一致性,这可以通过最小-最大归一化对及时性和一致性的范围进行重新调整至$(0, 1)$来获得。
% $\hat{\Theta_{i}}$和$\hat{\Psi_{i}}$的加权系数分别用$w_1$和$w_2$表示,可以根据ITS应用的不同要求进行相应的调整,$w_1+w_2=1$。
% $\operatorname{QV}_{i}$的值越高,说明构建的视图质量越高。
% 考虑车辆自动驾驶系统下,视图的时效性和一致性都是关键。一方面,汽车需要实时接收到周围车辆和行人的位置、速度和行为信息,以便能够做出准确的驾驶决策。如果视图数据延迟太高或过时,车辆可能无法及时识别出潜在的危险或变化情况,从而导致事故或冲突的发生。另一方面,一致性方面的重要性表现在视图数据的一致性建模。车辆需要确保从边缘节点接收到的视图数据是准确、完整且一致的,以便能够准确地理解周围环境并做出正确的决策。如果视图数据存在不一致或缺失,车辆可能会做出错误的判断,从而导致不安全的驾驶行为或导航错误。

% 进一步,基于视图质量定义车载信息物理融合质量如下:
% \begin{definition}
% VCPS质量$\mathscr{Q} \in (0, 1)$被定义为在调度期间$\mathbf{T}$的边缘节点中建模的每个视图的QV的平均值。
% 	\begin{equation}
% 		\mathscr{Q}=\frac{\sum_{\forall t \in \mathbf{T}} \sum_{\forall e \in \mathbf{E}} \sum_{\forall i \in \mathbf{I}_e^t} \operatorname{QV}_{i}}{\sum_{\forall t \in \mathbf{T}} \sum_{\forall e \in \mathbf{E}} |\mathbf{I}_e^t| }
% 	\end{equation}
% \end{definition}

% \subsection[\hspace{-2pt}VCPS开销]{{\CJKfontspec{SimHei}\zihao{4} \hspace{-8pt}VCPS开销}}

% 首先,由于同一物理实体的状态可能被多个车辆同时感应到,本章对信息$d$的冗余度定义如下:
% \begin{definition}
% 信息$d$的冗余度$\xi_d \in \mathbb{N}$定义为车辆感应到同一类型$\operatorname{type}_d$的额外信息数量。
% \begin{equation}
%     \xi_d= \left | \mathbf{D}_{d, i, e} \right| - 1, \forall d \in \mathbf{D}_j, \forall i \in \mathbf{I}_{e}^{t}, \forall e \in \mathbf{E}
% \end{equation}
% \noindent 其中$\mathbf{D}_{d, i, e}$是边缘节点$e$收到且被视图$i$需要,且类型为$\operatorname{type}_d$的信息集合,其由$\mathbf{D}_{d, i, e}=\left\{ d^* \vert \operatorname{type}_{d^*} = \operatorname{type}_{d}, \forall d^* \in \mathbf{D}_{i, e}^t \right \}$表示。

% \end{definition}
% \begin{definition}
% 视图$i$的冗余度$\Xi_j \in \mathbb{N}$定义为视图$i$中的总冗余度。
% 	\begin{equation}
%        \Xi_j =  \sum_{\forall d \in \mathbf{D}_{i^{\prime}}} \xi_d, \forall i \in \mathbf{I}_{e}^{t}, \forall e \in \mathbf{E}
%        \label{equ 4-20}
%     \end{equation}
% \end{definition}

% 其次,信息感知和传输需要消耗车辆的能量,本章定义视图$i$的感知开销和传输开销如下:
% \begin{definition}
% 视图$i$的感知开销$\Phi_{i} \in \mathbb{Q}^{+}$定义为视图$i$所需信息的总感知开销。
% 	\begin{equation}
%         \Phi_{i} = \sum_{\forall v \in \mathbf{V}_{e}^{t}} \sum_{\forall d \in \mathbf{D}_{i^{\prime}} \cap \mathbf{D}_v^t}{\phi_{d, v}}, \forall i \in \mathbf{I}_{e}^t, \forall e \in \mathbf{E}
%         \label{equ 4-21}
%     \end{equation}
%     其中$\phi_{d, v}$是信息$d$在车辆$v$中的感知开销。
% \end{definition}
% \begin{definition}
% 信息$d$在车辆$v$中的传输开销${\omega}_{d, v} \in \mathbb{Q}^{+}$定义为信息上传时消耗的传输功率。
% \begin{equation}
%     {\omega}_{d, v}= \pi_v^t \operatorname{g}_{d, v, e}^t, \forall d \in \mathbf{D}_v^t
% \end{equation}
% 其中$\pi_v^t$和$\operatorname{g}_{d, v, e}^t$分别为传输功率和传输时间。
% \end{definition}
% \begin{definition}
% 视图$i$的传输开销$\Omega_{i} \in \mathbb{Q}^{+}$定义为视图$i$所需的信息总传输开销。
% 	\begin{equation}
%         \Omega_{i} = \sum_{\forall v \in \mathbf{V}_{e}^{t}} \sum_{\forall d \in \mathbf{D}_{i^{\prime}} \cap \mathbf{D}_v^t} {\omega}_{d, v}, \forall i \in \mathbf{I}_{e}^t, \forall e \in \mathbf{E}
%        	\label{equ 4-23}
%     \end{equation}
% \end{definition}

% 最后,给出视图开销的正式定义,其综合了冗余度、感知开销和传输开销。
% \begin{definition}
% 视图的开销$\operatorname{CV}_{i} \in (0, 1)$定义为视图$i$的归一化冗余度、归一化感知开销和归一化传输开销的加权平均和。
% 	\begin{equation}
% 	    \operatorname{CV}_{i} = w_3  \hat{\Xi_{i}} +  w_4 \hat{\Phi_{i}} + w_5 \hat{\Omega_{i}}, \forall i \in \mathbf{I}_{e}^t, \forall e \in \mathbf{E}
% 	\end{equation}
% \end{definition}
% \noindent 其中 $\hat{\Xi_{i}}\in (0, 1)$、$\hat{\Phi_{i}} \in (0, 1)$和$\hat{\Omega_{i}} \in (0, 1)$ 分别表示视图$i$的归一化冗余度、归一化感知开销和归一化传输开销。
% $\hat{\Xi_{i}}$、$\hat{\Phi_{i}}$和$\hat{\Omega_{i}}$ 的加权系数分别表示为 $w_3$、$w_4$和 $w_5$。
% 同样地,$w_3+w_4+w_5=1$。
% 进一步,VCPS开销定义如下:
% \begin{definition}
% VCPS 开销$\mathscr{C} \in (0, 1)$定义为$\mathbf{T}$调度期间边缘节点中每个视图模型的CV的平均值。
% 	\begin{equation}
% 		\mathscr{C}=\frac{\sum_{\forall t \in \mathbf{T}} \sum_{\forall e \in \mathbf{E}} \sum_{\forall i \in \mathbf{I}_e^t}  \operatorname{CV}_{i}}{\sum_{\forall t \in \mathbf{T}} \sum_{\forall e \in \mathbf{E}} |\mathbf{I}_e^t| }
% 	\end{equation}
% \end{definition}

% \subsection[\hspace{-2pt}双目标优化问题]{{\CJKfontspec{SimHei}\zihao{4} \hspace{-8pt}双目标优化问题}}

% 给定解决方案$( \mathbf{C}, \bf\Lambda, \mathbf{P}, \bf\Pi, \mathbf{B} )$,其中$\mathbf{C}$表示确定的感知信息决策,$\bf\Lambda$表示确定的感知频率。$\mathbf{P}$表示确定的上传优先级,$\bf\Pi$表示确定的传输功率,$\mathbf{B}$表示确定的V2I带宽分配,其中$c_{d, v}^t$、$\lambda_{d, v}^{t}$和$p_{d, v}^{t}$分别为时间$t$内车辆$v$的信息$d$的感知信息决策、感知频率和上传优先级,$\pi_v^t$和$b_v^t$分别为时间$t$内车辆$v$的传输功率和V2I带宽。
% \begin{numcases}{}
% 	\mathbf{C} = \left \{ c_{d, v}^t \vert \forall d \in \mathbf{D}_{v}, \forall v \in \mathbf{V}, \forall t \in \mathbf{T} \right  \} \notag \\
% 	{\bf\Lambda} = \left \{ \lambda_{d, v}^{t} \vert \forall d \in \mathbf{D}_v^t  , \forall v \in \mathbf{V}, \forall t \in \mathbf{T} \right \} \notag \\ 
% 	\mathbf{P} = \left \{ p_{d, v}^{t} \vert \forall d \in \mathbf{D}_v^t  , \forall v \in \mathbf{V}, \forall t \in \mathbf{T}\right \}  \notag \\
% 	{\bf\Pi} = \left \{ \pi_v^t \vert \forall v \in \mathbf{V}, \forall t \in \mathbf{T} \right \} \notag \\
% 	\mathbf{B} = \left \{ b_v^t \vert \forall v \in \mathbf{V}, \forall t \in \mathbf{T}\right \}
% \end{numcases}
% 本章提出了双目标优化问题,旨在同时实现VCPS质量的最大化和VCPS 开销的最小化::
% \begin{align}
% 	\mathcal{P}4.1: & \max_{\mathbf{C}, \bf\Lambda, \mathbf{P}, \bf\Pi, \mathbf{B}} \mathscr{Q}, \min_{\mathbf{C}, \bf\Lambda, \mathbf{P}, \bf\Pi, \mathbf{B}} \mathscr{C} \notag \\
% 	\text { s.t. }
% 	& (\ref{equ 4-1}) \sim (\ref{equ 4-5}) \notag \\
%     &\mathcal{C}4.1: \sum_{\forall d \subseteq \mathbf{D}_v^t} \lambda_{d, v}^{t} \mu_d<1,\ \forall v \in \mathbf{V}, \forall t \in \mathbf{T} \notag \\
%     &\mathcal{C}4.2: \inf_{\mathbb{P} \in \tilde{p}} \operatorname{Pr}_{[\mathbb{P}]}\left(\operatorname{SNR}_{v, e}^{t} \geq \operatorname{SNR}_{v, e}^{\operatorname{tgt}}\right) \geq \delta, \forall v \in \mathbf{V}, \forall t \in \mathbf{T} \notag \\
%     &\mathcal{C}4.3: {\sum_{\forall v \in \mathbf{V}_e^{t}}b_v^t} \leq b_e, \forall t \in \mathbf{T}
% \end{align}
% 其中$\mathcal{C}4.1$保证队列稳定状态,$\mathcal{C}4.2$保证传输可靠性。
% $\mathcal{C}4.3$要求边缘节点$e$分配的V2I带宽之和不能超过其容量$b_e$。
% 基于CV的定义,视图的利润定义如下:
% \begin{definition}
% 视图的利润$\operatorname{PV}_{j} \in (0, 1)$定义为视图$i$的CV的补集。
% 	\begin{equation}
% 		\mathscr{P}= 1 - \operatorname{CV}_{i}
% 	\end{equation}
% \end{definition}
% \noindent 然后,本章将VCPS 利润定义如下:
% \begin{definition}
% VCPS 利润$\mathscr{P} \in (0, 1)$被定义为在调度期$\mathbf{T}$期间,边缘节点中每个视图模型的PV的平均值。
% 	\begin{equation}
% 		\mathscr{P}= \frac{\sum_{\forall t \in \mathbf{T}} \sum_{\forall e \in \mathbf{E}} \sum_{\forall i \in \mathbf{I}_e^t}   \operatorname{PV}_{j} }{\sum_{\forall t \in \mathbf{T}} \sum_{\forall e \in \mathbf{E}} |\mathbf{I}_e^t| }
% 	\end{equation}
% \end{definition}
% \noindent 因此,$\mathcal{P}4.1$问题可以改写如下:
% \begin{align}
% 	\mathcal{P}4.2: & \max_{ \mathbf{C}, \bf\Lambda, \mathbf{P}, \bf\Pi, \mathbf{B} } \left (\mathscr{Q}, \mathscr{P} \right ) \notag \\
% 		\text { s.t. }
% 	&(\ref{equ 4-1}) \sim (\ref{equ 4-5}), \mathcal{C}4.1 \sim \mathcal{C}4.3
% \end{align}

% \section[\hspace{-2pt}基于多目标的多智能体强化学习算法设计]{{\CJKfontspec{SimHei}\zihao{-3} \hspace{-8pt}基于多目标的多智能体强化学习算法设计}}\label{section 4-5}

% 本章节提出了基于多目标的多智能体深度强化学习算法,其模型如图\ref{fig 4-2}所示,由$K$分布式行动者、学习器和经验回放缓存组成。
% 具体地,学习器由四个神经网络组成,即本地策略网络、本地评论家网络、目标策略网络和目标评论家网络。
% 其中车辆的本地策略网络、本地评论家网络、目标策略网络和目标评论家网络参数分别表示为 $\theta_{\mathbf{V}}^{\mu}$、$\theta_{\mathbf{V}}^{Q}$、 $\theta_{\mathbf{V}}^{\mu^{\prime}}$和$\theta_{\mathbf{V}}^{Q^{\prime}}$。
% 同样地,边缘节点的本地策略网络、本地评论家网络、目标策略网络和目标评论家网络参数分别表示为 $\theta_{\mathbf{E}}^{\mu}$、$\theta_{\mathbf{E}}^{Q}$、$\theta_{\mathbf{E}}^{\mu^{\prime}}$和$\theta_{\mathbf{E}}^{Q^{\prime}}$。
% 本地策略和本地评论家网络的参数是随机初始化的。
% 目标策略和目标评论家网络的参数被初始化为相应的本地网络。
% 然后,启动$K$分布式行动者,每个分布式行动者独立地与环境进行交互,并将交互经验存储到重放经验缓存。
% 分布式行动者由本地车辆策略网络和本地边缘策略网络组成,其分别用$\theta_{\mathbf{V}, k}^{\mu}$和$\theta_{\mathbf{E}, k}^{\mu}$表示,其网络参数是从学习器的本地策略网络复制而来的。
% 同时,初始化了最大存储容量为$|\mathcal{B}|$的经验回放缓存以存储重放经验。
% 基于多目标的多智能体深度强化学习的具体步骤见算法4.1,分布式行动者与环境的交互将持续到学习器的训练过程结束,其具体步骤见算法4.2。

% \begin{figure}[h]
% \centering
%   \captionsetup{font={small, stretch=1.312}}\includegraphics[width=1\columnwidth]{Fig4-2-solution-model.pdf}
%   \bicaption[基于多目标的多智能体深度强化学习模型]{基于多目标的多智能体深度强化学习模型}[Multi-agent multi-objective deep reinforcement learning model]{Multi-agent multi-objective deep reinforcement learning model}
%   \label{fig 4-2}
% \end{figure}


% \SetKwInOut{KwIn}{输入}
% \SetKwInOut{KwOut}{输出}

% \begin{algorithm}[h]\small
% \setstretch{1.245} %设置具有指定弹力的橡皮长度(原行宽的1.2倍)
% \renewcommand{\algorithmcfname}{算法}
% 	\caption{基于多目标的多智能体深度强化学习}
% 	\KwIn{折扣因子 $\gamma$、批大小 $M$、回放经验缓存 $\mathcal{B}$、学习率$\alpha$和$\beta$、目标网络参数更新周期 $t_{\operatorname{tgt}}$、分布式行动者网络参数更新周期 $t_{\operatorname{act}}$、随机动作数量$N$}
% 	\KwOut{信息感知决策$\mathbf{C}_v^t$、信息感知频率决策$\lambda_{d, v}^{t}$、上传优先级决策$p_{d, v}^{t}$、传输功率$\pi_v^t$、V2I带宽分配$b_{v, e}^{t}$}
% 	初始化网络参数\\
% 	初始化经验回放缓存 $\mathcal{B}$\\
% 	启动 $K$ 分布式行动者并复制网络参数给行动者\\
% 	\For{\songti{迭代次数} $= 1$ \songti{到最大迭代次数}}{
% 		\For{\songti{时间片} $t = 1$ \songti{到} $T$}{
% 			从经验回放缓存$\mathcal{B}$随机采样$M$小批量\\
% 			通过目标评论家网络中DCN网络得到目标值\\
% 			基于分类分布的TD学习计算更新评论家网络\\
% 			更新本地策略和评论家网络\\
% 			\If{$t \mod t_{\operatorname{tgt}} = 0$}{
% 				更新目标网络\\
% 			}
% 			\If{$t \mod t_{\operatorname{act}} = 0$}{
% 				复制网络参数给分布式行动者\\
% 			}
% 		}
% 	}
% \label{algorithm 4-1}
% \end{algorithm}

% \begin{algorithm}[t]\small
% \setstretch{1.245} %设置具有指定弹力的橡皮长度(原行宽的1.2倍)
% \renewcommand{\algorithmcfname}{算法}
% 	\caption{分布式行动者}
% 	\KwIn{车辆探索常数 $\epsilon_{v}$、边缘节点探索常数$\epsilon_{e}$、车辆本地观测 $\boldsymbol{o}_{v}^{t}$、边缘节点本地观测 $\boldsymbol{o}_{e}^{t}$}
% 	\KwOut{车辆动作$\boldsymbol{a}_{\mathbf{V}}^{t}$、边缘节点动作$\boldsymbol{a}_{e}^{t}$}
% 	\While{\songti{学习器没有结束}}{
% 		初始化随机过程 $\mathcal{N}$ 以进行探索\\
% 		生成随机权重 $\boldsymbol{w}^{t}$\\
% 		接收初始系统状态 $\boldsymbol{o}_{1}$\\
% 		\For{\songti{时间片} $t = 1$ \songti{到} $T$}{
% 			\For{\songti{车辆} $v=1$ \songti{到} $V$}{
% 				接收车辆本地观测 $\boldsymbol{o}_{v}^{t}$\\
% 				选择车辆动作 $\mu_{\mathbf{V}}\left(\boldsymbol{o}_{v}^{t} \mid \theta_{\mathbf{V}}^{\mu}\right)+\epsilon_{v} \mathcal{N}_{v}^{t}$\\
% 			}
% 			接收边缘节点本地观测 $\boldsymbol{o}_{e}^{t}$\\
% 			选择边缘节点动作 $\boldsymbol{a}_{e}^{t}=\mu_{\mathbf{E}}\left(\boldsymbol{o}_{e}^{t},  \boldsymbol{a}_{\boldsymbol{\mathbf{V}}}^{t} \mid \theta_{\mathbf{E}}^{\mu}\right)+\epsilon_{e} \mathcal{N}_{e}^{t}$\\
% 			接收奖励 $\boldsymbol{r}^{t}$ 和下一个系统状态 $\boldsymbol{o}^{t+1}$\\
% 			\For{\songti{车辆} $v=1$ \songti{到} $V$}{
% 				根据公式\ref{equ 4-40}计算车辆的差分奖励\\
% 			}
% 			根据公式\ref{equ 4-41}计算边缘节点的归一化奖励\\
% 			存储 $\left(\boldsymbol{o}^{t}, \boldsymbol{a}_{\mathbf{V}}^{t}, \boldsymbol{a}_{e}^{t}, \boldsymbol{r}_{\mathbf{V}}^{t}, \boldsymbol{r}_{e}^{t}, \boldsymbol{w}^{t}, \boldsymbol{o}^{t+1}\right)$ 到经验回放缓存 $\mathcal{B}$
% 		}
% 	}
% \label{algorithm 4-2}
% \end{algorithm}

% \subsection[\hspace{-2pt}多智能体分布式策略执行]{{\CJKfontspec{SimHei}\zihao{4} \hspace{-8pt}多智能体分布式策略执行}}

% 在MAMO中,车辆和边缘节点分别通过本地策略网络分布式地决定动作。
% 车辆$v$在时间$t$上对系统状态的局部观测表示为:
% 	\begin{equation}
% 		\boldsymbol{o}_{v}^{t}=\left\{t, v, l_{v}^t, \mathbf{D}_{v}, \Phi_{v}, \mathbf{D}_{e}^{t}, \mathbf{D}_{\mathbf{I}_e^t}, \boldsymbol{w}^{t}\right\}
% 	\end{equation} 
% \noindent 其中$t$为时间片索引;
% $v$是车辆索引;$l_{v}^t$是车辆$v$的位置;
% $\mathbf{D}_{v}$表示车辆$v$可以感知的信息集合;
% $\Phi_{v}$代表$\mathbf{D}_{v}$中信息的感知开销;
% $\mathbf{D}_{e}^{t}$ 代表$e$在时间$t$的边缘节点的缓存信息集;
% $\mathbf{D}_{\mathbf{I}_e^t}$ 代表在时间$t$的边缘节点$e$中建模的视图所需的信息集合;
% $\boldsymbol{w}^{t}$ 代表每个目标的权重向量,其在每次迭代中随机生成。
% 具体地,$\boldsymbol{w}^{t} = \begin{bmatrix}  w^{(1), t}  &  w^{(2), t} \end{bmatrix}$,其中$w^{(1), t} \in (0, 1)$和$w^{(2), t} \in (0, 1)$分别是VCPS质量和VCPS利润的权重,$\sum_{\forall i \in \{1, 2\}} w^{(j), t} = 1$。
% 另一方面,边缘节点$e$在时间$t$上对系统状态的局部观测表示为
% \begin{equation}
% 	\boldsymbol{o}_{e}^{t}=\left\{t, e, \operatorname{\mathbf{Dis}}_{\mathbf{V}, e}^{t}, \mathbf{D}_{1}, \ldots, \mathbf{D}_{v}, \ldots, \mathbf{D}_{v}, \mathbf{D}_{e}^{t}, \mathbf{D}_{\mathbf{I}_e^t}, \boldsymbol{w}^{t} \right\}
% \end{equation}
% \noindent 其中$e$是边缘节点索引,$\operatorname{\mathbf{Dis}}_{\mathbf{V}, e}^{t}$代表车辆与边缘节点$e$之间的距离集合。
% 因此,系统在时间$t$的状态可以表示为$\boldsymbol{o}^{t}=\boldsymbol{o}_{e}^{t} \cup \boldsymbol{o}_{1}^{t} \cup \ldots \cup \boldsymbol{o}_{v}^{t} \cup \ldots \cup \boldsymbol{o}_{v}^{t}$。

% 车辆$v$的动作空间表示为:
% \begin{equation}
% 	\boldsymbol{a}_{v}^{t} = \{ \mathbf{C}_v^t,  \{ \lambda_{d, v}^{t}, p_{d, v}^{t} \mid \forall d \in \mathbf{D}_{v}^t \} , \pi_v^t   \}
% \end{equation}
% 其中,$\mathbf{C}_v^t$是感知决策;$\lambda_{d, v}^{t}$和$p_{d, v}^{t}$分别是信息$d$的感知频率和上传优先级,$\pi_v^t$是车辆$v$在时间$t$的传输功率。
% 车辆基于系统状态的本地观测,并通过本地车辆策略网络得到当前的动作。
% \begin{equation}
% 	\boldsymbol{a}_{v}^{t}=\mu_{\mathbf{V}}\left(\boldsymbol{o}_{v}^{t} \mid \theta_{\mathbf{V}}^{\mu}\right)+\epsilon_{v} \mathcal{N}_{v}^{t}
% \end{equation}
% \noindent 其中,$\mathcal{N}_{v}^{t}$为探索噪音,以增加车辆动作的多样性,$\epsilon_{v}$为车辆$v$的探索常数。
% 车辆动作的集合被表示为 $\boldsymbol{a}_{\mathbf{V}}^{t} = \left\{\boldsymbol{a}_{v}^{t}\mid \forall v \in \mathbf{V}\right\}$。
% 另一方面,边缘节点$e$的动作空间表示为:
% \begin{equation}
% 	\boldsymbol{a}_{e}^{t} = \{b_{v, e}^{t} \mid \forall v \in \mathbf{V}_{e}^{t}\}
% \end{equation}
% 其中$b_{v, e}^t$是边缘节点$e$在时间$t$为车辆$v$分配的V2I带宽。
% 同样地,边缘节点$e$的动作可以由本地边缘策略网络根据系统状态以及车辆动作得到。
% \begin{equation}
% 	\boldsymbol{a}_{e}^{t}=\mu_{\mathbf{E}}\left(\boldsymbol{o}_{e}^{t},  \boldsymbol{a}_{\boldsymbol{\mathbf{V}}}^{t} \mid \theta_{\mathbf{E}}^{\mu}\right)+\epsilon_{e} \mathcal{N}_{e}^{t}
% \end{equation}
% \noindent 其中$\mathcal{N}_{e}^{t}$和$\epsilon_{e}$分别为边缘节点$e$的探索噪声和探索常数。
% 此外,车辆和边缘节点的联合动作被表示为 $\boldsymbol{a}^{t}= \left\{\boldsymbol{a}_{e}^{t}, \boldsymbol{a}_{1}^{t}, \ldots, \boldsymbol{a}_{v}^{t}, \ldots, \boldsymbol{a}_{V}^{t}\right\}$。

% 环境通过执行联合动作获得系统奖励向量,其表示为:
% 	\begin{equation}
% 	\boldsymbol{r}^{t} = \begin{bmatrix}  r^{(1)}\left(\boldsymbol{a}_{\mathbf{V}}^{t},\boldsymbol{a}_{e}^{t} \mid \boldsymbol{o}^{t}\right)  &  r^{(2)}\left(\boldsymbol{a}_{\mathbf{V}}^{t},\boldsymbol{a}_{e}^{t} \mid \boldsymbol{o}^{t}\right) \end{bmatrix} ^{T}
% 	\end{equation}
% 	\noindent 其中 $r^{(1)}\left(\boldsymbol{a}_{\mathbf{V}}^{t},\boldsymbol{a}_{e}^{t} \mid \boldsymbol{o}^{t}\right)$ 和 $r^{(2)}\left(\boldsymbol{a}_{\mathbf{V}}^{t},\boldsymbol{a}_{e}^{t} \mid \boldsymbol{o}^{t}\right)$ 分别是两个目标(即实现的VCPS质量和VCPS 利润)的奖励,可以通过下式计算:  
% 	\begin{numcases}{}
% 			r^{(1)}\left(\boldsymbol{a}_{\mathbf{V}}^{t},\boldsymbol{a}_{e}^{t} \mid \boldsymbol{o}^{t}\right)=\frac{1}{\left|\mathbf{I}_e^t\right|} \sum_{\forall i \in \mathbf{I}_e^t}\operatorname{QV}_{i} \notag \\
% 			r^{(2)}\left(\boldsymbol{a}_{\mathbf{V}}^{t},\boldsymbol{a}_{e}^{t} \mid \boldsymbol{o}^{t}\right)=\frac{1}{\left|\mathbf{I}_e^t\right|} \sum_{\forall i \in \mathbf{I}_e^t} \operatorname{PV}_{j} 
% 	\end{numcases}
% 因此,车辆$v$在第$i$个目标中的奖励可以通过基于差分奖励的信用分配方案 \cite{foerster2018counterfactual} 得到,其为系统奖励和没有其动作所取得的奖励之间的差值,其表示为:
% \begin{equation}
% r_{v}^{(j), t}=r^{(j)}\left(\boldsymbol{a}_{\mathbf{V}}^{t},\boldsymbol{a}_{e}^{t} \mid \boldsymbol{o}^{t}\right)-r^{(j)}\left(\boldsymbol{a}_{\mathbf{V}-s}^{t},\boldsymbol{a}_{e}^{t} \mid \boldsymbol{o}^{t}\right), \forall i \in \{1, 2\}
% \label{equ 4-40}
% \end{equation}
% \noindent 其中 $r^{(j)}\left(\boldsymbol{a}_{\mathbf{V}-s}^{t},\boldsymbol{a}_{e}^{t} \mid \boldsymbol{o}^{t}\right)$ 是在没有车辆$v$贡献的情况下实现的系统奖励,它可以通过设置车辆$v$的空动作集得到。
% 车辆$v$在时间$t$的奖励向量表示为$\boldsymbol{r}_{v}^{t} = \begin{bmatrix}  r_{v}^{(1), t}  &  r_{v}^{(2), t} \end{bmatrix} ^{T}$。
% 车辆的差分奖励集合表示为 $\boldsymbol{r}_{\mathbf{V}}^{t}=\{ \boldsymbol{r}_{v}^{t} \mid \forall v \in \mathbf{V}\}$。

% 另一方面,系统奖励通过最小-最大归一化进一步转化为边缘节点的归一化奖励。
% 边缘节点$e$在时间$t$的第$i$个目标中的奖励由以下方式计算:
% \begin{equation}
% 	r_{e}^{(j), t}= \frac{r^{(j)}\left(\boldsymbol{a}_{\mathbf{V}}^{t},\boldsymbol{a}_{e}^{t} \mid \boldsymbol{o}^{t}\right) - \min \limits_{\forall {\boldsymbol{a}_{e}^{t}}^{\prime}} r^{(j)}\left(\boldsymbol{a}_{\mathbf{V}}^{t}, {\boldsymbol{a}_{e}^{t}}^{\prime} \mid \boldsymbol{o}^{t}\right)} {\max \limits_{\forall {\boldsymbol{a}_{e}^{t}}^{\prime}} r^{(j)}\left(\boldsymbol{a}_{\mathbf{V}}^{t}, {\boldsymbol{a}_{e}^{t}}^{\prime} \mid \boldsymbol{o}^{t}\right) - \min \limits_{\forall {\boldsymbol{a}_{e}^{t}}^{\prime}} r^{(j)}\left(\boldsymbol{a}_{\mathbf{V}}^{t}, {\boldsymbol{a}_{e}^{t}}^{\prime} \mid \boldsymbol{o}^{t}\right)}
% \label{equ 4-41}
% \end{equation}
% \noindent 其中 $\min \limits_{\forall {\boldsymbol{a}_{e}^{t}}^{\prime}} r^{(j)} (\boldsymbol{a}_{\mathbf{V}}^{t}, {\boldsymbol{a}_{e}^{t}}^{\prime} \mid \boldsymbol{o}^{t})$ 和 $\max \limits_{\forall {\boldsymbol{a}_{e}^{t}}^{\prime}} r^{(j)}(\boldsymbol{a}_{\mathbf{V}}^{t}, {\boldsymbol{a}_{e}^{t}}^{\prime} \mid \boldsymbol{o}^{t})$ 分别是在相同的系统状态$\boldsymbol{o}^{t}$下,车辆动作$\boldsymbol{a}_{\mathbf{V}}^{t}$不变时,可实现的系统奖励最小值和最大值。
% 边缘节点$e$在时间$t$的奖励向量表示为 $\boldsymbol{r}_{e}^{t} = \begin{bmatrix}  r_{e}^{(1), t}  &  r_{e}^{(2), t} \end{bmatrix} ^{T}$。
% 交互经验包括当前系统状态$\boldsymbol{o}^{t}$、车辆动作$\boldsymbol{a}_{\mathbf{V}}^{t}$、边缘节点动作$\boldsymbol{a}_{e}^{t}$、车辆奖励$\boldsymbol{r}_{\mathbf{V}}^{t}$、边缘节点奖励$\boldsymbol{r}_{e}^{t}$、权重$\boldsymbol{w}^{t}$,以及下一时刻系统状态$\boldsymbol{o}^{t+1}$都存储到经验回放缓存$\mathcal{B}$。

% \subsection[\hspace{-2pt}多目标策略评估]{{\CJKfontspec{SimHei}\zihao{4} \hspace{-8pt}多目标策略评估}}

% 本章节阐述了针对多目标的策略评估,具体地,提出了决斗评论家网络,根据状态的价值和动作的优势来评估智能体的动作。
% 在DCN中有两个全连接的网络,即动作优势网络和状态价值网络。
% 车辆和边缘节点的AA网络参数分别表示为 $\theta_{\mathbf{V}}^{\mathscr{A}}$ 和 $\theta_{\mathbf{E}}^{\mathscr{A}}$。
% 同样,车辆和边缘节点的SV网络的参数分别表示为 $\theta_{\mathbf{V}}^{\mathscr{V}}$ 和 $\theta_{\mathbf{E}}^{\mathscr{V}}$。
% 用$A_{\mathbf{V}}\left({o}_{v}^{m},  {a}_{v}^{m}, \boldsymbol{a}_{\boldsymbol{\mathbf{V}}-v}^{m}, \boldsymbol{w}^{m} \mid \theta_{\mathbf{V}}^{\mathscr{A}} \right)$表示车辆$v$中AA网络的输出标量, 其中 $\boldsymbol{a}_{\boldsymbol{\mathbf{V}}-v}^{m}$ 表示其他车辆动作。
% 同样地,以边缘节点$e$为输入的AA网络的输出标量表示为 $A_{\mathbf{E}}\left({o}_{e}^{m},  {a}_{e}^{m}, \boldsymbol{a}_{\boldsymbol{\mathbf{V}}}^{m}, \boldsymbol{w}^{m} \mid \theta_{\mathbf{E}}^{\mathscr{A}} \right)$, 其中 $\boldsymbol{a}_{\boldsymbol{\mathbf{V}}}^{m}$ 表示所有车辆动作。
% 车辆$v$的SV网络的输出标量表示为 $V_{\mathbf{V}}\left({o}_{v}^{m}, \boldsymbol{w}^{m} \mid \theta_{\mathbf{V}}^{\mathscr{V}} \right)$。
% 同样地,边缘节点$e$的SV网络的输出标量表示为 $V_{\mathbf{E}}\left({o}_{e}^{m}, \boldsymbol{w}^{m} \mid \theta_{\mathbf{E}}^{\mathscr{V}} \right)$。

% 多目标策略评估由三个步骤组成。
% 首先,AA网络基于观测、动作和权重输出智能体动作的优势。
% 其次,VS网络根据观测和权重,输出当前状态的价值。
% 最后,采用聚合模块,根据动作优势和状态价值,输出智能体动作的价值。
% 具体来说,在AA网络中随机生成$N$个动作并将智能体动作替换,以评估当前动作对于随机动作的平均优势。
% 用${a}_{v}^{m, n}$和${a}_{e}^{m, n}$分别表示车辆$v$和边缘节点$e$的第$n$个随机动作。
% 因此,车辆$v$和边缘节点$e$的第$n$个随机动作的优势可分别表示为 $A_{\mathbf{V}}\left({o}_{v}^{m},  {a}_{v}^{m, n}, \boldsymbol{a}_{\boldsymbol{\mathbf{V}}-v}^{m}, \boldsymbol{w}^{m} \mid \theta_{v}^{\mathscr{A}} \right)$ 和 $A_{\mathbf{E}}\left({o}_{e}^{m},  {a}_{e}^{m, n}, \boldsymbol{a}_{\boldsymbol{\mathbf{V}}}^{m}, \boldsymbol{w}^{m} \mid \theta_{\mathbf{E}}^{\mathscr{A}} \right)$。

% 进一步,通过评估智能体动作相对于随机动作的平均优势,对价值函数进行聚合。
% 因此,车辆$v\in\mathbf{V}$和边缘节点$e$的动作价值是通过下式计算: 
% \begin{align}
%     Q_{\mathbf{V}}\left({o}_{v}^{m}, {a}_{v}^{m}, \boldsymbol{a}_{\boldsymbol{\mathbf{V}}-v}^{m}, \boldsymbol{w}^{m} \mid \theta_{\mathbf{V}}^{Q} \right) &= V_{\mathbf{V}}\left({o}_{v}^{m}, \boldsymbol{w}^{m} \mid \theta_{\mathbf{V}}^{\mathscr{V}} \right) + A_{\mathbf{V}}\left({o}_{v}^{m},  {a}_{v}^{m}, \boldsymbol{a}_{\boldsymbol{\mathbf{V}}-v}^{m}, \boldsymbol{w}^{m} \mid \theta_{\mathbf{V}}^{\mathscr{A}} \right) \notag \\
%     &- \frac{1}{N} \sum_{\forall n} A_{\mathbf{V}}\left({o}_{v}^{m},  {a}_{v}^{m, n}, \boldsymbol{a}_{\boldsymbol{\mathbf{V}}-v}^{m}, \boldsymbol{w}^{m} \mid \theta_{\mathbf{V}}^{\mathscr{A}} \right)
% \end{align}
% \begin{align}
%     Q_{E}\left({o}_{e}^{m},  {a}_{e}^{m}, \boldsymbol{a}_{\boldsymbol{\mathbf{V}}}^{m}, \boldsymbol{w}^{m} \mid \theta_{\mathbf{E}}^{Q} \right) &= V_{\mathbf{E}}\left({o}_{e}^{m}, \boldsymbol{w}^{m} \mid \theta_{\mathbf{E}}^{\mathscr{V}} \right) + A_{\mathbf{E}}\left({o}_{e}^{m},  {a}_{e}^{m}, \boldsymbol{a}_{\boldsymbol{\mathbf{V}}}^{m}, \boldsymbol{w}^{m} \mid \theta_{\mathbf{E}}^{\mathscr{A}} \right) \notag \\
%     &- \frac{1}{N} \sum_{\forall n} A_{\mathbf{E}}\left({o}_{e}^{m},  {a}_{e}^{m, n}, \boldsymbol{a}_{\boldsymbol{\mathbf{V}}}^{m}, \boldsymbol{w}^{m} \mid \theta_{\mathbf{E}}^{\mathscr{A}} \right)
% \end{align}
% 其中,$\theta_{\mathbf{V}}^{Q}$ 和 $\theta_{\mathbf{V}}^{Q}$ 包含相应的AA和SV网络的参数。
% \begin{align}
% 	\theta_{\mathbf{V}}^{Q} = (\theta_{\mathbf{V}}^{\mathscr{A}}, \theta_{\mathbf{V}}^{\mathscr{V}}), \theta_{\mathbf{V}}^{Q^{\prime}} = (\theta_{\mathbf{V}}^{\mathscr{A}^{\prime}}, \theta_{\mathbf{V}}^{\mathscr{V}^{\prime}}) \\
% 	\theta_{\mathbf{E}}^{Q} = (\theta_{\mathbf{E}}^{\mathscr{A}}, \theta_{\mathbf{E}}^{\mathscr{V}}), \theta_{\mathbf{E}}^{Q^{\prime}} = (\theta_{\mathbf{E}}^{\mathscr{A}^{\prime}}, \theta_{\mathbf{E}}^{\mathscr{V}^{\prime}})
% \end{align}

% \subsection[\hspace{-2pt}网络学习和更新]{{\CJKfontspec{SimHei}\zihao{4} \hspace{-8pt}网络学习和更新}}

% 从经验回放缓存$\mathcal{B}$中抽出$M$小批量,以训练车辆和边缘节点的策略和评论家网络,其中单个样本表示为 $\left(\boldsymbol{o}_{\mathbf{V}}^{m}, {o}_{e}^{m}, \boldsymbol{w}^{m}, \boldsymbol{a}_{\mathbf{V}}^{m}, {a}_{e}^{m}, \boldsymbol{r}_{\mathbf{V}}^{m}, \boldsymbol{r}_{e}^{m}, \boldsymbol{o}_{\mathbf{V}}^{m+1}, {o}_{e}^{m+1}, \boldsymbol{w}^{m+1}\right)$。
% 车辆$v$的目标值表示为:
% \begin{equation}
% 	y_{v}^{m} = \boldsymbol{r}_{v}^{m} \boldsymbol{w}^{m} +\gamma Q_{\mathbf{V}}^{\prime}\left({o}_{v}^{m+1},  {a}_{v}^{m+1}, \boldsymbol{a}_{\boldsymbol{\mathbf{V}}-v}^{m+1}, \boldsymbol{w}^{m+1} \mid \theta_{\mathbf{V}}^{Q^{\prime}} \right)
% \end{equation}
% \noindent 其中 $Q_{\mathbf{V}}^{\prime}({o}_{v}^{m+1},  {a}_{v}^{m+1}, \boldsymbol{a}_{\boldsymbol{\mathbf{V}}-v}^{m+1}, \boldsymbol{w}^{m+1} \mid \theta_{\mathbf{V}}^{Q^{\prime}})$ 是目标车辆评论家网络产生的动作价值。
% $\gamma$是折扣因子。
% $\boldsymbol{a}_{\boldsymbol{\mathbf{V}}-v}^{m+1}$ 是没有车辆$v$的下一时刻车辆动作。
% \begin{equation}
% 	\boldsymbol{a}_{\boldsymbol{\mathbf{V}}-v}^{m+1} = \{ {a}_{1}^{m+1}, \ldots, {a}_{s-1}^{m+1}, {a}_{s+1}^{m+1}, \ldots, {a}_{v}^{m+1} \}
% \end{equation}
% 而 ${a}_{v}^{m+1}$ 是目标车辆策略网络根据对下一时刻系统状态的局部观测产生的车辆$v$的下一时刻动作。
% \begin{equation}
% 	{a}_{v}^{m+1} = \mu_{\mathbf{V}}^{\prime}(\boldsymbol{o}_{v}^{m+1} \mid \theta_{\mathbf{V}}^{\mu^{\prime}})
% \end{equation}
% 类似地,边缘节点$e$的目标值表示为:
% \begin{equation}
% 	y_{e}^{m} = \boldsymbol{r}_{e}^{m} \boldsymbol{w}^{m} +\gamma Q_{\mathbf{E}}^{\prime}\left({o}_{e}^{m+1},  {a}_{e}^{m+1}, \boldsymbol{a}_{\boldsymbol{\mathbf{V}}}^{m+1}, \boldsymbol{w}^{m+1} \mid \theta_{\mathbf{E}}^{Q^{\prime}} \right)
% \end{equation}
% \noindent 其中 $Q_{\mathbf{E}}^{\prime}({o}_{e}^{m+1},  {a}_{e}^{m+1}, \boldsymbol{a}_{\boldsymbol{\mathbf{V}}}^{m+1}, \boldsymbol{w}^{m+1} \mid \theta_{\mathbf{E}}^{Q^{\prime}})$ 表示由目标边缘评论家网络产生的动作价值。
% $\boldsymbol{a}_{\boldsymbol{\mathbf{V}}}^{m+1}$ 是下一时刻车辆动作。
% ${a}_{e}^{m+1}$表示下一时刻边缘节点动作,该动作可由目标边缘策略网络根据其对下一时刻系统状态的局部观测获得,即${a}_{e}^{m+1} = \mu_{\mathbf{E}}^{\prime}(\boldsymbol{o}_{e}^{m+1}, \boldsymbol{a}_{\mathbf{V}}^{m+1} \mid \theta_{\mathbf{E}}^{\mu^{\prime}})$。

% 车辆评论家网络和边缘评论家网络的损失函数是通过分类分布的时间差分(Temporal Difference, TD)学习得到的,其表示为:
% \begin{equation}
% 	\mathcal{L}\left(\theta_{\mathbf{V}}^{Q}\right)=\frac{1}{M} \sum_{m} \frac{1}{S} \sum_{v} {Y_v^{m}}
% \end{equation}
% \begin{equation}
% 	\mathcal{L}\left(\theta_{\mathbf{E}}^{Q}\right)=\frac{1}{M} \sum_{m} {Y_e^{m}}
% \end{equation}
% \noindent 其中$Y_v^{m}$和$Y_e^{m}$分别是车辆$v$和边缘节点$e$的目标值和局部评论家网络产生的动作价值之差的平方。
% \begin{equation}
% 	\begin{aligned}
% 		Y_v^{m} &= \left(y_{v}^{m}-Q_{\mathbf{V}}\left({o}_{v}^{m},  {a}_{v}^{m}, \boldsymbol{a}_{\boldsymbol{\mathbf{V}}-v}^{m}, \boldsymbol{w}^{m} \mid \theta_{\mathbf{V}}^{Q} \right)\right)^{2} \\
% 	\end{aligned}
% \end{equation}
% \begin{equation}
% 	\begin{aligned}
% 		Y_e^{m} &=\left(y_{e}^{m}-Q_{\mathbf{E}}\left({o}_{e}^{m},  {a}_{e}^{m}, \boldsymbol{a}_{\boldsymbol{\mathbf{V}}}^{m}, \boldsymbol{w}^{m} \mid \theta_{\mathbf{V}}^{Q} \right)\right)^{2} \\
% 	\end{aligned}
% \end{equation}
% 车辆和边缘策略网络参数通过确定性的策略梯度进行更新。
% \begin{equation}
% 	\nabla_{\theta_{\mathbf{V}}^{\mu}} \mathcal{J} (\theta_{\mathbf{V}}^{\mu}) \approx \frac{1}{M} \sum_{m} \frac{1}{S} \sum_{v} P_{v}^{m} 
% \end{equation}
% \begin{equation}
% 	\nabla_{\theta_{\mathbf{E}}^{\mu}} \mathcal{J} (\theta_{\mathbf{E}}^{\mu}) \approx \frac{1}{M} \sum_{m} P_{e}^{m} 
% \end{equation}
% \noindent 其中 
% \begin{equation}
% P_{v}^{m} = \nabla_{{a}_{v}^{m}} Q_{\mathbf{V}}\left({o}_{v}^{m}, {a}_{v}^{m}, \boldsymbol{a}_{\boldsymbol{\mathbf{V}}-v}^{m}, \boldsymbol{w}^{m} \mid \theta_{v}^{Q} \right) \nabla_{\theta_{\mathbf{V}}^{\mu}} \mu_{\mathbf{V}}\left({o}_{v}^{m} \mid \theta_{\mathbf{V}}^{\mu}\right)
% \end{equation}
% \begin{equation}
% P_{e}^{m} = \nabla_{{a}_{e}^{m}} Q_{\mathbf{E}}\left({o}_{e}^{m}, {a}_{e}^{m}, \boldsymbol{a}_{\boldsymbol{\mathbf{V}}}^{m}, \boldsymbol{w}^{m} \mid \theta_{\mathbf{E}}^{Q} \right) \nabla_{\theta_{\mathbf{E}}^{\mu}} \mu_{\mathbf{E}}\left({o}_{e}^{m}, {\boldsymbol{a}}_{\boldsymbol{\mathbf{V}}}^{m} \mid \theta_{\mathbf{E}}^{\mu}\right)
% \end{equation}

% 本地策略和评论家网络参数分别以$\alpha$和$\beta$的学习率更新。
% 特别地,车辆和边缘节点定期更新目标网络的参数,即当$t \mod t_{\operatorname{tgt}} = 0$, 其中 $t_{\operatorname{tgt}}$ 是目标网络的参数更新周期。
% \begin{align}
% 	\theta_{\mathbf{V}}^{\mu^{\prime}} \leftarrow n_{\mathbf{V}} \theta_{\mathbf{V}}^{\mu}+(1-n_{\mathbf{V}}) \theta_{\mathbf{V}}^{\mu^{\prime}}, \theta_{\mathbf{V}}^{Q^{\prime}} \leftarrow n_{\mathbf{V}} \theta_{\mathbf{V}}^{Q}+(1-n_{\mathbf{V}}) \theta_{\mathbf{V}}^{Q^{\prime}}\\
% 	\theta_{\mathbf{E}}^{\mu^{\prime}} \leftarrow n_{\mathbf{E}} \theta_{\mathbf{E}}^{\mu}+(1-n_{\mathbf{E}}) \theta_{\mathbf{E}}^{\mu^{\prime}}, \theta_{\mathbf{E}}^{Q^{\prime}} \leftarrow n_{\mathbf{E}} \theta_{\mathbf{E}}^{Q}+(1-n_{\mathbf{E}})  \theta_{\mathbf{E}}^{Q^{\prime}}
% \end{align}
% \noindent 其中 $n_{\mathbf{V}} \ll 1$ 和 $n_{\mathbf{E}} \ll 1$。
% 同样,分布式行动者的策略网络参数也会定期更新,即当$t \mod t_{\operatorname{act}} = 0$,其中 $t_{\operatorname{act}}$ 是分布式行动者的策略网络的参数更新周期。
% \begin{align}
% 	\theta_{\mathbf{V}, k}^{\mu} \leftarrow \theta^{{\mu}^{\prime}}_{\mathbf{V}}, \theta_{\mathbf{V}, k}^{Q} \leftarrow \theta_{\mathbf{V}}^{Q^{\prime}}, \forall k \in \{1, 2, \ldots, K\}\\
% 	\theta_{\mathbf{E}, k}^{\mu} \leftarrow \theta_{\mathbf{E}}^{\mu^{\prime}}, \theta_{\mathbf{E}, k}^{Q} \leftarrow \theta_{\mathbf{E}}^{Q^{\prime}}, \forall k \in \{1, 2, \ldots, K\}
% \end{align}

% \section[\hspace{-2pt}实验设置与结果分析]{{\CJKfontspec{SimHei}\zihao{-3} \hspace{-8pt}实验设置与结果分析}}\label{section 4-6}

% \subsection[\hspace{-2pt}实验设置]{{\CJKfontspec{SimHei}\zihao{4} \hspace{-8pt}实验设置}}

% 本章节使用Python 3.9.13和TensorFlow 2.8.0来搭建仿真实验模型以评估所提MAMO方案的性能,其运行在配备AMD Ryzen 9 5950X 16核处理器@ 3.4 GHz,两个NVIDIA GeForce RTX 3090 GPU和64 GB内存的Ubuntu 20.04服务器上。
% 实验仿真参数设置如下:
% V2I通信范围被设定为500 m。
% 传输功率被设定为100 mW。
% AWGN和可靠性阈值分别设置为-90 dBm和0.9\cite{wang2019delay}。
% V2I通信的信道衰减增益遵循高斯分布,其均值为2,方差为0.4\cite{sadek2009distributed}。
% $\hat{\Theta_{i}}$、$\hat{\Psi_{i}}$、$\hat{\Xi_{i}}$、$\hat{\Phi_{i}}$和$\hat{\Omega_{i}}$的加权系数分别设置为0.6、0.4、0.2、0.4和0.4。

% MAMO中策略和评论家网络的架构和超参数描述如下:
% 本地策略网络是有两层隐藏层的四层全连接神经网络,其中神经元的数量分别为256和128。
% 目标策略网络的结构与本地策略网络相同。
% 本地评论家网络是四层全连接神经网络,有两层隐藏层,其中神经元的数量分别为512和256。
% 目标评论家网络的结构与本地评论家网络相同。
% 折扣率、批大小和最大经验回放缓存大小分别为0.996、256和1$\times10^{6}$。
% 策略网络和评论家网络的学习率分别为1$\times10^{-4}$和1$\times10^{-4}$。

% 进一步,本章节实现了三个比较算法,其具体细节介绍如下:
% \begin{itemize}
% 	\item \textbf{随机分配}: 随机选择动作来确定感知信息、感知频率、上传优先级、传输功率和V2I带宽分配。
% 	\item \textbf{分布式深度确定性策略梯度}\cite{barth2018distributed}: 在边缘节点实现了一个智能体,根据系统状态,集中式地确定感知信息、感知频率、上传优先级、传输功率和V2I带宽分配。VCPS质量和VCPS 利润权重分别设定为0.5和0.5。
% 	\item \textbf{多智能体分布式深度确定性策略梯度}: 其为D4PG的多智能体版本,并在车辆上分布式实现,根据对物理环境的局部观测决定感知信息、感知频率、上传优先级和传输功率,边缘节点决定V2I带宽分配。VCPS质量和VCPS 利润权重分别设为0.5和0.5。
% \end{itemize}

% 为了评估算法在视图建模质量和有效性方面的表现,本章设计了以下两个新的指标。
% \begin{itemize}
% 	\item \textbf{单位开销质量}:其定义为花费单位开销实现的VCPS质量,其计算公式为:
% 		\begin{equation}
% 			\operatorname{QPUC}=\frac{\sum_{\forall t \in \mathbf{T}} \sum_{\forall e \in \mathbf{E}} \sum_{\forall i \in \mathbf{I}_e^t} \mathrm{QV}_i}{\sum_{\forall t \in \mathbf{T}} \sum_{\forall e \in \mathbf{E}} \sum_{\forall i \in \mathbf{I}_e^t} \mathrm{CV}_i}
% 		\end{equation}
% 		其中$\mathrm{QV}_i$和$\mathrm{CV}_i$分别是视图$i$的质量和开销。
% 	\item \textbf{单位质量利润}:其定义为单位VCPS质量所实现的VCPS 利润,其计算公式为:
% 		\begin{equation}
% 		\operatorname{PPUQ}=\frac{\sum_{\forall t \in \mathbf{T}} \sum_{\forall e \in \mathbf{E}} \sum_{\forall i \in \mathbf{I}_e^t}\mathrm{PV}_i}{\sum_{\forall t \in \mathbf{T}} \sum_{\forall e \in \mathbf{E}} \sum_{\forall i \in \mathbf{I}_e^t} \mathrm{QV}_i}
% 		\end{equation}
% 		其中$\mathrm{PV}_i$和$\mathrm{CV}_i$分别是视图$i$的利润和开销。
% \end{itemize}
% QPUC越高表明它能在相同的开销下实现更高的VCPS质量,而PPUQ越高表明它能更有效地使用感知和通信资源。上述指标全面显示了算法在同时最大化VCPS质量和最小化VCPS 开销的性能。
% 本章进一步基于公式\ref{equ 4-16}、\ref{equ 4-20}、\ref{equ 4-21}和\ref{equ 4-23}设计了四个指标,分别是\textbf{平均及时性}(Average Timeliness, AT)、\textbf{平均冗余度}(Average Redundancy, AR)、\textbf{平均感知开销}(Average Sensing Cost, ASC)和\textbf{平均传输开销}(Average Transmission Cost, ATC)。 

% \subsection[\hspace{-2pt}实验结果与分析]{{\CJKfontspec{SimHei}\zihao{4} \hspace{-8pt}实验结果与分析}}

% \textbf{1) 算法收敛性:}图\ref{fig 4-3}比较了四种算法的收敛性。其中,图\ref{fig 4-3}(a)和\ref{fig 4-3}(b)分别展示了四种算法的QPUC和PPUQ表现。X轴表示迭代次数,Y轴表示达到的QPUC和PPUQ。QPUC和PPUQ越高,表明算法在VCPS质量和VCPS开销方面表现越好。MAMO在大约850次迭代后,达到了最高的QPUC(约13.6)和最高的PPUQ(约1.13)。相比之下,RA、D4PG和MAD4PG分别实现了约2.29、7.34和2.58的QPUC,并分别实现了约0.87、0.99和0.81的PPUQ。与RA、D4PG和MAD4PG相比,MAMO在QPUC方面分别实现了约494.1\%、85.5\%和428.8\%的提升,在PPUQ方面分别实现了约30.6\%、14.2\%和40.7\%的改善。值得注意的是,MAMO是唯一能够同时改善QPUC和PPUQ的方案。这显示了MAMO在同时实现QPUC和PPUQ最大化方面的优势。

% \begin{figure}[h]
%  \centering
%  \captionsetup{font={small, stretch=1.312}}\includegraphics[width=1\columnwidth]{Fig4-3-different-algorithms.pdf}
%  \bicaption[算法收敛性比较]{算法收敛性比较,其显示与RA、D4PG和MAD4PG相比,MAMO在收敛后(约850次迭代)达到了最高的QPUC和最高的PPUQ。(a)单位开销质量(b)单位质量利润}[Convergence comparison]{Convergence comparison, which shows MAMO achieves the highest QPUC and the highest PPUQ compared with RA, D4PG, and MAD4PG after convergence (around 850 iterations). (a) Quality per unit cost (b) Profit per unit quality}
%  \label{fig 4-3}
% \end{figure}

% \begin{figure}[h]
%  \centering
%  \captionsetup{font={small, stretch=1.312}}\includegraphics[width=1\columnwidth]{Fig4-4-different-networks.pdf}
%  \bicaption[隐藏层中不同神经元数量下MAMO性能比较]{隐藏层中不同神经元数量下MAMO性能比较。(a)单位开销质量(b)单位质量利润}[Performance comparison of MAMO under different numbers of neurons in the hidden layers]{Performance comparison of MAMO under different numbers of neurons in the hidden layers. (a) Quality per unit cost (b) Profit per unit quality}
%  \label{fig 4-4}
% \end{figure}

% \textbf{2) 神经元数量的影响:}
% 图\ref{fig 4-4}比较了不同神经元数量下MAMO的性能。其中,X轴表示策略网络和评论家网络的两个隐藏层的神经元数量,分别设置为[64, 32] $\sim$ [1024, 512]和[128, 64] $\sim$ [2048, 1024]。如图\ref{fig 4-4}(a)所示,当策略网络和评论家网络的隐藏层的神经元数量设置为默认值(即[256, 128]和[512, 256])时,MAMO实现了最高的VCPS质量和利润。图\ref{fig 4-4}(b)比较了其他三个指标,包括AT、ASC和ATC。AT、ASC和ATC越低,说明在信息新鲜度、感知开销和传输开销方面表现越好。可以注意到,当每个隐藏层的神经元数量为默认设置时,MAMO在最小化AT、ASC和ATC方面表现最佳。

% \begin{figure}[h]
%  \centering
%  \captionsetup{font={small, stretch=1.312}}\includegraphics[width=1\columnwidth]{Fig4-5-different-scenarios.pdf}
%  \bicaption[不同交通场景下的性能比较]{不同交通场景下的性能比较。(a)单位开销质量(b)单位质量利润(c)平均感知开销(d)平均传输开销}[Performance comparison under different traffic scenarios]{Performance comparison under different traffic scenarios. (a) Quality per unit cost (b) Profit per unit quality (c) Average sensing cost (d) Average transmission cost}
%  \label{fig 4-5}
% \end{figure}

% \textbf{3) 交通场景的影响:}
% 图\ref{fig 4-5}比较了四种算法在不同交通场景下的性能。X轴表示交通场景,不同场景在不同的时间和空间中提取了现实的车辆轨迹作为输入,分别为:1)2016年11月16日8:00至8:05,中国成都市青羊区1平方公里区域;2)同日23:00至23:05,同一区域;3)2016年11月27日8:00至8:05,中国西安碑林区1平方公里区域。图\ref{fig 4-5}(a)比较了四种算法的QPUC,MAMO在所有场景下都取得了最高的QPUC。图\ref{fig 4-5}(b)比较了四种算法的PPUQ,MAMO在所有情况下都表现最好。与RA、D4PG和MAD4PG相比,MAMO分别提高了589.0\%、106.7\%和514.8\%的QPUC,并分别提高了约41.6\%、23.6\%和45.7\%的PPUQ。图\ref{fig 4-5}(c)比较了四种算法的ASC。MAMO的ASC低于RA、D4PG和MAD4PG,说明MAMO可以实现车辆协同感知以降低感知开销。图\ref{fig 4-5}(d)比较了四种算法的ATC,在不同情况下,MAMO的ATC最低。

% \begin{figure}[h]
%  \centering
%  \captionsetup{font={small, stretch=1.312}}\includegraphics[width=1\columnwidth]{Fig4-6-different-bandwidths.pdf}
%  \bicaption[不同V2I带宽下的性能比较]{不同V2I带宽下的性能比较。(a)单位开销质量(b)单位质量利润(c)平均及时性(d)平均冗余度(e)平均感知开销(f)平均传输开销}[Performance comparison under different V2I bandwidths]{Performance comparison under different V2I bandwidths. (a) Quality per unit cost (b) Profit per unit quality (c) Average timeliness (d) Average redundancy (e) Average sensing cost (f) Average transmission cost}
%  \label{fig 4-6}
% \end{figure}

% \begin{figure}[h]
%  \centering
%  \captionsetup{font={small, stretch=1.312}}\includegraphics[width=1\columnwidth]{Fig4-7-different-numbers.pdf}
%  \bicaption[不同视图需求下的性能比较]{不同视图需求下的性能比较。(a)单位开销质量(b)单位质量利润(c)平均及时性(d)平均冗余度(e)平均感知开销(f)平均传输开销}[Performance comparison under different digit twin requirements]{Performance comparison under different digit twin requirements. (a) Quality per unit cost (b) Profit per unit quality (c) Average timeliness (d) Average redundancy (e) Average sensing cost (f) Average transmission cost}
%  \label{fig 4-7}
% \end{figure}

% \textbf{4) V2I带宽的影响:}
% 图\ref{fig 4-6}比较了四种算法在不同V2I带宽下的性能。X轴表示V2I带宽,从1MHz增加到3MHz。较大的V2I带宽代表每辆车被分配的V2I带宽也随之增加。图\ref{fig 4-6}(a)比较了四种算法的QPUC。随着带宽的增加,MAMO的QPUC也相应增加。这是因为在带宽富余的场景中,MAMO中车辆之间的协同感知和上传更加有效。图\ref{fig 4-6}(b)显示了四种算法的PPUQ,可以进一步证明这一优势。如图\ref{fig 4-6}(b)所示,MAMO在不同的V2I带宽下实现了最高的PPUQ。特别地,与RA、D4PG和MAD4PG相比,MAMO分别提高了约453.3\%、131.4\%和437.6\%的QPUC,并使PPUQ提高了约33.0\%、18.3\%和48.4\%。图\ref{fig 4-6}(c)比较了四种算法的AT,MAMO实现了最低的AT。当带宽从2.5MHz增加到3MHz时,MAMO和D4PG的性能差距很小。这是因为随着带宽的增加,视图的及时性改善是有限的。图\ref{fig 4-6}(d)比较了四种算法的AR。AR越低意味着协同感知和上传的性能越好,MAMO实现了最低的AR。图\ref{fig 4-6}(e)和\ref{fig 4-6}(f)分别比较了四种算法的ASC和ATC。可以看出,当带宽增加时,这四种算法的ATC都会下降。原因是,当带宽增加时,信息上传时间减少,导致传输开销降低。MAMO的ASC和ATC在大多数情况下保持在最低水平。

% \textbf{5) 视图需求的影响:}
% 图\ref{fig 4-7}比较了四种算法在不同视图需求下的性能,其中X轴表示视图所需信息的平均数量从3增加到7。视图所需信息的平均数越大,说明车辆的感知和上传工作负荷越大。图\ref{fig 4-7}(a)比较了四种算法的QPUC。随着平均所需信息数的增加,四种算法的QPUC也相应减少。然而,MAMO在所有情况下保持最高的QPUC。图\ref{fig 4-7}(b)比较了四种算法的PPUQ。正如预期的那样,MAMO在所有情况下都取得了最高的PPUQ。特别地,与RA、D4PG和MAD4PG相比,MAMO的QPUC分别高出458.7\%、130.6\%和426.2\%,PPUQ分别高出31.5\%、18.2\%和40.7\%。图\ref{fig 4-7}(c)比较了四种算法的AT。MAMO在AT方面取得了最佳性能。图\ref{fig 4-7}(d)比较了四种算法的AR,表明MAMO可以在所有情况下实现最低的AR。图\ref{fig 4-7}(e)和\ref{fig 4-7}(f)分别比较了四种算法的ASC和ATC。值得注意的是,当平均信息数增加时,四种算法的ASC和ATC都会增加。原因是视图需要的平均信息量增加,导致车辆感应和传输开销提高。

% \section[\hspace{-2pt}本章小结]{{\CJKfontspec{SimHei}\zihao{-3} \hspace{-8pt}本章小结}}\label{section 4-7}

% 本章提出了协同感知与V2I上传场景,其中基于车辆协同感知与V2I协同上传构建逻辑视图。
% 具体地,基于多类M/G/1优先级队列构建了协同感知模型,并基于信道衰减分布和SNR阈值构建了V2I协同上传模型。
% 在此基础上,设计了两个指标QV和CV,以衡量在边缘节点建模的视图的质量和开销,并形式化定义了双目标优化问题,通过协同感知和上传,最大化VCPS质量的同时,最小化VCPS 开销。
% 进一步,提出了基于多目标的多智能体深度强化学习算法,其中采用了决斗评论家网络,根据状态价值和动作优势来评估智能体动作。
% 最后,进行了全面的性能评估,证明了所提MAMO算法的优越性。

\backmatter %%% 后置部分(致谢、参考文献、附录等)


%% 参考文献
% 顺序编码制:cqunumerical		
% 注意:至少需要引用一篇参考文献,否则下面两行会引起编译错误。
%\bibliographystyle{cqunumerical}
\bibliographystyle{gbt7714-numerical}
\bibliography{ref/refs}


%% 附录(按ABC...分节,证明、推导、程序、个人简历等)
\appendix
\chapter[附\hskip\ccwd{}\hskip\ccwd{}录]{{\CJKfontspec{SimHei}\zihao{3}附\hskip\ccwd{}\hskip\ccwd{}录}}
\section[\hspace{-2pt}作者在攻读学位期间的论文目录]{{\CJKfontspec{SimHei}\zihao{-3} \hspace{-8pt}作者在攻读学位期间的论文目录}}

%下面是盲审标记\cs{secretize}的用法,记得去\textsf{main.tex}开启盲审开关看效果:

\circled{1}已发表论文

% \begin{enumerate}
% 	\item \textbf{\secretize{XU X}}, \secretize{LIU K}, DAI P, et al. Joint task offloading and resource optimization in NOMA-based vehicular edge computing: A game-theoretic DRL approach[J]. Journal of Systems Architecture, 2023, 134: 102780. 影响因子: 5.836(2021), 4.497(5年) (中科院SCI 2区,对应本文第三章)
% 	\item \textbf{\secretize{许新操}}, \secretize{刘凯}, 刘春晖, 等. 基于势博弈的车载边缘计算信道分配方法[J]. 电子学报, 2021,49(5): 851-860. (EI 索引,CCF T1类中文高质量科技期刊,对应本文第三章)
% 	\item \textbf{ \secretize{XU X}}, \secretize{LIU K}, XIAO K, et al. Vehicular fog computing enabled real-time collision warning via trajectory calibration[J]. Mobile Networks and Applications, 2020, 25(6): 2482-2494. 影响因子: 3.077(2021), 2.92(5年) (中科院SCI 3区,对应本文第五章)
% 	\item \secretize{LIU K}, \textbf{\secretize{XU X}}, CHEN M, et al. A hierarchical architecture for the future Internet of Vehicles[J]. IEEE Communications Magazine, 2019, 57(7): 41-47. 影响因子: 9.03(2021), 10.892(5年) (中科院SCI 1区,对应本文第二章)
% 	\item \textbf{ \secretize{XU X}}, \secretize{LIU K}, ZHANG Q, et al. Age of view: A new metric for evaluating heterogeneous information fusion in vehicular cyber-physical systems[C]. Proceedings of IEEE International Conference on Intelligent Transportation Systems (IEEE ITSC’22), Macau, China, October 8-12, 2022. (EI 索引)
% 	\item \textbf{\secretize{许新操}}, 周易, \secretize{刘凯}, 等. 车载雾计算环境中基于势博弈的分布式信道分配[C]. 第十四届中国物联网学术会议(CWSN’20), 中国敦煌, 2020/9/18-9/21.
% 	\item \textbf{\secretize{XU X}}, \secretize{LIU K}, XIAO K, et al. Design and implementation of a fog computing based collision warning system in VANETs[C]. Proceedings of IEEE International Symposium on Product Compliance Engineering-Asia (IEEE ISPCE-CN’18), Hong Kong/Shengzhen, December 5-7, 2018. (EI 索引)
% 	\item LIU C, \secretize{LIU K}, REN H, \textbf{\secretize{XU X}}, et al. RtDS: Real-time distributed strategy for multi-period task offloading in vehicular edge computing environment[J]. Neural Computing and Applications, to appear. 影响因子: 5.102(2021), 5.13(5年) (中科院SCI 2区)
% 	\item XIAO K, \secretize{LIU K}, \textbf{\secretize{XU X}}, et al. Cooperative coding and caching scheduling via binary particle swarm optimization in software defined vehicular networks[J]. Neural Computing and Applications, 2021, 33(5): 1467-1478. 影响因子: 5.102(2021), 5.13(5年) (中科院SCI 2区)
% 	\item XIAO K, \secretize{LIU K}, \textbf{\secretize{XU X}}, et al. Efficient fog-assisted heterogeneous data services in software defined VANETs[J]. Journal of Ambient Intelligence and Humanized Computing, 2021, 12(1): 261-273. 影响因子: 3.662 (2021), 3.718 (5年) (中科院SCI 3区)
% 	\item LIU C, \secretize{LIU K}, \textbf{\secretize{XU X}}, et al. Real-time task offloading for data and computation intensive services in vehicular fog computing environments[C]. Proceedings of IEEE International Conference on Mobility, Sensing and Networking (IEEE MSN’20), Tokyo, Japan, December 17-19, 2020. (EI 索引,CCF C类国际会议)
% 	\item ZHOU Y, \secretize{LIU K}, \textbf{ \secretize{XU X}}, et al. Multi-period distributed delay-sensitive tasks offloading in a two-layer vehicular fog computing architecture[C]. Proceedings of International Conference on Neural Computing and Applications (NCAA’20), Shenzhen, China, July 3-6, 2020. (EI 索引)
% 	\item ZHOU Y, \secretize{LIU K}, \textbf{ \secretize{XU X}}, et al. Distributed scheduling for time-critical tasks in a two-layer vehicular fog computing architecture[C]. Proceedings of IEEE Consumer Communications and Networking Conference (IEEE CCNC’20), Las Vegas, USA, January 11-14, 2020. (EI 索引)
% \end{enumerate}

\circled{2}已投稿论文

% \begin{enumerate}
% 	\item \textbf{\secretize{XU X}}, \secretize{LIU K}, DAI P, et al. Cooperative sensing and heterogeneous information fusion in VCPS: A multi-agent deep reinforcement learning approach[J]. IEEE Transactions on Intelligent Transportation Systems, under major revision. 影响因子: 9.551 (2021), 9.502 (5年) (中科院SCI 1区,对应本文第二章)
% 	\item \secretize{LIU K},\textbf{\secretize{XU X}}, DAI P, et al. Cooperative sensing and uploading for quality-cost tradeoff of digital twins in VEC[J]. IEEE Transactions on Consumer Electronics, under minor revision. 影响因子: 4.414 (2021), 3.565 (5年) (中科院SCI 2区,对应本文第四章) 
% \end{enumerate}

\section[\hspace{-2pt}作者在攻读学位期间取得的专利]{{\CJKfontspec{SimHei}\zihao{-3} \hspace{-8pt}作者在攻读学位期间取得的专利}}
% \begin{enumerate}
% 	\item \textbf{\secretize{许新操}}, \secretize{刘凯}, 李东. 一种针对软件定义车联网的控制平面视图构建方法. 发明专利. ZL202110591822.1.
% 	\item \secretize{刘凯}, 张浪, \textbf{\secretize{许新操}}, 任华玲, 周易. 一种基于边缘计算的盲区车辆碰撞预警方法. 发明专利. ZL201910418745.2.
% 	\item 任华玲, \secretize{刘凯}, 陈梦良, 周易, \textbf{\secretize{许新操}}. 一种基于雾计算的信息采集、计算、传输架构. 发明专利. ZL201910146357.3.
% \end{enumerate}

\section[\hspace{-2pt}作者在攻读学位期间参与的科研项目目录]{{\CJKfontspec{SimHei}\zihao{-3} \hspace{-8pt}作者在攻读学位期间参与的科研项目目录}}
% \begin{enumerate}
% 	\item 国家自然科学基金面上项目,面向车联网边缘智能的计算模型部署与协同跨域优化,项目编号: 62172064,2022/01–2025/12.(项目参与人员)
% 	\item 国家自然科学基金面上项目,面向大规模数据服务的异构融合车联网架构与协议研究,项目编号: 61872049,2019/01–2022/12.(项目参与人员)
% \end{enumerate}

% \section[\hspace{-2pt}学位论文相关代码]{{\CJKfontspec{SimHei}\zihao{-3} \hspace{-8pt}学位论文相关代码}}
% \begin{enumerate}
% 	\item 基于差分奖励的多智能体深度强化学习源代码\\https://github.com/neardws/Multi-Agent-Deep-Reinforcement-Learning
% 	\item 基于博弈理论的多智能体深度强化学习源代码\\https://github.com/neardws/Game-Theoretic-Deep-Reinforcement-Learning
% 	\item 基于多目标的多智能体深度强化学习源代码\\https://github.com/neardws/MAMO-Deep-Reinforcement-Learning
% 	\item 基于车载信息物理融合系统优化的碰撞预警源代码\\https://github.com/neardws/fog-computing-based-collision-warning-system
% 	\item 基于C-V2X通信的碰撞预警原型系统源代码\\https://github.com/neardws/V2X-based-Collision-Warning
% 	\item 基于DSRC通信的碰撞预警原型系统源代码\\https://github.com/cqu-bdsc/Collision-Warning-System
% 	\item 滴滴 GAIA 数据集处理源代码\\https://github.com/neardws/Vehicular-Trajectories-Processing-for-Didi-Open-Data
% \end{enumerate}

\newpage
\section[\hspace{-2pt}学位论文数据集]{{\CJKfontspec{SimHei}\zihao{-3} \hspace{-8pt}学位论文数据集}}

\begin{table}[h]
\resizebox{\columnwidth}{!}{%
\begin{tabular}{|cllcclclcl|}
\hline
\multicolumn{4}{|c|}{\heiti{关键词}}             & \multicolumn{2}{c|}{\heiti{密级}}   & \multicolumn{4}{c|}{\heiti{中图分类号}}                                    \\ \hline
\multicolumn{4}{|c|}{\begin{tabular}[c]{@{}c@{}}车载信息物理融合系统;\\异构车联网; 车载边缘计算;\\资源优化; 多智能体深度强化学习\end{tabular}} & \multicolumn{2}{c|}{公开} & \multicolumn{4}{c|}{TP} \\ \hline
\multicolumn{3}{|c|}{\heiti{学位授予单位名称}} & \multicolumn{3}{c|}{\heiti{学位授予单位代码}}    & \multicolumn{2}{c|}{\heiti{学位类别}}  & \multicolumn{2}{c|}{\heiti{学位级别}}        \\ \hline
\multicolumn{3}{|c|}{\secretize{重庆大学}}     & \multicolumn{3}{c|}{\secretize{10611}}       & \multicolumn{2}{c|}{学术学位}  & \multicolumn{2}{c|}{博士}          \\ \hline
\multicolumn{4}{|c|}{\heiti{论文题名}}            & \multicolumn{2}{c|}{\heiti{并列题名}} & \multicolumn{4}{c|}{\heiti{论文语种}}                                     \\ \hline
\multicolumn{4}{|c|}{\begin{tabular}[c]{@{}c@{}}车载信息物理融合系统建模与优化关键技术研究\end{tabular}}               & \multicolumn{2}{c|}{无}   & \multicolumn{4}{c|}{中文} \\ \hline
\multicolumn{3}{|c|}{\heiti{作者姓名}}     & \multicolumn{3}{c|}{\secretize{许新操}}         & \multicolumn{2}{c|}{\heiti{学号}}    & \multicolumn{2}{c|}{\secretize{20191401452}} \\ \hline
\multicolumn{6}{|c|}{\heiti{培养单位名称}}                                      & \multicolumn{4}{c|}{\heiti{培养单位代码}}                                   \\ \hline
\multicolumn{6}{|c|}{\secretize{重庆大学}}                                        & \multicolumn{4}{c|}{\secretize{10611}}                                    \\ \hline
\multicolumn{3}{|c|}{\heiti{学科专业}}     & \multicolumn{3}{c|}{\heiti{研究方向}}        & \multicolumn{2}{c|}{\heiti{学制}}    & \multicolumn{2}{c|}{\heiti{学位授予年}}       \\ \hline
\multicolumn{3}{|c|}{计算机科学与技术} & \multicolumn{3}{c|}{车联网}         & \multicolumn{2}{c|}{4年}     & \multicolumn{2}{c|}{\secretize{2023年}}        \\ \hline
\multicolumn{3}{|c|}{\heiti{论文提交日期}}   & \multicolumn{3}{c|}{\secretize{2023年6月}}     & \multicolumn{2}{c|}{\heiti{论文总页数}} & \multicolumn{2}{c|}{\pageref{LastPage}}         \\ \hline
\multicolumn{3}{|c|}{\heiti{导师姓名}}     & \multicolumn{3}{c|}{\secretize{刘凯}}          & \multicolumn{2}{c|}{\heiti{职称}}    & \multicolumn{2}{c|}{教授}          \\ \hline
\multicolumn{6}{|c|}{\heiti{答辩委员会主席}}                                     & \multicolumn{4}{c|}{\secretize{雒江涛}}                                      \\ \hline
\multicolumn{10}{|c|}{\heiti{\begin{tabular}[c]{@{}c@{}} 电子版论文提交格式\\ 文本(\checkmark) 图像() 视频()音频()多媒体()其他()\end{tabular}}}                              \\ \hline
\end{tabular}%
}
\end{table}


%% 致谢
\chapter[致\hskip\ccwd{}\hskip\ccwd{}谢]{{\CJKfontspec{SimHei}\zihao{3}致\hskip\ccwd{}\hskip\ccwd{}谢}}

% 这里用盲审环境包裹致谢,在开启盲审开关时,环境内部的内容不予渲染。
\begin{secretizeEnv}

% 提笔之际,已到了在重庆生活学习的第六个年头,而我的博士研究生学习阶段也可以说算是告一段落。回想读博期间一路走来,其中有欣喜,也有难过;有深深的孤独,也有现在的恋恋不舍。如今终于到了要道别的时候,所以想借此机会给每一个支持和帮助我的人们好好说一声感谢与有缘再见。

% 首先,我要衷心感谢我的导师刘凯教授。您是我学术道路上的引路人,您的悉心指导对我产生了巨大影响。您的专业知识、学术见解和研究激情都激发了我不断超越自我的动力。您耐心地解答我的问题,指导我的实验,并对我的论文提出宝贵的建议。您对我的信任和鼓励使我更加自信地迈向学术领域的新阶段。我将永远铭记您对我的慷慨付出和关心。

% 其次,我要感谢西南交通大学戴朋林老师、重庆邮电大学张浩老师、重庆师范大学肖颗老师、重庆大学国家卓越工程师学院李楚照老师,以及实验室廖成武、金飞宇、任华玲、刘春晖、晏国志、胡峻菠、钟成亮、吴峻源等同学在本学位论文撰写和校对过程中提供的宝贵意见与无私帮助。

% 再次,我要感谢我的母亲刘菁女士。您生育抚养了我,感谢您对我的无私包容与关爱支持,如果可以,我希望把这篇论文献给您。您是我所见过最坚强的人,都说\qthis{为母则刚},但我也希望能有一天,您能放下心中的重担,为自己好好生活。

% 此外,我也要感谢实验室的师弟师妹们。在毕业之际,我们一起欢聚于重庆大学国家卓越工程师学院,一起度过了许多个日夜,也为我带来了难忘的回忆。

% 最后,我要特别感谢答辩主席中国科学院重庆绿色智能技术研究院尚明生教授和所有委员重庆大学郭松涛教授、西北工业大学王柱教授、重庆邮电大学高陈强教授、重庆大学古富强教授的仔细审查和评估。感谢你们在繁忙的工作中抽出时间来对我的研究进行评价,并给予我宝贵的意见和建议。同时,我还要感谢论文评审专家们,你们在匿名评审的过程中,以专业、客观的态度审查了我的论文。你们对我的研究提出的批评和建议,帮助我更好地认识到研究的不足之处,并鼓励我在今后的学术探索中不断进步和改进。衷心感谢各位论文评审专家与答辩委员专家的辛勤工作和付出。

% 感谢你们陪伴我度过漫长岁月,世界因你们更美好。
\vfill
\begin{flushright}
{\CJKfontspec{STXingkai} \Large 赵方瑜} \hspace*{3.5em}
\\  \hspace*{\fill} \\
{二〇二三年五月\hspace*{1em}于重庆}
\end{flushright}
\end{secretizeEnv}

\end{document}